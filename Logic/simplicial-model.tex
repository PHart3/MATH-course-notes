\documentclass[10pt,letterpaper,cm]{nupset}
\usepackage[bmargin=1in, tmargin=1in, lmargin=1.5in, rmargin=1in, includefoot]{geometry}
\linespread{1.1}
\usepackage{graphicx}
 \usepackage{enumitem}
 \usepackage{setspace}
 \usepackage[utf8]{inputenc}
\usepackage[english]{babel}
 \usepackage{stmaryrd}
 \expandafter\def\csname opt@stmaryrd.sty\endcsname
{only,shortleftarrow,shortrightarrow}
 \usepackage{bm}
\usepackage{amsfonts}
\usepackage{amssymb}
\usepackage{quiver}
\usepackage{pgfplots}
\usepackage{amsmath,amsthm}
\usepackage{lmodern}
\usepackage{tikz-cd}
\usepackage{faktor}
\usepackage{extpfeil}
\usepackage{xfrac}
\usepackage{mathtools}
\usepackage{bm}
\usepackage{xcolor}
\usepackage{soul}
\usepackage{mathpartir}
\usepackage{ dsfont }
\usepackage{mathrsfs}
\usepackage{hyperref}
\usepackage{url}
\usepackage{thmtools}
\usepackage[capitalise]{cleveref}
\usepackage{scrextend}
\usepackage{mdframed}
\usepackage{comment}
\usepackage{tcolorbox}


\usepackage[backend=bibtex]{biblatex}

\hypersetup{colorlinks=true, linkcolor=red,          % color of internal links (change box color with linkbordercolor)
    citecolor=black,        % color of links to bibliography
    filecolor=magenta,      % color of file links
    urlcolor=cyan           }
    
\theoremstyle{definition}
\newtheorem{definition}{Definition}[subsection]
\newtheorem{exmp}[definition]{Example}
\newtheorem{non-exmp}[definition]{Non-example}
\newtheorem{note}[definition]{Note}

\theoremstyle{theorem}
\newtheorem{theorem}[definition]{Theorem}
\newtheorem{lemma}[definition]{Lemma}
\newtheorem{corollary}[definition]{Corollary}
\newtheorem{prop}[definition]{Proposition}
\newtheorem{conj}[definition]{Conjecture}
\newtheorem*{claim}{Claim}
\newtheorem*{fact}{Fact}
\newtheorem{exercise}[definition]{Exercise}

\theoremstyle{remark}
\newtheorem{remark}[definition]{Remark}
\newtheorem*{todo}{To do}
\newtheorem*{conv}{Convention}
\newtheorem*{aside}{Aside}
\newtheorem*{notation}{Notation}
\newtheorem*{term}{Terminology}
\newtheorem*{background}{Background}
\newtheorem*{further}{Further reading}
\newtheorem*{sources}{Sources}

\newcommand{\vequiv}{\rotatebox[origin=c]{-90}{$\equiv$}}

\makeatletter
\renewcommand*\env@matrix[1][*\c@MaxMatrixCols c]{%
  \hskip -\arraycolsep
  \let\@ifnextchar\new@ifnextchar
  \array{#1}}
\makeatother
\pgfplotsset{unit circle/.style={width=4cm,height=4cm,axis lines=middle,xtick=\empty,ytick=\empty,axis equal,enlargelimits,xmax=1,ymax=1,xmin=-1,ymin=-1,domain=0:pi/2}}
\DeclareMathOperator{\Ima}{Im}
\newcommand{\A}{\mathcal A}
\newcommand{\C}{\mathbb C}
\newcommand{\D}{\mathcal D}
\newcommand{\E}{\vec E}
\newcommand{\CP}{\mathbb CP}
\newcommand{\F}{\mathbb F}
\newcommand{\G}{\vec G}
\renewcommand{\H}{\mathbb H}
\newcommand{\HP}{\mathbb HP}
\newcommand{\K}{\mathcal K}
\renewcommand{\L}{\mathcal L}
\newcommand{\RI}{\mathcal R}
\newcommand{\M}{\mathbb M}
\renewcommand{\O}{\mathbf O}
\newcommand{\OP}{\mathbb OP}
\renewcommand{\P}{\mathbf P}
\newcommand{\Q}{\mathbb Q}
\newcommand{\I}{\mathbb I}
\newcommand{\un}{\mathbf{U}}
\newcommand{\R}{\mathbb R}
\newcommand{\RP}{\mathbb RP}
\renewcommand{\S}{\mathbf S}
\newcommand{\X}{\mathbf X}
\newcommand{\Z}{\mathbb Z}
\newcommand{\B}{\mathcal{B}}
\newcommand{\ds}{\displaystyle}
\newcommand{\ran}{\right>}
\newcommand{\lan}{\left<}
\newcommand{\bmat}[1]{\begin{bmatrix} #1 \end{bmatrix}}
\newcommand{\sra}{\shortrightarrow}
\newcommand{\hooklongrightarrow}{\lhook\joinrel\longrightarrow}

\newlist{steps}{enumerate}{1}
\setlist[steps, 1]{label = Step \arabic*:}

\newcommand{\h}{\vec h}
\newcommand{\f}{\vec f}
\newcommand{\g}{\vec g}
\renewcommand{\i}{\vec i}
\renewcommand{\k}{\vec k}
\newcommand{\n}{\vec n}
\newcommand{\p}{\vec p}
\newcommand{\q}{\vec q}
\renewcommand{\r}{\vec r}
\newcommand{\s}{\vec s}
\renewcommand{\t}{\vec t}
\renewcommand{\u}{\vec u}
\renewcommand{\v}{\vec v}
\newcommand{\w}{\vec w}
\newcommand{\x}{\vec x}
\newcommand{\z}{\vec z}
\DeclareMathOperator*{\Span}{span}
\DeclareMathOperator*{\GL}{GL}
\DeclareMathOperator*{\SL}{SL}
\DeclareMathOperator*{\SO}{SO}
\DeclareMathOperator*{\SU}{SU}
\DeclareMathOperator{\rng}{range}
\DeclareMathOperator{\ft}{ft}
\DeclareMathOperator{\gemu}{gemu}
\DeclareMathOperator{\almu}{almu}
\DeclareMathOperator{\Char}{\mathsf{char}}
\DeclareMathOperator{\im}{im}
\DeclareMathOperator{\graph}{Graph}
\DeclareMathOperator{\gal}{Gal}
\DeclareMathOperator{\tr}{Tr}
\DeclareMathOperator{\norm}{N}
\DeclareMathOperator{\aut}{Aut}
\DeclareMathOperator{\Int}{Int}
\DeclareMathOperator{\ext}{Ext}
\DeclareMathOperator{\stab}{Stab}
\DeclareMathOperator{\orb}{Orb}
\DeclareMathOperator{\inn}{Inn}
\DeclareMathOperator{\out}{Out}
\DeclareMathOperator{\fix}{Fix}
\DeclareMathOperator{\ab}{ab}
\DeclareMathOperator{\sgn}{sgn}
\DeclareMathOperator{\syl}{syl}
\DeclareMathOperator{\Syl}{Syl}
\DeclareMathOperator{\ob}{Ob}
\DeclareMathOperator{\mor}{Mor}
\DeclareMathOperator{\sym}{sym}
\DeclareMathOperator{\red}{red}
\DeclareMathOperator{\ev}{ev}
\DeclareMathOperator{\Ty}{Ty}
\DeclareMathOperator{\Tm}{Tm}



%Our function symbols for our signature (not including universe closure symbols, \mathsf{0}, or lambda symbol):
\newcommand{\J}{\mathsf{J}}
\newcommand{\id}{\mathsf{Id}}
\newcommand{\refl}{\mathsf{refl}}
\newcommand{\app}{\mathsf{app}}
\renewcommand{\split}{\mathsf{split}}
\newcommand{\ind}{\mathsf{ind}}
\newcommand{\pair}{\mathsf{pair}}
\newcommand{\case}{\mathsf{case}}
\newcommand{\U}{\mathsf{U}}
\newcommand{\el}{\mathsf{el}}
\newcommand{\univv}{\mathsf{univ}}
\newcommand{\0}{\mathbf{0}}
\newcommand{\1}{\mathbf{1}}
\newcommand{\2}{\mathbf{2}}

%Primitive strings in our meta-language:
\DeclareMathOperator{\ctx}{\mathtt{ctx}}
\DeclareMathOperator{\type}{\mathtt{type}}

%Our abbreviations used in our meta-language:

\DeclareMathOperator{\inv}{\mathtt{inv}}
\DeclareMathOperator{\isprop}{\mathtt{is\_prop}}
\DeclareMathOperator{\retr}{\mathtt{retr}}
\DeclareMathOperator{\sect}{\mathtt{sec}}
\DeclareMathOperator{\ac}{\mathtt{ac}}
\DeclareMathOperator{\assoc}{\mathtt{assoc}}
\DeclareMathOperator{\lunit}{\mathtt{l\_unit}}
\DeclareMathOperator{\runit}{\mathtt{r\_unit}}
\DeclareMathOperator{\linv}{\mathtt{l\_inv}}
\DeclareMathOperator{\rinv}{\mathtt{r\_inv}}
\DeclareMathOperator{\concat}{\mathtt{concat}}
\DeclareMathOperator{\cons}{\mathtt{cons}}
\DeclareMathOperator{\comp}{\mathtt{comp}}
\DeclareMathOperator{\idmap}{\mathtt{idmap}}
\DeclareMathOperator{\transport}{\mathtt{transport}}
\DeclareMathOperator{\ap}{\mathtt{ap}}
\DeclareMathOperator{\apd}{\mathtt{apd}}
\DeclareMathOperator{\happly}{\mathtt{hApply}}
\DeclareMathOperator{\isequiv}{\mathtt{is\_equiv}}
\DeclareMathOperator{\hfiber}{\mathtt{hFiber}}
\DeclareMathOperator{\iscont}{\mathtt{is\_contr}}
\DeclareMathOperator{\iscontmap}{\mathtt{is\_contr\_map}}
\DeclareMathOperator{\iso}{\mathtt{iso}}
\DeclareMathOperator{\hpyconcat}{\mathtt{htpy\_concat}}
\DeclareMathOperator{\pr}{\mathtt{pr}}
\DeclareMathOperator{\isset}{\mathtt{is\_set}}
\DeclareMathOperator{\isgrp}{\mathtt{is\_group}}
\DeclareMathOperator{\isoeq}{\mathtt{iso\_eq}}
\DeclareMathOperator{\homm}{\mathtt{hom}}
\DeclareMathOperator{\equiveq}{\mathtt{idtoequiv}}

\DeclareMathOperator{\fv}{FV}

\DeclareMathOperator{\cdtt}{\mathrm{CDTT}}
\DeclareMathOperator{\univ}{\mathrm{Univ}}
\DeclareMathOperator{\wfe}{\mathrm{WFE}}
\DeclareMathOperator{\sfe}{\mathrm{FE}}
\DeclareMathOperator{\uip}{\mathrm{UIP}}
\DeclareMathOperator{\err}{\mathrm{ERR}}




\DeclareMathOperator{\op}{op}
\DeclareMathOperator{\sset}{\mathbf{sSet}}
\DeclareMathOperator{\expr}{exp}
\DeclareMathOperator{\set}{\mathbf{Set}}
\DeclareMathOperator{\Ab}{\mathbf{Ab}}
\DeclareMathOperator{\Cmon}{\mathbf{CMon}}
\DeclareMathOperator{\spec}{Spec}
\DeclareMathOperator{\rank}{rank}
\DeclareMathOperator{\rk}{rk}
\DeclareMathOperator{\colimm}{colim}
\DeclareMathOperator{\hocolimm}{hocolim}
\DeclareMathOperator{\holimm}{holim}
\DeclareMathOperator{\diag}{diag}
\DeclareMathOperator{\Ar}{Arr}
\DeclareMathOperator{\pb}{Pb}
\DeclareMathOperator{\trr}{tr}
\DeclareMathOperator{\sk}{sk}
\DeclareMathOperator{\Sk}{Sk}
\DeclareMathOperator{\Lan}{Lan}
\DeclareMathOperator{\fp}{fp}

\makeatletter
\newcommand{\colim}{\gen@colim{\colimm}}
\newcommand{\gen@colim}[1]{%
  \@ifnextchar_{\gen@@colim{#1}}{\mathbin{#1}}%
}
\def\gen@@colim#1_#2{%
  \mathpalette\gen@@@colim{{#1}{#2}}%
}
\newcommand\gen@@@colim[2]{\mathbin{\gen@@@@colim#1#2}}
\newcommand\gen@@@@colim[3]{%
  \ifx#1\displaystyle
    \mathop{#2}\limits_{#3}%
  \else
    {#2}_{#3}%
  \fi
}
\makeatother

\makeatletter
\newcommand{\hocolim}{\gen@colim{\hocolimm}}
\newcommand{\gen@hocolim}[1]{%
  \@ifnextchar_{\gen@@hocolim{#1}}{\mathbin{#1}}%
}
\def\gen@@hocolim#1_#2{%
  \mathpalette\gen@@@hocolim{{#1}{#2}}%
}
\newcommand\gen@@@hocolim[2]{\mathbin{\gen@@@@hocolim#1#2}}
\newcommand\gen@@@@hocolim[3]{%
  \ifx#1\displaystyle
    \mathop{#2}\limits_{#3}%
  \else
    {#2}_{#3}%
  \fi
}
\makeatother

\makeatletter
\newcommand{\holim}{\gen@colim{\holimm}}
\newcommand{\gen@holim}[1]{%
  \@ifnextchar_{\gen@@holim{#1}}{\mathbin{#1}}%
}
\def\gen@@holim#1_#2{%
  \mathpalette\gen@@@holim{{#1}{#2}}%
}
\newcommand\gen@@@holim[2]{\mathbin{\gen@@@@holim#1#2}}
\newcommand\gen@@@@holim[3]{%
  \ifx#1\displaystyle
    \mathop{#2}\limits_{#3}%
  \else
    {#2}_{#3}%
  \fi
}
\makeatother

\newcommand{\RomanNumeralCaps}[1]
    {\MakeUppercase{\romannumeral #1}}

\renewcommand{\a}{\mathscr{A}}
\renewcommand{\b}{\mathscr{B}}
\renewcommand{\c}{\mathscr{C}}
\renewcommand{\d}{\mathscr{D}}
\newcommand{\e}{\mathscr{E}}
\renewcommand{\j}{\mathscr{J}}
\newcommand{\y}{\mathscr{Y}}
\newcommand{\rr}{\mathscr{R}}

\newcommand{\N}{\mathbb N}
\newcommand{\T}{\mathbb T}

\DeclareMathOperator{\Sp}{Sp}
\DeclareMathOperator{\Hom}{Hom}
\DeclareMathOperator{\Fun}{Fun}
\DeclareMathOperator{\cone}{cone}
\DeclareMathOperator{\idd}{id}
\DeclareMathOperator{\fib}{\mathnormal{Fib}}
\DeclareMathOperator{\cof}{\mathnormal{Cof}}
\DeclareMathOperator{\we}{\mathnormal{W}}
\DeclareMathOperator{\dom}{dom}
\DeclareMathOperator{\cod}{cod}
\DeclareMathOperator{\cell}{cell}
\DeclareMathOperator{\ret}{ret}
\DeclareMathOperator{\rel}{rel}
\DeclareMathOperator{\rlp}{rlp}
\DeclareMathOperator{\llp}{llp}
\DeclareMathOperator{\nondeg}{nondeg}
\DeclareMathOperator{\trans}{\mathnormal{trans}}
\DeclareMathOperator{\ch}{\mathbf{Ch}}
\DeclareMathOperator{\Mod}{\mathbf Mod}


\newmdenv[leftline=false,rightline=false]{topbot}


\def \RightTirName #1{\rm\hbox {\hskip 1ex (#1)}}

\newcommand{\bi}{\begin{itemize}}
\newcommand{\ei}{\end{itemize}}

\newcommand{\be}{\begin{enumerate}}
\newcommand{\ee}{\end{enumerate}}

\newcommand{\bmp}{\begin{mathpar}}
\newcommand{\emp}{\end{mathpar}}

\setlength{\parindent}{0pt}

\makeatletter
\def\th@plain{%
  \thm@notefont{}% same as heading font
  \itshape % body font
}
\def\th@definition{%
  \thm@notefont{}% same as heading font
  \normalfont % body font
}
\makeatother

\addbibresource{simplicial.bib}


% info for header block in upper right hand corner

\name{Perry Hart}
\class{June 7, 2020}

%\pagenumbering{roman}
%\setcounter{page}{1}

\begin{document}
\begin{abstract}
%\doublespacing
 After presenting our variant of constructive dependent type theory ($\cdtt$), we develop the language necessary to postulate Voevodsky's univalence axiom ($\univ$), which formally encodes the identification of equivalent objects in any categorical model of $\cdtt + \univ$. Afterwards, we describe the original construction of a model of $\cdtt+\univ$ in the (Quillen)  model category of simplicial sets, due mainly to Voevodsky.
\end{abstract}

%\pagebreak
\tableofcontents
\addtocontents{toc}{\protect\thispagestyle{empty}}
\pagenumbering{gobble}
\pagebreak

\addcontentsline{toc}{section}{Overview}
\section*{Overview}
\pagenumbering{arabic}
\setcounter{page}{1}

Our ultimate goal is to construct a certain  model category  in which every theorem of $\cdtt +\univ$ is true.
As we shall see, this model category interprets a dependent type as a fibration. The univalence axiom is so named because in any model category satisfying it, the canonical fibration over a chosen universe of types $\U$ is univalent, i.e.,  every fibration with small enough fibers is an essentially unique pullback of it. In the language of $\infty$-category theory, this means that a univalent fibration is a classifier for the class of all such fibrations.

\medskip

It need \emph{not} be the case, though, that $\cdtt+\univ$ is modeled by any model category with object classifiers. In the categorical semantics of $\cdtt$, the syntactic substitution of a term $t$ for a variable occurring in a dependent type $B$ is interpreted as a pullback of the fibration interpreting $B$ along the morphism interpreting $t$. But substitution is strictly functorial, whereas pullback is merely functorial up to isomorphism. Thus, any model of $\cdtt+\univ$ needs at least enough structure to make pullbacks in it along certain fibrations strictly associative. Our chosen  model category will have such structure, as well as enough structure to model the strict behavior of the type constructors of $\cdtt$. Finally, we must find an object classifier in our model category that is strictly ``closed" under these type constructors. 

\section{Martin-L\"of  dependent  type theory (MLDTT)}

In this section and \cref{itypes}, we present a particular variant of (intensional)  Martin-L\"of  dependent  type theory, another name for constructive dependent type theory, in honor of the Swedish logician Per Martin-L\"of. In the type theory literature, there are many other variants of the same theory. These differ from ours only in which logical/type constructors they include. The more one includes, the more expressive it is. Usually, other variants include  at least our constructors for the unit type, the dependent product, and the dependent sum.

\subsection{Syntax}

A \textit{Martin L\"of dependent type theory} is a system of natural deduction whose object- and meta-languages are defined as follows. (See \cref{natded} for a review of deductive systems.) 

\subsection*{Object language}

Our presentation of the object language is inspired by \cite{Gambino}.

\bigskip

First of all, we are given a countably infinite set of \textit{variables} $$\mathcal{V} \coloneqq \left\{v_0, v_1, v_2, \ldots \right\}$$ along with the auxiliary symbols \lq{:}\rq,  \lq{,}\rq , \lq{(}\rq, and \lq{(}\rq.  Note that $\mathcal{V} $ inherits the well-ordering $\leq$ of $\N$.

\medskip

We want to build our language out of variables and auxiliary symbols

\begin{definition}\label{sig} $ $
\be
\item An \textit{arity} is a tuple of the form $$ \left(\left(n_{1}, \beta_{1}\right),\left(n_{2}, \beta_{2}\right), \ldots,\left(n_{k}, \beta_{k}\right), \beta\right)     $$ where $k\in \N$ and $\beta, \beta_i \in \{0,1\}$ for each $1\leq i \leq k$. Let $\mathit{Ar}$ denote the set of all arities. 
\item A \textit{signature} is a pair $\left(\Sigma^{\sym}, \alpha\right)$ consisting of a set $\Sigma^{\sym}$ of \textit{logical symbols/constructors} and a function $\alpha : \Sigma^{\sym} \to \mathit{Ar}$. The value $$\alpha(s) = \left(\left(n_{1}, \beta_{1}\right),\left(n_{2}, \beta_{2}\right), \ldots,\left(n_{k}, \beta_{k}\right), \beta\right) $$ is called the \textit{arity of $s$}.  If $s$ has arity $0$, then it is called a \textit{term-valued symbol}. If it has arity $1$, then it is called a \textit{type-valued symbol}.
\ee
\end{definition}



\begin{definition}[$0$- and $1$-expressions]
Let $\left(\Sigma^{\sym}, \alpha\right)$ be a signature. By mutual recursion, define the set $\Sigma_0^{\sym}$ of \textit{$0$-expressions} and the set $\Sigma_1^{\sym}$ of \textit{$1$-expressions} so that
\be[label =(\roman*)]
\item every variable is a $0$-expression and
\item if $s\in \Sigma^{\sym}$ has arity $\left(\left(n_{1}, \beta_{1}\right), \ldots,\left(n_{k}, \beta_{k}\right), \beta\right)$ and $M_i$ is a $\beta_i$-expression and $x^i_1, x^i_2, \ldots, x^i_{n_i}$ is a list of pairwise distinct variables for each $i \in \{1, 2, \ldots, k\}$, then 
$$
s\big(x^1_1.x^1_2.\ldots.x^1_{n_1}.M_{1},\ \ldots,\  x^k_1.x^k_2.\ldots.x^k_{n_k}.M_{k}\big)
$$
is a $\beta$-expression.
\ee
If $k=0$, then we write $s$ instead of $s()$ and say that $s$ is a \textit{constant symbol}.
\end{definition}

\begin{term}
Other names for a $0$-expression and $1$-expression are \textit{term expression/constructor} and \textit{type expression/constructor}, respectively.
\end{term}

\medskip

Think of the arity $$\alpha(s) = \left(\left(n_{1}, \beta_{1}\right),\left(n_{2}, \beta_{2}\right), \ldots,\left(n_{k}, \beta_{k}\right), \beta\right)$$ of $s$ as specifying an operation that 
\be[label=(\alph*)]
\item
takes $k$ expressions as inputs (the sort of each indicated by $\beta_i$) with $n_i$ pairwise distinct variables bound in the $i$-th input  and 
\item outputs a new expression whose sort is indicated by $\beta$. 
\ee

\medskip

Let us add one more kind of expression to the object language.

\begin{definition}[Context]
A \textit{context} is a list of the form 
\[
 x_{1} : A_{1}, x_{2} : A_{2},\ldots, x_{n} : A_{n} 
\] such that 
\bi
\item each $x_i$ denotes a variable,
\item each $A_i$ denotes a $1$-expression, and
 \item for any $i, j\geq 1$ with $i\ne j$, we have that $x_i$ and $x_j$ denote distinct variables, i.e., each variable in the list is a \textit{fresh} variable.
 \ei
  We say that the context \textit{declares} the variables $x_1, \ldots, x_n$.
\end{definition}

\medskip
The set of raw terms / expressions is taken to be $\Sigma_0^{\expr} \cup \Sigma_1^{\expr} \cup \mathcal{X}$ where $\mathcal{X}$ denotes the set of all contexts. 

\medskip

 Thus, the object language of a particular dependent type theory is determined by its signature. 
 \smallskip
 
 All but one logical symbol in \emph{our specific} MLDTT will appear in at least one of the inference rules postulated in \cref{logrules}, \cref{univ}, or \cref{itypes}. The other one (namely $\univv$) will appear in the univalence axiom (\cref{uaxiom}). The arity of each logical symbol will be evident. 

\begin{table}[h!]
\centering
\caption{A fragment of our signature}
\label{table:1}
\begin{tabular}{||c c||} 
 \hline
Symbol & Arity \\ [0.5ex] 
 \hline\hline
 $\Pi$ & ((0,1), (1,1), 1) \\
 $ \lambda$ &  ((0,1), (1,1), (1,0), 0) \\
  $\Sigma$ & ((0,1), (1,1), 1) \\
   $\0$ & (1) \\
 $\1$ & (1) \\ 
  $\2$ & (1) \\ 
   $0_{\2}$ & (0) \\  
 $\U$ & (1) \\ 
 $\el$ & ((0,0), 1) \\
 $\id$ & ((0,1), (0,0), (0,0), 1) \\ [1ex] 
 \hline
\end{tabular}
\end{table}

\begin{definition}[Free variable] $ $
\be
\item Let $t$ be a raw term other than a context. If $t$ is a variable, then we say that $t$ is \textit{free in itself}. 

Otherwise, by construction, $t$ is of the form  $$s\left(x^1_1.x^1_2.\ldots.x^1_{n_1}.M_{1},\ \ldots,\  x^k_1.x^k_2.\ldots.x^k_{n_k}.M_{k}\right)  .$$  If $x$ is a variable occurring in $t$ and $x\notin \left\{x_j^i \mid i\in \{1, \ldots, k\},\ j\in \{1, \ldots, n_i\}\right\}$, then $x$ is \textit{free in $t$}. 
If $x = x_j^i$ for some $i$ and $j$, then we say that $x$ is \textit{bound in $M_i$}.
\item For any context $\underbrace{x_1:A_1, \ldots, x_n :A_n}_{\Gamma}$, a variable $x$ is \textit{free in $\Gamma$} if  $x= x_i$ for some $1\leq i \leq n$ or $x$ is free in $A_i$ for some $i$.
\ee
The set of free variables in an expression $\rho$ is denoted by $\fv(\rho)$.
\end{definition}

\begin{exmp} $ $
\be
\item If $A$ and $B$  are $1$-expressions and $x$ is a variable, then the raw term $\Pi_{x:A}B \coloneqq \Pi(A, x.B)$ is a $1$-expression with $x$ bound in $B$. 

This is similar to a first-order formula such as  $\forall{x}.\psi$, where $x$ is bound in $\psi$.  
\item If $A$ and $B$ are $1$-expressions, $t$ is a term, and $x$ is a variable, then the raw term $\lambda(x:A).t \coloneqq \lambda(A, x.A, x.t)$ is a $0$-expression with $x$ bound in $A$ and $t$. 

This is similar to a first-order definition of a function such as $f(x) = x+yz$, where $x$ is bound in $x+yz$.
\ee
\end{exmp}

\subsection*{Meta-language}

By assumption, the meta-language contains a countably infinite set of \textit{meta-variables}, which range over raw terms.  
\begin{notation} $ $
\bi 
\item The symbols $x_i$, $y_i$, $x$, $y$, $z$, $x'$, etc.\  will refer to arbitrary variables. 
\item The symbols $\Gamma$, $\Delta$, $\Theta$, $\Gamma'$, $\Gamma_i$, etc.\  will refer to arbitrary contexts.
\item The symbols $a$, $b$, $c$, $d$, $e$, $f$, $g$, $h$, $m$, $t$, $s$, $t'$, $t_i$, $s_i$, $\tau$, etc.\ will refer to arbitrary $0$-expressions. 
\item The symbols $A$, $B$, $C$, $A'$, $B'$, $C'$, $A_i$, etc.\  will refer to arbitrary $1$-expressions.
\ei
For readability, we may write $A\left(x_1, \ldots, x_n\right)$ in the meta-language to indicate that the variables occurring in the $1$-expression denoted by $A$ include $x_1, \ldots, x_n$.
\end{notation}

\bigskip

Any judgment will have one of six forms.

\begin{topbot}
\begin{minipage}{5.5 in}
\be
\item (well-formed context) ``$\Gamma$ is a well-formed context," formally, $$ \ctx(\Gamma)   .$$
\item (equality of contexts) ``$\Gamma$ and $\Delta$ are judgmentally equal well-formed contexts," formally, $$ \Gamma \equiv \Delta \ctx  .$$ 
\item (typehood) ``$A$ is a  well-formed type in  context $\Gamma$," formally, $$\Gamma \vdash A \type   .$$
\item (typing declaration) ``$a$ is a (well-formed) term of type $A$ / inhabiting $A$ in context $\Gamma$,"  formally, $$\Gamma \vdash a : A.$$
\item (equality of types) ``$A$ and $B$ are judgmentally equal well-formed types in context $\Gamma$,"  formally, $$ \Gamma \vdash A \equiv B \type  .$$
\item (equality of terms) ``$a$ and $b$ are judgmentally equal well-formed terms of type $A$ in context $\Gamma$,"  formally, $$ \Gamma \vdash a\equiv b : A    .$$
\ee
\end{minipage}
\end{topbot}

\medskip 

A \textit{generic judgment} refers to any judgment with one of the last four forms. A generic judgment  consists of an antecedent $\Gamma$ and a consequent, e.g., $A \type$. We call such a judgment a \textit{hypothetical judgment}, thinking of $\Gamma$ as a list of hypotheses. For example, the theory of a category $\c$ has as an axiom $x:\mathsf{obj}, y:\mathsf{obj} \vdash \mathsf{hom}(x,y) \type$. Intuitively, this is intended to mean that $\mathsf{hom}(x,y)$ is a well-formed type whenever $x$ and $y$ are objects in $\c$.

\begin{notation}
The symbol $\K$ will denote the consequent of a generic judgment.
\end{notation}

\medskip

Note that the inference rules of a MLDTT induce six relations on the set of all raw terms, which determine the subset of raw terms that are \textit{well-formed}. 

\bigskip

Finally, we define a family of total operations on expressions and then define another such family in terms of it.

\begin{definition}[Capture-free substitution] Let $y$ be a variable and $t$ be a term expression. Let $\rho$ be any expression. Define the \textit{(capture-free) substitution of $t$ for (free occurrences of) $y$ in $\rho$}, denoted by $$ \rho\left[t/y\right] ,   $$ as follows.
\be
\item Suppose that $\rho$ is not a context.  Then $\rho\left[t/x\right]$ is the finite string obtained recursively by
\be
\item $x\left[t/y\right] = \begin{cases} t & x = y \\ x & x\ne y\end{cases} , $
\item $\kappa\left[t/y\right]  = \kappa$ where $\kappa$ denotes a constant symbol, and
\item 

 $s\left(x^1_1.x^1_2.\ldots.x^1_{n_1}.M_{1},\ \ldots,\  \underbrace{x^k_1.x^k_2.\ldots.x^k_{n_k}.M_{k}}_{k\geq 1}\right)\left[t/y\right]   =
  s\left(N_1, \ldots, N_k \right)$

where for each $i\in \{1, \ldots, k\}$,
\[
N_i = 
\begin{cases}
\tilde{x}^i_1.\tilde{x}^i_2.\ldots.\tilde{x}^i_{n_i}.M_{i}\left[t/y\right]  & y \notin \left\{x^i_1, \ldots, x^i_{n_i}\right\} \\ 
x^i_1.x^i_2.\ldots.x^i_{n_i}.M_{i} &  \text{otherwise}
\end{cases}
\]
such that for any $1\leq j \leq n_i$, we choose $\tilde{x}_j^i$ to be the least variable $z \geq x_j^i$ with \linebreak $z \notin \{\tilde{x}^i_1, \ldots, \tilde{x}_{j-1}^i\} \cup \fv(M_i) \cup \fv(t)$.
\ee
\item If $\rho$ is a context, then $ \rho\left[t/y\right]$ is the context obtained recursively by
\be
\item $\epsilon\left[t/y\right] = \epsilon$, where $\epsilon$ denotes the empty list, and
\item $\left(x_1 : A_1, \ldots, \underbrace{x_k :A_k}_{k\geq 1}\right)\left[t/y\right] = $
\[ \begin{cases}
(x_1 :A _1, \ldots, x_{k-1} : A_{k-1})\left[t/y\right], \tilde{x}_k :A_k\left[t/y\right] & x_i \ne y, \ i = 1, \ldots, k
\\
\epsilon & \text{otherwise}
\end{cases}
\]  where we choose $\tilde{x}_k$ to be the least variable $z \geq x_k$ such that $z\notin \{x_1, \ldots, x_{k-1}\} \cup \fv(t)$.
\ee
\ee
We extend this definition in the obvious way to a definition of the \textit{substitution of $t$ for $x$ in $\K$}, denoted by $\K\left[t/x\right]$.
\end{definition}

\begin{remark}
We have defined substitution so as to avoid obtaining a raw term in which variables intended to be free are \textit{captured}, i.e., become bound. 
For example, in the language of arithmetic, naively substituting the variable $y$ for $x$ in the formula $\exists y(x+y=1) $ results in $\exists y(y+y=1)$, which is not logically equivalent to the original formula. Rather, one ought to first convert $\exists y(x+y=1) $ to something like $\exists z(x+z=1)$ and then naively substitute $y$ for $x$ in it.
\end{remark}

\begin{definition}[Simultaneous substitution]\label{ssub} $ $
Let $x_1, \ldots, x_n$ be pairwise distinct variables and let $t_1, \ldots, t_n$ be term expressions such that $$\left(\fv(t_1) \cup \cdots \cup \fv(t_n)\right) \cap \left\{x_1, \ldots, x_n\right\} =\emptyset.$$ Let $\rho$ be any expression. Define the \textit{simultaneous substitution of   $t_1, \ldots, t_n$ for $x_1, \ldots, x_n$ in $\rho$}, denoted by $$ \rho\left[t_1, \ldots, t_n / x_1, \ldots, x_n\right]   ,$$  as the term  $$ \rho\left[t_1/x_1\right]\left[t_2/x_2\right]\cdots \left[t_n/x_n\right]     ,$$ which is obtained by iterated substitution. We extend this definition in the obvious way to a definition of the \textit{simultaneous substitution of $t_1, \ldots, t_n$ for $x_1, \ldots, x_n$  in $\K$}, denoted   by $\K\left[t_1, \ldots, t_n / x_1, \ldots, x_n\right]$.
\end{definition}

\begin{exmp} $ $
\be
\item $\left(\lambda (x:A).y\right)[z/x] =\lambda (x:A).y.$
\item $\left(\lambda (v_1:A).v_2\right)\left[v_0, v_3/ v_2, v_1\right] = \lambda(v_1:A).v_0$. 
\ee
\end{exmp}

\subsection{Structural rules} 
We also require any  MLDTT to include certain inference rules known as \textit{structural rules}, which we now list.

\smallskip

First, we postulate four structural rules, which govern the formation and equality of well-formed contexts:
\bmp
\def \RightTirName #1{\rm\hbox {\hskip 1ex (#1)}}
\inferrule*[right={where $\epsilon$ denotes the empty list}]{ }{\ctx(\epsilon)}
\\
\inferrule*[right={when $x_n\notin \left\{x_1, \ldots, x_{n-1}\right\}$}]
{x_1:A_1, \ldots, x_{n-1}:A_{n-1} \vdash A_n \type}{\ctx\left(x_1 :A_1, \ldots, x_n:A_n\right)}
\\
\inferrule*{ }{  \epsilon \equiv \epsilon \ctx } \and
\inferrule*[right={when $x\notin \fv(\Gamma)$ and $y\notin \fv(\Delta)$}]{ \Gamma \equiv \Delta \ctx \\ \Gamma \vdash A \equiv B \type}{ \Gamma, x: A \equiv \Delta, y:B \ctx}
\emp
where $$x_1:A_1, \ldots, x_{n-1}:A_{n-1}$$ is, by convention, the empty context when $n=1$.
It follows that a context is well-formed exactly when it is either empty or an expression of the form $$ x_{1} : A_{1}, x_{2} : A_{2},\ldots, x_{n} : A_{n}  $$ such that 
\bi 
\item if $2 \leq k \leq n$, then $$ x_{1} : A_{1}, x_{2} : A_{2}, \ldots, x_{k-1} : A_{k-1} \vdash A_{k}  \type$$ and
\item if $n=1$, then $$\vdash A_1 \type.$$ In this case, we say that $A_1$ is a \textit{closed type}. Also, if $a$ is a term of type $A$ in the empty context, then we say that $a$ is a \textit{closed term}.
\ei

\begin{note}
It turns out that, in light of all of our inference rules to be presented,  the following   meta-theoretic property will be true of our system:
\smallskip
\begin{addmargin}{2em}
For any well-formed context $$x_1 :A_1, \ldots, x_n :A_n, $$ we have that $\fv(A_i) \subset \{x_1, \ldots, x_{i-1}\}$. Moreover, if we can derive both $\ctx(\Gamma)$ and $\Gamma \vdash a: A$, then any free variable in either  $a$ or $A$ must be declared by $\Gamma$. 
\end{addmargin}
\end{note}

\medskip

In addition, we postulate the rules 
\bmp
\inferrule*{\ctx(\Gamma)}{ \Gamma \equiv \Gamma \ctx}
\and
\inferrule*{ \Gamma \equiv \Delta \ctx}{ \Delta \equiv \Gamma \ctx}
\and
\inferrule*{ \Gamma \equiv \Delta \ctx \\  \Delta \equiv \Theta \ctx}{ \Gamma \equiv \Theta \ctx}
, 
\emp
which together assert that equality of contexts is an equivalence relation.


\bigskip

To be able to manipulate variables for convenience in our derivations, we postulate four more structural rules:  
\bmp
\inferrule*[Left = Vble]
{\ctx(\Gamma, x:A, \Delta) }
{\Gamma, x:A, \Delta \vdash x:A}
\and
\inferrule*[Left = Subst]
{\Gamma \vdash a:A \\ \Gamma, x:A, \Delta \vdash \K }
{\Gamma, \Delta[a/x]\vdash \K[a/x]}
\\
\inferrule*[Left = Wkg, right={when $x$ is not free in $\Gamma, \Delta$}]
{\Gamma \vdash A \type  \\ \Gamma,  \Delta \vdash \K }
{\Gamma, x:A, \Delta \vdash \K}
\\
\inferrule*[Left = Exchange, right={when $x$ is not free in $B$}]
{\Gamma, x: A, y: B, \Delta \vdash \K }
{\Gamma, x:B, y:A, \Delta \vdash  \K}
\\
\inferrule*[Left = $\alpha$-conv-ctx, right={when $x'$ is not free in $\Gamma, x:A, \Delta$}]
{\Gamma, x:A , \Delta \vdash \K}{\Gamma, x' :A, \Delta[x'/x] \vdash \K[x'/x]}.
\emp
\be[label=(\roman*)]
\item The \textit{variable rule \textsc{Vble}} asserts that each declared variable in a well-formed context is well-typed. 
\item The \textit{substitution rule \textsc{Subst}} asserts that substituting a declared variable with a term of the same type preserves $\K$.
\item The \textit{weakening rule \textsc{Wkg}} asserts that expanding the context by a fresh variable of type $A$ (known as \textit{weakening by $A$}) preserves $\K$.
\item The \textit{exchange rule \textsc{Exchange}} asserts that certain permutations of the context preserve $\K$.
\item The \textit{context $\alpha$-conversion} $\alpha$-\textsc{conv}-\textsc{ctx}  rule asserts that renaming a declared variable as a fresh variable preserves $\K$. 
\ee
Additionally, we postulate the structural rules
\bmp
\inferrule{\Gamma \vdash a \equiv b : A \\ \Gamma, x:A , \Delta \vdash B \type}{\Gamma, \Delta[a/x]\vdash B[a/x]\equiv B[b/x]\type}
\and
\inferrule{\Gamma \vdash a \equiv a' : A \\ \Gamma, x:A , \Delta \vdash b: B}{\Gamma, \Delta[a/x]\vdash b[a/x]\equiv b[a'/x]: B[a/x]}
,\emp
which assert certain congruence conditions for judgmental equality of terms.

\medskip
Next, we postulate those structural rules governing judgmental equality of types. Specifically, we have the rules
\bmp
\inferrule{\Gamma \vdash  A  \type  }{\Gamma \vdash A \equiv A \type} \and 
\inferrule{\Gamma \vdash A \equiv B \type    }{\Gamma \vdash B \equiv A \type} \and 
\inferrule{\Gamma \vdash A \equiv B \type   \\ \Gamma \vdash B \equiv C \type  }{\Gamma \vdash A \equiv C \type}
, \emp
 which together assert that judgmental equality of types is an equivalence relation, along with the \textit{variable conversion} rule
 \[
 \inferrule{\Gamma \vdash  A \equiv B  \type \\ \Gamma, x :A, \Delta \vdash \K }{\Gamma, x:B, \Delta \vdash \K}
 .\] 
 Finally, we postulate  those structural rules governing  judgmental equality of terms.
\bmp
\inferrule{\Gamma \vdash a : A    }{\Gamma \vdash a \equiv a : A} \and 
\inferrule{\Gamma \vdash a \equiv b: A    }{\Gamma \vdash b \equiv a : A} \and 
\inferrule{\Gamma \vdash a \equiv b: A   \\ \Gamma \vdash b \equiv c : A  }{\Gamma \vdash a \equiv c : A}
\\
\inferrule{\Gamma \vdash a : A \\ \Gamma \vdash A \equiv B  \type   }{\Gamma \vdash a : B}
\and
\inferrule{\Gamma \vdash a \equiv b: A \\ \Gamma \vdash A \equiv B  \type   }{\Gamma \vdash a \equiv b: B}
. \emp
Together, these assert that judgmental equality of terms is an equivalence relation respected by typing. 

\subsection{Logical rules}\label{logrules}

In addition to our structural rules, our MLDTT postulates certain inference rules known as \textit{logical rules}. These allow us to define various types inductively. We can describe five main kinds of logical rules.
\be
\item A \textit{type formation} rule, which asserts those conditions under which we can use a type constructor to form a new type $B$.
\item A \textit{term introduction} rule, which asserts those conditions under which we can use a  a term constructor to form a term of type $B$.

This term is called a \textit{canonical term} of type $B$.
\item A \textit{term elimination} rule (sometimes called an \textit{induction principle}), which asserts that to define a ``section over $B$" it is both necessary and sufficient to define it on the term constructors for $B$. That is, it is enough to define it on the canonical terms of type $B$.

\item A \textit{computation rule}, whose conclusion is a judgmental equality allowing us to rewrite the result  of applying term elimination to a term formed by term introduction. 
\item A \textit{congruence rule}, which asserts that a given logical constructor preserves judgmental equality in each of its arguments.
\ee
An \textit{inductive type} is governed by at least a type formation rule and an induction principle. An inductive type is \textit{non-degenerate} if it is governed by all five kinds of rule (among others). 

\subsection*{Dependent products ($\Pi$-types)}

\bmp
\inferrule*[Left = $\Pi$-form]{\Gamma \vdash A \type \\ \Gamma, x:A \vdash B(x) \type}{\Gamma \vdash \Pi_{x:A}B(x) \type}
\\
\inferrule*[Left = $\Pi$-intro]{\Gamma, x:A \vdash B(x) \type \\ \Gamma, x:A \vdash b(x) : B(x)}{\Gamma \vdash \lambda(x:A).b(x) : \Pi_{x:A}B(x)}
\\
\inferrule*[Left = $\Pi$-elim]{\Gamma \vdash f: \Pi_{x:A}B(x) \\ \Gamma \vdash a:A}{\Gamma \vdash \app(f,a):B[a/x]}
\\ 
\inferrule*[Left = $\Pi$-comp]{\Gamma, x:A \vdash B(x) \type \\ \Gamma, x:A \vdash b(x) : B(x) \\ \Gamma \vdash a:A}
{\Gamma \vdash \app\left(\lambda(x:A).b(x), a\right) \equiv b[a/x] : B[a/x]}
\\
\inferrule*[Left = $\Pi$-cong(1)]{\Gamma \vdash A \equiv A' \type \\ \Gamma, x:A \vdash B(x) \equiv B'(x) \type}{\Gamma \vdash \Pi_{x:A}B(x) \equiv \Pi_{x:A'}B'(x) \type}
\\
\inferrule*[Left = $\Pi$-cong(2)]{\Gamma \vdash A \equiv A' \type \\ \Gamma, x:A \vdash B(x) \equiv B'(x) \type \\ \Gamma, x:A \vdash b(x) \equiv b'(x) : B(x)}{\Gamma \vdash \lambda(x:A).b(x) \equiv \lambda(x:A').b'(x) : \Pi_{x:A}B(x)}
\\
\inferrule*[Left = $\Pi$-$\eta$]{\Gamma \vdash f: \Pi_{x:A}B(x)}{\Gamma \vdash f \equiv \lambda(x:A).\app(f,x) : \Pi_{x:A}B(x)}
\emp


\medskip

A judgment $\Gamma \vdash A \type$ together with a judgment  $$\Gamma, x:A \vdash B(x) \type$$ is called  a \textit{type family $B$ over $A$}. Informally, this corresponds to the fiber bundle \linebreak $\pi : {\coprod_{x\in A}B(x)} \twoheadrightarrow A$.
Moreover, we can think of an inhabitant $f$ of the dependent product $\Pi_{x:A}B(x)$ as a set-theoretic function $f: A \to \bigcup_{x\in A}B(x)$ where $f(x) \in B(x)$ for each $x\in A$ (i.e., $f$ is a choice function). We can also think of $f$ as a section of the fiber bundle determined by ${\Gamma, x:A \vdash B(x) \type}$. In any case, we shall call such an $f$ a \textit{section of $B$ over $A$} or a \textit{dependent function on $A$}. 

\smallskip

\begin{notation} $ $
\bi
\item We may write $\lambda{x}.b(x)$ for the expression $\lambda(x:A).b(x)$. 
\item We may write $f(a)$ and $f(x)$ for the expressions $\app(f,a)$ and $\app(f,x)$, respectively.
\ei
\end{notation}
 
 
\begin{exmp}[Function types]
Using the weakening rule, we get the derivation
\bmp
\inferrule*{\Gamma \vdash A \type \\  \Gamma \vdash B \type}{
\inferrule*{\Gamma, x:A \vdash B \type}{\Gamma \vdash \Pi_{x:A}B\type}}.
\emp 
In context $\Gamma$, the expression $\Pi_{x:A}B$ is called the \textit{type of (non-dependent) functions from $A$ to $B$}. Thus, a non-dependent function is a special case of a dependent one.
\end{exmp}

\begin{notation}
We may write $A\to B$ or $B^A$ for the type of functions from $A$ to $B$. To avoid ambiguity, we stipulate that the symbol $\to$ is right associative. 
\end{notation}

 \begin{exmp}[Swap function]
 We can apply the exchange rule together with the context $\alpha$-conversion rule to obtain the derivation
 \bmp
 \inferrule*{\Gamma \vdash A \type \\ \Gamma \vdash B \type \\ \Gamma, x:A, y:B \vdash C(x,y)\type \\\\  \ctx(\Gamma, f : \Pi_{(x : A)} \Pi_{(y : B)} C(x, y), x:A, y: B)}
{\inferrule*{
 \Gamma, f : \Pi_{(x : A)} \Pi_{(y : B)} C(x, y), x:A, y:B \vdash f(x)(y) : C(x,y)
 }{\inferrule*{ \Gamma, f : \Pi_{(x : A)} \Pi_{(y : B)} C(x, y), y:B, x:A \vdash f(y)(x) : C(x,y)   }
 {\inferrule*{ \Gamma, f : \Pi_{(x : A)} \Pi_{(y : B)} C(x, y) \vdash   \lambda y. \lambda x.f(y)(x) :   \Pi_{(y : B)} \Pi_{(x : A)} C(x, y)     }{
 {\Gamma \vdash  \lambda f. \lambda y. \lambda x.f(y)(x): \left(\Pi_{(x : A)} \Pi_{(y : B)} C(x, y)\right) \rightarrow\left(\Pi_{(y : B)} \Pi_{(x : A)} C(x, y)\right)}}}}}.
 \emp
Intuitively, this shows that we can switch the order of two independent arguments of a dependent function.
 \end{exmp}

\begin{note}[$\alpha$-equivalence]
Moreover, we shall postulate certain $\alpha$-conversion rules when defining a new inductive type. In the case of $\Pi$-types, these are precisely
\bmp
\inferrule*[right={when $x'$ is not free in $\Gamma, x:A$}]{\Gamma, x:A \vdash B(x) \type}{\Gamma \vdash \Pi_{x:A}B(x) \equiv \Pi_{x':A}B[x'/x] \type}
\\
\inferrule*[right={when $x'$ is not free in $\Gamma, x:A$}]{\Gamma, x :A \vdash b(x) : B(x)}{\Gamma \vdash \lambda{x}.b(x) \equiv \lambda{x'}.b[x'/x] : \Pi_{x:A}B(x)}
\emp 
These assert that we can always rename bound variables in $\lambda$- or $\Pi$-expressions so long as we avoid variable capture. We shall tacitly assert similar rules for each subsequent logical constructor.
\end{note}


\begin{definition} $ $
\be
\item Using the variable rule, we get the derivation 
\bmp
\inferrule*{\ctx(\Gamma, x:A) \\ \Gamma \vdash A \type}{
\inferrule*{\Gamma, x:A \vdash x:A}{\Gamma \vdash \lambda x. x:A \to A}}. 
\emp 
We call $ \lambda x. x$ the \textit{identity map on $A$}, written as $\idmap_A$. 
\item Using both the variable and weakening rules, we get the derivation
\bmp
\inferrule*{\ctx(\Gamma, y:B, x:A) \\ \Gamma \vdash A \type}{
\inferrule*{\inferrule*{\Gamma, y:B \vdash y:B}{ \Gamma, y:B, x:A \vdash y:B   }}{\Gamma, y:B \vdash \lambda x. y:A \to B}}. 
\emp  We call $\lambda x.y$ the \textit{constant map at $y$}, denoted by $\cons_y$.
\item  It is straightforward yet tedious to derive the rule 
\bmp
\inferrule*{\Gamma \vdash A \type \\ \Gamma \vdash B \type \\ \Gamma \vdash C \type \\\\ 
\ctx(\Gamma, g : C^{B}, f : B^{A}, x : A, y : B)}
{\Gamma \vdash \lambda g. \lambda f. \lambda x.g(f(x)) : C^B \to (B^A \to C^A)}
.\emp  We write $j\circ h$ for $$\app\left(\app\left(\lambda g. \lambda f. \lambda x.g(f(x)), j\right), h\right),$$ called the \textit{composition of $j$ with $h$}.\footnote{\cite[Definition 2.2.4]{Rijke} includes a full derivation.} 
\ee
\end{definition}

\begin{theorem}\label{assoc}
The rule
\bmp
\inferrule*{     \Gamma \vdash f :A \to B \\ \Gamma \vdash g : B \to C  \\ \Gamma \vdash h : C \to D    }
{
\Gamma \vdash \left(h \circ g\right) \circ f \equiv h \circ \left(g \circ f\right) : A \to D
}
\emp
is derivable.\footnote{\cite[Lemma 2.2.5]{Rijke}.}
Hence function composition is associative. 
\end{theorem}


\subsection*{Dependent sums ($\Sigma$-types)}

\begin{remark}
From now on, we shall postulate \emph{tacitly}  a congruence rule $\textsc{cong}$ for each new logical constructor that we define. This rule will be like that found in our definition of $\Pi$-types.
\end{remark}

\bmp
\inferrule*[Left = $\Sigma$-form]{\Gamma \vdash A \type \\ \Gamma, x:A \vdash B(x) \type}{\Gamma \vdash \Sigma_{x:A}B(x) \type}
\\
\inferrule*[Left = $\Sigma$-intro]{\Gamma, x:A \vdash B(x) \type \\ \Gamma \vdash a:A \\ \Gamma \vdash b:B[a/x]}{\Gamma \vdash \pair(a,b) : \Sigma_{x:A}B(x)}
\\
\inferrule*[Left = $\Sigma$-elim]{\Gamma, z:\Sigma_{x:A}B(x) \vdash C(z)\type \\ \Gamma, x:A, y:B(x)\vdash d(x,y) : C\left[\pair(x,y)/z\right] \\ \Gamma \vdash p:\Sigma_{x:A}B(x)}{\Gamma \vdash \split(z.C, x.y.d, p) : C[p/z]}
\\
\inferrule*[Left = $\Sigma$-comp]{\Gamma, z:\Sigma_{x:A}B(x) \vdash C(z) \type \\\\ \Gamma, x:A , y: B(x) \vdash d(x,y) : C\left[\pair(x,y)/z\right] \\\\
\Gamma \vdash a :A \\ \Gamma \vdash b : B[a/x]}
{\Gamma  \vdash \split(z.C, x.y.d, \pair(x,y)) \equiv d[a,b/ x,y] : C\left[\pair(a,b)/z\right]}
\emp
\smallskip
Informally, we can think of a dependent sum $\Sigma_{x:A}B(x)$ as a set-theoretic disjoint union $\coprod_{x\in A}B(x)$. 

\begin{notation}
We may write $\pair(x,y)$ as $\left(x,y\right)$.
\end{notation}

\begin{note}
Intuitively, $\Sigma$-\textsc{elim} asserts that to construct a dependent function out of $\Sigma_{x:A}B(x)$, it suffices to construct,  for each canonical element $\left(a, b\right)$, a term of  type $C\left[(a,b)/z\right]$. 
\end{note}

\pagebreak

\begin{definition} Suppose that we have a type family $B$ over $A$.
\be
\item Define the \textit{first projection map} $\pr_1 : \left(\Sigma_{x:A}B(x)\right)\to A     $ inductively by $$\pr_1(x,y) \coloneqq x. $$
\begin{notation}
In the style of programming languages, the symbol $\coloneqq$ here means that $\pr_1$ is syntactic sugar for $\split(A, x.y.x, (x,y)).$ In particular, $\pr_1$ is \emph{not} a symbol in our object language. We shall make use of sugaring throughout.
\end{notation} 
\item Define the \textit{second projection map} $\pr_2 : \Pi_{\left(p : \Sigma_{(x : A)} B(x)\right)} B\left(\pr_{1}(p)\right)$ inductively by $$ \pr_2(x,y) \coloneqq y     .$$
\ee 
\end{definition}

\begin{exmp}[Product types] 
Using the weakening rule, we get the derivation
\bmp
\inferrule*{\Gamma \vdash A \type \\  \Gamma \vdash B \type}{
\inferrule*{\Gamma, x:A \vdash B \type}{\Gamma \vdash \Sigma_{x:A}B\type}}.
\emp 
In context $\Gamma$, the expression $\Sigma_{x:A}B$ is called the \textit{(cartesian) product of $A$ and $B$}.
\end{exmp}

\begin{notation} We may write $A\times B$ for the product of $A$ and $B$. \end{notation}

\subsection*{Empty type ($\0$)}

\bmp
\inferrule*[Left = $\0$-form]{\ctx(\Gamma)}{\Gamma \vdash \0 \type}
\and
\inferrule*[Left = $\0$-elim]{\Gamma, x : \0 \vdash C(x) \type \\ \Gamma \vdash a :\0}{\Gamma \vdash \ind_{\0}(x.C, a) :C[a/x]}
\emp
\medskip
Note that the empty type is a degenerate inductive type. In particular, if one can derive a typing declaration of the form $\Gamma \vdash t : \0$, then one can derive any typing declaration with context $\Gamma$. Thus, the empty type corresponds, informally, to the empty set in set theory for if one can prove  $\exists x\left(x\in \emptyset\right)$, then one can prove any sentence in the language of set theory.

\subsection*{Unit type ($\1$)}

\bmp
\inferrule*[Left = $\1$-form]{\ctx(\Gamma)}{\Gamma \vdash \1 \type}
\and
\inferrule*[Left = $\1$-intro]{\ctx(\Gamma)}{\Gamma \vdash \star : \1}
\\
\inferrule*[Left = $\1$-elim]{\Gamma, x:\1 \vdash C(x) \type \\ \Gamma \vdash c:C\left[\star /x\right] \\ \Gamma \vdash a :\1}
{\Gamma \vdash \ind_{\1}(x.C, c, a) :C[a/x]}
\\ 
\inferrule*[Left = $\1$-comp]{\Gamma, x:\1 \vdash C(x) \type \\ \Gamma \vdash c:C\left[\star /x\right]}{\Gamma \vdash \ind_{\1}(x.C, c, \star)\equiv c:C\left[\star /x\right]}
\emp

Note that in any well-formed context $\Gamma$, we have that $\star$ is the unique term of type $\1$. Thus, the unit type corresponds, informally, to a singleton set in set theory.

\subsection*{Boolean type ($\2$)}

\bmp
\inferrule*[Left = $\2$-form]{\ctx(\Gamma)}{\Gamma \vdash \2 \type}
\and
\inferrule*[Left = $\2$-intro(1)]{\ctx(\Gamma)}{\Gamma \vdash 0_{\2} : \2}
\and 
\inferrule*[Left = $\2$-intro(2)]{\ctx(\Gamma)}{\Gamma \vdash 1_{\2} : \2}
\\
\inferrule*[Left = $\2$-elim]{\Gamma, x:\2 \vdash C(x) \type \\ \Gamma \vdash c:C\left[0_{\2} /x\right] \\  \Gamma \vdash d:C\left[1_{\2} /x\right] \\\\ \Gamma \vdash a :\2}
{\Gamma \vdash \ind_{\2}(x.C, c,d, a) :C\left[a/x\right]}
\\ 
\inferrule*[Left = $\2$-comp(1)]{\Gamma, x:\2 \vdash C(x) \type \\ \Gamma \vdash c:C\left[0_{\2} /x\right]}{\Gamma \vdash \ind_{\2}(x.C, c, 0_{\2})\equiv c:C\left[0_{\2} /x\right]}
\\ 
\inferrule*[Left = $\2$-comp(2)]{\Gamma, x:\2 \vdash C(x) \type \\ \Gamma \vdash d:C\left[1_{\2} /x\right]}{\Gamma \vdash \ind_{\2}(x.C, d, 1_{\2})\equiv d:C\left[1_{\2} /x\right]}
\emp

Informally, the Boolean type corresponds to the set of truth values $\left\{F, T\right\}$ in propositional logic. 

\subsection{Propositions as types}


The following table describes the so-called Curry-Howard correspondence.

\begin{table}[h!]
\centering
\caption{Logical and set-theoretic interpretations of type theory}
\label{table:2}
\begin{tabular}{||c c c||} 
 \hline
 FOL (with bounded quantifiers) & Set theory & Type theory \\ [0.5ex] 
 \hline\hline
 Proposition &  Set & Well-formed type \\ 
   Proof        &  Element  & Inhabitant  \\
 $\neg{A}$ &  $A^c$  &   $A \to \0$ \\
 $A \land B$ & $A \times B$ & $A \times B$  \\ 
 $A \to B$ & $B^A$ &  $A \to B$  \\
 $\forall_{x:A}B(x)$ & $\prod_{x\in A}B(x)$ &  $\Pi_{x:A}B(x)$  \\
 $\exists_{x:A} B(x)$ & $\coprod_{x\in A}B(x)$ & $\Sigma_{x:A}B(x)$ \\ 
  $\top$    & $\left\{0\right\}$    & $\1$ \\
  $\bot$    & $\emptyset$  & $\0$ \\ [1ex] 
 \hline
\end{tabular}
\end{table}

In particular, this correspondence between first-order logic and type theory, called the \textit{propositions-as-types doctrine}, encodes a system of constructive logic inside our  type theory. (It is constructive in the sense that proving a proposition $P$ corresponds to constructing a term of type $P$.) For example, negation in our MLDTT corresponds to the principle of explosion in constructive logic, and $\Pi$-\textsc{elim} corresponds to modus ponens. In addition, the propositions-as-types doctrine automatically provides us with a type-theoretic notion of logical equivalence.


\begin{definition}[Logical equivalence]
Suppose that we have derived  a rule of the form
\bmp 
\inferrule*{ J_1 \\ \ldots \\ J_n }
{\Gamma \vdash A \type \\ \Gamma  \vdash B \type \\\\ \Gamma \vdash  f : A \to B \\ \Gamma \vdash g: B \to A}
\emp where each $J_i$ denotes a judgment. 
 Then we say that the type expressions $A$ and $B$ are \textit{logically equivalent}. 
\end{definition}


\subsection{The universe ($\U$)}\label{univ}

With the following logical rules, we define a closed type called the \textit{universe (type)} such that it is closed under all of our logical constructors.

\bmp
\inferrule*{ }{\vdash \U \type}
\and 
\inferrule*{ }{x: \U \vdash \el(x) \type}
\\
\inferrule*{\Gamma \vdash  a:\U \\ \Gamma, x: \el(a) \vdash b(x):\U}{\Gamma \vdash \hat{\Pi}(a, x.b) : \U}
\and 
\inferrule*{\Gamma \vdash a:\U \\ \Gamma, x: \el(a) \vdash b(x):\U}{\Gamma \vdash \el(\hat{\Pi}(a, x.b)) \equiv \Pi_{x:\el(a)} \el(b(x)) \type}
\\
\inferrule*{\Gamma \vdash a:\U \\ \Gamma, x: \el(a) \vdash b(x):\U}{\Gamma \vdash \hat{\Sigma}(a, x.b) : \U}
\and 
\inferrule*{\Gamma \vdash a:\U \\ \Gamma, x: \el(a) \vdash b(x):\U}{\Gamma \vdash \el(\hat{\Sigma}(a, x.b)) \equiv \Sigma_{x:\el(a)} \el(b(x)) \type}
\\ 
\inferrule*{ }{\vdash \hat{\0} : \U} \and \inferrule*{ }{\vdash \el(\hat{\0}) \equiv \0 \type} 
\\
\inferrule*{ }{\vdash \hat{\1} : \U} \and \inferrule*{ }{\vdash \el(\hat{\1}) \equiv \1 \type} 
\\
\inferrule*{ }{\vdash \hat{\2} : \U} \and \inferrule*{ }{\vdash \el(\hat{\2}) \equiv \2 \type} 
\emp


\begin{definition}
We say that a well-formed type $A$ in context $\Gamma$ is \textit{small} if we can derive $$\Gamma \vdash \hat{A} : \U \qquad  \Gamma \vdash \el(\hat{A}) \equiv A \type$$ for some expression $\hat{A}$. 
\end{definition}

From a set-theoretic viewpoint, $\hat{A}$ is a  lift of $A$ under $\el$.

\begin{exmp}
If $A$ is a small type in context $\Gamma$ and $B(x)$ is a small type in context $\Gamma, x:A$, then $\Pi_{x:A}B(x)$ is a small type in context $\Gamma$.
\end{exmp}

We can view $\el$ as an interpretation operator (in a semantic sense) so that each inhabitant $X$ of $\U$ in context $\Gamma$ is a name for the genuine type $\el(X)$.

\bigskip

\begin{aside}
We could not assert that every type has type $\U$, in which case $\U$ inhabits $\U$. For then we'd obtain an encoding of Russell's paradox known as Girard's paradox, so that our MLDTT would be inconsistent (i.e., we could construct a closed term of type $\0$). We could postulate a sequence of universes $ \left(\U_i\right)_{i\in \N} $ governed by the rule schemata
\bmp
\inferrule
{\ctx(\Gamma) }
{\Gamma \vdash  \U_i :\U_{i+1}}
\and
\inferrule
{\Gamma \vdash A: \U_i}
{\Gamma \vdash A: \U_{i+1}}.
\emp Such a sequence is called a \textit{cumulative hierarchy}.\footnote{Each stage of this hierarchy is called a \textit{universe \`a la Russell}. The universe in our MLDTT is called a \textit{universe \`a la Tarski}.} In this case, we would alter our MLDTT by removing judgments of typehood and expressing that $A$ is a well-formed type via a typing declaration such as $\Gamma \vdash A : \U_i$. This approach is taken by \cite{HoTT}. 
\end{aside}

\section{Homotopy type theory}\label{synth}

In this section, our goal is to develop enough classical homotopy theory \emph{within} our MLDTT to motivate and state the univalence axiom. This form of homotopy theory is known as \textit{synthetic homotopy theory}. This is precisely the area that both \cite{HoTT} and \cite{Rijke} cover. Synthetic homotopy theory has produced some new proofs of old theorems, such as the Freudenthal suspension theorem.

\medskip

Throughout this section, we shall mention a new, informal interpretation of Martin-L\"of  dependent type theory in which each well-formed type represents (the homotopy type of) a topological space. This interpretation will be made precise by way of categorical semantics.

\medskip
\begin{remark} 
After defining identity types, we shall mainly use informal notation so that our presentation matches the ordinary style of mathematics. (The type theory literature normally does the same.) Still, all of the definitions, theorems, proofs, etc.\ internal to our system can be syntactically formalized.\footnote{For example, there are ongoing implementations of \cite{HoTT} in the proof assistants Coq and Agda available on GitHub.}
\end{remark}

\subsection{Identity types ($\id_{-}({-}, {-})$)}\label{itypes}

So far, our sole concept of equality is judgmental equality. But this fails to caputure our ordinary concept of equality, as in the formula ${\forall{x,y}\in \R\left(x+y = y+x\right)}$. Indeed, this is a first-order sentence and thus should correspond to a well-formed type \`a la Curry-Howard. By contrast, judgmental equality, as a relation on the set of all raw terms, determines a rewriting system, i.e., a set of rules for replacing one raw term with another.

\smallskip

To reason about mathematical equality in our MLDTT, we define our final inductive type, the \textit{(propositional) identity type}, as follows.

\bmp
\inferrule*[Left = $\id$-form]{\Gamma \vdash A \type}{\Gamma, x:A, y:A \vdash \id_A(x,y) \type}
\and
\inferrule*[Left = $\id$-intro]{\Gamma \vdash A \type}{\Gamma, x:A \vdash \refl(A,x) : \id_A(x,x)}
\\
\inferrule*[Left = $\id$-elim]{\Gamma, x:A, y:A, p: \id_A(x,y) \vdash C(x,y,p) \type \\\\
\Gamma, z:A \vdash c(z):C[z, z, \refl(A,z)/x,y,p]}
{\Gamma, x:A, y:A, p:\id_A(x,y) \vdash \J(x.y.p.C, z.c, x, y, p) : C(x,y,p)}
\\
\inferrule*[Left = $\id$-comp]{\Gamma, x:A, y:A, p: \id_A(x,y) \vdash C(x,y,p) \type \\ \Gamma, z:A \vdash c(z) :C[z, z, \refl(A,z)/x,y,p]}{\Gamma, x:A \vdash \J(x.y.p.C, z.c, x, x, \refl(A,x)) \equiv c[x/z] : C[x, x, \refl(A,x) /x,y,p]   }
\emp

\begin{notation}
We shall write $\refl_a$ for $\refl(A,a)$ when its first argument can easily be inferred. 
\end{notation}

\begin{remark}\label{exttypes}
Our MLDTT is \textit{intensional} in that it leaves out both of the following inference rules.

\be
\item{(\textit{equality reflection rule} ($\err$))}
\bmp
\inferrule*{ \Gamma \vdash p: \id_A(x,y)}{\Gamma \vdash x\equiv y :A  }.
\emp

\item{(\textit{uniqueness of identity proofs} ($\uip$))}
\bmp
\inferrule*{\ctx(\Gamma) }{\Gamma \vdash \mathsf{uip} : \prod_{A:\U} \prod_{x,y: \el(A)}\prod_{p,q: \id_{\el(A)}(x,y)} \id_{\id_{\el(A)}(x,y)}(p,q)}.
\emp
\ee

A Martin-L\"of dependent  type theory is \textit{extensional} if it includes $\err$. It is a \textit{set-level} type theory if it includes $\uip$.\footnote{It is easy to prove that $\uip$ is incompatible with the univalence axiom.} In an extensional type theory, propositional equality implies judgmental equality. (The converse is always true.) In a set-level type theory,  for any two  well-formed terms, there is at most one proof that they are propositionally equal.  If $\err$ is assumed, then $\uip$ is provable with $\id$-\textsc{elim}.
\end{remark}

\medskip


Now, consider the logical rule $\id$-\textsc{elim}. Intuitively, this  asserts that if we know that 
\be[label=(\alph*)] 
\item for any $x, y: A$ and $p: \id_A(x,y)$, we have a type $C(x,y,p)$ and 
\item  whenever $x=y$ and $p = \refl(A,x)$, we have a term $t$ of type $C(x,x, \refl(A,x))$, 
\ee
then we can construct a certain term $\J$ of type $C\left(x',y',p'\right)$ for \emph{any} $x',y': A$ and $p':\id_A(x,y)$.


\begin{term}
Another name for $\id$-\textsc{elim} is \textit{path induction}.
\end{term}

Indeed, we can interpret a well-formed type $A$ as a topological space and each inhabitant of $\id_A(x,y)$ as a path in $A$ from the point $x$ to the point $y$. In this case, we think of the term $\refl_x$ as the constant path at the point $x$.

\medskip

Moreover, under our propositions-as-types doctrine, we view a path in $A$ from $x$ to $y$ as a proof of the proposition that $x=y$. 

\medskip
With  the following two rules, we postulate that the universe is closed under the logical constructor $\id$.

\bmp
\inferrule*{\Gamma \vdash a:\U \\ \Gamma \vdash b, c :\el(a)}{\Gamma \vdash \hat{\id}_a(b,c) : \U}
\and 
\inferrule*{\Gamma \vdash a:\U \\ \Gamma \vdash b, c :\el(a)}{\Gamma \vdash \el(\hat{\id}_a(b,c)) \equiv \id_{\el(a)}(b,c) \type}
\emp

\begin{notation}
We may write $x \leadsto_A y$ for $\id_A(x,y)$, omitting the subscript $A$ when it can be easily inferred.
\end{notation}

The type $x \leadsto_A y$ can be viewed as the path space of $A$, which consists of the set of paths in $A$ equipped with the compact-open topology. 

\begin{table}[h]
\centering
\caption{Homotopy interpretation of type theory}
\label{table:3}
\begin{tabular}{||c c||} 
 \hline
 Type theory & Homotopy theory \\ [0.5ex] 
 \hline\hline
 Type & Space \\ 
 Inhabitant & Point  \\
 Identity type & Path space  \\
 Type family & Fibration  \\
 Dependent sum & Total space  \\ [1ex] 
 \hline
\end{tabular}
\end{table}



\bigskip

In our propositions-as-types doctrine, we intuitively interpret \emph{any} type as a certain logical proposition. Using identity types, we can define an alternative notion of proposition within our MLDTT as follows.

\begin{definition}\label{h-prop}
A type $A$ is an \textit{h-proposition} (or a \textit{mere proposition}) if there is some term of type $$\isprop(A) \coloneqq \prod_{x, y : A} x\leadsto y     .$$
\end{definition}

Not every well-formed  type $A$ is necessarily an  h-proposition within our MLDTT. It is one exactly when any two of its inhabitants are propositionally equal.  According to our propositions-as-types doctrine, this means that any two of its proofs are the same. In this case, $A$ is true exactly when it is inhabited by a single term and false exactly when its inhabitation leads to a contradiction. 
\smallskip

 As a result, the constructive logic encoded in our MLDTT now has the property that a specific proof of a proposition $P$ carries no mathematical data other than the fact that $P$ is true. Likewise, in propositional logic, a proposition is interpreted as nothing more than a truth value. 
 \smallskip

\begin{aside} 
Suppose that we treated as propositions exactly those  types $A$ such that $\isprop(A)$ is inhabited. Further, suppose that we postulated  the rule
\bmp
\inferrule*{\Gamma \vdash A \type }{\mathsf{dn}_A: \isprop(A)\to \left({\neg{\neg{A}}} \to A\right)},
\emp
called the \textit{law of double negation}.\footnote{As it turns out, this is compatible with the univalence axiom, but it would not be if we omitted the antecedent ``$\isprop(A)$." See \cite[Theorem 3.2.2]{HoTT}.}  
  Then the logic encoded in our MLDTT would become classical. Thus, we could recover classical mathematical reasoning, if desired, at the expense of constructiveness.  
\end{aside}


\subsection{Basic properties of (type-theoretic) paths}

\begin{lemma}[Path inversion]
Let $A$ be a type and let $x$ and $y$ be inhabitants of $A$ (in context $\Gamma$). Then there is some function $$\inv:\left(x\leadsto y\right) \to \left(y \leadsto x\right)$$ such that $\inv(\refl_z) \equiv \refl_z$ for each $z:A$. For any $p : x \leadsto y$, let $p^{-1} \coloneqq \inv(p)$.
\end{lemma}
\begin{proof}
By path induction, it suffices to construct a term $$\inv(\refl_x) : x \leadsto x.$$ Take $\inv(\refl_x)$ to be $\refl_x$. 
\end{proof}

\begin{lemma}[Path concatenation]\label{concat}
Let $A$ be a type and $x,y : A$. Then there is some function $$\concat : \left(x\leadsto y\right) \to \left(y \leadsto z\right) \to \left(x \leadsto z\right)$$ such that $\concat( \refl_x, q) \equiv q$ for any $q: x \leadsto z$. Let $p \ast q \coloneqq \concat(p,q)$.
\end{lemma}
\begin{proof}
Again, it suffices to construct, for any $z: A$, a term 
\[
\concat(\refl_x) :  \left(x\leadsto z\right) \to \left(x\leadsto z\right)  .
\] Take $\concat(\refl_x) $ to be $\idmap_{x \leadsto z}$. The fact that $\concat( \refl_x, q) \equiv q$ follows automatically from $\id$-\textsc{comp}.
\end{proof}

It follows that propositional equality is like a set-theoretic equivalence relation.

\begin{lemma} Let $A$ be a type. Let $x,y:A$ and  $p: x \leadsto y$.
\be[label=(\arabic*)]
\item Let $z, w : A$. Let  $q : y\leadsto z$ and $r : z \leadsto w$. Then there is some path $$ \assoc(p,q,r) : \left(p \ast q\right) \ast r \leadsto p \ast \left(q \ast r\right)    .$$
\item  There exist certain paths
\begin{align*}
\lunit(p) & : \refl_x \ast p \leadsto p
\\ \runit(p) &: p \ast \refl_x \leadsto p.
\end{align*}
\item There exist certain paths
\begin{align*}
\linv(p) & : p^{-1}\ast p \leadsto 	\refl_y
\\ \rinv(p) &: p \ast p^{-1} \leadsto \refl_x.
\end{align*}
\ee
\end{lemma}
\begin{proof}
Let us just prove (1) for both (2) and (3) will follow similarly. By path induction, it suffices to construct a term of type
\[
\left(\refl_x \ast q\right) \ast r \leadsto \refl_x \ast \left(q \ast r\right)
\] where $q: x \leadsto z$. Note that $\left(\refl_x \ast q\right) \ast r \equiv q \ast r$ and  $\refl_x \ast \left(q \ast r\right) \equiv q \ast r   $ due to \cref{concat}.   Thus, by some of our structural rules, we can just choose $\refl_{q\ast r}$.
\end{proof}

\begin{corollary}\label{groupoid} 
Every type has the structure of a fundamental groupoid of a topological space.
\end{corollary}

Our next result  shows that any non-dependent function is continuous in a certain sense.

\begin{lemma}[Functoriality] \label{functor}
Any non-dependent function $f: A \to B$ preserves paths, i.e., for any path $p: x \leadsto_A y$, there is some path $\ap_{f,p} : f(x) \leadsto_B f(y)$.
\end{lemma}
\begin{proof}
It suffices to construct a term $\ap_{f, p}(\refl_x) : f(x) \leadsto_B f(x)$. Choose $\refl_{f(x)}$.
\end{proof}

\smallskip

\begin{lemma}[Transport]
Let $P$ be a type family over $A$. Let $p: x\leadsto_A y$. Then there is some function $$\transport(p) : P(x) \to P(y)$$ such that $\transport(\refl_x)(u) \equiv u$ for any $u: P(x)$. Let $p \cdot u \coloneqq \transport(p)(u)$.
\end{lemma}
\begin{proof}
It suffices to construct a term $ \transport(\refl_x)  : P(x) \to P(x)  .$ Choose $\idmap_{P(x)}$.
\end{proof}

\begin{lemma}
Let $f: \prod_{x:A}P(x)$ and $p: x\leadsto_A y$. Then there is some path $$\apd_f(p):p \cdot f(x) \leadsto_{P(y)} f(y).$$
\end{lemma}
\begin{proof}
It suffices to construct a term $ \apd_f(\refl_x) : f(x) \leadsto_{P(x)}f(x)    .$ Choose $\refl_{f(x)}$.
\end{proof}

As fiber bundles in topology possess the homotopy lifting property,  our next lemma leads us to interpret the dependent sum $\sum_{x:A}P(x)$ as the total space of a fiber bundle over $A$.

\begin{lemma}[Path lifting]
Let $P$ be a type family over $A$. Suppose that $p: x\leadsto_A y$ and that $u: P(x)$. Then there is some path $$p_{\Sigma}(u) : \left(x, u\right) \leadsto_{\Sigma_{x:A}P(x)} \left(y, p\cdot u\right).$$
\end{lemma}
\begin{proof}
It suffices to construct a term $\left(\refl_x\right)_{\Sigma}(u) : (x, u) \leadsto (x, u)$. Choose $\refl_{(x,u)}$.
\end{proof}



\begin{lemma}\label{noname1}
Suppose that $x, x':A$, $y :B(x)$, and $y' : B(x')$. Suppose that we have both a path from $\left(x,y\right)$ to $\left(x', y'\right)$ and a path from $y'$ to $z$ where $z: B(x')$. Then there is some term of type $$\id_{\Sigma_{x:A} B(x)}\left(\left(x,y\right), \left(x',z\right)\right).$$
\end{lemma}
\begin{proof}
Another easy application of path induction. 
\end{proof}

\medskip

As one may expect, if two dependent functions of the same type are propositionally equal, then they are pointwise propositionally equal. 

\begin{lemma}
Let $f,g: \prod_{x:A} P(x)$ where $P$ is a type family over $A$. Let $\alpha : f \leadsto g$. Then $\alpha$ induces a path $\alpha(x) :f(x) \leadsto_{P(x)} g(x)$ for each $x:A$. Therefore, we have a function $$\happly_{A,P} : \prod_{f, g:  \Pi_{x:A} P(x)} \left(f\leadsto g\right) \to \left(\prod_{x:A} f(x)\leadsto g(x) \right).$$
\end{lemma}
\begin{proof}
It suffices to construct a term $$\happly_{A,P}(f, f)(\refl_f) : \prod_{x:A} f(x) \leadsto f(x).$$ Choose $\lambda x. \refl_{f(x)}$.
\end{proof}

\begin{definition}[Homotopy]
Let $f,g: \prod_{x:A} P(x)$ where $P$ is a type family over $A$. A \textit{homotopy from $f$ to $g$} is a term $H$ of type $$f\approx g \coloneqq \prod_{x:A} f(x) \leadsto_{P(x)} g(x)  .$$ If this type is inhabited, then we say that $f$ and $g$ are \textit{homotopic}, written as $f\sim g$.
\end{definition}

As it turns out, if our MLDTT assumes the univalence axiom, then any model of it must satisfy so-called functional extensionality, which ensures that every type-theoretic homotopy induces a continuous \emph{choice} of paths $p_x : f(x) \leadsto g(x)$, just as a  homotopy in  the classical sense.

\medskip

The following notion corresponds to homotopy equivalence in topology, i.e., an isomorphism in the homotopy category of $\mathbf{Top}$.

\begin{definition}[Isomorphism]
Let $f: A \to B$. We say that $f$ is a \textit{(homotopy) isomorphism} if there is some $g:  B \to A$ such that $f \circ g \sim \idmap_B$ and $g\circ f \sim \idmap_A$. To be precise, $$\iso(f) \coloneqq \sum_{g:B \to A} \left(\idmap_B \approx f\circ g\right) \times  \left(\idmap_A \approx g\circ f\right).$$ We call such a $g$ a \textit{homotopy inverse of $f$}.

\end{definition} 

\begin{definition} Let $B$ be a type family over $A$.
 Define \textit{homotopy concatenation} as the function $$\hpyconcat : \prod_{f, g, h : \Pi_{(x : A)} B(x)}\left(f \approx g\right) \rightarrow \left(g \approx h\right) \rightarrow \left(f \approx h\right)$$ where $\hpyconcat(H, K) \coloneqq \lambda x. H(x) \ast K(x)$. We write $H\bullet K$ for $\hpyconcat(H, K)$. 
\end{definition}

\begin{lemma}\label{htpy-comp}
Suppose that $f,g: A\to B$ and $f',g' : B \to C$. Suppose that we have inhabitants $H$ and $H'$ of $f \approx g$ and $H' : f' \approx g'$, respectively. Then $f' \circ f \sim g' \circ g$.  
\end{lemma}
\begin{proof}
The term $\lambda a .g^{\prime}(H(a)) \bullet H^{\prime}(f(a))$ has type $f' \circ f \approx g' \circ g$.
\end{proof}

\smallskip

\begin{definition}[Contractible type]
Given a type $A$, we say that $A$ is \textit{contractible} if  the type
\[
\iscont(A)\coloneqq \sum_{a:A}\prod_{x:A}x \leadsto a
\] is inhabited.  For any inhabitant $\left(c, C\right)$ of $\iscont(A)$, we call $c$ a \textit{center of contraction} for $A$ and $C$ a \textit{contraction} of $A$.
\end{definition}

To preserve our topological intuition, we should interpret this as saying that $A$ is contractible exactly when we can construct a term $a:A$ and a homotopy from $\lambda x.x$ to $\lambda x.a$.


\begin{exmp} 
 The unit type is contractible.
\end{exmp}
\begin{proof}
Define $f: \prod_{x: \1} x \leadsto \star$ inductively by $f(\star) \coloneqq \refl_{\star}$. Then  $\left(\star, f\right)$ has type $\iscont(\1)$. 
\end{proof}



The following result reveals that a type where each inhabitant is potentially non-contractible can collectively form a contractible type.

\begin{lemma}\label{total-cont}
For any type $A$ and any $a:A$, the type $\sum_{x:A}x \leadsto a$ is contractible.
\end{lemma}
\begin{proof}
 We claim that $\left(a, \refl_a\right)$ inhabits $\iscont(\sum_{x:A}x \leadsto a)$. We must show that there is some  path from $\left(x,p\right)$ to $\left(a, \refl_a\right)$ for any $\left(x,p\right) : \sum_{x:A}x \leadsto a$. By the path lifting lemma along with \cref{noname1}, it suffices to construct a term $q: a \leadsto x$ such that $q\cdot \refl_a$ and $p$ are propositionally equal. By path induction, it is easy to show that $p \cdot \refl_a$ and $p$ are propositionally equal. Hence take $q$ to be $p$.
\end{proof}

The following notion corresponds to the fiber of a point under a continuous map between spaces.

\begin{definition} Let $f: A \to B$ and $b: B$.
 The \textit{homotopy fiber of $b$} is the type $$\hfiber(f,b) \coloneqq \sum_{x:A} f(x) \leadsto b .$$
\end{definition}

We can now extend our notion of contractibility to functions. 

\begin{definition}
We say that  a function $f: A \to B$ is \textit{contractible} if $\hfiber(f,b)$ is contractible for each $b: B$, i.e., there is some term of type $$\iscontmap(f) \coloneqq \prod_{y: B} \iscont(\hfiber(f,y)).$$
\end{definition}

\subsection{Type-theoretic equivalence}


\begin{definition} Consider a function $f: A \to B$.
\be
\item Let $$\retr(f) \coloneqq \sum_{g: B \to A} g \circ f \approx \idmap_A    .$$  For any term $\left(g, G\right) : \retr(f)$, we call $g$ a \textit{retraction of $f$}.
\item Let $$\sect(f) \coloneqq \sum_{h: B \to A} f \circ h \approx \idmap_B    .$$ For any term $\left(h, H\right) : \sect(f)$, we call $h$ a \textit{section of $f$}.
\item We say that $f$ is an \textit{equivalence from $A$ to $B$} if we have functions $g : B \to A$ and $h: B\to A$ such that $g \circ f \sim \idmap_A$ and $f \circ h \sim \idmap_B$, i.e., there is some term of type $$ \isequiv(f) \coloneqq \retr(f) \times \sect(f).  $$
Then the type of equivalences from $A$ to $B$ is precisely $$A \simeq B \coloneqq  \sum_{f:A \to B}\isequiv(f).$$ If this is inhabited, then we say that $A$ and $B$ are \textit{equivalent}. 
\ee
\end{definition}

\begin{remark}
We have defined logical equivalence and equivalence between types differently.
\end{remark}

\begin{exmp}\label{idequiv}
For any type $A$, the map $\idmap_A$ is clearly an equivalence.
\end{exmp}

\begin{corollary}
Let $P$ be a type family over $A$ and $p: x\leadsto_A y$. Then $\transport(p)$ is an equivalence from $P(x)$ to $P(y)$. 
\end{corollary}
\begin{proof}
An easy application of path induction.
\end{proof}

\begin{lemma}\label{mere-prop}
Suppose that $P$ and $Q$ are h-propositions and  that we have terms $f : P \to Q$ and $g: Q \to P$. Then $P$ and $Q$ are equivalent. 
\end{lemma}
\begin{proof}
It is easy to see that $f$ and $g$ are homotopy inverses of each other.
\end{proof}

We now proceed to establish two new ways of logically characterizing $\isequiv(f)$. 

\begin{lemma}
 A function $f : A \to B$ is an equivalence if and only if it is an isomorphism, i.e., $\iso(f)$ and $\isequiv(f)$ are \emph{logically} equivalent types.
\end{lemma}
\begin{proof}
The ($\Longleftarrow$) direction is obvious. Conversely, suppose that $f$ is an equivalence. Then we have a term $\left(g, G\right) : \retr(f)$ and a term $\left(h, H\right) : \sect(f)$. For any $y: B$, we can apply \cref{assoc} to get a chain of paths 
\[
\begin{tikzcd}[column sep = large]
h(y) \arrow[r, "G(h(y))^{-1}", squiggly] & (g \circ f)(h(y))  \arrow[r, squiggly] & g ((f\circ h)(y))  \arrow[r, "\apd_g(H(y))", squiggly] & g(y)
\end{tikzcd}.
\] This shows that we can construct a homotopy $K : h \approx g$. It follows that $h \circ f \sim \idmap_A$, so that $h$ is a homotopy inverse of $f$. 
\end{proof}


\begin{corollary}
Any two contractible types are equivalent. In particular, every contractible type is equivalent to the unit type.
\end{corollary}

\begin{corollary}\label{contr-equiv}
If the types $A$ and $B$ are contractible and $f: A \to B$, then $f$ is an equivalence.
\end{corollary}
\begin{proof}
It suffices to show that $f$ is an isomorphism.  By assumption, we have inhabitants  $\left(c, C\right)$ and $\left(c', C'\right)$ of $\iscont(A)$ and $\iscont(B)$, respectively. It is easy to check that  $f \circ \cons_c \sim \idmap_B$ and $\cons_c \circ f \sim \idmap_A$.
\end{proof}

\begin{theorem}\label{equiv-cont}
 A function $f : A \to B$ is an equivalence if and only if it is contractible, i.e.,  $\isequiv(f)$ and $\iscontmap(f)$ are \emph{logically} equivalent types.
\end{theorem}
\begin{proof}
For the ($\Longrightarrow$) direction (which is much more difficult), see \cite[Theorem 6.3.3]{Rijke}.

\smallskip
Conversely, suppose that $f$ is contractible. Then for each $y:B$, we get a term $$\left(h(y), H(y)\right) :\iscont(\hfiber(t,y)).$$ Therefore, we have a term $$ \lambda y. \left(h(y), H(y)\right) : \prod_{y:B}\hfiber(f,y)    .$$ From this we can construct a function $h: B \to A$ and a homotopy $$H: \prod_{y:B} f(h(y)) \leadsto_B y.$$ Now, we also can construct a term of type $\prod_{x : A} h(f(x)) \leadsto_A x$. Indeed, for each $x:A$, we have a path $$p: f(h(f(x))) \leadsto_B f(x),$$ so that $f(h(f(x)))$ inhabits $\hfiber(f, f(x))$. Since $\hfiber(f, f(x))$ is contractible, we get a path $$q : \left(h(f(x)), p\right) \leadsto \left(x, \refl_{f(x)}\right).$$ Then $\pr_1(h(f(x)), p) \leadsto \pr_1((x, \refl_{f(x)}))$ is inhabited due to \cref{functor}. Hence $h(f(x)) \leadsto x$ is inhabited as well. It follows that $h$ is both a section and a retraction of $f$. In particular, $f$ is an equivalence. 
\end{proof}


\begin{definition}
Let $P$ and $Q$ be dependent families over $A$. A function $\gamma : \prod_{x:A}P(x) \to Q(x)$ is a \textit{fiberwise equivalence from $P$ to $Q$} if each $\gamma(x)$ is an equivalence from  $P(x)$ to $Q(x)$. 
\end{definition}

\begin{theorem}[Voevodsky]\label{fwe}

Let $P$ and $Q$ be type families over $A$. Consider a term $$\tau : \prod_{x:A} P(x) \to Q(x)$$ with the property that $$\sigma_{\tau} \coloneqq \lambda w .\left(\pr_{1}(w), \tau\left(\pr_{1}(w)\right)\left(\pr_{2}(w)\right)\right) :    \sum_{x:A}P(x) \to \sum_{x:A}Q(x)  $$ is an equivalence. Then $\tau$ is a fiberwise equivalence.\footnote{\cite[Theorem 2.4.19]{Rijke2}.} 
\end{theorem}

\subsection*{Functional extensionality}

So far, we have intuitively regarded dependent functions as set-theoretic functions, i.e., ordinary mathematical functions.  At this point, however, our deductive system is too weak to prove that any two dependent functions $f$ and $g$ that are pointwise propositionally equal are themselves propositionally equal. 
 In this way, they are more like algorithms than ordinary functions. Indeed, two different algorithms may have the same output on each input. To avoid this issue, we may consider so-called functional extensionality principles.

\begin{definition} $ $
\begin{enumerate} 
\item The \textit{weak functional extensionality principle} ($\wfe$) asserts that for any type family $P$ over $A$, we have a term $$\left( \prod_{x:A}\iscont(P(x))   \right) \to \iscont\left(\prod_{x:A}P(x)\right)  .\footnote{This corresponds to the fact that any product of contractible spaces is contractible in classical homotopy theory.} $$
\item The \textit{functional extensionality principle} ($\sfe$) asserts that for any type family $P$ over $A$, there is some term of type $$\prod_{f,g: \Pi_{x:A} P(x)} \isequiv(\happly(f,g)).$$
\end{enumerate}
\end{definition}

In particular, $\sfe$ asserts that for any two dependent functions $f, g: \prod_{x:A}P(x)$,  if $f$ are $g$ are homotopic, then they are propositionally equal.   
From a  set-theoretic perspective, this corresponds to the fact that if two functions $f, g: X \to Y$ agree at each point in $X$, then $f=g$. 
From a topological perspective, it corresponds to the fact that a homotopy between $f$ and $g$ induces a path between $f$ and $g$ in the mapping space $M(X, Y)$. In this case, our type-theoretic notion of homotopy agrees with the classical notion. 

\begin{exmp}
To see that functional extensionality is useful, let $P$ be a mere proposition (as in \cref{h-prop}). We want to show that $\neg{P}$ is also a mere proposition, i.e., that $\prod_{x,y : \neg{P}} x\leadsto y$ is inhabited. 
To this end, assume $\sfe$ and let $x,y: P \to \0$. We must construct a term of type $x\leadsto y$. Note that $x(z) \leadsto y(z)$ is inhabited for any $z: P$ by virtue of $\0$-\textsc{elim}. Thus, we get a term of type $\Pi_{z:P}x(z)\leadsto y(z)$. By $\sfe$, it follows that $x\leadsto y$ is inhabited, as desired.
\end{exmp}

\medskip

It turns out that, despite our terminology, $\wfe$ is logically at least as strong as $\sfe$. Before proving this, we state a few intermediate results.

\bigskip
Whereas the axiom of choice is not provable in $\mathsf{ZF}$, its type-theoretic formulation \`a la propositions-as-types is easily derivable in our MLDTT.

\begin{theorem}[Axiom of choice]
Let $P$ be a type family over $A$. Also, for each $x:A$, let $C(x)$ be a type family over $P(x)$. Then we have a term $$\ac: \left(\prod_{x:A}\sum_{y:P(x)} C(x, y)\right) \to \left( \sum_{s:  \Pi_{x:A} P(x)}\prod_{x:A} C\left(x, s(x)\right) \right) .\footnote{This is called the axiom of choice because it is a direct translation of the set-theoretic axiom of choice under the Curry-Howard correspondence. Despite this formal similarity, the set-theoretic version is much  stronger in FOL than the type-theoretic version is in our deductive system. In fact, a suitably strong type-theoretic version is not derivable in our MLDTT.}$$
\end{theorem}
\begin{proof}
Take $\ac$ to be the term $\lambda h .\left(\lambda x . \pr_{1}(h(x)), \lambda x . \pr_{2}(h(x))\right)$.
\end{proof}

Under the propositions-as-types doctrine, the following result states that we can always switch the order of bounded quantifiers in a proposition of the form $\forall x \exists y C(x,y)$.

\begin{theorem}\label{acequiv}
Assume $\wfe$. Then  $\ac$ is an equivalence.\footnote{\cite[Lemma 2.5.6]{Rijke2}. This relies on $\Pi$-$\eta$.}
\end{theorem}

\begin{corollary}\label{contr}
Assume $\wfe.$ Let $P$ be a type family over $A$ and $f: \prod_{x:A} P(x)$. Then $$  \sum_{g: \Pi_{x:A} P(x)} g \approx f    $$ is contractible. 
\end{corollary}
\begin{proof}
By \cref{total-cont}, we know that the type $\sum_{y: P(x)} y \leadsto f(x)$ is contractible for each $x:A$. By $\wfe$, it follows that $$\prod_{x:A} \sum_{y: P(x)} y \leadsto f(x) $$ is also contractible. 
It is easy to see that any type that is equivalent (hence isomorphic) to a contractible type is contractible.
Thus, it follows from \cref{acequiv} that $\sum_{g: \Pi_{x:A} P(x)} g \approx f$ is contractible. 
\end{proof}

\begin{theorem}\label{wfe-sfe}
If $\wfe$ is derivable in our MLDTT, then so is $\sfe$.
\end{theorem}
\begin{proof}
Assume $\wfe.$  Let $P$ be a type family over $A$ and $f: \prod_{x:A}P(x)$.  
We want to show that $$\prod_{g: \Pi_{x:A} P(x)} \isequiv\left(\happly(f,g)\right)$$ is inhabited.
\smallskip

Consider the function $$  \lambda g. \happly_{A,P}(f,g) : \prod_{x:A}P(x) \to \left(\left(f \leadsto g\right) \to \left( \prod_{x:A} f(x) \leadsto g(x) \right) \right).$$
By \cref{fwe}, it suffices to show that $$\sigma_{ \lambda g. \happly_{A,P}(f,g)} : \left(\sum_{g : \prod_{x:A} P(x)} f \leadsto g \right) \to \sum_{g: \prod_{x:A} P(x)} f\approx g $$ is an equivalence.   Note that \cref{total-cont} and \cref{contr} imply that the ``domain" type and the ``codomain" type here are contractible, respectively.  Therefore, $\sigma_{ \lambda g. \happly_{A,P}(f,g)}$ is an equivalence  by \cref{contr-equiv}.
\end{proof}

The derivability of $\wfe$ in our system is useful for proving many inferences about propositional equality of functions. The next two results are small examples of this.

\begin{exmp}\label{bot-cont}
Assume $\wfe$. For any type family $P$ over $\0$, we have that $\prod_{x:\0} P(x)$ is contractible. 
\end{exmp}
\begin{proof}
By induction, there is some dependent function $f : \prod_{x: \0} P(x)$. For any $g : \prod_{x:\0}P(x)$, another use of induction shows that $f \sim g$. By $\sfe$, it follows that $f \leadsto g$ is inhabited. Therefore, $\prod_{x:\0} P(x)$ is contractible, with center of contraction $f$. 
\end{proof}

\begin{exmp}\label{prod-prop}
Assume $\wfe$. Let $B$ be a type family over $A$. Suppose that for each $x:A$, the type $B(x)$ is an h-proposition. Then $\prod_{x:A}B(x)$ is an h-proposition.
\end{exmp}
\begin{proof}
Let $f, g: \prod_{x:A}B(x)$. For any $x:A$, we see that $f(x)$ and $g(x)$ are propositionally equal. By $\sfe$, it follows that $f$ and $g$ are propositionally equal. 
\end{proof}


\begin{corollary}
Assume $\wfe$. For any function $f: A \to B$, the type $$\isequiv(f) \simeq \iscontmap(f)$$ is inhabited. 
\end{corollary}
\begin{proof}
By \cref{mere-prop} along with \cref{equiv-cont}, it suffices to show that both $\isequiv(f)$ and $\iscontmap(f)$ are h-propositions. The fact that the former is an h-proposition is precisely \cite[Theorem 4.3.2]{HoTT}. Moreover,  \cite[Lemma 3.1..4]{HoTT} states that $\iscont(E)$ is an h-proposition for any type $E$. By \cref{prod-prop}, it follows that $\iscontmap(f)$ is an h-proposition, as desired. 
\end{proof}

\begin{remark}
 By contrast, even if $\wfe$ holds, it is \emph{not} the case that for any function $f: A \to B$, $\iso(f) \simeq \isequiv(f)$. Indeed, $\iso(f)$ need not be an h-proposition \cite[p. 77]{HoTT}.
\end{remark}

\subsection{Univalence}

\begin{lemma}\label{considequiv}
Let $B$ be a type family over $A$. For any $x,y:A$,  there is some function $$\equiveq_{{x:A}; B(x)}(x,y) : \left(x \leadsto  y\right) \to \left(B(x) \simeq B(y)\right).$$  
\end{lemma}
\begin{proof}
By path induction, it suffices to construct a term $\equiveq_{x,y}(\refl_x) : B(x) \simeq B(x)$. We can choose $\idmap_{B(x)}$ thanks to \cref{idequiv}.
\end{proof}

\begin{definition}[Univalence axiom ($\univ$)]\label{uaxiom}
For any $A, B : \U$, the function $$   \equiveq_{{x:\U}; \el(x)}(A, B) : \left(A \leadsto_{\U} B\right) \to \left(\el(A) \simeq \el(B)\right) $$ is an equivalence. 
Formally, we postulate the logical rule
\bmp
\inferrule*{ }{ \vdash \mathsf{univ} :  \prod_{x,y:\U} \isequiv\left(\equiveq_{{x:\U}; \el(x)}(x,y)\right)  }
.\emp
\end{definition}

\smallskip

In general, we say that   a type family $B$ over $A$ is \textit{univalent} if the function $\equiveq_{{x:A}; B(x)}(a,b)$ is an equivalence for any $a,b:A$. Therefore, the univalence axiom states that the type family $x:\U \vdash \el(x) \type$ is univalent.
In particular, there is some homotopy inverse $$\equiveq(A, B)^{-1}: \left(\el(A) \simeq \el(B)\right) \to \left(A \leadsto_{\U} B\right).$$ Informally, this means that whenever two types are equivalent, they are propositionally equal. 

\smallskip

\begin{theorem}\label{nfe}
Let $A$ and  $B$  be types. Then $\univ$ implies that there is some term of type $$\prod_{f,g :A\to B} \left(f \approx g\right) \to \left(f\leadsto g\right).\footnote{\cite[Lemma 2.7.6]{Rijke2}.}$$
\end{theorem}

\begin{corollary}[Voevodsky]
$\univ \implies \wfe \implies \sfe$.
\end{corollary}
\begin{proof}
In light of \cref{wfe-sfe}, it just remains to prove that if $\univ$ is derivable, then so is $\wfe.$ 
\medskip



Assume $\univ$. Suppose that $P$ is a type family over $A$ and that we have a term of type $$ \prod_{x:A}\iscont(P(x)).$$ Since $\U$ is closed under all logical constructors, we may assume that $A$ as well as each $P(x)$ is a small type. We must show that $\prod_{x:A} P(x)$ is contractible.  Define $F: A \to \U$ as the constant map $\lambda x.\hat{\1}$. Since both $P(x)$ and $ \1$ are contractible for any $x:A$, we have that $P(x) \simeq  \1$ is inhabited. By $\univ$, it follows that $\widehat{P(x)} \leadsto_{\U}  \hat{\1}$.
By path induction, we see that the type $$  \left(\hat{\prod}_{x:A} \widehat{P(x)} \right) \leadsto_{\U} \left(\hat{\prod}_{x:A} \overbrace{F(x)}^{\hat{1}} \right)  $$ is inhabited, so that $$  \prod_{x:A}P(x) \simeq \prod_{x:A}\el(F(x))   $$ is also inhabited. 

\medskip
Hence it suffices  to show that the righthand type is contractible.  But any function $f: A \to \1$ is homotopic to $\lambda x.\star$. \Cref{nfe} thus shows  that $f$ and $\lambda x.\star$ are, in fact, propositionally equal. It follows that $\prod_{x:A} F(x) $ is contractible, with center of contraction $\lambda x.\star$.
\end{proof}

For a concrete application of $\univ$ to algebra within Martin-L\"of  dependent  type theory, see \cref{groups}.


\section{Categorical semantics}\label{catsem}

Any variant of CDTT is purely a formal language. To give its well-formed expressions meanings, we interpret them as certain mathematical objects. Specifically, we define an interpretation of them as certain structures within a suitable category, thereby providing the CDTT with a \textit{categorical semantics}.

\medskip  

This section develops those notions from categorical semantics which  \cref{models} will rely on. First of all, it is worth summarizing our set-theoretic foundations for category theory.

\begin{definition}
A \textit{Grothendieck universe} is a transitive set $U$ such that
\be[label=(\roman*)]
\item $\N \in U$,
\item $x\in U \implies \mathcal{P}(x)\in U$, and
\item for any $I\in U$ and function $u: I \to U$, $\bigcup_{i\in I}u(i) \in U$.
\ee 
\end{definition}

By a combination of (ii) and (iii), any subset of an element of $U$ belongs to $U$. As a result, $U$ is also closed under intersections, unions, and cartesian products. 

\begin{exmp}\label{inacc}
For any (strongly) inaccessible cardinal $\kappa$, the $\kappa$-th stage $V_{\kappa}$ of the rank hierarchy is a Grothendieck universe.
\end{exmp}


In fact, any Grothendieck universe $U$ satisfies $\mathsf{ZFC}$. Thus, by G\"odel's incompleteness theorems, it is impossible to prove the existence of a Grothendieck universe in $\mathsf{ZFC}$. This leads us to the first-order \textit{axiom of universes}:

\[
\text{For every set $s$, there exists a Grothendieck universe $U$ such that $s\in U$}.
\]

This guarantees that every class definable from $U$-small sets for some Grothendieck universe $U$ is $U'$-small for some larger    universe $U'$.

\medskip

Conversely, it is provable in $\mathsf{ZFC}$ that any Grothendieck universe $U$ has the form $V_{\kappa}$ for some  inaccessible cardinal $\kappa$.\footnote{See \autocite{Williams}.} By \cref{inacc}, it follows  that there exists a Grothendieck universe if and only if there exists an inaccessible cardinal. This means that the axiom of universes is equivalent to the large cardinal axiom that there exist arbitrarily large inaccessible cardinals. 

\medskip


For us, category theory will be formulated in the extension of $\mathsf{ZFC}$ by this axiom (which, of course, has \emph{not} been proven inconsistent). Moreover, the category $\mathbf{Set}$ will consist of all $U$-small sets for a sufficiently large Grothendieck universe $U$. In particular, this means that if $\c$ is a locally small category, then we can pass to a universe $U'$ larger than $U$ via the axiom of universes so that  $\c$ itself is a $U'$-small category.\label{ssetsmall}  For this reason, we may regard, for example, $\sset$ as a small category.

\begin{remark}
Most of our results, however, do hold in $\mathsf{ZFC}$.
\end{remark}

\smallskip

\begin{remark}
Throughout this section, all categories are assumed to be locally small. 
\end{remark}

\subsection{The syntactic category}

To begin with,  we build a category directly out of our MLDTT. In \cref{CC}, we shall see  this category determines the canonical semantics of $\cdtt$.  

\begin{notation}
Let $\T$ denote our MLDTT (without $\univ$).
\end{notation}

\begin{definition}[Context morphism]
Let $\Gamma$ and $\Delta \coloneqq {x_1 :A_1, \ldots, x_n :A_n}$ be well-formed contexts in $\T$. Further, let $t_1$, \ldots, $t_n$ denote $0$-expressions.
\bi
\item Recalling \cref{ssub}, suppose that the following $n$ judgments are derivable in $\T$:
\begin{align*}
\Gamma & \vdash t_1 : A_1 \type
\\ \Gamma & \vdash t_2 : A_2[t_1/x_1] \type
\\ & \vdots
\\ \Gamma & \vdash  t_n : A_n[t_1/x_1][t_2/x_2]\cdots[t_{n-1}/x_{n-1}] \type.
\end{align*}
Then the tuple $f\coloneqq \left(t_1, \ldots, t_n \right)$ is a \textit{context morphism from $\Gamma$ to $\Delta$}, written as $f : \Gamma \sra \Delta$.
\item Let $f \coloneqq \left(t_1, \ldots, t_n \right)$ and $g\coloneqq \left(s_1, \ldots, s_n \right)$ be context morphisms from $\Gamma$ to $\Delta$. Suppose that the following $n$ judgments are derivable in $\T$:
\begin{align*}
\Gamma & \vdash t_1 \equiv s_1 : A_1 
\\ \Gamma & \vdash t_2 \equiv s_2 : A_2[t_1/x_1]
\\ & \vdots
\\ \Gamma & \vdash  t_n\equiv s_n : A_n[t_1/x_1][t_2/x_2]\cdots[t_{n-1}/x_{n-1}] .
\end{align*}
Then $f$ and $g$ are \textit{judgmentally equal context morphisms from $\Gamma$ to $\Delta$}, written as \linebreak $f \equiv g :  \Gamma \sra \Delta$.
\ei 
\end{definition}

\begin{notation}
$\K\left[f/\Delta\right] \coloneqq \K[t_1/x_1][t_2/x_2]\cdots[t_n/x_n]$.
\end{notation}

\begin{exmp}[Display map]
 Suppose that $\ctx\left(\Gamma, x: A\right)$ is derivable in $\T$ where \linebreak $\Gamma \coloneqq x_1:A_1, \ldots, x_n:A_n$. Then each of
\begin{align*}
\Gamma, x:A & \vdash x_1 :A_1 \type 
\\ \Gamma, x:A & \vdash x_2 :A_2 \type 
\\ & \vdots
\\ \Gamma, x:A & \vdash x_n :A_n \type 
\end{align*}
is derivable in $\T$ due to the structural rule \textsc{Vble}. Thus, $p_A \coloneqq \left(x_1, \ldots, x_n\right)$ is a context morphism from $\Gamma, x: A$ to $\Gamma$, called the \textit{display map of $A$}. 

\smallskip

From a syntactic perspective, this represents a type family $A$ over the types appearing in $\Gamma$. From a topological perspective, it represents a fiber bundle over $\Gamma$.
\end{exmp}

\begin{definition}[Syntactic category]\label{syncat}
Define the \textit{syntactic category $\c(\T)$ of $\T$} as follows.
\bi
\item Let $\ob{\c(\T)}$ be the quotient of the set of  all well-formed contexts in $\T$  by the equivalence relation $\sim$ where $$\Gamma \sim \Delta \iff \Gamma \equiv \Delta \ctx \text{ is derivable in }\T.$$  
\item For any $\left[\Gamma\right], \left[\Delta\right] \in \ob{\c(\T)}$, let  $\c(\T)(\Gamma, \Delta)$ be the quotient of the set of all context morphisms from  $\Gamma$ to $\Delta$ by the equivalence relation ${\sim'}$ where 
$$\left[f\right] {\sim'} \left[g\right] \iff f\equiv g : \Gamma \sra \Delta  \text{ is derivable in }\T.$$ This is well-defined because of the derived rule  
\bmp
\inferrule*{\Gamma_1 \equiv \Gamma_2 \\ \Gamma_3 \equiv \Gamma_4 \\ f: \Gamma_1 \sra \Gamma_3 }{f : \Gamma_2 \sra \Gamma_4 }.
\emp
\item For any two (equivalence classes of) context morphisms $f  : \Gamma \sra \Delta$ and $g\coloneqq \left(s_1, \ldots, s_n \right):\linebreak \Delta \sra \Theta$,  let $$g \circ f  = \left(s_1\left[f/\Delta\right], \ldots, s_n\left[f/\Delta\right] \right).$$ 
\ei
\end{definition}

It is straightforward yet tedious to verify that the operation $\circ$ is well-defined and associative and that for any well-formed context $\Gamma \coloneqq {x_1 :A_1, \ldots, x_n :A_n}$, the morphism $\left(x_1, \ldots, x_n\right)$ is well-defined and constitutes the identity morphism $\idd_{\Gamma}$ for (the equivalence class of) $\Gamma$. It follows that $\c(\T)$ is, indeed, a category. 

\begin{note}\label{terms} $ $
\be[label=(\arabic*)]
\item Let $\Gamma \coloneqq x_1:A_1, \ldots, x_n:A_n$. We have a one-to-one correspondence between typing declarations $\Gamma \vdash a : A$ and sections of the display map $p_A$ given by mapping $\Gamma \vdash a : A$ to $\left(x_1, \ldots, x_n, a\right)$.
\item  The set $\ob{\c(\T)}$ is $\N$-graded in that there is a natural bijection $\ob{\c(\T)} \cong \coprod_{n\in \N}C_n$ where $C_n$ denotes the quotient of the set of all well-formed contexts in $\T$ of length $n$ by $\sim$. Thus, $\ob{\c(\T)}$ may be viewed as a rooted tree with the following properties.  
\be
\item Its root is precisely the empty context.
\item Its $n$-th level is precisely $C_n$.
\item For any node $\Gamma, x:A$ of degree $n\geq 1$, its parent is precisely $\Gamma$. 
\ee
In particular, the empty context is the terminal object of $\c(\T)$ as well as the unique object of degree zero.
\ee
\end{note}


\medskip

Let $f \coloneqq \left(t_1, \ldots, t_n\right) : \Delta \sra \Gamma$ be a morphism in $\c(\T)$. Suppose that both $\ctx\left(\Delta, y : A\left[f/\Gamma\right]\right)$ and $\ctx\left(\Gamma, x :A\right)$ are derivable in $\T$. Note that $$q(f,A) \coloneqq \left(t_1, \ldots, t_n, y\right)$$ is a morphism from $\Delta, y : A\left[f/\Gamma\right]$ to $\Gamma, x :A$ because $\Delta, y : A\left[f/\Gamma\right] \vdash y : A\left[f/\Gamma\right]$ is derivable in $\T$ by the structural rule  \textsc{Vble}.

\begin{lemma}
The commutative square
\begin{equation}
\label{eqn: cansyn} \begin{tikzcd}[column sep=large]
{\Delta, y : A\left[f/\Gamma\right]} \arrow[d, "{p_{A\left[f/\Gamma\right]}}"'] \arrow[r, "{q(f,A)}"] & {\Gamma, x :A} \arrow[d, "p_A"] \\
\Delta \arrow[r, "f"']                                                                                  & \Gamma                           
\end{tikzcd} \tag{1}
\end{equation}
 is a pullback  in $\c(\T)$.
\end{lemma}
\begin{proof}
Suppose that
\[
\begin{tikzcd}[column sep=large]
\Theta \arrow[rdd, "g_1"', bend right] \arrow[rrd, "g_2", bend left] &                                                                                                          &                                   \\
                                                                     & {\Delta, y : A\left[f/\Gamma\right]} \arrow[d, "{p_{A\left[f/\Gamma\right]}}"'] \arrow[r, "{q(f,A)}"] & {\Gamma, x :A} \arrow[d, "p_A"] \\
                                                                     & \Delta \arrow[r, "f"']                                                                                   & \Gamma                           
\end{tikzcd}
\] commutes in $\c(\T)$. We must show that there exists a unique morphism $g : \Theta \to {\Delta, y : A\left[f/\Gamma\right]}$  that fits into this diagram. Since $p_A \circ g_2 = f \circ g_1$, we see that $g_2 = \left( f \circ g_1, \tau \right)$ for some term expression $\tau$ such that $\Theta \vdash \tau : A\left[f \circ g_1/\Gamma \right]$ is derivable in $\T$. But $A\left[f \circ g_1/\Gamma \right] = A\left[f/\Gamma\right]\left[g_1/\Delta\right],$ so that $g \coloneqq \left(g_1, \tau\right)$ is a morphism $\Theta \to {\Delta, y :A[f/\Gamma]}$. This satisfies 
\begin{align*}
 p_{A\left[f/\Gamma\right]} \circ g  & = g_1 
\\ & 
\\ \label{eqn:one} q(f,A) \circ g  & =  \left(f \circ g_1, \tau\right) \tag{$\ast$}
\\ & = g_2,
\end{align*}
and thus $g$ fits into our diagram. To prove that $g$ is unique, let $\tilde{g}$ be any other such morphism $\Theta \to {\Delta, y : A\left[f/\Gamma\right]}$. As $p_{A\left[f/\Gamma\right]} \circ \tilde{g} = g_1$, we have that $\tilde{g}$ is of the form $\left(g_1, \tilde{\tau}\right)$ for some  $\tilde{\tau}$ such that $\theta \vdash \tilde{\tau} : A[g_1/\Delta]$ is derivable in $\T$. Moreover, using \eqref{eqn:one}, we have that
\begin{align*}
 \left(f \circ g_1, \tau\right) & = g_2 
\\ & = q(f,A) \circ \hat{g} 
\\ & = \left(f, y\right) \circ \left(g_1, \tilde{\tau}\right)
\\ & = \left(f \circ g_1, \tilde{\tau}\right).
\end{align*}
This implies that $\Theta \vdash \tau \equiv \tilde{\tau} : \underbrace{A[f\circ g_1/\Gamma]}_{A\left[f/\Gamma\right]\left[g_1/\Delta\right]}$ is derivable in $\T$. This means that $$\tilde{g} = \left(g_1, \tilde{\tau}\right) = \left(g_1, \tau\right) = g,$$ as required. 
\end{proof}


\subsection{Contextual categories}\label{CC}

There are at least three reasons for looking at contextual categories for our categorical semantics. First, the class of objects in a contextual category has a tree-like structures, just as the set of well-formed contexts of our MLDTT. Second, contextual categories are suitable for interpreting type equality judgments of our MLDTT as their objects can be compared for a kind of equality, not just for isomorphism. Third, they possess a  class of distinguished pullbacks, which must be strictly functorial and must commute strictly with logical constructors such as dependent products, thereby resembling syntactic substitution.

\medskip 

Overall, the notion of a contextual category is designed to abstract the key structure from \cref{syncat}.

\begin{definition}[Contextual category]\label{ccat} A \textit{contextual category $\c$} is a category  such that
\be[label=(\roman*)]
\item $\c$ comes equipped with a terminal object $1$,
\item $\ob{\c}$ is $\N$-graded, i.e., is of the form $\coprod_{n\in \N}\ob_n{\c}$,
\item $\ob_0(\c) = \left\{1\right\}$,
\item for each $n$, $\c$ comes equipped with a (class) function $\ft_n: \ob_{n+1}{\c} \to \ob_n{\c}$,
\item for each $X\in \ob_{n+1}{\c}$, $\c$ comes equipped with a morphism $p_X : X \to \ft_n(X)$ (called the \textit{display map of $X$}),
\item for each $X\in \ob_{n+1}{\c}$ and map $f: Y \to \ft_n(X)$, $\c$ comes equipped with an object $f^{\ast}{X}$ and a morphism $q(f,X)$ such that $\ft\left(f^{*} X\right)=Y$ and 
\[
\begin{tikzcd}[column sep=large]
f^{\ast}{X} \arrow[d, "p_{f^{\ast}{X}}"'] \arrow[r, "{q(f,X)}"] & X \arrow[d, "p_X"] \\
Y \arrow[r, "f"']                                               & \ft_n(X)          
\end{tikzcd}
\] is a pullback square (called the \textit{canonical pullback} of $X$ along $f$), and
\item every canonical pullback is strictly functorial in the sense that
\begin{align*}
\idd_{\ft_n(X)}^{\ast}{X}& = X
\\ q\left(\idd_{\ft_n{X}}, X\right) & = \idd_X
\\ \left(fg\right)^{\ast}{X} & = g^{\ast}\left(f^{\ast}{X}\right) 
\\ q\left(fg, X\right) & = q\left(f, X\right)q\left(g, f^{\ast}{X}\right)
\end{align*}
for any $X\in \ob_{n+1}{\c}$ and morphisms $f: Y \to \ft_n(X)$ and $g:Z \to Y$.
\ee
\end{definition}


\begin{remark}
To motivate condition (vii), let $\Gamma \vdash A \type$ be derivable in $\T$ and let $f: \Delta \to \Gamma$ and $g: \Theta \to \Delta$ be context morphisms in $\T$. On the one hand, it is easy to check that \[
A\left[f/\Gamma\right]\left[g/\Delta\right] = A\left[fg/\Gamma\right] 
\] as strings. On the other hand, consider the commutative diagram
\[
\begin{tikzcd}[row sep=large]
\left(fg\right)^{\ast}{X} \arrow[rd, "\gamma", dashed] \arrow[rdd, "p_{\left(fg\right)^{\ast}{X}}"', bend right] \arrow[rrrd, "{q(fg, X)}", bend left] &                                                                                                                                 &                                                                            &                                \\
                                                                                                                                                       & g^{\ast}\left(f^{\ast}{X}\right) \arrow[d, "p_{g^{\ast}\left(f^{\ast}{X}\right)}" description] \arrow[r, "{q(g, f^{\ast}{X})}"] & f^{\ast}{X} \arrow[d, "p_{f^{\ast}{X}}" description] \arrow[r, "{q(f,X)}"] & X \arrow[d, "p_X" description] \\
                                                                                                                                                       & \ft(g^{\ast}\left(f^{\ast}{X}\right)) \arrow[r, "g"']                                                                           & \ft(f^{\ast}{X}) \arrow[r, "f"']                                           & \ft(X)                        
\end{tikzcd}
.\] If (vii) is omitted, then the induced map $\gamma$, though an isomorphism in $\c$, need not be the identity morphism.  Therefore, without \emph{strict} functoriality, interpreting typehood judgments of the form $\Delta \vdash A\left[f/\Gamma\right]$ as canonical pullbacks in $\c$ may be unsound. Yet, we want a contextual category to carry a structure similar to that of the syntactic category of $\T$, in which $\Delta \vdash A\left[f/\Gamma\right]$ is correctly interpreted as the canonical pullback \eqref{eqn: cansyn}.    
\end{remark}

\begin{exmp}
The category $\set$ of sets carries the data of a contextual category $\d$. Indeed, define the grading of $\ob{\d}$ by mutual recursion with a function $\delta : \ob{\d} \to \set$ as follows.
\bi
\item $\ob_0{\d}\equiv \left\{\emptyset\right\}$, and $\delta(\emptyset)\equiv \left\{\emptyset\right\}$.
\item The class $\ob_{n+1}{\d}$ consists of pairs $\left(X,A\right)$ where $X\in \ob_n{\d}$ and $A: \delta(X)\to \set$, and $\delta(X,A) \equiv \left\{\left(x,a\right) \mid x\in \delta(X), a\in A(x)\right\}$.
\ei
Let $\Hom_{\d}(X, Y)$ consist of all set-theoretic functions $\delta(X) \to \delta(Y)$. The remaining data of a contextual category are given as follows. For any $\left(X,A\right) \in \ob_{n+1}{\d}$ and $f \in \Hom_{\d}(Y,X)$,
\bi
\item $\ft(X, A) \equiv X$.
\item $p_{\left(X, A\right)}(x,a) \equiv x$.
\item $f^{\ast}(X,A) \equiv \left(Y, A \circ f\right)$, and $q(f, \left(X, A\right))(x,a)\equiv \left(f(x), a\right)$.
\ei
Let $g: \delta(Y) \to \delta(X)$ be any set-theoretic function. Define $A: \delta(X) \to \set$ by $A(x) =g^{-1}(x)$. It is easy to check that the function $\psi : \delta(Y) \cong \delta(X,A)$ given by $y \mapsto \left(g(y), y\right)$ is bijective and that $p_{\left(X,A\right)} \circ \psi = g$. Thus, every map in $\d$ is isomorphic to a canonical projection. In an arbitrary contextual category, then, we may think of canonical projections as resembling set maps.
\end{exmp}

\begin{note}[Binary products]
Let $\c$ be a contextual category. For any object $X$ in $\c$, we can form the pullback square 
\[
\begin{tikzcd}[column sep=large]
p_X^{\ast}{X} \arrow[d, "p_{p_X^{\ast}}"'] \arrow[r, "{q(p_X, X)}"] & X \arrow[d, "p_X"] \\
X \arrow[r, "p_X"']                                                 & \ft(X)            
\end{tikzcd}
.\] Then $p_X^{\ast}{X}$ is the product $X\times X$ in $\c$ with projection map $p_{p_X^{\ast}}$.
\end{note}

\begin{note}[Pullback section]
Suppose that $s$ is a section of $p_X$. Then, by the universal property of pullback squares, there exists a unique morphism $f^{\ast}{s} : Y \to f^{\ast}{X}$ such that
\[
\begin{tikzcd}[column sep=large]
Y \arrow[rd, "f^{\ast}{s}"] \arrow[rdd, "\idd_Y"', bend right] \arrow[rrd, "s\circ f", bend left] &                                                                  &                    \\
                                                                                              & f^{\ast}{X} \arrow[d, "p_{f^{\ast}{X}}"'] \arrow[r, "{q(f,X)}"] & X \arrow[d, "p_X"'] \\
                                                                                              & Y \arrow[r, "f"']                                                & \ft(X) \arrow[u, "s"', bend right=49]         
\end{tikzcd}
\] commutes. This means that $f^{\ast}{s}$ is a section of $p_{f^{\ast}{X}}$.
\end{note}

\begin{notation}\label{bignot} $ $
\be
\item For any $X\in \ob_{n}{\c}$, we may write $\left(X, A\right)$ for any object $Y \in \ob_{n+1}{\c}$ such that $\ft(Y) = X$. Also, we may write $\left(X, A, B\right)$ for any object $Y \in \ob_{n+2}{\c}$ such that $\ft(Y) = \left(X, A\right)$. 
\item For any such $X$ and any map $f: Y \to X$ in $\c$, we may write $\left(Y, f^{\ast}{A}\right)$ for the canonical pullback $f^{\ast}(X, A)$.
\item For any such $X$ and $f$, we may write $p_A$ for the display map $p_{(X,A)} : \left(X, A\right) \to X$ and $q(f, A)$ for the map $q(f,X,A)$.
\item For any such $X$, we may write $p_{A,B}$ for the composite of display maps $p_A\circ p_B : \left(X, A, B\right) \to X$.
\item For any such $X$, we may write $s_{A}$ for any section of $p_A$.  
\item For any commutative diagram of the form
\[
\begin{tikzcd}[column sep=large]
Y  \arrow[rdd, "g"', bend right] \arrow[rrd, "h", bend left] &                                                                  &                    \\
                                                                                              & f^{\ast}{X} \arrow[d, "p_{f^{\ast}{X}}"'] \arrow[r, "{q(f,X)}"] & X \arrow[d, "p_X"] \\
                                                                                              & Y \arrow[r, "f"']                                                & \ft(X)          
\end{tikzcd}
,\] we may write $\left\langle f^{\ast}{X}, g, h \right\rangle$ for the unique morphism $Y \to f^{\ast}{X}$ fitting into it.
\ee
\end{notation}

\smallskip

Using the setting of contextual categories, let us begin formally defining a notion of truth of a judgment in $\T$. Let $\c$ be a contextual category. We want $\c$ to carry a structure for each logical constructor in $\T$ as well as a structure for the universe type in $\T$. Defining such a structure on $\c$ amounts to 
\be[label=(\alph*)] 
\item translating the main logical rules (except congruence rules) governing the given constructor or universe into the language of contextual categories and
\item stipulating a so-called stability condition so that  canonical pullbacks commute strictly with the given constructor or universe.
\ee In the interest of space, we describe here the structures for just $\Pi$, $\0$, $\id$, and $\U$. For the other logical constructors, see \cite[Appendix B]{KL}.  

\medskip

A \textit{$\Pi$-type structure on $\c$} consists of the following data:\label{pitype}
\be[label=(\roman*)]
\item for each $\left(\Gamma, A, B\right) \in \ob_{n+2}{\c}$, an object $\left(\Gamma, \Pi(A, B)\right)\in \ob_{n+1}{\c}$;  
\item for each such $\left(\Gamma, A, B\right)$ and each section $b$ of $p_B$, a section $\lambda(b)$ of $p_{\Pi(A, B)}$;
\item for each such $\left(\Gamma, A, B\right)$, each section $k$ of $p_{ \Pi(A, B)}$, and each section $a$ of $p_A$, a section $\app(k,a)$ of the composite $p_{A,B}$ such that $p_B \circ \app(k,a)=a$,
\item $\app(\lambda(b), a)=b\circ a$, and
\item for each map $f: \Gamma' \to \Gamma$ in $\c$, we have that
\begin{align*}
f^{\ast}(\Gamma, \Pi(A,B)) & = \left(\Gamma', \Pi(f^{\ast}{A}, f^{\ast}{B})\right)
\\ f^{\ast}\lambda(b) & = \lambda(f^{\ast}{b})
\\ f^{\ast}\app(k,a) & = \app(f^{\ast}{k}, f^{\ast}{a}).
\end{align*}
\ee

\begin{remark}
We have left out the data for $\Pi$-$\eta$ for simplicity. Nevertheless, they are present in every model of $\T$ studied in \cref{models}.
\end{remark}

\medskip

A \textit{$\0$-type structure on $\c$} consists of the following data:
\be[label=(\roman*)]
\item for each $\Gamma \in \ob{\c}$, an object $\left(\Gamma, \0_{\Gamma}\right)$; 
\item for each object $\left(\Gamma, \0_{\Gamma}, A\right)$, a section $\ind_{\0}(A)$ of $p_A$ such that
\item for each $f: \Gamma' \to \Gamma$, we have that
\begin{align*}
f^{\ast}(\Gamma, \0_{\Gamma}) & = \left(\Gamma', \0_{\Gamma'}\right)
\\ f^{\ast}(\ind_{\0}(A)) & = \ind_{\0}(f^{\ast}{A}).
\end{align*}
\ee

\medskip

A \textit{$\id$-type structure on $\c$} consists of the following data:\label{idtype}
\be[label=(\roman*)]
\item for any $\left(\Gamma, A\right)\in \ob_{n+1}{\c}$ and sections $a$ and $b$ of $p_A$, an object $\left(\Gamma, A, p_A^{\ast}{A}, \id_A\right)$;
\item for each such $\left(\Gamma, A\right)$, a morphism $\refl_A :\left(\Gamma, A\right) \to \left(\Gamma, A, p_A^{\ast}{A}, \id_A\right)$ such that $p_{\id_A} \circ \refl_A = \Delta_A$;
\item for each $\left(\Gamma, A, p_A^{\ast}{A}, \id_A, B\right)\in \ob_{n+4}{\c}$ and map $d: \left(\Gamma, A\right) \to \left(\Gamma, A, p_A^{\ast}{A}, \id_A, B\right)$ satisfying $p_B \circ d = \refl_A$, a section $\J_{B,d}$ of $p_B$ such that $\J_{B,d} \circ \refl_A =d$ and
\item for each $f: \Gamma' \to \Gamma$, we have that
\begin{align*}
f^{\ast}(\Gamma, A, p_A^{\ast}{A}, \id_A)  & = \left(\Gamma', f^{\ast}{A}, \left(p_{f^{\ast}{A}}\right)^{\ast}(f^{\ast}{A}), \id_{f^{\ast}{A}}\right)
\\  f^{\ast}{\refl_A} & = \refl_{f^{\ast}{A}}
\\ f^{\ast}{\J_{B,d}} & = \J_{f^{\ast}{B}, f^{\ast}{d}}.
\end{align*}
\ee

\begin{note} $ $
\be
\item Consider the commutative diagram
\[
\begin{tikzcd}
{\left(\Gamma, A\right)} \arrow[rdd, bend right, equal] \arrow[rrd, "\refl_A", bend left] \arrow[rd, dashed] &                                                                      &                                                          \\
                                                                                                      & {\left(\Gamma, A, \Delta_A^{\ast}{\id_A}\right)} \arrow[d] \arrow[r] & {\left(\Gamma, A, p_A^{\ast}{A}, \id_A\right)} \arrow[d] \\
                                                                                                      & {\left(\Gamma, A\right)} \arrow[r, "\Delta_A"']                      & {\left(\Gamma, A, p_A^{\ast}{A}\right)}                 
\end{tikzcd}
\] where $\Delta_A$ stands for the diagonal morphism $\Delta_{\left(\Gamma, A\right)}$. Note that the dashed arrow $$\left\langle \left(\Gamma, A, \Delta_A^{\ast}{\id_A}\right), \idd_{\left(\Gamma, A\right)}, \refl_A\right\rangle$$ is a section of $p_{\Delta_A^{\ast}{\id_A}}$. 

\item Consider the pullback square
\[
\begin{tikzcd}
{\left(\Gamma, A, \refl_A^{\ast}{B}\right)} \arrow[d] \arrow[r] & {\left(\Gamma, A, p_A^{\ast}{A}, \id_A, B\right)} \arrow[d] \\
{\left(\Gamma, A\right)} \arrow[r, "\refl_A"']           & {\left(\Gamma, A, p_A^{\ast}{A}, \id_A\right)}             
\end{tikzcd}.
\] Then for any section $s$ of $p_{ \refl_A^{\ast}{B}}$, we have that $p_B \circ \underbrace{q(\refl_A, B) \circ s}_{d} = \refl_A$. Thus, $d$ yields a distinguished section $\J_{B, d}$.  
\ee
\end{note}

\medskip

A \textit{$\U$-type structure on $\c$} is a distinguished object $\left(\U, \el\right) \in \ob_{2}{\c}$ that is closed under each type structure on $\c$. For example, it is closed under $\Pi$-types in the sense that
\be[label=(\roman*)]
\item  for any two maps $a: \Gamma \to \U$ and $b: \left(\Gamma, a^{\ast}{\el}\right) \to \U$, $\c$ comes equipped with a morphism $\hat{\Pi}(a,b): \Gamma \to \U$ such that 
\item $\left(\Gamma, \hat{\Pi}(a,b)^{\ast}{\el}\right) = \left(\Gamma, \Pi(a^{\ast}{\el}, b^{\ast}{\el})\right)$ and
\item for each $f: \Gamma' \to \Gamma$, we have that 
$$\hat{\Pi}(a,b) \circ f
= 
\hat{\Pi}\big(a\circ f, b \circ q(f, a^{\ast}{\el})\big).$$
\ee

\begin{notation} 
Consider the diagrams
\[
\begin{tikzcd}
{\left(\Gamma, !_\Gamma^{\ast}{\U}\right)} \arrow[r] \arrow[d] & \U \arrow[d] &  & {\left(\Gamma, (q(!_\Gamma, \U)\circ s)^{\ast}{\el}, !^{\ast}{\U}\right)} \arrow[d] \arrow[r]                  & \U \arrow[d] \\
\Gamma \arrow[r, "!_\Gamma"'] \arrow[u, "s", bend left]       & 1            &  & {\left(\Gamma, (q(!_\Gamma, \U)\circ s)^{\ast}{\el}\right)} \arrow[r, "!"'] \arrow[u, "\tilde{s}", bend left] & 1          
\end{tikzcd}
\] where both $s$ and $\tilde{s}$ denote sections. We shall refer to the composite maps $q\circ s : \Gamma \to \U$ and $q\circ \tilde{s} : \left(\Gamma, (q(!_\Gamma, \U)\circ s)^{\ast}{\el}\right) \to \U$ as $\upsilon_{s, \Gamma}$ and $\upsilon_{\tilde{s}, s, \Gamma}$, respectively. 
\end{notation}

\begin{definition}[$\T$-structure]\label{model1}
A \textit{$\T$-structure} is a contextual category equipped with a structure for each logical constructor in $\T$ and a structure for the universe type in $\T$.
\end{definition}

Let $\c$ be a $\T$-structure. We proceed to define a partial function $\left\llbracket{-}\right\rrbracket$ (called an \textit{interpretation function}) on the class of all judgments in $\T$\label{intfunct} such that  $\left\llbracket{-}\right\rrbracket$  has values of the forms

\begin{alignat*}{2}
\left\llbracket{\ctx(\Gamma)}\right\rrbracket & = X, \qquad && X\in \ob{\c}
\\  \left\llbracket{\Gamma \vdash C \type}\right\rrbracket & = p_{A}, \qquad && \left(X, A\right) \in \ob{\c}
\\ \left\llbracket{\Gamma \vdash c :C}\right\rrbracket  & =  s, \qquad && p_A \circ s = \idd_X.
\end{alignat*}

Intuitively, this means that 
\bi
\item objects in $\c$ represent well-formed contexts in $\T$,
\item display maps in $\c$ represent well-formed types, and
\item sections of display maps represent inhabitants of well-formed types.
\ei  Our motivation for interpreting typing declarations as sections of display maps is precisely part (1) of \cref{terms}. 

\medskip

Specifically, $\left\llbracket{-}\right\rrbracket$ is defined, in part, via mutual recursion as follows.\footnote{Our definition is an adaptation and extension of \cite[Section 6.4]{Pitts}.}


\begin{center}
Well-formed contexts
\end{center}
\bmp
\inferrule*{ }{\left\llbracket{\ctx(\epsilon)}\right\rrbracket = 1}
\\
\inferrule*[right={when $x\notin \fv(\Gamma)$}]
{\left\llbracket{\ctx(\Gamma)}\right\rrbracket = X \\ \left\llbracket{\Gamma \vdash C \type}\right\rrbracket = p_A}{\left\llbracket{\ctx(\Gamma, x: C)}\right\rrbracket = \left(X, A\right)}
\emp

\smallskip

\begin{center}
Typing declarations
\end{center}
\bmp
\inferrule*
{\left\llbracket{\ctx(\Gamma)}\right\rrbracket = X \\ \left\llbracket{\Gamma \vdash C \type}\right\rrbracket = p_A \\\\ \left\llbracket{  \Gamma, x:C \vdash C' \type  }\right\rrbracket = p_{\left(p_A\right)^{\ast}{A}}}{ \left\llbracket{  \Gamma, x:C \vdash x:C'  }\right\rrbracket =  \left\langle \left(X, A, p_A^{\ast}{A}\right), \idd_{\left(X,A\right)}, \idd_{\left(X,A\right)}\right\rangle}
\\
\inferrule*{\left\llbracket{\ctx(\Gamma)}\right\rrbracket = X \\ \left\llbracket{\Gamma \vdash C_1 \type}\right\rrbracket = p_A \\ \left\llbracket{\Gamma, x: C_1 \vdash C_2 \type}\right\rrbracket = p_{\left(X, A, B\right)} \\\\ \left\llbracket{ \Gamma, x: C_1, x': C_2 \vdash C_3 \type}\right\rrbracket = p_{\left(p_{B,A}\right)^{\ast}{A} }  }{\left\llbracket{\Gamma, x: C_1, x':C_2 \vdash x:C_3}\right\rrbracket =  \left\langle \left(X, A, B, \left(p_{A, B}\right)^{\ast}{A}\right) , \idd_{\left(X, A, B\right)}, p_{B}\right\rangle }
\emp


\begin{center}
Dependent products
\end{center}
\bmp
\inferrule*{\left\llbracket{\Gamma \vdash C \type}\right\rrbracket = p_{\left(X,A\right)} \\ \left\llbracket{\Gamma, x: C \vdash C'(x) \type}\right\rrbracket = p_{\left(X, A, B\right)} }{\left\llbracket{\Gamma \vdash \Pi_{x:C}{C'(x)} \type}\right\rrbracket = p_{\Pi(A, B)}}
\\
\inferrule*{\left\llbracket{\Gamma \vdash C \type}\right\rrbracket = p_{\left(X,A\right)} \\ \left\llbracket{\Gamma, x: C \vdash C'(x) \type}\right\rrbracket = p_{\left(X, A, B\right)} \\\\
\left\llbracket{\Gamma, x:C \vdash c'(x) : C'(x)}\right\rrbracket = s_{B} }
{\left\llbracket{\Gamma \vdash \lambda(x:C).c'(x) : \Pi_{x:C}{C'(x)} }\right\rrbracket = \lambda(s_{B})}
\\
\inferrule*{\left\llbracket{\Gamma \vdash C \type}\right\rrbracket = p_{\left(X,A\right)} \\ \left\llbracket{\Gamma, x: C \vdash C'(x) \type}\right\rrbracket = p_{\left(X, A, B\right)} \\
\left\llbracket{\Gamma \vdash k: \Pi_{x:C}{C'(x)}}\right\rrbracket = s_{\Pi(A, B)} \\\\ \left\llbracket{\Gamma \vdash a : C}\right\rrbracket = s_A }
{\left\llbracket{ \Gamma \vdash \app(k, a) : C'[a/x]}\right\rrbracket =\left\langle \left(X, s_A^{\ast}{B}\right), \idd_X, \app(s_{\Pi(A, B)}, s_A)\right\rangle }
\emp

\smallskip

\begin{center}
Zero type
\end{center}
\bmp
\inferrule*{\left\llbracket{\ctx(\Gamma)}\right\rrbracket = X }{\left\llbracket{\Gamma \vdash \0 \type}\right\rrbracket = p_{\0_X}}
\\
\inferrule*{\left\llbracket{\Gamma, x: \0 \vdash C(x) \type}\right\rrbracket = p_{\left(X, \0_X, A\right)} \\ \left\llbracket{\Gamma \vdash a:\0}\right\rrbracket = s_{\0_X} }{\left\llbracket{\Gamma \vdash \ind_{\0}(x.C,a): C[a/x]}\right\rrbracket = \left\langle \left(X, s_{\0_X}^{\ast}{A}  \right), \idd_X, \ind_{\0}(A) \circ s_{\0_X}\right\rangle }
\emp

\pagebreak

\begin{center}
Identity types
\end{center}
\bmp
\inferrule*{\left\llbracket{\Gamma \vdash C \type}\right\rrbracket = p_{\left(X,A\right)}}{\left\llbracket{\Gamma, x:C, y:C \vdash \id_C(x,y) \type}\right\rrbracket = p_{\left(X, A, p_A^{\ast}{A}, \id_A\right)}}
\\
\inferrule*{\left\llbracket{\Gamma \vdash C \type}\right\rrbracket = p_{\left(X,A\right)}} 
{\left\llbracket{ \Gamma, x:C \vdash \refl(C, x) : \id_C(x,x)}\right\rrbracket =  \left\langle \left(X, A, \Delta_A^{\ast}{\id_A}\right), \idd_{\left(X, A\right)}, \refl_A\right\rangle}
\\
\inferrule*{\left\llbracket{\Gamma, x: C, y:C,  p: \id_C(x,y) \vdash C'(x,y,p) \type}\right\rrbracket = p_{\left(X, A, p_A^{\ast}{A}, \id_A, B\right)}  \\\\ \left\llbracket{\Gamma, z:C \vdash c: C'[z,z,\refl(C,z)/x,y,p]}\right\rrbracket =  s_{\refl_A^{\ast}{B}} }
{\left\llbracket{\Gamma, x:A, y:A, p : \id_C(x,y) \vdash \J(z.c, x,y, p) : C'(x,y,p)}\right\rrbracket = \J_{B,q(\refl_A, B) \circ s_{\refl_A^{\ast}{B}} }}
\emp

\smallskip

\begin{center}
Universe type
\end{center}
\bmp
\inferrule*{ }{\left\llbracket{{ }\vdash \U \type}\right\rrbracket = {!: \U \to 1}}
\\
\inferrule*{ }{\left\llbracket{x:\U \vdash \el(x) \type}\right\rrbracket = p_{\el}}
\\
\inferrule*{\left\llbracket{\Gamma \vdash a:\U}\right\rrbracket = s_{!_X^{\ast}{\U}}  \\ \left\llbracket{\Gamma, x:\el(a) \vdash b(x): \U}\right\rrbracket=  s_{!_{ \upsilon_{s_{!_X^{\ast}{\U}}, X}^{\ast}{\el}  }^{\ast}{\U}  }} 
{\left\llbracket{ \Gamma \vdash \hat{\Pi}(a,x.b) :\U}\right\rrbracket = \left\langle \left(X, !_X^{\ast}{\U}\right) , \idd_X, \hat{\Pi}(\upsilon, \upsilon) \right\rangle }
\\
\vdots
\emp

Due to its hideous form, we ought to describe the map $S \coloneqq s_{!_{ \upsilon_{s_{!_X^{\ast}{\U}}, X}^{\ast}{\el}  }^{\ast}{\U}  }$ explicitly:
\[
\begin{tikzcd}
{\left(X, !^{\ast}{\U}\right)} \arrow[r, "q"] \arrow[d] & \U \arrow[d] &  & {\left(X, \upsilon_{s_{!_X^{\ast}{\U}}, X}^{\ast}{\el} \right)} \arrow[d] \arrow[r] & {\left(\U, \el\right)} \arrow[d] &  & {!^{\ast}{\left(X, \upsilon_{s_{!_X^{\ast}{\U}}, X}^{\ast}{\el} \right)}} \arrow[d] \arrow[r]       & \U \arrow[d] \\
X \arrow[r] \arrow[u, " s_{!_X^{\ast}{\U}}", bend left] & 1            &  & X \arrow[r, "q\circ  s_{!_X^{\ast}{\U}}"']                                          & \U                               &  & {\left(X, \upsilon_{s_{!_X^{\ast}{\U}}, X}^{\ast}{\el} \right)} \arrow[r] \arrow[u, "S", bend left] & 1           
\end{tikzcd}
\]

\medskip

The reason that we forego a separate group of semantic rules for typehood is that these correspond to the formation rules for $\0$, $\1$, $\2$, and $\U$, i.e., our four non-dependent types. In the interest of space, we have omitted any rule defining $\left\llbracket{-}\right\rrbracket$ for $\Sigma$-types, the unit type, or the Boolean type.

\medskip

\begin{notation}
If $\varphi$ is a judgment in $\T$, then the notation $\left\llbracket{\varphi}\right\rrbracket_{\checkmark}$ means that $\left\llbracket{-}\right\rrbracket$ \emph{or an extension of it} is defined at $\varphi$.
\end{notation}

\begin{definition}[Model of type theory]\label{satis1}
Let $\c$ be a $\T$-structure and let $\varphi$ be a judgment in $\T$. We say that \textit{$\c$ satisfies $\varphi$} or \textit{$\varphi$ is true in $\c$}, written as $\c \models \varphi$, according to the following conditions.
\bi
\item $\c \models {\ctx(\Gamma)}$ if and only if  $\left\llbracket{\ctx(\Gamma)}\right\rrbracket_{\checkmark}$.
\item $\c \models {\Gamma \vdash C \type}$ if and only if $\left\llbracket{\Gamma \vdash C \type}\right\rrbracket_{\checkmark}$.
\item $\c \models {\Gamma \vdash c:C}$ if and only if $\left\llbracket{\Gamma \vdash c:C}\right\rrbracket_{\checkmark}$.
\item  $\c \models { \Gamma \equiv \Gamma'  \ctx}$ if and only if $\left\llbracket{\ctx(\Gamma)}\right\rrbracket = \left\llbracket{\ctx(\Gamma')}\right\rrbracket$.
\item $\c \models {\Gamma \vdash  C \equiv C' \type}$ if and only if $\left\llbracket{\Gamma \vdash C \type}\right\rrbracket = \left\llbracket{\Gamma \vdash C' \type}\right\rrbracket$.
\item $\c \models { \Gamma \vdash t \equiv t' : C}$ if and only if $\left\llbracket{\Gamma \vdash t : C}\right\rrbracket = \left\llbracket{\Gamma \vdash t' : C}\right\rrbracket$
\ei
where $\left\llbracket{\varphi}\right\rrbracket=\left\llbracket{\varphi'}\right\rrbracket $ means that $\left\llbracket{\varphi}\right\rrbracket$ and $\left\llbracket{\varphi'}\right\rrbracket$ are defined and equal.  
We say that a $\T$-structure $\c$ \textit{models} $\T$ if every theorem of $\T$ is satisfied by $\c$.
\end{definition}

\begin{exmp}[Tautological model]\label{gen}
The syntactic category $\c(\T)$ is clearly a $\T$-structure. By its design, for any judgment $\varphi$ in $\T$, $\c(\T)$ satisfies $\varphi$ if and only if $\varphi$ is a theorem of $\T$. 
\end{exmp}


\begin{theorem}[Completeness]\label{comp}
For  any judgment $\varphi$ in $\T$, if $\varphi$ is satisfied by every $\T$-structure, then $\varphi$ is a theorem of $\T$.
\end{theorem}
\begin{proof}
If $\varphi$ is satisfied by every $\T$-structure, then it is satisfied  in particular by $\c(\T)$, which only satisfies theorems of $\T$.
\end{proof}

\begin{definition}\label{cxtfun}
Let $\c$ and $\d$ be contextual categories. A  \textit{contextual functor $F: \c \to \d$} is a functor $\c \to \d$ that preserves the structure of a contextual category on the nose, i.e.,
\be[label=(\roman*)]
\item if $\Gamma$ is the parent of $\Gamma'$, then $F(\Gamma)$ is the parent of $F(\Gamma')$,
\item $F(1_{\c}) = 1_{\d}$,
\item $F(p_{\Gamma}) = p_{F(\Gamma)}$ for any $\Gamma \in \ob_{n+1}{\c}$, 
\item $F(f^{\ast}{\Gamma}) = F(f)^{\ast}(F(\Gamma))$ for any $f: \Gamma' \to \Gamma$, and 
\item $F(q(f, \Gamma))= q(F(f), F(\Gamma))$.
\ee
Likewise, if both $\c$ and $\d$ are $\T$-structures, then a contextual functor $F: \c \to \d$ is a \textit{logical contextual functor} if it preserves all of the logical structures on the nose. 
\end{definition}

\begin{comment}
\begin{remark}
\Cref{cxtfun} is stronger than the notion of a homomorphism in classical model theory, which merely must preserve the relations and functions of a structure
\end{remark}
\end{comment}

\medskip

For each $\T$-structure $\c$, the function $\left\llbracket{-}\right\rrbracket$ induces a functor 
\begin{align*}
\left\llbracket{-}\right\rrbracket^{\c} & : \c(\T) \to \c
\\ x_1 : C_1, \ldots, x_n : C_n  & \ \mapsto \ \left(1, X_1, \ldots, X_n\right), \ \left\llbracket{x_1 : C_1, \ldots x_{i-1} : C_{i-1} \vdash C_i}\right\rrbracket = p_{X_i} 
\\ \Gamma \  \xrightarrow{\left(t_1, \ldots, t_n\right)} \  x_1 : C_1, \ldots, x_n : C_n  & \ \mapsto \  \left\llbracket{\Gamma}\right\rrbracket^{\c} \xrightarrow{\tau}  \left(1, X_1, \ldots, X_n\right),
\end{align*} where the morphism $\tau$ is defined as follows.
\[
\begin{tikzcd}[column sep = huge]
	{\left(\Gamma, X_1\right)} & {\left(\Gamma, X_1\right)} && {\left(\Gamma, X_i\right)} & {\left(\Gamma, X_1, \ldots, X_i\right)} \\
	\Gamma & \Gamma && \Gamma & {\left(\Gamma, X_1, \ldots, X_{i-1}\right)} \\
	\\
	\Gamma & {\left(\Gamma, X_n\right)} && {\left(\Gamma, X_1, \ldots, X_n\right)} & {\left(X_1, \ldots, X_n\right)} \\
	&&& \Gamma & 1
	\arrow["{p_{X_1}}", from=1-2, to=2-2]
	\arrow["{\idd_{\Gamma}}"', from=2-1, to=2-2]
	\arrow[from=1-1, to=2-1]
	\arrow["{q(\idd_{\Gamma},X_1)}", from=1-1, to=1-2]
	\arrow["\scalebox{1.5}{\color{black}$\lrcorner$}"{anchor=center, pos=0.125}, draw=none, from=1-1, to=2-2]
	\arrow["{t_1}", bend left, from=2-1, to=1-1]
	\arrow[from=1-4, to=2-4]
	\arrow["{p_{X_i}}", from=1-5, to=2-5]
	\arrow["{q(t_{i-2}, X_{i-1}) \circ t_{i-1}}"', from=2-4, to=2-5]
	\arrow["{q( q(t_{i-2}, X_{i-1}) \circ t_{i-1} , X_i)}", from=1-4, to=1-5]
	\arrow["{t_i}", bend left, from=2-4, to=1-4]
	\arrow["\scalebox{1.5}{\color{black}$\lrcorner$}"{anchor=center, pos=0.125}, draw=none, from=1-4, to=2-5]
	\arrow[dashed, from=4-4, to=4-5]
	\arrow[from=4-5, to=5-5]
	\arrow[from=5-4, to=5-5]
	\arrow[dashed, from=4-4, to=5-4]
	\arrow["\scalebox{1.5}{\color{black}$\lrcorner$}"{anchor=center, pos=0.125}, draw=none, from=4-4, to=5-5]
	\arrow["{t_n}"', from=4-1, to=4-2]
	\arrow["{q( q(t_{n-2}, X_{n-1}) \circ t_{n-1} , X_n)}"', from=4-2, to=4-4]
	\arrow["\tau", bend left=11, from=4-1, to=4-5]
\end{tikzcd}
\]

\begin{conj}[Initiality of syntax]\label{initial}
If $\c$ is a $\T$-structure, then $\left\llbracket{-}\right\rrbracket^{\c}$ is a logical contextual functor $\c(\T) \to \c$. Moreover, it is unique.
\end{conj}

Informally, this means that every $\T$-structure correctly interprets the syntax of $\T$.

\medskip

We shall assume that \cref{initial} is true. Most, but not all, type theorists accept that it is a straightforward adaptation of the Correctness Theorem in \cite{Str}, which proves the conjecture for a type theory smaller than ours (called the Calculus of Constructions).\footnote{ Currently, there is  a communal project called the \textit{Initiality Project}, hosted on nLab, that aims to rigorously establish the initiality of $\c(\T)$.    }  
 If we do not assume \cref{initial}, then the soundness of the semantics of our MLDTT is left unverified.\footnote{Note, however, that \cite[Section 3.5]{Hof} briefly sketches a direct proof of soundness.} In this case, proving that a contextual category (such as $\mathbf{sSet}$) has suitable logical structure is insufficient to prove that it correctly interprets the syntax.  

\medskip

We are now in position to state a crucial consequence of \cref{initial} (whose counterpart  in classical FOL, by contrast, holds easily).

\begin{theorem}[Soundness]\label{sound}
For  any judgment $\varphi$ of $\T$, if $\varphi$ is a theorem of $\T$, then $\varphi$ is satisfied by every $\T$-structure. Hence every $\T$-structure is a model for $\T$.
\end{theorem} 


We say that $\T$ is \textit{consistent} if no judgment of the form $\vdash a:\0$ is derivable in $\T$. Otherwise, we say that $\T$ is \textit{inconsistent}.

\begin{corollary}[Consistency]
$\T$ is consistent if and only if it has a model in which the display map $p_{\0_1}$ has no section.
\end{corollary}
\begin{proof}
For the forward direction, simply observe that if $\T$ is consistent, then $\c(\T)$ is a model of it and contains no section of $p_{\0_1}$. The backward direction follows easily from \cref{sound} (just as it does for classical FOL).
\end{proof}

\bigskip

Before moving on, let us formulate the univalence axiom in any sufficiently structured contextual category.

\medskip

Let $\c$ be a $\T$-structure and let $\Gamma$ be an object in $\c$. Since $$ \prod_{x,y:\U} \isequiv\left(\equiveq_{{x:\U}; \el(x)}(x,y)\right)$$ is a closed type in $\T$, we can find, by virtue of our existing semantics, a certain object $\left(1, \Pi(A_u, B_u)\right) \in \ob_{1}{\c}$ such that $$\left\llbracket{{ }\vdash \prod_{x,y:\U} \isequiv\left(\equiveq_{{x:\U}; \el(x)}(x,y)\right) \type}\right\rrbracket = p_{ \Pi(A_u, B_u)} .$$ Note that $\c$ satisfies a typing declaration of the form $$ \vdash \tau :  \prod_{x,y:\U} \isequiv\left(\equiveq_{{x:\U}; \el(x)}(x,y)\right)$$ if and only if $\left\llbracket{-}\right\rrbracket$ maps this judgment to a section of $p_{ \Pi(A_u, B_u)}$. Therefore, we make the following definition.

\begin{definition}
We say that $\c$ \textit{satisfies the univalence axiom} if it comes equipped with a section $\univv$ of $p_{ \Pi(A_u, B_u)}$.
\end{definition}

\subsection{Universe categories}


\begin{definition}\label{univ1}
Let $\c$ be any category. A \textit{universe in $\c$} is an object $U$ in $\c$ equipped with a morphism $p: \tilde{U} \to U$ and, for each map $f: X \to U$, a distinguished pullback square
\[
\label{eqn:uobj} \begin{tikzcd}
\left(X; f\right) \arrow[d, "{P_{X,f}}"'] \arrow[r, "Q(f)"] \arrow[dr, phantom, "\scalebox{1.5}{\color{black}$\lrcorner$}", very near start] & \tilde{U} \arrow[d, "p"] \\
X \arrow[r, "f"']                                           & U                       
\end{tikzcd}. \tag{$\ast$}
\]
\end{definition} 

Intuitively, $U$ corresponds to the universe type $\U$, and the morphism $p$ corresponds to the type family $\Gamma, x: \U \vdash \el(x) \type$ over $\U$. Further, any map $\alpha : Y \to X$ isomorphic to $P_{X,f}$ in the over category $\c/X$ corresponds to a judgement of the form $\Gamma \vdash \el(a) \type$, i.e., a well-formed type.  

\begin{notation} $ $
\be
\item For any maps $f_1 : X \to U$ and $f_2 : \left(X; f_1\right) \to U$, write $\left(X; f_1, f_2\right)$ for the object $\left(\left(X; f_1\right); f_2\right)$ and $P_{X, f_1, f_2}$ for the map $P_{\left(X; f_1\right), f_2}$.
\item $\left(X; \ \right) \coloneqq X$.
\ee
\end{notation}

\begin{definition}\label{ucat} $  $
\be
\item A \textit{universe category} is a triple $\left(\c, U, 1\right)$ where $\c$ is a category, $U$ is a universe in $\c$, and $1$ is a terminal object in $\c$. 
\item A \textit{functor of universe categories} from $\left(\c, U, 1\right)$ to $\left(\c', U', 1'\right)$ is a triple $\left(\Phi, \varphi, \tilde{\varphi}\right)$ where $\Phi : \c \to \c'$ is a functor and $\varphi: \Phi(U) \to U'$ and $ \tilde{\varphi}: \Phi(\tilde{U}) \to \tilde{U}'$ are morphisms such that
\be
\item $\Phi$ maps distinguished pullback squares in $\c$ to pullbacks squares in $\c'$,
\item $\Phi$ maps $1$ to a terminal object in $\c$, and
\item the diagram
\[
\begin{tikzcd}
\Phi(\tilde{U}) \arrow[d, "\Phi(p)"'] \arrow[r, "\tilde{\varphi}"] & \tilde{U}' \arrow[d, "p'"] \\
\Phi(U) \arrow[r, "\varphi"']                                      & U'                        
\end{tikzcd}
\]
is a pullback square in $\c'$.
\ee
\ee
\end{definition}

The notions of a universe category and a functor of universe categories are due to Voevodsky and can be found in \cite{voev}.

\begin{remark}
\Cref{ucat} makes the class of all universe categories into a \textit{precategory}, i.e., a category except that each composition operation is partial rather than total. For simplicity, we shall refer to this precategory as a category.
\end{remark}

\pagebreak

Now, consider any universe category $\left(\c, U, 1\right)$. Define the contextual category $\c_U$ as follows.\label{indcxtcat}\footnote{\cite[Definition 1.3.2]{KL}.}
\bi
\item $\ob_n{\c_U} \equiv \left\{\left(f_1, \ldots, f_n\right) \in \left(\mor(\c)\right)^n \mid f_i : \left(1; f_1, \ldots, f_{i-1}\right) \to U, \ 1 \leq i \leq n \right\}$.
\item $\Hom_{\c_{U}}\big(\left(f_1, \ldots, f_n\right), \left(g_1, \ldots, g_m\right)\big) \equiv \Hom_{\c}\big(\left(1; f_1, \ldots, f_n\right), \left(1; g_1, \ldots, g_m\right)    \big)$.
\item $1_{\c_U}\equiv \left({}\right)$, the empty sequence.
\item $\ft_n(f_1, \ldots, f_{n+1}) \equiv \left(f_1, \ldots, f_n\right)$.
\item Take $P_{\left(1; f_1, \ldots, f_n\right), f_{n+1}}$ (as in \eqref{eqn:uobj}) to be the display map $p_{\left(f_1, \ldots, f_{n+1}\right)}$.
\item For any object $\left(f_1, \ldots, f_{n+1}\right)$ and map $\alpha : \left(g_1, \ldots, g_m\right) \to \left(f_1, \ldots, f_n\right)$ in $\c_{U}$, the canonical pullback of $\left(f_1, \ldots, f_{n+1}\right)$ along $\alpha$ is given by $\left(g_1, \ldots, g_m, f_{n_1}\circ \alpha \right)$, i.e., the canonical pullback square is precisely the lefthand square in
\[
\label{eqn:CU} \begin{tikzcd}[row sep=large]
{\left(1; g_1, \ldots, g_m, f_{n+1} \circ \alpha\right)} \arrow[rr, "Q(f_{n+1}\circ \alpha)", bend left] \arrow[r, dashed] \arrow[d, "{p_{\left(1; g_1, \ldots, g_m, f_{n+1} \circ \alpha\right)}}"']  \arrow[dr, phantom, "\scalebox{1.5}{\color{black}$\lrcorner$}", very near start] & {\left(1; f_1, \ldots, f_{n+1}\right)} \arrow[r, "Q(f_{n+1})"'] \arrow[d, "{p_{\left(1; f_1, \ldots, f_{n+1}\right)}}"']  \arrow[dr, phantom, "\scalebox{1.5}{\color{black}$\lrcorner$}", very near start] & \tilde{U} \arrow[d, "p"] \\
{\left(1; g_1, \ldots, g_m\right)} \arrow[r, "\alpha"']                                                                                                                                               & {\left(1; f_1, \ldots, f_{n}\right)} \arrow[r, "f_{n+1}"']                                                              & U                       
\end{tikzcd}. \tag{$\star$}
\]
\ei



\begin{theorem}\label{C-U}
There exists a certain functor $\mathcal{CC}$ from the category of universe categories to the category of contextual categories such that $\mathcal{CC}(\c, U, 1)=\c_U$.\footnote{\autocite[Construction 4.7]{voev}.  }
\end{theorem}

Suppose that $U$ is a universe in $\c$ when equipped with either of two choices $C$ and $C'$ of distinguished pullback squares. Further, suppose that both $1$ and $1'$ are terminal objects in $\c$. These conditions yield two universe categories $\mathit{UC}$ and $\mathit{UC}'$.

\begin{corollary}
 $\mathcal{CC}(\mathit{UC})$ and  $\mathcal{CC}(\mathit{UC}')$ are isomorphic as contextual categories.
\end{corollary}
\begin{proof}
The triple $\left(\idd_{\c}, \idd_U, \idd_{\tilde{U}}\right)$ is a functor of universe categories $\mathit{UC} \to \mathit{UC}'$. Clearly, it is its own inverse and thus is an isomorphism. This implies that $\mathcal{CC}\big(\idd_{\c}, \idd_U, \idd_{\tilde{U}}\big)$ is an isomorphism, thereby completing our proof.
\end{proof}

This means that, up to canonical isomorphism, $\c_{U}$ is independent of our choice of distinguished pullback squares and terminal  object in $\c$.

\subsection{Logical structure on a universe category}\label{log}

Let $\left(\c, U, 1\right)$ be a universe category with morphism $p: \tilde{U} \to U$. As we did on contextual categories, we would like to endow $U$ with logical structures  based on the logical rules in $\T$. Moreover, we would like these structures to induce corresponding logical structures on the contextual category $\c_U$, thereby making $\c_U$ into a $\T$-structure. (We can think of the logical structure on $\c_U$ as internal to that on $\c$.)  This section is devoted to describing certain structures on $U$ that achieve these goals.

\smallskip

\begin{remark}
Throughout this section, we shall assume that $\c$ is a locally cartesian closed category (LCCC). See \cref{LCC} for a review of this kind of category. Additionally, we shall assume that $\c$ has both an initial object $0$ and the coproduct $1\coprod 1$.
\end{remark}

\smallskip

For convenience, Table~\ref{tab:ustr} summarizes our results of this section. 

\begin{table}[h!]
\centering
\caption{Logical structure on $U$}
\label{table:3}
\begin{tabular}{||c c||} 
 \hline
 $\cdtt$ & LCCC \\ [0.5ex] 
 \hline\hline
 Dependent product &  Dependent product \\ 
 Dependent sum &  Dependent sum  \\
  Empty type & Initial object  \\
   Unit type & Terminal object  \\
    Boolean type & $1 \coprod 1$  \\
 Identity type & Path space object  \\
 Universe type & Internal universe  \\ [1ex] 
 \hline
\end{tabular}
\label{tab:ustr}
\end{table}



\subsection*{Dependent products}
 Consider a distinguished pullback $P_{X,A}$ in $\c$, i.e., a  well-formed type $A$. Consider also a distinguished pullback $P_{X, A, B}$, i.e., a type family $B$ over $A$. Intuitively, to form the dependent product $\prod_{x:A}B(x)$ in $\c$, we need a map $\hat{\Pi}(A,B) : X\to U$ along with an isomorphism $P_{X, \hat{\Pi}(A,B)} \cong  \Pi_{P_{X,A}}{P_{X, A, B}}$ such that $\left(X; \hat{\Pi}(A,B)\right) $ is stable under pullback.  
 
 \smallskip
 
 To this end, we endow $\c$ with such a map  $ \hat{\Pi}(A,B)$ under the assumption that $X$ equals
 $$ \Pi(U) \coloneqq  \Sigma_{!_U}\Pi_{p}\big(\pi_2^U : U \times \tilde{U}\to \tilde{U}\big)$$ and 
$\left(A, B\right)$ equals a special pair $\left(A_g, B_g\right)$ of maps determined by $\Pi(U)$, called the \textit{generic} pair.
As it turns out, this is enough to produce in $\c_U$ \emph{any} dependent product $\prod_{x:A}B(x)$ as well as ensure stability. 

\begin{note}[Functoriality of $\Pi({-})$]\label{Ufunct}
Let $\pb(\c)$ denote the category with morphisms in $\c$ as objects and squares of the form 
\[
\begin{tikzcd}
X_1 \arrow[d, "x"'] \arrow[r, dashed]  \arrow[dr, phantom, "\scalebox{1.5}{\color{black}$\lrcorner$}", very near start] & Y_1 \arrow[d, "y"]  \\
X_2 \arrow[r, dashed]                & Y_2               
\end{tikzcd}
\]
 as morphisms $x\to y$.  (This is a wide subcategory of the arrow category $\Ar(\c)$ of $\c$.) Now, let $U'$ be another universe in $\c$ and let $\left(f, g\right)$ be a map $p' \to p$ in $\pb(\c)$:
\[
\label{eqn:pbk} \begin{tikzcd}
\tilde{U}' \arrow[d, "p'"'] \arrow[r, "f"]  \arrow[dr, phantom, "\scalebox{1.5}{\color{black}$\lrcorner$}", very near start] & \tilde{U} \arrow[d, "p"] \\
U' \arrow[r, "g"']                         & U                       
\end{tikzcd}. \tag{$\star$}
\]
We can define another map $\pi_2^{U'} \to \pi_2^U$ in $\Ar(\c)$ by the commutative square  
\[
\begin{tikzcd}[column sep=huge]
U'\times \tilde{U}' \arrow[d, "\pi_2"'] \arrow[r, "{\left(g\circ \pi_1, f \circ \pi_2\right)}"] & U\times \tilde{U} \arrow[d, "\pi_2"] \\
\tilde{U}' \arrow[r, "f"']                                                                      & \tilde{U}                           
\end{tikzcd}
.\] This induces a unique map $H(f,g): \pi_2^{U'} \to f^{\ast}{\pi_2^U}$ in the over category $\c/\tilde{U}'$ thanks to the universal property of pullback squares. Applying the functor $\Pi_{p'}$ to $H$ yields a new map 
$$ \Pi_{p'}\left(\pi_2^{U'}\right) \to \Pi_{p'}\left(f^{\ast}{\pi_2^U}\right) $$ 
in $\c/U'$. In light of \cref{C-B}, there exists an isomorphism  $$\Pi_{p'}\left(f^{\ast}{\pi_2^U}\right) \cong g^{\ast}\Pi_p\left( \pi_2^U\right)$$ natural in $\pi_2^U$. From this, we get yet another map $ \Pi_{p'}\left(\pi_2^{U'}\right) \to  g^{\ast}\Pi_p\left(\pi_2^U\right)$ in $\c/U'$ and thus a map $$\Pi_{p'}\left(\pi_2^{U'}\right) \to \Pi_p\left(\pi_2^U\right)$$ in $\Ar(\c)$. This is our desired map $\Pi(g): \Pi(U') \to \Pi(U) $ under the identification $\c/1 \cong \c$. 
\end{note}

\medskip

Now, let us return to describing the map $\hat{\Pi}(A_g, B_g) : \Pi(U) \to U$. The map $\Pi_p{\pi_2}$ can be viewed as a projection map $\pr: \Pi(U) \to U$ in $\c$. Take the pullback $\Pi(U) \times_U \tilde{U}$ to be $A_g$ and let $\alpha_g$ denote the corresponding projection map $A_g \to \Pi(U)$. Note that $\alpha_g = p^{\ast}\left(\Pi_p{\pi_2}\right)$. Consider now the counit $\epsilon$ of the adjunction $\Pi_p \dashv p^{\ast}$ and define $B_g$ and $\beta_g$ so that
\[
\begin{tikzcd}[column sep=large]
B_g \arrow[r, dashed] \arrow[d, "\beta_g"', dashed]  \arrow[dr, phantom, "\scalebox{1.5}{\color{black}$\lrcorner$}", very near start] & \tilde{U} \arrow[d, "p"] \\
A_g \arrow[r, "\pi_1 \circ \epsilon_{\pi_2}"']          & U                       
\end{tikzcd}
.\]


\smallskip

At last, we define a \textit{$\Pi$-structure on $U$} as a map $\bar{\Pi} : \Pi(U) \to U $ together with an isomorphism $\bar{\Pi}^{\ast}{p}\cong \Pi_{\alpha_g}\beta_g$.\label{depprods}

\subsection*{Empty type}

Recall that we can think of the empty type as the empty set, i.e., the initial object in $\set$. This leads us to define a \textit{$\0$-structure on $U$} as a map $\bar{\0}: 1 \to U$ together with an isomorphism $\bar{\0}^{\ast}{\tilde{U}} \cong 0$.

\subsection*{Identity types}

Let $\d$ be a contextual category. For simplicity, we shall refer to an object $\left(\Gamma, A\right)$ in $\d$ by simply $A$ and to the product $\left(\Gamma, A\right) \times \left(\Gamma, A\right)$ in $\d$ by $A \times A$. Recall the notion of an $\id$-type structure on $\d$ (p.~\pageref{idtype}).  Note that condition (ii) exhibits a section
\[
\begin{tikzcd}
\Delta_A^{\ast}{\id_A} \arrow[d] \arrow[r]                                        & \id_A \arrow[d, "p_{\id_A}"] \\
A \arrow[r, "\Delta_A"'] \arrow[u, "s_{\Delta_A^{\ast}\id_A}", dashed, bend left] & A \times A                  
\end{tikzcd}
\] by way of $\refl_A$. This is equivalent to exhibiting a lift
\[
\label{eqn:factr}\begin{tikzcd}
                                                       & \id_A \arrow[d, "p_{\id_A}"] \\
A \arrow[r, "\Delta_A"'] \arrow[ru, "\refl_A", dashed] & A \times A                  
\end{tikzcd} \tag{$\dagger$}
.\] Further, condition (iii) exhibits a diagonal fill-in 
\[
\begin{tikzcd}[row sep=large]
A \arrow[r, "d"] \arrow[d, "\refl_A"']           & B \arrow[d, "p_B"] \\
\id_A \arrow[r, equal] \arrow[ru, "{\J_{B,d}}", dashed] & \id_A             
\end{tikzcd}.
\] This implies that $\refl_A$ has the left lifting property against \emph{any} display map $p_C$ in $\d$. Indeed, for any map $f: \id_A \to \ft(C)$ and commutative square
\[
\begin{tikzcd}
A \arrow[d, "\refl_A"'] \arrow[r, "g"] & C \arrow[d, "p_C"] \\
\id_A \arrow[r, "f"']             & \ft(C)            
\end{tikzcd}
,\] we have a lift 
\[
\begin{tikzcd}
A \arrow[r, dotted] \arrow[d, "\refl_A"'] \arrow[rr, "g", bend left] & f^{\ast}{C} \arrow[r] \arrow[d, "p_{f^{\ast}{C}}"] & C \arrow[d, "p_C"] \\
\id_A \arrow[r, equal] \arrow[ru, dashed]                                   & \id_A \arrow[r, "f"']                              & \ft(C)            
\end{tikzcd}.
\] In light of \eqref{eqn:factr},  we see that $\Delta_A$ factors as a map $A \to \id_A$ having the left lifting property against all display maps followed by a display map $ \id_A \to A\times A$. 

\begin{remark}\label{fibfact}
In the language of model categories (\cref{modcat}), if we view a  display map  as a fibration and a map having the left lifting property against all fibrations as a trivial cofibration, then $\id_A$ is precisely a \textit{path space object} of $A$.
\end{remark}

\medskip

With this in mind, an \textit{$\id$-structure on $U$} consists of maps  $\overline{\id}: \tilde{U} \times_U \tilde{U} \to U$ and $\zeta : \tilde{U} \to \overline{\id}^{\ast}{\tilde{U}}$ such that 
\[
\begin{tikzcd}
\tilde{U} \arrow[rd, "\Delta_p"'] \arrow[r, "\zeta"] & \overline{\id}^{\ast}{\tilde{U}} \arrow[d, "\overline{\id}^{\ast}{p}"] \\
                                                              & \tilde{U}\times_U\tilde{U}                                            
\end{tikzcd}
\] commutes and $\zeta$ is \textit{stably orthogonal} to $p \times U$ over $U$, i.e., for any map $X\to U$, every lifting problem
\[
\begin{tikzcd}
X\times_U \tilde{U} \arrow[d, "X\times \zeta"'] \arrow[r, "a"] & \tilde{U}\times_U U \arrow[d, "p \times U"] \\
X\times_U \bar{\id}^{\ast}{\tilde{U}} \arrow[r, "b"']         & \underbrace{U \times_U U}_{U}                               
\end{tikzcd}
\] between $X \times \zeta$ and $p \times U$ comes equipped with a solution $D(a,b)$ 
such that for any map $f: Y \to X$ over $U$,
\[
D\left(a,b\right) \circ \left(f \times \bar{\id}^{\ast}{U} \right) = D\left(a \circ \left(f \times \tilde{U}\right), b\circ \left(f \times \bar{\id}^{\ast}{U}\right)\right)
.\]


\medskip

This structure can be thought of as the ``generic" identity type just as a $\Pi$-structure is seen as the generic dependent product. This means that it produces  in $\c_U$ \emph{every} identity type thanks to the universal properties of a LCCC. 

\subsection*{Universe type}\label{internaluniv}

A $\U$-structure on $U$ roughly amounts to a universe $U_0$ in $\c$ that is ``nested" in $U$. To be precise, a \textit{nested universe in $U$} is a pair $\left(u_0, \iota\right)$ where $u_0$ is a map $1 \to U$ and $\iota$ is a map $U_0 \coloneqq u_0^{\ast}{\tilde{U}} \to U$. 

\begin{term}
$U_0$ is called an \textit{internal universe}, and $U$ a \textit{meta-universe}.
\end{term}

\medskip

Note that $U_0$ is, indeed, a universe in $\c$ when equipped with the morphism $$\underbrace{\iota^{\ast}{p}}_{p_0} :  \underbrace{\iota^{\ast}{\tilde{U}}}_{\tilde{U}_0} \to U_0 $$ and, for each map $f: X \to U_0$, the lefthand square in the commutative diagram
\[
\begin{tikzcd}
\left(X; \iota \circ f\right) \arrow[d, "{p_{X, \iota \circ f}}"'] \arrow[r, "Q(f)"', dashed] \arrow[rr, "Q(\iota \circ f)", bend left] & \tilde{U}_0 \arrow[d, "p_0"'] \arrow[r]  \arrow[dr, phantom, "\scalebox{1.5}{\color{black}$\lrcorner$}", very near start] & \tilde{U} \arrow[d, "p"] \\
X \arrow[r, "f"']                                                                                                                       & U_0 \arrow[r, "\iota"']                & U                       
\end{tikzcd}
\] as the distinguished pullback square. We say that $U_0$ is \textit{closed under $\Pi$-types in $U$} if it has a $\Pi$-structure $\bar{\Pi}_0 : \Pi(U_0) \to U_0$ such that 
\[
\begin{tikzcd}
\Pi(U_0) \arrow[d, "\bar{\Pi}_0"'] \arrow[r, "\Pi(\iota)"] & \Pi(U) \arrow[d, "\bar{\Pi}"] \\
U_0 \arrow[r, "\iota"']                                 & U                         
\end{tikzcd}
\]
 commutes (with $\Pi(i)$ as in \cref{Ufunct}). We say that $U_0$ is \textit{closed under $\Sigma$-types, etc.} under similar circumstances.

\bigskip

Again, in the interest of space, we have omitted the definitions of \textit{$\Sigma$-structure}, \textit{$\1$-structure}, and \textit{$\2$-structure}.

\subsection*{Induced logical structure on $\c_U$}

The following result will ensure that the induced $\Pi$- and $\Sigma$-type structures on $\c_U$ are stable under substitution.

\begin{lemma}\label{generic}
Let $B \overset{f_2}{\longrightarrow} A \overset{f_1}{\longrightarrow} \Gamma$ be a composite of morphisms in $\c$. Let $g_1: \Gamma \to U$, $g_2 : A\to U$, $h_1: A\to \tilde{U}$, and $h_2 : B \to \tilde{U}$ be morphisms in $\c$ such that 
\[
\begin{tikzcd}
A \arrow[r, "h_1"] \arrow[d, "f_1"'] & \tilde{U} \arrow[d, "p"] &  & B \arrow[r, "h_2"] \arrow[d, "f_2"'] & \tilde{U} \arrow[d, "p"] \\
\Gamma \arrow[r, "g_1"']             & U                        &  & A \arrow[r, "g_2"']                  & U                       
\end{tikzcd}
\] are pullback squares. Then there exists a unique map $\left(A, B\right)$ such that 
\[
\label{eqn:doubpb} \begin{tikzcd}
B \arrow[d, "f_2"'] \arrow[r]   \arrow[dr, phantom, "\scalebox{1.5}{\color{black}$\lrcorner$}", very near start] & B_g \arrow[d, "\beta_g"]  \\
A \arrow[d, "f_1"'] \arrow[r]   \arrow[dr, phantom, "\scalebox{1.5}{\color{black}$\lrcorner$}", very near start] & A_g \arrow[d, "\alpha_g"] \\
\Gamma \arrow[r, "{\left(A, B\right)}"']              & \Pi(U)                   
\end{tikzcd} \tag{$\ast$}
\] in $\c$.
\end{lemma}
\begin{proof}
We have a unique mediating map
\[
\begin{tikzcd}
A \arrow[rrd, "g_2", bend left] \arrow[rdd, "g_1\times p"', bend right] \arrow[rd, "k", dashed] &                                                  &             \\
                                                                                                & U\times U \arrow[r, "\pi_2"] \arrow[d, "\pi_1"'] & U \arrow[d] \\
                                                                                                & U \arrow[r]                                      & 1          
\end{tikzcd}
\] by the universal property of pullback squares. Note that $k$ is a map $g_1 \times p \to \pi_1$ in $\c/U$. Thus, we can take its exponential transpose $\tilde{k} : g_1 \to \left(\pi_1\right)^p$. Since $$\left(\pi_1\right)^p \simeq \Pi_pp^{\ast}{\pi_1}= \Pi_p{\pi_2^U}$$ by \cref{inthom}, we see that  $\tilde{k}$ is a map $g_1 \to  \Pi_p{\pi_2^U}$, i.e., a map $\left(A, B\right) : \Gamma \to \Pi(U)$ over $U$. It is easy to check that this map satisfies \eqref{eqn:doubpb}. The fact that it is unique is clear from the way in which  it is constructed.
\end{proof}

Of course, this universal property fails to imply that the induced $\id$-type structure  is stable. For this, we use instead the requirement that $\zeta$ be stably orthogonal to $p \times U$.

\begin{theorem}\label{univlog}
Any given logical structure on $U$ induces a corresponding logical structure on $\c_U$.\footnote{\cite[Theorem 1.4.15]{KL}.}
\end{theorem}
\begin{proof}
Other than the data for $\id$-\textsc{elim} (i.e., condition (iii) and part of condition (iv) for a $\id$-type structure (p.~\pageref{idtype})), each corresponding structure flows from the ``generic" structure on $U$ in a predictable way. To understand our method for verifying this, it is enough to consider just the data for $\Pi$-\textsc{form}, i.e., condition (i) and part of condition (v) for a $\Pi$-type structure (p.~\pageref{pitype}).

\smallskip

Suppose that $U$ has a $\Pi$-structure $\bar{\Pi}: \Pi(U)\to U$. Let $\left(\Gamma, A, B\right)\in \ob_{n+2}{\c_U}$. This is precisely a pair of pullback squares
\[
\begin{tikzcd}
{\left(1; \Gamma, A\right)} \arrow[d, "{P_{\Gamma, A}}"'] \arrow[r, "Q(A)"] & \tilde{U} \arrow[d, "p"] &  & {\left(1; \Gamma, A, B\right)} \arrow[d, "{P_{\Gamma, A, B}}"'] \arrow[r, "Q(B)"] & \tilde{U} \arrow[d, "p"] \\
\left(1;\Gamma\right) \arrow[r, "A"']                                       & U                        &  & {\left(1; \Gamma, A\right)} \arrow[r, "B"']                                       & U                       
\end{tikzcd}
\] in $\c$, which yields another diagram
\[
\begin{tikzcd}
{\left(1; \Gamma, A, B\right)} \arrow[d, "{P_{\Gamma, A, B}}"'] \arrow[r] & B_g \arrow[d, "\beta_g"]  \\
{\left(1; \Gamma, A\right)} \arrow[r] \arrow[d, "{P_{\Gamma, A}}"']       & A_g \arrow[d, "\alpha_g"] \\
\left(1; \Gamma\right) \arrow[r, "{\left(A, B\right)}"']       & \Pi(U)                   
\end{tikzcd}
\]  consisting of pullback squares via \cref{generic}. Take the composite $\bar{\Pi} \circ \left(A, B\right)$ to be $\Pi(A,B)$ in $\c_U$. 

\smallskip

It remains to check that $\Pi(A,B)$  is stable under substitution in $\c_U$. Let $f: \left(1; \Gamma'\right) \to \left(1; \Gamma\right)$ be a map in $\c$. On the one hand, we have that $$f^{\ast}(\Gamma, \Pi(A, B)) = \left(\Gamma', \bar{\Pi} \circ  \left(A, B\right) \circ f\right).$$ On the other hand, 
\begin{align*}
\left(\Gamma', \Pi(f^{\ast}{A}, f^{\ast}{B})\right) & = \left(\Gamma', \bar{\Pi}\circ \left(f^{\ast}{A},f^{\ast}{B}\right) \right)
\\ \left(f^{\ast}{A},f^{\ast}{B}\right)& =  \left(A, B\right) \circ f 
\end{align*}
by the uniqueness of $\left(f^{\ast}{A},f^{\ast}{B}\right)$. Hence $\Pi(A,B)$ is stable.
\end{proof}

\begin{remark}
It is easy to show that, under this logical structure, $\c_U$ also satisfies $\Pi$-$\eta$.
\end{remark}

\medskip

In conclusion, we have a method for building a model of $\T$ in a given LCCC: finding a ``logically structured" universe in it. This is precisely the method we employ in \cref{models}.


\subsection{Presheaf universes}\label{PrUn}

At this time, it is worth outlining a modest extension of the map $U \mapsto \c_{U}$ due to \cite{Nat}. This requires a certain concept from algebraic geometry. Suppose that $\c$ is any locally small category with a terminal object $1$. At this point, we may pass to a larger Grothendieck universe than our current one so that $\c$ is small (see p.~\pageref{ssetsmall}).

\begin{notation}
Let $\mathcal{Y}: \c \to \widehat{\c}$ denote the Yoneda embedding, as in \cref{Yoneda}.
\end{notation}

\begin{definition}[Grothendieck]\label{univ2}
 Let $\widetilde{\mathcal{U}}$ and $\mathcal{U}$  be objects in $\widehat{\c}$. A natural transformation $$\rho: \widetilde{\mathcal{U}} \to \mathcal{U}$$ is \textit{representable} if for any object $C$ in $\c$ and any map $T : \mathcal{Y}_C \to \mathcal{U}$, $\rho$ comes equipped with a distinguished pullback square
\[
\begin{tikzcd}[column sep=large]
\mathcal{Y}_{C\ldotp{T}} \arrow[d, "{\mathcal{Y}(p_T)}"']  \arrow[r, "q_T"] \arrow[dr, phantom, "\scalebox{1.5}{\color{black}$\lrcorner$}" , very near start, color=black] & \widetilde{\mathcal{U}} \arrow[d, "\rho"] \\
\mathcal{Y}_C \arrow[r, "T"']                                      & \mathcal{U}                              
\end{tikzcd}
\] in $\widehat{\c}$, which means that each fiber of $\rho$ is chosen to be a representable object. In this case, we say that $\mathcal{U}$ is a \textit{universe in $\widehat{\c}$}.
\end{definition}

Since the functor $\Hom_{\c}(C, {-}): \c \to \set$ is limit preserving for each $C\in \ob{\c}$, so is the Yoneda embedding. Therefore, we can apply $\mathcal{Y}$ to \cref{univ1} to get a special case of \cref{univ2}. As $h$  is fully faithful, this means that any universe in $\c$ may be viewed as a universe in $\widehat{\c}$.

\medskip

As it turns out, a representable natural transformation $\rho$ over $\c$ makes $\c$ into a \textit{category with families}, equivalently a \textit{category with attributes} \cite[Definition 6.3.3]{Pitts}. This is the same as a contextual category with the $\N$-grading of $\ob{\c}$ replaced by a chosen class $\Ty_{\c}(X)$ of \textit{semantic types} as well as a chosen \textit{total object $X\ltimes A$ of $A$} for each $X\in \ob{\c}$ and $A\in \Ty_{\c}(X)$. 

\begin{remark}
 We assume that \cref{initial} also holds for categories with attributes.
\end{remark}

In particular, a canonical pullback square looks like
\[
\begin{tikzcd}
Y \ltimes f^{\ast}{A} \arrow[d, "p_{f^{\ast}{A}}"'] \arrow[r, "{q(f, A)}"] & X\ltimes A \arrow[d, "p_A"] \\
Y \arrow[r, "f"']                                                          & X                          
\end{tikzcd}.
\]

If we regard $C\ldotp{T} \in \ob{\c}$ as the total object of $T$, then any such pullback square exists in $\c$. Indeed, consider the cospan 
\[
\begin{tikzcd}
                  & C\ldotp{T} \arrow[d, "{p}_T"] \\
B \arrow[r, "f"'] & C                                  
\end{tikzcd}
\] in $\c$. We have two pullbacks of $\rho$ fitting into a commutative diagram
\[
\begin{tikzcd}
\mathcal{Y}_{B\ldotp{T \circ \mathcal{Y}(f)}} \arrow[d, "\mathcal{Y}\left(p_{T\circ \mathcal{Y}(f)}\right)"'] \arrow[r, "g"', dashed] \arrow[rr, "q_{T \circ \mathcal{Y}(f)}", bend left] & \mathcal{Y}_{C\ldotp{T}} \arrow[d, "\mathcal{Y}({p}_T)"] \arrow[r, "q_T"'] & \widetilde{\mathcal{U}} \arrow[d, "\rho"] \\
\mathcal{Y}_B \arrow[r, "\mathcal{Y}(f)"']                                                                                                    & \mathcal{Y}_C \arrow[r, "T"']                                 & \mathcal{U}                              
\end{tikzcd}.
\] As both the total rectangle and the righthand square are pullbacks, so is the lefthand square. Further,  the induced map $g$ has the form $\mathcal{Y}(q)$ for some unique map $q : B\ldotp{T\circ \mathcal{Y}(f)} \to C\ldotp{T}$ in $\c$ since the Yoneda embedding is fully faithful.  It also reflects all limits for the same reason. Therefore, we can take
\[
\begin{tikzcd}
B\ldotp{T\circ \mathcal{Y}(f)} \arrow[d, "p_{T\circ \mathcal{Y}(f)}"'] \arrow[r, "q"] & C\ldotp{T} \arrow[d, "{p}_T"] \\
B \arrow[r, "f"']                                                 & C                            
\end{tikzcd}
\] as a canonical pullback square in $\c$.

\medskip

Moreover, \cite{Nat} defines dependent products, dependent sums, and identity types on $\mathcal{U}$ just as in \cref{log} and proves a corresponding result to \cref{univlog}. It is straightforward to extend that result to the empty type, the unit type, and boolean types on $\mathcal{U}$ again with similar definitions  to those found in \cref{log}. 
In conclusion, if $\mathcal{U}$ is logically structured enough, then $\c$ models our type theory $\T$ excluding the universe type $\U$.

\medskip

Let $\mathbb{D}$ be a subclass of $\mor(\c)$. We want to specify conditions on $\mathbb{D}$ guaranteeing the existence of a universe $\mathcal{U}$ in $\widehat{\c}$ with sufficient logical structure. 

\begin{definition} We say that $\mathbb{D}$ is \textit{stable} if
\be[label=(\alph*)]
\item for any $f\in \mathbb{D}$, the pullback of $f$ along any map in $\c$ exists and 
\item for any pullback square
\[
\begin{tikzcd}
X_4 \arrow[r] \arrow[d, "e"']  \arrow[dr, phantom, "\scalebox{1.5}{\color{black}$\lrcorner$}" , very near start, color=black] & X_1 \arrow[d, "g"] \\
X_3 \arrow[r]                 & X_2               
\end{tikzcd}
\] in $\c$, we have that $g\in \mathbb{D} \implies e\in \mathbb{D}$.
\ee
\end{definition}

Define the presheaves $\mathbb{D}_1$ and $\mathbb{D}_0$ on $\c$ as follows, where $C/\c$ denotes the under category.
\begin{align*}
& \mathbb{D}_1(C)  \equiv \left\{\left(a,d\right) \in C/\c \times \mathbb{D} \mid \cod(a) = \dom(d)\right\}
\\[2pt] &\mathbb{D}_0(C)  \equiv \left\{\left(a,d\right) \in C/\c  \times \mathbb{D} \mid \cod(a) = \cod(d)\right\} 
\\[6pt]  & \mathbb{D}_1(s: D \to C)(a,d)  \equiv \left(a \circ s, d\right) 
\\[2pt]   &\mathbb{D}_0(s: D \to C)(a,d)  \equiv \left(a \circ s, d\right) 
\end{align*}
Further, define the natural transformation $\zeta(\mathbb{D}) :\mathbb{D}_1 \to \mathbb{D}_0$ componentwise by
\[
\zeta(\mathbb{D})_C(a,d) \equiv \left(d \circ a, d\right).
\] 

\begin{lemma}
Suppose that $\mathbb{D}$ is stable. Then the natural transformation $\zeta(\mathbb{D})$ is representable.
\end{lemma}
\begin{proof}[Proof sketch]
Let $C\in \ob{\c}$ and consider any map $T: h_C \to \mathcal{U}$. We must exhibit a pullback square in $\widehat{\c}$ of the form
\[ \label{eqn:romone}
\begin{tikzcd}
\mathcal{Y}_{C\ldotp{T}} \arrow[d, "\mathcal{Y}(p_T)"'] \arrow[r, "q_T"]  \arrow[dr, phantom, "\scalebox{1.5}{\color{black}$\lrcorner$}" , very near start, color=black] & \mathbb{D}_1 \arrow[d, "\zeta(\mathbb{D})"] \\
\mathcal{Y}_C \arrow[r, "T"']                                                & \mathbb{D}_0                               
\end{tikzcd}
. \tag{$\diamond$}
\] Note that $T$ is an element of $\mathbb{D}_0(C)$ via the Yoneda lemma, so that it is a cospan of the form
\[
\begin{tikzcd}
                  & B \arrow[d, "d_T"] \\
C \arrow[r, "a_T"'] & A               
\end{tikzcd}
\] where $d_T\in \mathbb{D}$. Take the pullback square
\[
\begin{tikzcd}
a_T^{\ast}(B) \arrow[d, "{p}_T"'] \arrow[r, "\tilde{q}_T"] & B \arrow[d, "d_T"] \\
C \arrow[r, "a_T"']                                      & A                 
\end{tikzcd}
\]
in $\c$. Let $C\ldotp{T} = a_T^{\ast}(B)$ and let $q_T$ be the map corresponding to the pair $\left(\tilde{q}_T,  d_T\right) \in \mathbb{D}_1(C\ldotp{T})$ under the Yoneda lemma. A straightforward yet tedious argument, omitted here, confirms that \eqref{eqn:romone} is a pullback square.
\end{proof}

\medskip

For any $X\in \c$, let $\mathbb{D}(X)$ denote the full subcategory of $\c/X$ consisting of all maps in $\mathbb{D}$ with codomain $X$.

\begin{definition}\label{closedmaps}
We say that $\mathbb{D}$ is \textit{closed} if 
\be[label=(\alph*)]
\item it is stable,
\item it is closed under composition,
\item every map of the form $C \to 1$ belongs to $\mathbb{D}$,
\item for any map $a: D \to C$ in $\c$, the base change functor $a^{\ast} : \mathbb{D}(C)\to \mathbb{D}(D)$ has a right adjoint, and
\item the inclusion functor $\mathbb{D}(C) \hookrightarrow \c/C$ preserves exponentials.  
\ee
\end{definition}

\begin{definition}
We say that $\mathbb{D}$ is \textit{factorizing} if every map $a: C \to D$ in $\c$ factors as $a =d\circ f$ where $d\in \mathbb{D}$ and $f$ has the left lifting property against all maps in $\mathbb{D}$.
\end{definition}

\begin{theorem}\label{Awthm}
Suppose that $\mathbb{D}$ is both closed and factorizing. Then the universe $\mathbb{D}_0$ in $\widehat{\c}$ carries enough logical structure  that $\c$ (as a category with families) models $\mathbb{T}$ excluding $\U$.\footnote{\cite[Theorem 32]{Nat}.}
\end{theorem}

\section{Homotopy theory}\label{htpy}

This section develops those notions from classical homotopy theory which  \cref{models} will rely on.

\subsection{Simplicial sets}

Here, we gather a number of standard concepts and properties about the category $\mathbf{sSet}$ of simplicial sets, i.e., the functor category $\left[\varDelta^{\op}, \set\right]$ where $\varDelta$ denotes the category of all nonempty finite  ordinals with order-preserving functions as morphisms, known as the \textit{simplex category}.

\bigskip

Recall that any simplicial set $X$ admits a nice combinatorial description. Specifically, for any $n\in \N$ and integer $0 \leq i \leq n+1$, consider the $i$-th coface morphism $\delta_i^n: \left[n\right] \to \left[n+1\right]$ defined by 
\[
\delta_i^n(m) = \begin{cases}
m &  m < i
\\ m+1 & m \geq i
\end{cases}.
\]
Also, for any integer $0 \leq i \leq n$, consider the $i$-th codegeneracy morphism $\sigma_i^n: \left[n+1\right] \to \left[n\right]$ defined by
\[
\sigma_i^n(m) = \begin{cases}
m &  m \leq  i
\\ m-1 & m > i
\end{cases}.
\] 

The following properties of $\mor(\varDelta)$ are easy to verify yet quite useful.

\begin{lemma}\label{mordelt} $ $
\be[label=(\arabic*)]
\item Any morphism $ \left[n\right] \to \left[m\right]$ in $\varDelta$ factors uniquely as the composite of an epimorphism $ \left[n\right] \to \left[p\right]$ (i.e., an order-preserving surjection) and a monomorphism $ \left[p\right] \to \left[m\right]$ (i.e., an order-preserving injection).
\item Any epimorphism $\epsilon : \left[n\right] \to \left[p\right]$ in $\varDelta$ factors uniquely as $\epsilon = \sigma_{j_{1}} \cdots \sigma_{j_{t}}$ where $t \equiv n-p$ and \linebreak $0 \leq j_{1}<\cdots<j_{t}<n$.
\item Any monomorphism $\mu : \left[p\right] \to \left[m\right]$ factors uniquely as $\mu=\delta_{i_{r}} \cdots \delta_{i_{1}}$ where $r \equiv m-p$ and \linebreak $0 \leq i_{1}<\cdots<i_{r} \leq m$.
\ee
\end{lemma}

\smallskip

Now, let us form the $i$-th face operator $d_i \coloneqq X(\delta_i^n) :X_{n+1}\to X_n$ and $i$-th degeneracy operator $s_i \coloneqq X(\sigma_i^n) : X_n \to X_{n+1}$ in $X$.

\begin{term}
The \textit{boundary} of an $n$-simplex $x\in X_n$ is the tuple $\partial{x} \coloneqq \left(d_0{x}, \ldots, d_n{x}\right)$.
\end{term}

For any $n$-simplex $x$ in $X$, we can view $d_i(x)$ as the $\left(n-1\right)$-simplex, or face, in $X$ missing the $i$-th vertex of $x$. Moreover, we can view $s_i(x)$ as the $\left(n+1\right)$-simplex having $x$ as its $i$-th and $\left(i+1\right)$-th faces so that collapsing the edge between its $i$ and $\left(i+1\right)$-th vertices yields $x$. For example, the map $s_1$ acts on $1$-simplices by
\[
\left(\begin{tikzcd}
\cdotp \arrow[r, "x", no head] & \boldsymbol{\cdot}
\end{tikzcd}\right) \mapsto
\left( \begin{tikzcd}
                                  & \boldsymbol{\cdot}\arrow[d, equal] \\
\cdotp \arrow[r, no head, "x"'] \arrow[ru, no head, "x"] &\boldsymbol{\cdot}          
\end{tikzcd}\right).
\]

\begin{term}
A simplex of the form $s_i(x)$ is called \textit{degenerate}. 
\end{term}

Since any epimorphism other than an identity morphism in $\varDelta$ factors as a composite of degeneracy operators, we may say equivalently that $x$ is degenerate if there exist an epimorphism $s: \left[n\right] \to \left[m\right]$ with $m<n$ and an $m$-simplex $y\in X_m$ such that $x= X(s)(y)$.

\begin{lemma}[Simplicial identities]\label{Simpid}
\[
\left\{\begin{array}{ll}{d_{i} d_{j}=d_{j-1} d_{i}} & {\text { for } i<j} \\ {d_{i} s_{j}=s_{j-1} d_{i}} & {\text { for } i<j} \\ {d_{i} s_{j}= \idd_{X_n}} & {\text { for } j \leq i \leq j+1} \\ {d_{i} s_{j}=s_{j} d_{i-1}} & {\text { for } j+1<i} \\ {s_{i} s_{j}=s_{j+1} s_{i}} & {\text { for } i \leq j}\end{array}\right..
\]
\end{lemma}
\begin{proof}
It is easy to verify that the \textit{cosimplicial identities}, those dual to the simplicial identities, hold in $\varDelta$, e.g., $$\delta_j\delta_i =\delta_i\delta_{j-1}, \ \quad i<j.$$ Applying now the  contravariant functor $X$ on $\varDelta$ to the cosimplicial identities yields the simplicial ones.
\end{proof}

Conversely, a family of set maps $$\left\{ d_i^n : X_{n+1} \to X_n \mid 0 \leq i \leq n+1, \ n \in \N \right\} \cup \left\{s_j^n : X_n \to X_{n+1} \mid 0 \leq j \leq n, \ n \in \N\right\}$$ satisfying the simplicial identities completely determines a simplicial set $X$.\footnote{\autocite[Lemma 6.2.8]{Rognes}.} In other words, a simplicial set amounts to an $\N$-graded set $\left(X_n\right)$ equipped with such a family of maps.


\begin{lemma}[Eilenberg-Zilber]\label{E-Z}
For any $x\in X_n$, there exists a unique pair $\left(s, y\right)$ where $s: \left[n\right] \to \left[m\right]$ is an epimorphism and $y $ is a non-degenerate $m$-simplex in $X$ satisfying $x= X(s)(y)$.
\end{lemma}
\begin{proof}
To see that such a pair exists, we have two cases to consider.
\bi
\item If $x$ is non-degenerate, then simply take the pair $\left(\idd_{\left[n\right]}, x\right)$.
\item If $x$ is degenerate, then by definition we can find  an epimorphism $t: \left[n\right] \to \left[m\right]$ with $m<n$ and an $m$-simplex $w\in X_m$ such that $x= X(t)(w)$. In this case, if $w$ is non-degenerate, then we are done. Otherwise, we can find another pair $\left(r, z\right)$ witnessing the degeneracy of $w$, so that $r : \left[m\right] \to \left[p\right]$ for some $p<m$. But this process must terminate in finitely many steps, resulting in our desired pair since any composite of epimorphisms is epic.
\ei
To see that $\left(s,y\right)$ is unique, suppose that $\left(s', y'\right)$ is another such pair. 
\begin{claim}
Any epimorphism in $\varDelta$ is a split epimorphism.
\end{claim}
\begin{proof}
Note that any surjective map in $\varDelta$ has a set-theoretic section,\footnote{This is provable in $\mathsf{ZF}$ as it follows from the finite version of the axiom of choice.} which is easily seen to be order-preserving. But the epimorphisms in $\varDelta$ are precisely the order-preserving surjections, which completes our proof.
\end{proof}
Thus, we may choose sections $\sigma$ and $\sigma'$ of $s$ and $s'$ in $\varDelta$, respectively.  This implies that 
\[
y = X(\sigma)(x) = X(\sigma)X(s')(y') = X(s'\sigma)(y').
\] Since $y'$ is non-degenerate, $s'{\sigma}$ must be an automorphism. But the identity map is the only  automorphism of a well-ordered set. Hence $y' = y$, and any section of $s$ is a section of $s'$. This means that $s' = s$, so that $\left(s, y\right) = \left(s', y'\right)$.
\end{proof}

\smallskip

Before moving on, let us record a definition that will appear in \cref{modcat}.

\begin{definition}\label{finsimp}
A simplicial set is \textit{finite} if it has only finitely many non-degenerate simplices.
\end{definition}


\medskip

Next, for any integer $n\geq 1$ and any simplicial set $X$, consider the discrete diagram $D$ valued in the over category $\sset/X$ consisting of all simplicial subsets $Y \hookrightarrow X$  that contain every non-degenerate simplex in $X$ of degree $<n$. The limit $\sk_n(X) \hookrightarrow X$ of $D$, which exists because $\sset$ is complete, is called the \textit{intersection} of the $Y$. We can view the \textit{$n$-skeleton $\sk_n(X)$ of $X$} as the smallest simplicial subset of $X$ that contains every non-degenerate simplex in $X$ of degree $<n$. Note that $\sk_n({-})$ determines a functor $\sset \to \sset$.

\begin{comment}
Next, for any integer $n\geq 0$, consider the full subcategory $\varDelta_{\leq n}$ of $\varDelta$ on the set of all ordinals $\left[m\right]$ such that $m\leq n$. We call a functor of the form $\varDelta_{\leq n}^{\op} \to \set$ an \textit{$n$-truncated simplicial set}. The full inclusion $\iota : \varDelta_{\leq n}^{\op} \hookrightarrow \varDelta^{\op}$ induces a \textit{truncation} functor 
\[
\trr_n : \sset \to \left[\varDelta_{\leq n}^{\op}, \set\right], \ \quad K \mapsto K \circ \iota.
\]
This has a left adjoint $\sk_n : \left[\varDelta_{\leq n}^{\op}, \set\right] \to \sset$ called the \textit{$n$-skeleton}, which is given pointwise by the formula 
\[
\sk_n{X}[p] = {\int^{[m]: \varDelta_{\leq n}} \varDelta\left(\left[p\right], \left[m\right]\right)\times  X_m }.
\] (See \cref{coend} below for the meaning of $\int$.)
\end{comment}

\bigskip

Consider the Yoneda embedding $\mathcal{Y}: \varDelta \to \widehat{\varDelta}$ (\cref{Yoneda}). For any $n\in \N$, we call the simplicial set
\[
\Delta[n]\coloneqq \mathcal{Y}{\left[n\right]} =\varDelta\left({-}, \left[n\right]\right)
\] the \textit{standard (combinatorial) $n$-simplex}. 

\begin{remark}\label{std}
Notice that 
\[
\Delta[n]_k \cong \left\{\left(x_0, \ldots, x_k\right) \mid 0\leq x_i \leq x_j \leq n, \ i \leq j\right\}
\]  for any $k \in \N$. Thus, we can view $\Delta[n]$ as the simplicial complex whose $k$-simplices are precisely the nonempty subsets of $\left\{0, 1, \ldots, n\right\}$ endowed with their natural orders.
\end{remark}

Now, the $i$-th face operator $d_i : \Delta[n]_{k+1} \to \Delta[n]_k$ is given by
\[
\left(\left[k+1\right] \stackrel{f}{\longrightarrow}\left[n\right]\right) \mapsto \left(\left[k\right] \stackrel{\delta_i}{\longrightarrow}\left[k+1\right] \stackrel{f}{\longrightarrow}\left[n\right]\right).
\]
Additionally, the $i$-th degeneracy operator $s_i : \Delta[n]_{k} \to \Delta[n]_{k+1}$ is given by 
\[
\left(\left[k\right] \stackrel{f}{\longrightarrow}\left[n\right]\right) \mapsto \left(\left[k+1\right] \stackrel{\sigma_i}{\longrightarrow}\left[k\right] \stackrel{f}{\longrightarrow}\left[n\right]\right).
\] 

Using \cref{mordelt}, we see that the non-degenerate $k$-simplices in $\Delta[n]$ are precisely the monomorphisms belonging to $\varDelta\left(\left[k\right], \left[n\right]\right)$.  In particular, the unique non-degenerate $n$-simplex in $\Delta[n]$ is precisely $\idd_{\left[n\right]}$. Thus, for any simplicial set $X$, there is a natural one-to-one correspondence 
\[
x\in X_n \longleftrightarrow \Delta[n] \overset{x}{\to} X
\] where the map $x$ sends the unique non-degenerate $n$-simplex in $\Delta[n]$ to the element $x$.

\medskip


Suppose that $n\in \Z_{\geq 1}$. For each $i\in \left\{0,1, \ldots, n\right\}$, The \textit{$i$-th face} of $\Delta[n]$ is the simplicial subset (i.e., subfunctor) $\partial^i{\Delta[n]}$ of $\Delta[n]$ with $$\partial^i{\Delta[n]}_k \equiv \im(\Delta(\delta_i^{n-1})_k) \subset \Delta[n]_k$$ for each $k\in \N$. The simplicial subset
\[
\partial{\Delta[n]} \coloneqq \bigcup_{i=0}^n\partial^i{\Delta[n]}
\] of $\Delta[n]$, computed pointwise in $\set$,  is called the \textit{simplicial $\left(n-1\right)$-sphere} or \textit{boundary of $\Delta[n]$}. 

\begin{remark}
The $k$-simplices of $\partial{\Delta[n]}$ are precisely the non-surjective morphisms $\left[k\right]\to \left[n\right]$ in $\varDelta$. Thus, in light of \cref{std}, we can view $\partial{\Delta[n]}$ as the simplicial complex whose $k$-simplices are precisely the nonempty \emph{proper} subsets of $\left\{0, 1, \ldots, n\right\}$ endowed with their natural orders.
\end{remark}

\subsection*{Geometric realization}

Let $\d$ be any cocomplete, locally small category and let $\c$ be any small category. Suppose that $F$ is a covariant functor $ \c \to \d$. For each $X \in \ob{\d}$, consider the presheaf $$R_F(X)\equiv \Hom_{\d}(F({-}), X) : \c^{\op} \to \set.$$ Also, for any map $f : X\to Y$ in $\d$, define the natural transformation $R_F{f} : R_F(X)  \to R_F(Y)$ componentwise by
\[
R_F(X)_c \to R_F(Y)_c,\ \quad \varphi \mapsto f \circ \varphi.
,\] thereby yielding a functor $R_F: \d \to \left[\c^{\op}, \set\right]$. 

\begin{exmp}\label{nerve}
We can view each nonempty finite ordinal  as an order category. This means that $\varDelta$ is precisely the full subcategory of $\mathbf{Cat}$ on all nonempty finite ordinals. Consider the full inclusion $\iota : \varDelta \hookrightarrow \mathbf{Cat}$. Then for each small category $\e$, $N{\e} \coloneqq R_{\iota}(\e)$ is a simplicial set (called the \textit{nerve of $\e$}) with $N{\e}_n = \Hom_{\mathbf{Cat}}(\left[n\right], \e)$ for each $n\in \N$, i.e.,  the set of all sequences of $n$ composable morphisms in $\e$.
The $i$-th face operator $d_i : N{\e}_{n+1}\to N{\e}_n$ is given by
\[
{x_0 \to x_1 \to \cdots \to x_{n+1} \to x_{n+2}} \mapsto \begin{cases}
{x_1 \to x_2\to \cdots \to x_{n+1} \to x_{n+2}} & i=0
\\ {x_0 \to x_1 \to \cdots \to x_n\to x_{n+1}} & i= n+1
\\ {x_0  \to \cdots \to x_i \to x_{i+2} \to \cdots  \to x_{n+2}} & \text{otherwise}
\end{cases}
,\] 
and the $j$-th degeneracy operator $s_j : N{\e}_n \to N{\e}_{n+1}$ is given by
\[
\left( x_0 \to x_1 \to \cdots \to x_{n} \to x_{n+1}\right) \mapsto  \left({x_0 \to \cdots \to x_j \xrightarrow{\idd_{x_j}} x_j \to x_{j+1} \to \cdots \to x_{n+1}}\right)  
\] for each $j=0,1, \ldots, n$.
\end{exmp}

In general, $R_F$ has a left adjoint, which we now begin constructing. To this end, consider the following generalization of \cref{tens-hom}. 

\begin{comment}
For each $x\in \ob{\d}$ and $n\in \N$, let $$R{x_n}= \Hom_{\d}(F[n], x).$$ Define the face operator $d_i : R{x_{n+1}} \to R{x_n}$ and the degeneracy operator $s_i : R{x_n} \to R{x_{n+1}}$ as the set maps $\Hom_{\d}(F(\delta_i^n), x)$ and $\Hom_{\d}(F(\sigma_i^n), x)$, respectively. By functoriality of $F$, the $F(\delta_i^n)$ and $F(\sigma_i^n)$ satisfy the cosimplicial identities. Since $\Hom_{\d}(F(-), x)$ is a contravariant functor $ \varDelta \to \set$, it follows that the $d_i$ and $s_i$ satisfy the simplicial identities, thereby making $\left(R{x_n}\right)$ into a simplicial set.
\end{comment}

\begin{definition}
Let $\a$ be a closed monoidal category and $\b$ a category enriched over $\a$. Let $b\in \ob{\b}$ and $S\in \ob{\a}$. The \textit{copower of $b$ by $S$} is an object $S\odot b$ in $\b$ together with a natural isomorphism
\[
\Hom_{\b}(S\odot b, y) \cong \left[S, \Hom_{\b}(b,y)\right]
\] in $y \in \ob{\b}$.
\end{definition}

For example, since $\d$ is enriched over $\set$ and has all coproducts by assumption, the copower of $x\in \ob{\d}$ by a set $S$ is precisely the coproduct $\coprod_{s\in S}{x}$ of $\left\lvert{S}\right\rvert$ many copies of $x$ along with the natural isomorphism
\[
\Hom_{\d}\left(\coprod_{s\in S}{x}, y\right) \cong \prod_{s\in S}\Hom_{\d}\left(x, y\right) \cong \Hom_{\set}\left(S, \Hom_{\d}\left(x, y\right) \right)
\] witnessing the fact that $\Hom_{\d}({-}, y) : \d^{\op} \to \set$ is a continuous functor for each $y\in \ob{\d}$.


\makeatletter
\newcommand*{\doublerightarrow}[2]{\mathrel{
  \settowidth{\@tempdima}{$\scriptstyle#1$}
  \settowidth{\@tempdimb}{$\scriptstyle#2$}
  \ifdim\@tempdimb>\@tempdima \@tempdima=\@tempdimb\fi
  \mathop{\vcenter{
    \offinterlineskip\ialign{\hbox to\dimexpr\@tempdima+1em{##}\cr
    \rightarrowfill\cr\noalign{\kern.5ex}
    \rightarrowfill\cr}}}\limits^{\!#1}_{\!#2}}}
\newcommand*{\triplerightarrow}[1]{\mathrel{
  \settowidth{\@tempdima}{$\scriptstyle#1$}
  \mathop{\vcenter{
    \offinterlineskip\ialign{\hbox to\dimexpr\@tempdima+1em{##}\cr
    \rightarrowfill\cr\noalign{\kern.5ex}
    \rightarrowfill\cr\noalign{\kern.5ex}
    \rightarrowfill\cr}}}\limits^{\!#1}}}
\makeatother
   
 \newcommand{\xdasharrow}[2][->]{
% correct vertical setting by egreg:
% http://tex.stackexchange.com/a/59660/13304
\tikz[baseline=-\the\dimexpr\fontdimen22\textfont2\relax]{
\node[anchor=south,font=\scriptsize, inner ysep=1.5pt,outer xsep=2.2pt](x){#2};
\draw[shorten <=3.4pt,shorten >=3.4pt,dashed,#1](x.south west)--(x.south east);
}
}


\begin{definition}\label{coend}
Suppose that $\a$ is a small category and $\b$ is cocomplete. Let $G: \a^{\op}\times \a \to \b$ be a functor. The \textit{coend of $G$} is the coequalizer of the diagram
\[ \label{eqn:coeq}
\coprod_{f: a' \to a}G(a, a') \doublerightarrow{G(a,f)}{G(f,a')} \coprod_{a \in \ob{\a}}G(a,a) \xdasharrow{\hphantom{2mm}} \int^{a:\a}G(a,a)  \tag{$\bullet$}
\] in $\b$.
\end{definition}

Here is an equivalent description of the coend of $G$. A \textit{cowedge $k: G \to w$ for $G$} is an object $w$ in $\b$ together with a family of morphisms $\left\{k_a : G(a,a) \to w \mid a\in \ob{\a}\right\}$ such that for each map $f:a' \to a$ in $\a$, the square
\[
\begin{tikzcd}[column sep = large]
w                         & {G\left(a',a'\right)} \arrow[l, "k_{a'}"']                         \\
{G(a,a)} \arrow[u, "k_a"] & {G(a,a')} \arrow[l, "{G(a,f)}"] \arrow[u, "{G(f,a')}"']
\end{tikzcd}
\] commutes. The coend $\int^{a:\a}G(a,a)$ of $G$ is defined as a universal cowedge $\left\{\tilde{k}_a : G(a,a) \to w \mid a\in \ob{\a}\right\}$ for $G$ in the sense that there exists a unique map $ \int^{a:\a}G(a,a) \to w$ in $\b$ such that any diagram of the form
\[
\begin{tikzcd}[column sep=large]
{\int^{a:\a}G(a,a)} \arrow[rd, dashed] &                                                                 &                                                                            \\
                                       & w                                                               & {G\left(a',a'\right)} \arrow[l, "k_{a'}"'] \arrow[llu, "\tilde{k}_{a'}"', bend right] \\
                                       & {G(a,a)} \arrow[u, "k_a"] \arrow[luu, "\tilde{k}_a", bend left] & {G(a,a')} \arrow[u, "{G(f,a')}"'] \arrow[l, "{G(a,f)}"]                                  
\end{tikzcd}.
\]
commutes.
We have a category $\mathbf{Cwd}(G)$ with cowedges for $G$ as objects and maps $u: w \to w'$ in $\b$ for which the triangle
\[
\begin{tikzcd}
w \arrow[r, "u"]                              & w' \\
{G(a,a)} \arrow[u, "k_a"] \arrow[ru, "k'_a"'] &   
\end{tikzcd}
\] commutes for every $a\in \ob{\a}$ as morphisms $w\to w'$. Note that the coend $\int^{a:\a}G(a,a)$ is precisely the initial object of $\mathbf{Cwd}(G)$. Thus, it may be viewed as either a colimit in $\b$ or a colimit in $\mathbf{Cwd}(G)$.

\begin{comment}
To see that this is isomorphic to the coequalizer in \eqref{eqn:coeq}, consider the \textit{twisted arrow category $\mathbf{Tw}(\a)$ of $\a$} with morphisms in $\a$ as objects and commutative squares of the form
\[
\begin{tikzcd}
A \arrow[d, "f"'] & C \arrow[d, "g"] \arrow[l, "p"'] \\
B \arrow[r, "q"'] & D                               
\end{tikzcd}
\] as morphisms $\left(p,q\right): f \to g$. Define the functor $\underline{G}: \mathbf{Tw}(\a^{\op})^{\op} \to \b$ by $$\left(f: B \to A\right)\mapsto G(B,A).$$ Note that any cowedge $\left\{k_a : G(a,a) \to w \mid a\in \ob{\a}\right\}$ for $G$ induces a cocone $$\left\{k_a\circ G(a,f) \mid f \in \Hom_{\a^{\op}}(a,a'), \ a,a'\in \ob{\a}\right\}$$ under $\underline{G}$. Conversely, any cocone $\left\{c_f : \underline{G}(f) \to A \mid f\in \mor(\a^{\op}) \right\}$ under $\underline{G}$ induces a cowedge $\left\{c_{\idd_a} \mid a\in \ob{\a} \right\}$ for $G$. This defines an equivalence $E$ from the category of cocones under $\underline{G}$ to  $\mathbf{Cwd}(G)$. Since equivalences preserve initial objects, it follows that $$E\left(\colim_{\mathbf{Tw}(\a^{\op})^{\op}}\underline{G}\right) \cong \int^{a:\a}G(a,a).$$
\end{comment}

\smallskip

\begin{definition}\label{gext}
The  \textit{Yoneda extension of $F$} is the functor 
\[
\widetilde{F} : \left[\c^{\op}, \set\right] \to \d, \ \quad X \mapsto \int^{c:\c} X(c) \odot F(c). 
\]
We call $\left\lvert{X}\right\rvert \coloneqq \widetilde{F}(X)$ the \textit{geometric realization} of $X$ with respect to $F$. 
\end{definition}

If $F$ denotes the functor $\varDelta \to \mathbf{Top}$ mapping $\left[n\right]$ to the standard topological $n$-simplex $\Delta^n$, then we recover the familiar definition of $\left\lvert{\left(X_n\right)}\right\rvert$ as a quotient space
$$ \faktor{\coprod_{m\geq 0}{\left(X_m \times \Delta^m\right)}}{\sim}$$ where each set $X_m$ is endowed with the discrete topology.

\begin{prop}\label{GCW}
For any simplicial set $X$, $\left\lvert{X}\right\rvert$ is a CW-complex.
\end{prop}
\begin{proof}[Intuitive proof]
For each integer $m\geq 0$, consider the subset $Y_m \subset X_m$ of all non-degenerate $m$-simplices. Then $\left\lvert{X}\right\rvert$ is obtained by gluing together countably many disjoint unions $Y_m \times \Delta^{m} \cong \coprod_{y\in Y_m}\Delta^m$ of topological simplices $\Delta^m$ along their individual boundaries $\partial{\Delta^m}$. Therefore, $\left\lvert{X}\right\rvert$ carries the structure of a CW-complex.
\end{proof}

This means that the functor $\left\lvert{-}\right\rvert : \sset \to \mathbf{Top}$ takes values in the full subcategory $k\mathbf{Top}$ of \textit{$k$-spaces}, i.e., quotient spaces of disjoint unions of compact Hausdorff spaces.

\medskip

The subcategory $k\mathbf{Top}\subset \mathbf{Top}$ has a coreflection $k: \mathbf{Top} \to k\mathbf{Top}$, known as \textit{$k$-ification}. For any topological space $U$, the $k$-space $k(U)$ is given by the set $U$ topologized so that a subset $A\subset U$ is closed if and only if  $A \cap H$ is closed in $U$ for any compact Hausdorff subspace of $U$. Thus, $k(U)$ has a finer topology than $U$.

\medskip

Since right adjoints preserve limits and $\mathbf{Top}$ is complete, we have that $k\mathbf{Top}$ is complete. Indeed, let $D : J \to k\mathbf{Top}$ be a functor. Then
\[
k\left(\lim_J{\left(\iota \circ D\right)_j}\right) \cong \lim_J{\left(k\circ \iota \circ D\right)_j}
\] where $\iota : k\mathbf{Top} \hookrightarrow \mathbf{Top}$ denotes inclusion.
\begin{prop}
Consider any adjoint pair $\left(F: \c \to \d, G: \d \to \c\right)$ of functors. If $F$ is fully faithful, then the unit $\eta : \idd_{\c}\to  G\circ F $ is an isomorphism.
\end{prop}
\begin{proof}
Let the natural isomorphism $\varphi : \Hom_{\d}(F({-}), {-}) \overset{\cong}{\longrightarrow} \Hom_{\c}({-}, G({-}))$ witness our adjunction. Suppose that $F$ is fully faithful. Then we have a composite of isomorphisms
\[
\begin{tikzcd}
{\Hom_{\c}(x,y)} \arrow[r, "\underset{\cong}{F({-})}"] & {\Hom_{\d}(F{x}, F{y})} \arrow[r, "\underset{\cong}{\varphi_{x, F{y}}}"] & {\Hom_{\d}(x, G{F{y}})}
\end{tikzcd}
\] for any $x,y\in \ob{\c}$. For any map $f: x\to y$ in $\c$, we have that 
\begin{align*}
 \varphi_{x,F{y}}(F(f)) & =G(F(f)) \circ \eta_x \tag{unit identity}
 \\ & = \eta_y \circ f \tag{naturality of $\eta$}.
 \end{align*} Thus, our composite isomorphism is given by $f\mapsto \eta_y \circ f$, i.e., $\Hom_{\c}(x, \eta_y)$. As the Yoneda embedding reflects isomorphisms, we see that $\eta_y : y \overset{\cong}{\longrightarrow} G{F{y}}$ is an isomorphism, i.e., $\eta$ is an isomorphism.
\end{proof}
As a consequence, $k\circ \iota$ is naturally isomorphic to the identity functor because. Hence $\lim_J{D_j}$ is given by $k\left(\lim_J{\left(\iota \circ D\right)_j}\right)$. Note that the limit of a diagram in $k\mathbf{Top}$ is obtained by $k$-ifying the limit of the same diagram in $\mathbf{Top}$. 

\begin{theorem}\label{realpres} $ $
\be[label=(\arabic*)]
\item The functor $\left\lvert{-}\right\rvert : \sset \to k\mathbf{Top}$ preserves finite products.\footnote{\cite[Lemma 3.1.8]{Hovey}.}
\item The functor $\left\lvert{-}\right\rvert : \sset \to k\mathbf{Top}$ preserves equalizers, hence all finite limits.\footnote{\cite[Lemma 3.2.4]{Hovey}.}
\ee
\end{theorem}

\medskip

Let us now return to our general setting.

\begin{lemma}
$\left(\widetilde{F}, R_F\right)$ is an adjoint pair.\footnote{Cf.  \cite[Section 4]{Riehl}.}
\end{lemma}
\begin{proof}
Let $x\in \ob{\c}$ and consider the Yoneda embedding $\mathcal{Y} : \c \to \widehat{\c}$. For each $c\in \ob{\c}$, define the map $k_c :\coprod_{s\in \mathcal{Y}_x(c)}F(c) \to F(x)$ in $\d$ by the copairing $\left(F(s) :F(c)\to F(x)\right)_{s\in \mathcal{Y}_x(c)}$. For any map $f: c' \to c$ in $\c$ and any $s\in \mathcal{Y}_x(c)$, we have that 
$$F(s) \circ F(f) = F(s \circ f) =  F(\mathcal{Y}_x(f)(s)) = F(\mathcal{Y}_x(f)(s)) \circ \idd_{F(c')}.$$ This implies that the square
\[
\begin{tikzcd}[column sep=huge, row sep = large]
F(x)                                       & \coprod_{s\in \mathcal{Y}_x(c')}F(c') \arrow[l, "k_{c'}"']                                                \\
\coprod_{s\in \mathcal{Y}_x(c)}F(c) \arrow[u, "k_c"] & \coprod_{s\in \mathcal{Y}_x(c)}F(c') \arrow[l, "\mathcal{Y}_x(c)\odot F(f)"] \arrow[u, "\mathcal{Y}_x(f) \odot F(c')"']
\end{tikzcd}
\] commutes, so that $\left\{k_c\right\}$ is a cowedge  to $F(x)$. Let $\left\{k'_c : \coprod_{s\in \mathcal{Y}_x(c)}F(c) \to w \right\}$ be another cowedge. Consider the composite $k'_x \circ i_{\idd_x} : F(x) \to w$, where $ i_{\idd_x}$ denotes inclusion. For any map $s: c\to x$ in $\c$, the fact that
\[
\begin{tikzcd}[column sep=huge, row sep = large]
w                                           & \coprod_{t\in \mathcal{Y}_x(c)}F(c) \arrow[l, "k_{c}'"']                                          \\
\coprod_{t\in \mathcal{Y}_x(x)}F(x) \arrow[u, "k_x'"] & \coprod_{t\in \mathcal{Y}_x(x)}F(c) \arrow[l, "\mathcal{Y}_x(x)\odot F(s)"] \arrow[u, "\mathcal{Y}_x(s) \odot F(c)"']
\end{tikzcd}
\] commutes yields
\begin{align*}
k'_x \circ i_{\idd_x} \circ k_c \circ i_s & = k'_x \circ i_{\idd_x} \circ  F(s) 
\\ & =k'_c \circ \left( \mathcal{Y}_x(s) \odot F(c) \right) \circ i_{\idd_x}
\\ & =  k'_c \circ i_{\mathcal{Y}_x(s)(\idd_x)} \circ \idd_{F(c)} 
\\ & =k'_c \circ i_s.
\end{align*}
It follows that $F(x)$ is a universal cowedge. By uniqueness of colimits, we have a natural isomorphism $$F(x) \cong \int^{c:\c} \mathcal{Y}_x(c) \odot F(c) = \widetilde{F}(\mathcal{Y}_x)$$ in $x$. Thanks to this as well as the Yoneda lemma, we have a sequence of isomorphisms
\[
\label{eqn:adji} \Hom_{\widehat{\c}}(\mathcal{Y}_x, R_F(d)) \cong R_F(d)_x = \Hom_{\d}(F(x), d) \cong \Hom_{\d}(\widetilde{F}(\mathcal{Y}_x), d)
\tag{$\blacklozenge$}
\] natural in both $x$ and $d$. By \cref{psclr}, every presheaf $X : \c^{\op} \to \set$ is naturally isomorphic to a small colimit of representable presheaves. Further, colimits commute with colimits, and thus $\widetilde{F}$ commutes with all colimits. Since the hom-functor of any locally small category is continuous in its first variable, we can conclude that \eqref{eqn:adji} holds with $\mathcal{Y}_x$ replaced by any presheaf $X$.
\end{proof}

\subsection*{Kan fibrations}

Let $n\in \Z_{\geq 1}$. For each integer $0\leq k\leq n$, the simplicial subset
\[
\Lambda^k[n]\coloneqq \bigcup_{i\in \left\{0,\ldots, k-1, k+1, \ldots, n\right\}}\partial^i{\Delta[n]}
\] of $\Delta[n]$, computed pointwise in $\set$,  is called the \textit{(simplicial) $\left(n,k\right)$-horn}. For any simplicial set $X$, an \textit{$\left(n,k\right)$-horn in $X$} is a simplicial map $\Lambda^k[n] \to X$. 
\begin{term}
If $0<k<n$, then such a map is called an \textit{inner horn} in $X$. Otherwise, it is called an \textit{outer horn} in $X$.
\end{term}

\begin{exmp}
The geometric realization of the inner horn in $\Delta[2]$ looks like
\[
\begin{tikzcd}[column sep=small, row sep=small]
             & 1 \arrow[rd, no head] &   \\
0 \arrow[ru, no head] &              & 2
\end{tikzcd},
\]
 whereas  the geometric realizations of the two outer horns in $\Delta[2]$ look like
 \[
\begin{tikzcd}[column sep=small, row sep=small]
                        & 1 &   &  &              & 1 \arrow[rd, no head] &   \\
0 \arrow[ru, no head] \arrow[rr, no head] &   & 2 &  & 0 \arrow[rr, no head] &              & 2
\end{tikzcd}
 \]
\end{exmp}

\smallskip

\begin{comment}
\medskip

Now is a good time to list a couple more discrete-geometric objects in $\sset$. Let $m\in \Z_{\geq 1}$.

\bi
\item A \textit{prism of dimension $m+n$} is the cartesian product $\Delta[m,n]\coloneqq \Delta[m]\times \Delta[n]$.

The \textit{boundary} of this simplicial set is given by the Leibniz rule:
\[
\partial{\Delta[m+n]} \coloneqq \left(\partial{\Delta[m]} \times \Delta[n]\right)\cup \left(\Delta[m]\times \partial{\Delta[n]}\right)
.\] 
\item By removing the face $\Delta[m]\times \partial^k{\Delta[n]}$ from $\partial{\Delta[m+n]}$, we obtain an \textit{open box}
\[
\Lambda^{m+1+k}[m,n]\coloneqq \left(\partial{\Delta[m]}\times \Delta[n]\right) \cup \left(\Delta[m] \times \Lambda^k[n]\right).
\]
\ei

\medskip
\end{comment}

We say that $X$ is a \textit{Kan complex} if every horn in $X$ has a filler, i.e., can be extended to $\Delta[n]$ along the inclusion map:
\[
\begin{tikzcd}
{\Lambda^k[n]} \arrow[d, hook] \arrow[r] & X \\
\Delta[n] \arrow[ru, dashed]              &  
\end{tikzcd}
.\]
Intuitively, by the Yoneda lemma, such an extension picks out a unique $n$-simplex in $X$ all of whose faces but one are determined by the given horn in $X$. 


\begin{definition}[Kan fibration]\label{KF}
A map $p: X\to Y$ of simplicial sets is a \textit{Kan fibration} if any commutative square of the form
\[
\begin{tikzcd}
{\Lambda^k[n]} \arrow[d, hook] \arrow[r] & X \arrow[d, "p"] \\
\Delta[n] \arrow[r]                       & Y               
\end{tikzcd}
\] admits a lift
\[
\begin{tikzcd}
{\Lambda^k[n]} \arrow[d, hook] \arrow[r] & X \arrow[d, "p"] \\
\Delta[n] \arrow[r] \arrow[ru, dashed]    & Y               
\end{tikzcd}.
\]
\end{definition}

This means that $X$ is a Kan complex if and only if the unique map $X \to 1$ from $X$ to the terminal object $\Delta[0]$ is a Kan fibration.

\medskip

Now, let $\c$ be any cocomplete category. We say that a subclass of $\mor(\c)$ is \textit{saturated} if it
\be[label=(\roman*)]
\item contains all isomorphisms in $\c$,
\item is closed under pushouts,
\item is closed under retracts (see \eqref{eqn:ret} below), and
\item is closed under transfinite compositions (see \cref{TrC} below).
\ee

\begin{term}
The smallest saturated class containing a given class $K$ of morphisms in $\c$ is called the \textit{saturated class generated by $K$}.
\end{term}

\begin{definition}[Anodyne extension]\label{anodyne}
A map of simplicial sets is an \textit{anodyne extension} if it belongs to the saturated class  generated by the set 
\[
\left\{\Lambda^k[n] \hookrightarrow \Delta[n] \mid n\geq 1, \ 0\leq k \leq n\right\}
\] of horn inclusions.
\end{definition}

\begin{notation}
$\mathbb{A}$ will denote the class of all anodyne extensions. 
\end{notation}

Thanks to \cref{Satd} below, we see that a simplicial map is a Kan fibration if and only if it has  the right lifting property against $\mathbb{A}$.

\bigskip

We want to look at a certain class of Kan fibrations that will play a key role in our interpretation of $\cdtt +\univ$ in $\sset$. For this, we must first gather some standard concepts of simplicial homotopy theory.

\begin{definition}[Simplicial homotopy]
Let $f, g: X \to Y$ be maps of simplicial sets. A \textit{(simplicial) homotopy $f \overset{\simeq}{\longrightarrow} g$ from $f$ to $g$} is a map $h: X \times \Delta[1] \to Y$ of simplicial sets fitting into a commutative diagram
\[
\begin{tikzcd}[column sep = large]
\overbrace{X\times \Delta[0]}^{X} \arrow[rd, "f"] \arrow[d, "\idd_X \times \delta_1"'] &   \\
X\times \Delta[1] \arrow[r, "h"]                                                       & Y \\
\underbrace{X\times \Delta[0]}_{X} \arrow[ru, "g"'] \arrow[u, "\idd_X\times \delta_0"] &  
\end{tikzcd}.
\]
\end{definition}

\begin{notation} $ $
\bi
\item $h_0 \coloneqq h \circ  \left(\idd_X \times \delta_1\right)$.
\item $h_1 \coloneqq h \circ  \left(\idd_X \times \delta_0\right)$.
\ei
\end{notation}

The cospan 
\begin{tikzcd}
\Delta[0] \arrow[r, "0 \coloneqq \delta_1"] & \Delta[1] & \Delta[0] \arrow[l, "1 \coloneqq \delta_0"']
\end{tikzcd} is the standard \textit{interval object} in $\sset$, analogous to the standard interval object
\[
\begin{tikzcd}
\left\{0\right\} \arrow[r, hook] & {\left[0,1\right]} & \left\{1\right\} \arrow[l, hook']
\end{tikzcd} 
\] in $\mathbf{Top}$.
From this perspective, any simplicial homotopy $h: f \overset{\simeq}{\longrightarrow} g$ satisfies  $h(x,0) = f(x)$ and $h(x,1) = g(x)$ for all simplices $x$ in $X$.


\medskip

Simplicial homotopy ${}\simeq{}$ is \emph{not} an equivalence relation in general. For example, let $n\in \Z_{\geq 1}$ and consider the simplicial maps $\iota_0, \iota_1 : \Delta[0] \to \Delta[n]$ induced by the monomorphisms $0 \mapsto 0$ and $0 \mapsto 1$, respectively, in $\varDelta$. In light of \cref{std}, it is easy to see, on the one hand, that the map sending, say, $\left(0, 0, 1, 1\right)$ to itself determines a homotopy from $\iota_0$ to $\iota_1$. On the other hand, there is no homotopy from $\iota_1$ to $\iota_0$, because $0\leq 1$. This shows that ${}\simeq{}$ is not symmetric in general. 

\begin{lemma}
Simplicial homotopy is an equivalence relation on the class of all simplicial maps $X\to Y$ with $Y$ a Kan complex.
\end{lemma}
\begin{proof}
For the moment, assume that $X = \Delta[0]$. Then, by the Yoneda lemma, there is a homotopy $f \overset{\simeq}{\longrightarrow} g$ if and only if there is a $1$-simplex $v\in Y_1$ such that $\partial{v} = \left(g, f\right)$. Thus, the equation $\partial{s_0{f}} = \left(f, f\right)$ witnesses the fact that ${}\simeq{}$ is reflexive.

\medskip

Next, to see that ${}\simeq{}$  is symmetric, let $\partial{v_2} = \left(g,f\right)$. Let $v_1 = s_0{f}$. Then $d_1{v_1} = d_1{v_2}$. From this, we get a $\left(2,0\right)$-horn $\left(v_1, v_2\right)$  in $Y$ where $v_i$ acts on the $i$-th face of $ \Lambda^0[2]$ for each $i=1,2$. As $Y$ is a Kan complex by hypothesis, this has a filler
\[
\begin{tikzcd}[column sep=large]
{\Lambda^0[2]} \arrow[d, hook] \arrow[r, "{\left(v_1, v_2\right)}"] & Y \\
{\Delta[2]} \arrow[ru, "\theta"', dashed]                          &  
\end{tikzcd}
.\] By the simplicial identities, we have that 
\begin{align*}
\partial\left(d_{0} \theta\right) &=\left(d_{0} d_{0} \theta, d_{1} d_{0} \theta\right) \\
&=\left(d_{0} d_{1} \theta, d_{0} d_{2} \theta\right) \\
&=\left(f, g\right)
,\end{align*} so that there is a homotopy from $g$ to $f$.

\medskip

Finally, to see that ${}\simeq{}$  is transitive, let $\partial{v_2} = \left(g,f\right)$ and $\partial{v_0} = \left(j, g\right)$. This means that $d_1{v_0} = d_0{v_2}$, thereby yielding a $\left(2,1\right)$-horn $\left(v_0, v_2\right)$ in  $Y$. As $Y$ is a Kan complex, this has a filler:
\[
\begin{tikzcd}[column sep=large]
{\Lambda^1[2]} \arrow[d, hook] \arrow[r, "{\left(v_0, v_2\right)}"] & Y \\
{\Delta[2]} \arrow[ru, "\theta'"', dashed]                           &  
\end{tikzcd}
.\] We have that
\begin{align*}
\partial\left(d_{1} \theta'\right) &=\left(d_{0} d_{1} \theta', d_{1} d_{1} \theta'\right) \\
&=\left(d_{0} d_{0} \theta', d_{1} d_{2} \theta'\right) \\
&=\left(j, f\right),
\end{align*} so that $f$ is homotopic to $j$.

\medskip

Now, assume that $X$ is arbitrary. Since $\sset$ is cartesian closed by \cref{pcc}, any simplicial map $f: X \times \Delta[0] \cong X \to Y$  naturally corresponds to a map $\tilde{f} : \Delta[0] \to Y^X$. Likewise, any simplicial homotopy $h: X \times \Delta[1] \to Y$ naturally corresponds to a map $\tilde{h}: \Delta[0] \times \Delta[1] \cong \Delta[1] \to Y^X$. Therefore, any homotopy $h: f \overset{\simeq}{\longrightarrow} g$ naturally corresponds to a homotopy $\tilde{h}: \tilde{f} \overset{\simeq}{\longrightarrow} \tilde{g}$. In this case, we have shown that ${}\simeq{}$ is an equivalence relation, and thus our proof is done.
\end{proof}

It follows at once that for any $n\in \Z_{\geq 1}$, $\Delta[n]$ is \emph{not} a Kan complex.

\medskip


For any simplicial set $X$, the set $\pi_0(X)$ of \textit{connected components} of $X$ is precisely the coequalizer in the diagram

\[
X_1 \doublerightarrow{d_0}{d_1} X_0 \xdasharrow{\hphantom{2mm}} \pi_0(X).
\] 

Explicitly, $\pi_0(X)$ equals the set of all connected components of the undirected graph 
\[\left(X_0, E_{X_0}\right), \ \quad E_{X_0} \equiv \left\{\left\{d_0(x), d_1(x)\right\} \mid x\in X_1\right\}.
\] By viewing a $1$-simplex in $X$ as a homotopy, we thus have $\pi_0(X)$ as the quotient of the set of all vertices in $X$ by the equivalence relation $\simeq$. 

\begin{remark}\label{conncomp}
We can generalize this notion a bit. Let $\c$ be a category enriched over the cartesian monoidal category $\sset$. Then the \textit{category $\pi_0(\c)$ of components of $\c$} is given by
\begin{gather*}
\ob{\pi_0(\c)} \equiv \ob{\c}
\\ \Hom_{\pi_0(\c)}(a,b) \equiv \pi_0(\Hom_{\c}(a,b)).
\end{gather*}
\end{remark}
It follows easily from the universal property of coequalizers that any $\sset$-enriched functor $F: \c \to \d$ induces a functor $\pi_0(F) : \pi_0(\c) \to \pi_0(\d)$.

\bigskip

Moving on, for any two simplicial maps $i : A \to B$ and $k : Y \to Z$, consider the \textit{pushout product} 
\[
i \mathbin{\ast} k  : \left(A \times Z\right) \cup_{A\times Y} \left(B \times Y \right) \to B \times Z,
\] i.e., the unique map fitting into a commutative diagram
\[
\begin{tikzcd}
A\times Y \arrow[d, "A\times k"'] \arrow[r, "i\times Y"]  & B\times Y \arrow[d] \arrow[rdd, "B\times k", bend left]                                                   &           \\
A\times Z \arrow[r] \arrow[rrd, "i\times Z"', bend right] &  \left(A \times Z\right) \cup_{A\times Y} \left(B \times Y \right) \arrow[rd, "i\mathbin{\ast}k", dashed] &           \\
                                                          &                                                                                                           & B\times Z
\end{tikzcd}
.\]
Notice that $B\times Z \cong Z \times B$ and $\left(A \times Z\right) \cup_{A\times Y} \left(B \times Y \right) \cong \left(Y \times B\right) \cup_{Y\times A} \left(Z \times A\right)$. Moreover, it is easy to check that if both $i$ and $k$ are monic (i.e., levelwise injections), then so is $i \mathbin{\ast} k$.


\begin{theorem}[Gabriel-Zisman]
If $i$ and $k$ are monic and $i$ is anodyne, then $i \mathbin{\ast} k$ is anodyne.\footnote{\cite[Theorem 3.2.2]{Joyal}.}
\end{theorem}
\begin{proof}
Let $k : Y \to Z$ be any simplicial map. Let $\D$ denote the class of all monomorphisms $i: A \to B$ such that $i\mathbin{\ast}k$ is an anodyne extension. We must show that $\mathbb{A} \subset \D$.
\begin{fact}
Let $\mathcal{C}$ denote the saturated class generated by 
\[
\left\{ {\iota_e}\mathbin{\ast}{m} : \left(\left\{e\right\}\times Z'\right) \cup \left(\Delta[1] \times Y'\right) \to \Delta[1] \times Z'\mid m : Y' \xrightarrow{\text{monic}} Z', \ e=0,1   \right\}
\] where $\iota_e$ denotes the horn inclusion  $\left\{e\right\}\hookrightarrow \Delta[1]$ for each $e=0,1$. Then $\mathbb{A} = \mathcal{C}$.\footnote{\cite[Theorem 3.2.3]{Joyal}.}
\end{fact}
Therefore, it suffices to show that $\mathcal{C}\subset \D$.

\medskip

We have that $\D$ is saturated. For example, to see that $\D$ is closed under pushouts, let $i\in \D$ and consider the pushout square
\[ \label{eqn:gz1}
\begin{tikzcd}
A \arrow[d, "i"']
 \arrow[dr, phantom, "\scalebox{1.5}{\color{black}$\ulcorner$}", very near end, color=black] \arrow[r, "f"] & C \arrow[d, "f_{\ast}{i}"] \\
B \arrow[r, "f'"']               & D                  
\end{tikzcd} \tag{1}
,\] where  $f_{\ast}{i}$ is monic since pushouts preserve monomorphisms in $\set$.
This square induces another pushout square
\[ \label{eqn:gz2}
\begin{tikzcd}[column sep=large]
A\times Z \arrow[d, "i\times Z"'] 
 \arrow[dr, phantom, "\scalebox{1.5}{\color{black}$\ulcorner$}", very near end, color=black] 
 \arrow[r, "f\times Z"] & C\times Z \arrow[d, "f_{\ast}{i}\times Z"] \\
B\times Z \arrow[r, "f'\times Z"']                       & D\times Z                                 
\end{tikzcd} \tag{2}
.\] 
Let $p: U \to V$ be a Kan fibration. We must show any lifting problem of the form
\[ \label{eqn:gz3}
\begin{tikzcd}
\left(C \times Z\right) \cup \left(D \times Y \right) \arrow[d, "{f_{\ast}{i}}\mathbin{\ast}{k}"'] \arrow[r] & U \arrow[d, "p"] \\
D\times Z \arrow[r]                                                                         & V               
\end{tikzcd} \tag{3}
\] has a solution. To this end, note that the commutative diagram 
\[
\begin{tikzcd}
\left(A \times Z\right) \cup \left(B \times Y \right) \arrow[d, "i\mathbin{\ast}{k}"'] \arrow[r] & \left(C \times Z\right) \cup \left(D \times Y \right) \arrow[r] \arrow[d, "{f_{\ast}{i}}\mathbin{\ast}{k}"', dotted] & U \arrow[d, "p"] \\
B\times Z \arrow[r]                                                                                & D\times Z \arrow[r]                                                                                                    & V               
\end{tikzcd}
\] induced by \eqref{eqn:gz2} admits a lift $B\times Z \to U$. By the universal property of \eqref{eqn:gz2}, we obtain a unique mediating map $D\times Z \to U$, which is a  solution to \eqref{eqn:gz3}.

\medskip

Thus, it suffices to show that for each monomorphism $m : Y' \to Z'$ and each integer $0\leq e\leq 1$, the map $${\iota_e}\mathbin{\ast}{m} : \left(\left\{e\right\} \times Z'\right)\cup \left(\Delta[1]\times Y'\right) \to \Delta[1]\times Z'$$ belongs to $\D$. It is easy to see that the maps
\begin{align*}
\left(\iota_e \mathbin{\ast}m \right) \mathbin{\ast} k &:\left( \left(\left(\left\{e\right\} \times Z^{\prime}\right) \cup\left(\Delta[1] \times Y^{\prime}\right)\right) \times Z\right) \cup \left(\left(\Delta[1] \times Z^{\prime}\right) \times Y\right) \longrightarrow\left(\Delta[1] \times Z^{\prime}\right) \times Z
\\ 
\iota_e \mathbin{\ast} \left(m\mathbin{\ast}k\right)& : \left(\left\{e\right\} \times Z^{\prime} \times Z\right) \cup \left(\Delta[1] \times\left(Y^{\prime} \times Z \cup Z^{\prime} \times Y\right)\right) \longrightarrow \Delta[1] \times\left(Z^{\prime} \times Z\right)
\end{align*} are isomorphic in $\Ar(\sset)$. But $\iota_e \mathbin{\ast} \left(m\mathbin{\ast}k\right)$ belongs to $\mathcal{C}$ and thus is anodyne. This implies that $\left(\iota_e \mathbin{\ast}m \right) \mathbin{\ast} k$ is also anodyne since $\mathbb{A}$ is closed under retracts. Hence the monomorphism $\iota_e \mathbin{\ast}m$  belongs to $\D$, as desired.
\end{proof}


\begin{corollary}[Covering homotopy extension property]\label{chep}
Let $p : U \to V$ be a Kan fibration and $k: Y\to Z$ be a monomorphism. Any commutative diagram of the form
\[
\begin{tikzcd}
{\left(Y \times \Delta[1]\right)\cup \left(Z\times \left\{e\right\}\right)} \arrow[d, "k\mathbin{\ast}\iota_e"'] \arrow[r] & U \arrow[d, "p"] \\
{Z\times \Delta[1]} \arrow[r]                                                                                        & V               
\end{tikzcd}
\] has a diagonal fill-in for each $e=0,1$.
\end{corollary}

\medskip


\newextarrow{\xbigtoto}{{20}{20}{20}{20}}
   {\bigRelbar\bigRelbar{\bigtwoarrowsleft\rightarrow\rightarrow}}

Suppose that $L$ is a simplicial subset of $X$ and that $f\restriction_L = g\restriction_L$.  We say that a homotopy $h: f\overset{\simeq}{\longrightarrow} g$ is a \textit{simplicial homotopy from $f$ to $g$ relative $L$} if the square
\[
\begin{tikzcd}
L\times \Delta[1] \arrow[r, "\pi_1"] \arrow[d, hook, "\iota\times \idd_{\Delta[1]}"'] & L \arrow[d, "f\restriction_L=g\restriction_L"] \\
X\times \Delta[1] \arrow[r, "h"']                                & Y                                             
\end{tikzcd}
\] commutes.  The relation ${}\simeq{}\rel{L}$ is also an equivalence relation when $Y$ is a Kan complex.

\medskip

Suppose that $p: X \to Y$ is a Kan fibration. We say that $p$ is \textit{minimal} if for any commutative diagram of the form
\[
\begin{tikzcd} \label{eqn:fibhom}
{\partial{\Delta[n]}\times \Delta[1]} \arrow[d, hook] \arrow[r, "\pi_1"] & {\partial{\Delta[n]}} \arrow[d] \\
{\Delta[n]  \times \Delta[1]} \arrow[d, "\pi_1"'] \arrow[r, "h"]   & X \arrow[d, "p"]                \\
{\Delta[n]} \arrow[r]                                              & Y                              
\end{tikzcd} \tag{$\ast$}
,\]
the diagram
\[ 
\Delta[n]  \xbigtoto[\left(\idd_{\Delta[n]}, \delta_1\right)]{\left(\idd_{\Delta[n]}, \delta_0\right)} \Delta[n]\times \Delta[1] \overset{h}{\longrightarrow} X
\] commutes. The bottom square of \eqref{eqn:fibhom} exhibits $h$ as a \textit{fiberwise} homotopy. Note that $p$ is minimal precisely when
\bi
\item $h$ is a homotopy relative boundary,
\item  $p\circ h_0 =p\circ h_1$ (i.e., $h_0$ and $h_1$ are in the same fiber of $p$), and
\item  whenever $h$ is a fiberwise homotopy relative $\partial{\Delta[n]}$, we have that $h_0 = h_1$.
\ei

In general, we say that two $n$-simplices $e, e' : \Delta[n] \to X$ of $X$ are \textit{$p$-fiberwise homotopic relative boundary}, written as $e \simeq_p e' \rel{\partial{\Delta[n]}}$, if there is a diagram of the form \eqref{eqn:fibhom} such that $h_0=e$ and $h_1=e'$.
Then $p$ is a minimal fibration if and only if $e \simeq_p e' \implies e=e'$.

\begin{prop}
$\simeq_p{}\rel{\partial{\Delta[n]}}$ is an equivalence relation.
\end{prop}

\smallskip


\begin{definition}[Deformation retraction]\label{defret}
Let $i: A \to B$ be a map of simplicial sets and let $r: B \to A$ be a retraction of $i$.
\be
\item We say that $r$ is a \textit{deformation retraction of $i$} if there is a homotopy $h: B \times \Delta[1] \to B$ from $\idd_B$ to $i \circ r$.
\pagebreak
\item We say that $r$ is a \textit{strong deformation retraction of $i$} if it is a deformation retraction of $i$ and the homotopy $h$ is stationary on $A$ in the sense that 
\[
\begin{tikzcd}[column sep=large]
{A\times \Delta[1]} \arrow[r, "{i\times \idd_{\Delta[1]}}"] \arrow[d, "\pi_1"'] & {B\times \Delta[1]} \arrow[d, "h"] \\
A \arrow[r, "i"']                                                               & B                                 
\end{tikzcd}
\] commutes.
\ee
\end{definition}



\begin{theorem}\label{minfact}
Let  $p: X \to Y$ be any Kan fibration. There exists a commutative diagram
\[ \label{eqn:minimal}
\begin{tikzcd}[row sep=large]
Z \arrow[rd, "p'"'] \arrow[r, "j"] & X \arrow[r, "g"] \arrow[d, "p"] & Z \arrow[ld, "p'"] \\
                                   & Y                               &                   
\end{tikzcd}
\tag{$\bullet$}
\] such that $g$ is a fiberwise strong deformation retraction of $j$ and $p'$ is a minimal fibration.\footnote{\cite[Theorem 3.3.3]{Joyal}.}
\end{theorem}


\begin{lemma}[Quillen]\label{Quillen}
The map $g$ in \eqref{eqn:minimal} has the right lifting property against the inclusion $i_n: \partial{\Delta[n]} \hookrightarrow \Delta[n]$ for every $n\in \N$.\footnote{\cite[Lemma 10.11]{goerss}.}
\end{lemma}
\begin{proof}
Since $g$ is a fiberwise strong deformation retraction of $j$ by hypothesis, we have a homotopy $h:  X\times \Delta[1] \to X$ fitting into commutative diagrams
\[
\begin{tikzcd}
X \arrow[d, "\delta_1"'] \arrow[r, "g"]            & Z \arrow[d, "j"] &  & {X\times \Delta[1]} \arrow[r, "h"] \arrow[d, "\pi_1"'] & X \arrow[d, "p"] \\
{X\times \Delta[1]} \arrow[r, "h"]            & X                &  & X \arrow[r, "p"']                                      & Y                \\
X \arrow[u, "\delta_0"] \arrow[ru, "\idd_X"'] &                  &  &                                                        &                 
\end{tikzcd}
.\] Now, consider any lifting problem
\[
\begin{tikzcd}
{\partial{\Delta[n]}} \arrow[d, "i_n"', hook] \arrow[r, "a"] & X \arrow[d, "g"] \\
\Delta[n] \arrow[r, "b"']                                     & Z               
\end{tikzcd}
\] between $i_n$ and $g$ with $n\geq 1$. It is easy to check that these two squares along with \eqref{eqn:minimal} yield additional commutative diagrams
\[
\begin{tikzcd}
{\partial{\Delta[n]}} \arrow[r, "a"] \arrow[d, "i_n"', hook] & X \arrow[r, "g"] & Z \arrow[r, "j"]                   & X \arrow[d, "p"] &  & {\partial{\Delta[n]}\times \Delta[1]} \arrow[d, "{i_n\times \idd_{\Delta[1]}}"', hook] \arrow[rr, "{a\times \idd_{\Delta[1]}}"] &  & {X\times \Delta[1]} \arrow[r, "h"]   & X \arrow[d, "p"] \\
{\Delta[n]} \arrow[rr, "b"']                                 &                  & Z \arrow[ru, "j"] \arrow[r, "p'"'] & Y                &  & {\Delta[n]\times \Delta[1]} \arrow[rr, "\pi_1"']                                                                                &  & {\Delta[n]} \arrow[r, "p' \circ b"'] & Y               
\end{tikzcd}
\]
\[
\Downarrow
\]
\[
\begin{tikzcd}[column sep=large]
{\left(\partial{\Delta[n]}\times \Delta[1]\right)\cup \left(\Delta[n]\times \left\{0\right\}\right)} \arrow[rr, "{\left(h \circ \left(a\times \idd_{\Delta[1]}\right), j\circ b \right)}"] \arrow[d, hook] &                                     & X \arrow[d, "p"] \\
{\Delta[n]\times \Delta[1]} \arrow[r, "\pi_1"']                                                                                                                                                     & {\Delta[n]} \arrow[r, "p'\circ b"'] & Y               
\end{tikzcd}.
\] Since $p$ is a Kan fibration, our last diagram admits a diagonal fill-in $H: \Delta[n] \times \Delta[1] \to X$ by \cref{chep}. Consider the $n$-simplex $v$ in $X$ given by
\[
\begin{tikzcd}
{\Delta[n]\times \Delta[0]} \arrow[r, "\delta_0"] & {\Delta[n]\times \Delta[1]} \arrow[r, "H"] & X
\end{tikzcd}.
\] Then the triangle
\[
\begin{tikzcd}
{\partial{\Delta[n]}} \arrow[d, "i_n"', hook] \arrow[r, "a"] & X \\
{\Delta[n]} \arrow[ru, "v"']                                 &  
\end{tikzcd}
\] commutes  since $h_1 = \idd_X$. Therefore, the composite $t \coloneqq g \circ h \circ \left(v \times \idd_{\Delta[1]}\right): \Delta[n]\times \Delta[1] \to Z$ is a homotopy $\underbrace{g \circ j \circ g \circ v}_{g\circ v} \overset{\simeq}{\longrightarrow} g\circ v$ such that $t\restriction_{\partial{\Delta[n]}\times \Delta[1]}= g \circ h \circ\left(a \times \idd_{\Delta[1]}\right)$. Note that $g\circ H$ and $t$ together determine a map $\left\langle g\circ H, t\right\rangle : \Delta[n]\times \Lambda^0[2] \to Z$, with 
\begin{align*}
\left\langle g\circ H, t\right\rangle\restriction_{\Delta[n] \times \partial^1{\Delta[2]}} &= g \circ H
\\ \left\langle g\circ H, t\right\rangle\restriction_{\Delta[n] \times \partial^2{\Delta[2]}} &= t.
\end{align*}
We now have a commutative diagram
\[
\begin{tikzcd}[column sep=large]
{\left(\partial{\Delta[n]}\times \Delta[2]\right)\cup \left(\Delta[n]\times \Lambda^0[2]\right)} \arrow[d, hook] \arrow[rr, "{\left(g\circ h\circ \left(a\times \sigma_1\right),\left\langle g\circ H, t\right\rangle\right)}"] &                                     & Z \arrow[d, "p'"] \\
{\Delta[n]\times \Delta[2]} \arrow[r, "\pi_1"']                                                                                                                                                                      & {\Delta[n]} \arrow[r, "p'\circ b"'] & Y               
\end{tikzcd}
,\] which admits a diagonal fill-in $H' : \Delta[n] \times \Delta[2] \to Z$ thanks to \cref{chep}.
This yields another commutative diagram
\[
\begin{tikzcd}[column sep=huge]
{\partial{\Delta[n]}\times \Delta[1]} \arrow[d, hook] \arrow[r, "\pi_1"]                                             & {\partial{\Delta[n]}} \arrow[d, "g\circ a"] \\
{\Delta[n]\times \Delta[1]} \arrow[d, "\pi_1"'] \arrow[r, "{H'\circ \left(\idd_{\Delta[n]} \times \delta_0\right)}"] & Z \arrow[d, "p'"]                           \\
{\Delta[n]} \arrow[r, "p'\circ b"']                                                                                  & Y                                          
\end{tikzcd}.
\] This shows that $g\circ v \simeq_{p'} b$. As $p'$ is minimal, it follows that  $g\circ v = b$, so that $v$ is a solution to our lifting problem. This completes our proof.
\end{proof}

\subsection*{Weak homotopy equivalences}

Recall from \cref{gext} the geometric realization functor $\left\lvert{-}\right\rvert : \sset \to  k\mathbf{Top}$. We say that a simplicial map $f: X \to Y$ is a \textit{weak homotopy equivalence} if $\left\lvert{f}\right\rvert : \left\lvert{X}\right\rvert \to \left\lvert{Y}\right\rvert$ is a homotopy equivalence of topological spaces. 

\begin{remark}
By the Whitehead theorem, $f$ is a weak homotopy equivalence in $\sset$ if and only if $\left\lvert{f}\right\rvert$ is a weak homotopy equivalence in  $k\mathbf{Top}$.
\end{remark}

\begin{definition}
A simplicial set $X$ is \textit{contractible} if the unique map $X \to 1$ is a weak homotopy equivalence.
\end{definition}

\begin{exmp}
Any horn $\Lambda^k[n]$ is contractible.
\end{exmp}

\smallskip

Suppose that $p: Y \to X$ is a map of simplicial sets. Let $F$ be a simplicial set such that for each vertex $x_0 \in X_0$,  the fiber 
\[
\begin{tikzcd}
{Y\times_X \Delta[0]} \arrow[d] \arrow[r] & {\Delta[0]} \arrow[d, "x_0"] \\
Y \arrow[r, "p"']                         & X                           
\end{tikzcd}
\] of $p$ over $x_0$ is isomorphic to $F$. In this case, we say that $p$ is a \textit{fiber bundle with standard fiber $F$}. 


\begin{theorem}\label{minisopull}
If $p$ is a minimal fibration  with two pullback squares
\[
\begin{tikzcd}
f_1^{\ast}{Y} \arrow[d] \arrow[r] & Y \arrow[d, "p"] &  & f_2^{\ast}{Y} \arrow[d] \arrow[r] & Y \arrow[d, "p"] \\
A \arrow[r, "f_1"']            & X                &  & A \arrow[r, "f_2"']            & X               
\end{tikzcd}
\] and a homotopy $f_0\overset{\simeq}{\longrightarrow} f_1$, then there is a commutative triangle of the form
\[
\begin{tikzcd}
f_1^{\ast}{Y} \arrow[r, "\cong"] \arrow[rd] & f_2^{\ast}{Y} \arrow[d] \\
                                            & A                      
\end{tikzcd}
.
\]
\end{theorem}
\begin{proof}[Proof reference]
See \cite[Corollary 10.7]{goerss}.
\end{proof}

\begin{corollary}
If $p$ is a minimal fibration and $X$ is connected (i.e., $\pi_0(X)=1$),  then it is a fiber bundle.
\end{corollary}
\begin{proof}
Suppose that $p$ is a minimal fibration and that $X$ is connected. Consider any two vertices $v_1, v_2 : \Delta[0] \to X$ in $X$. Since $X$ is connected, there is a $1$-simplex $z$ in $X$ whose boundary $\partial{z}$ equals $\left(v_2,v_1\right)$. Then $z$ is precisely a homotopy $v_1 \overset{\simeq}{\longrightarrow} v_2.$ By \cref{minisopull}, the fiber of $p$ over $v_1$ is thus isomorphic to that over $v_2$.
\end{proof}

\begin{theorem}\label{trivifibbund}
If $p$ is a minimal fibration and $X$ is contractible, then $p$ is \textit{trivializable}, i.e., isomorphic to the trivial bundle 
\[
\begin{tikzcd}
F\times X \arrow[d, "\pi_2"] \\
X                           
\end{tikzcd}
\] over $X$ with fiber $F$.\footnote{\cite[Corollary III.5.6]{Barr}.}
\end{theorem}

\begin{lemma}\label{trivbKF}
If $p$ is merely a Kan fibration, then the trivial bundle $F\times X \overset{\pi_2}{\longrightarrow} X$ is a Kan fibration. 
\end{lemma}
\begin{proof}
Note that the unique map $F \to \Delta[0]$ is precisely the pullback
\[
\begin{tikzcd}
F \arrow[r] \arrow[d]         \arrow[dr, phantom, "\scalebox{1.5}{\color{black}$\lrcorner$}" , very near start, color=black]
& 
Y \arrow[d, "p"] \\
{\Delta[0]} \arrow[r, "x_0"'] & X               
\end{tikzcd}
.\] 
Thus, $F$ is a Kan complex by \cref{CP} below. Moreover, $\pi_2$ is precisely the pullback
\[
\begin{tikzcd}
F\times X \arrow[d, "\pi_2"'] \arrow[r, "\pi_1"] 
\arrow[dr, phantom, "\scalebox{1.5}{\color{black}$\lrcorner$}" , very near start, color=black]
& F \arrow[d] \\
X \arrow[r]                                      & {\Delta[0]}
\end{tikzcd}
.\] Hence $\pi_2$ is a Kan fibration again by \cref{CP}. 
\end{proof}

\subsection{Quillen model categories}\label{modcat}

This section begins to develop the theory of model categories, a generalized setting for homotopy theory (whether simplicial or topological) originally developed by Quillen. This will provide the background for \cref{cmodsset}, which examines the classical model structure on $\sset$. This, in turn, will play a key role in  \cref{models}.

\smallskip

Our treatment of model categories is mainly based on \cite{Hirsch} and \cite{Hovey}.

\bigskip

\begin{definition}\label{WE}
We say that a category $\c$ is a \textit{category with weak equivalences} if it is equipped with a subclass $\we$ of $\mor(\c)$ consisting of \textit{weak equivalences} such that every isomorphism in $\c$ belongs to $\we$ and $\we$ satisfies \textit{two-out-of-three}, i.e., for any commutative triangle
\[
\begin{tikzcd}
                        & Y \arrow[rd] &   \\
X \arrow[ru] \arrow[rr] &              & Z
\end{tikzcd}
\] in $\c$, if two of these three morphisms are in $\we$, then so is the third.
\end{definition}

\begin{definition}[Model category]\label{MC} Let $\c$ be a category with all small limits and colimits.
\be
\item A \textit{weak factorization system (WFS) on $\c$} is a pair $\left(\L, \RI\right)$ of subclasses of $\mor(\c)$ such that 
\be[label=(\roman*)]
\item any morphism in $\c$ factors as a morphism in $\L$ followed by a morphism in $\RI$,
\item $\L$ consists of all \textit{$\RI$-projective} morphisms in $\c$, i.e., those with the left lifting property against every morphism in $\RI$, and
\item $\RI$ consists of all \textit{$\L$-injective} morphisms in $\c$, i.e., those with the right lifting property against every morphism in $\L$.
\ee
\item A category $\left(\c, \we\right)$ with weak equivalences is a \textit{model category} if $\c$ is equipped with two additional subclasses $\fib$ and $\cof$ of $\mor(\c)$ consisting of \textit{fibrations} and \textit{cofibrations}, respectively, such that the pairs
\begin{gather*}
\left(\overbrace{\we \cap \cof}^{\textit{trivial cofibrations}} , \fib  \right)
\\ \left(\cof, \underbrace{\we \cap \fib}_{\textit{trivial fibrations}}   \right)
\end{gather*}
 are both weak factorization systems on $\c$.
 
 In this case,  the triple $\left(\fib, \cof, \we\right)$ is called a \textit{model structure on $\c$}.
\ee
\end{definition}

\begin{remark}
For any WFS $\left(\L, \RI\right)$ on $\c$, both $\L$ and $\RI$ contain all isomorphisms in $\c$. Hence $\fib$, $\cof$, and $\we$ contain all isomorphisms in $\c$. Thus, the condition of \cref{WE} that $\we$ must contain all isomorphisms in $\c$ is superfluous for \cref{MC}.
\end{remark}

\smallskip

\begin{term} Let $\c$ be a model category and let $X\in \ob{\c}$.
\be
\item We call $X$ \textit{cofibrant} if the unique map $0\to X$ is a cofibration.
\item We call $X$ \textit{fibrant} if the unique map $X\to 1$ is a fibration.
\ee
\end{term}

\smallskip

\begin{definition}[Properness] Let $\left(\c, \fib, \cof, \we\right)$ be a model category.
\be
\item We say that $\c$ is \textit{right proper} if for any pullback square
\[
\begin{tikzcd}
X' \arrow[r]
\arrow[dr, phantom, "\scalebox{1.5}{\color{black}$\lrcorner$}" , very near start, color=black]
 \arrow[d, "w'"'] & X \arrow[d, "w"] \\
Y' \arrow[r, "f"']            & Y               
\end{tikzcd}
\] where $f\in \fib$ and $w \in \we$, we have $w' \in \we$. 
\item We say that $\c$ is \textit{left proper} if for any pushout square
\[
\begin{tikzcd}
X \arrow[r, "f"]
 \arrow[dr, phantom, "\scalebox{1.5}{\color{black}$\ulcorner$}", very near end, color=black]
 \arrow[d, "w"'] & X' \arrow[d, "w'"] \\
Y \arrow[r]            & Y'               
\end{tikzcd}
\] where $f\in \cof$ and $w \in \we$, we have $w' \in \we$. 
\ee
We say that $\c$ is \textit{proper} if it is both right proper and left proper.
\end{definition}

\medskip

\begin{lemma}\label{trivcofp} Suppose that both $\c$ and $\d$ are model categories. Let $\left(F: \c \to \d, G :\d \to \c \right)$ be an adjoint pair of functors.
\be[label=(\alph*)]
\item $F$ preserves cofibrations if and only if $G$ preserves trivial fibrations.
\item Dually, $G$ preserves fibrations if and only if $F$ preserves trivial cofibrations.
\ee 
\end{lemma}
\begin{proof}[Proof sketch] 
Suppose that $F$ preserves cofibrations. Let $f:A \to B$ be a cofibration in $\c$ and $g: C\to D$ be a trivial fibration in $\d$. Any lifting problem between $f$ and $G(g)$ induces a lifting problem between $F(f)$ and $g$ by adjunction.
\[
\begin{tikzcd}
A \arrow[r] \arrow[d, "f"'] & G(C) \arrow[d, "G(g)"] & \implies & F(A) \arrow[r] \arrow[d, "F(f)"'] & C \arrow[d, "g"] \\
B \arrow[r]                 & G(D)                   & \implies & F(B) \arrow[r]                    & D               
\end{tikzcd}
\] Since $g$ is a trivial fibration, there is a solution $\hat{g} : F(B) \to C$ to our righthand lifting problem. The conjugate of $\hat{g}$ under our adjunction is a solution to our lefthand lifting problem. This proves that $G(g)$ has the right lifting property against every cofibration in $\c$, i.e., that $G(g)$ is a trivial fibration in $\c$.

\smallskip 

A similar argument proves that if $G$ preserves trivial fibrations, then $F$ preserves cofibrations. 
\end{proof}

\medskip

\begin{lemma}\label{CP}
Let $\left(\L, \RI\right)$ be a WFS on $\c$. Then $\L$ is closed under pushouts. Dually, $\RI$ is closed under pullbacks.
\end{lemma}
\begin{proof}[Proof sketch]
Consider any pushout square
\[
\begin{tikzcd}
A \arrow[d, "x"'] \arrow[r, "y"] & C \arrow[d, "y_{\ast}{x}"] \\
B \arrow[r]                      & B\cup_A C              
\end{tikzcd}
\] where $x\in \L$.  Let 
\[
\begin{tikzcd}
C \arrow[d, "y_{\ast}{x}"'] \arrow[r, "t"] & X \arrow[d, "f"] \\
B\cup_A C \arrow[r]                & Y               
\end{tikzcd}
\] be a commutative diagram with $f\in \RI$. We must find a lift $B\cup_A C \to X$. To this end, note that the commutative diagram
\[
\begin{tikzcd}
A \arrow[r, "y"] \arrow[d, "x"'] & C \arrow[r, "t"]         & X \arrow[d, "f"] \\
B \arrow[r]                      & B\cup_A C \arrow[r] & Y               
\end{tikzcd}
\] admits a lift  $s: B \to X$ because $x\in \L$ and $f\in \RI$. By the universal property of pushout squares, the pair $\left(s, t\right)$ induces a unique mediating map $B\cup_A C \to X$, which is a solution to our lifting problem between $y_{\ast}{x}$ and $f$.
\end{proof}

\begin{lemma}[Ken Brown]\label{KB}
Suppose that $\left(\c, \fib, \cof, \we\right)$ is a model category and $\left(\d, \we\right)$ is a category with weak equivalences. Let $F: \c \to \d$ be a functor sending any trivial fibration  of fibrant objects to a weak equivalence. Then $F$ sends any weak equivalence of fibrant objects to a weak equivalence. 
\end{lemma}
\begin{proof}
Let $f : A \to B$ be a weak equivalence of fibrant objects in $\c$. The pullback square
\[
\begin{tikzcd}
A\times B \arrow[d, "\pi_1"'] \arrow[r, "\pi_2"] 
\arrow[dr, phantom, "\scalebox{1.5}{\color{black}$\lrcorner$}" , very near start, color=black]& B \arrow[d] \\
A \arrow[r]                                      & 1          
\end{tikzcd}
\] exhibits the projections $\pi_1$ and $\pi_2$ as fibrations in $\c$ by \cref{CP}. Now, factor the  map $h\coloneqq \left(\idd_A, f\right) : A \to A \times B$ as
\[
\begin{tikzcd}
A \arrow[r, "p_1"'] \arrow[rr, "h", bend left] & C \arrow[r, "p_2"'] & {A\times B}
\end{tikzcd}
\] where $p_1 \in \we \cap \cof$ and $p_2\in \fib$. Both $\pi_1 \circ p_2$ and $\pi_2 \circ p_2$ are weak equivalences by two-out-of-three. They are also fibrations since $\fib$ is closed under composition. For the same reason, $C$ is fibrant. Therefore, both $\pi_1 \circ p_2$ and $\pi_2 \circ p_2$ are trivial fibrations of fibrant objects. By hypothesis, it follows that both $F(\pi_1 \circ p_2)$ and $F(\pi_2 \circ p_2)$ are weak equivalences in $\d$. As $$F(\pi_1 \circ p_2 \circ p_1) = F(\idd_A) = \idd_{F(A)}$$ is  a weak equivalence in $\d$ as well, we have that $F(p_1)$ is a weak equivalence by two-out-of-three. Applying two-out-of-three yet again shows that $$F(\pi_2 \circ p_2\circ p_1) =F(f)$$ is a weak equivalence, as desired.  
\end{proof}

\smallskip

Let  $I$ denote the \textit{interval category} $\left\{0 \to 1\right\}$. The functor category $\c^I \coloneqq \left[I, \c\right]$ is isomorphic to the arrow category $\Ar(\c)$ of $\c$.

\begin{lemma}\label{CC}
Let $\left(\L, \RI\right)$ be a WFS on $\c$. Then $\L$ is closed under coproducts in $\c^I$. Dually, $\RI$ is closed under products in $\c^I$.
\end{lemma}
\begin{proof}
Let $\left\{f_s : A_s \to B_s\right\}_{s\in S}$ be any set of elements of $\L$. As colimits in $\c^I$ are computed pointwise, their coproduct is precisely the map
\[
\begin{tikzcd}[column sep=large]
\coprod_{s\in S}A_s \arrow[r, "\left(f_s\right)_{s\in S}"] & \coprod_{s\in S}B_s
\end{tikzcd}
\] induced by the universal property of coproducts. Let
\[
\begin{tikzcd}
\coprod_{s\in S}A_s \arrow[r] \arrow[d, "\left(f_s\right)_{s\in S}"'] & X \arrow[d, "p"] \\
\coprod_{s\in S}B_s \arrow[r]                                         & Y               
\end{tikzcd}
\] be  any lifting problem such that $p\in \RI$. By the universal property of coproducts, this naturally corresponds to the set
\[
\left\{
\begin{tikzcd}
A_s \arrow[r] \arrow[d, "f_s"'] & X \arrow[d, "p"] \\
B_s \arrow[r]                                         & Y               
\end{tikzcd} \mid s\in S
\right\}
\] of lifting problems, each of which has  a solution $\ell_s: B_s \to X$ by hypothesis. Again, by the universal property of coproducts, the induced map $\left(\ell_s\right)_{s\in S}$ is a solution to our original lifting problem. Hence $\left(f_s\right)_{s\in S}$ belongs to $\L$.
\end{proof}

\medskip

Consider a commutative diagram of the form
\[ \label{eqn:ret}
\begin{tikzcd}
A \arrow[d, "f"'] \arrow[r] \arrow[rr, "\idd_A", bend left] & C \arrow[d, "g"] \arrow[r] & A \arrow[d, "f"] \\
B \arrow[r] \arrow[rr, "\idd_B"', bend right]               & D \arrow[r]                & B               
\end{tikzcd} \tag{$\dagger$}
\] in a category $\c$. In this situation, we say that $f$ is a \textit{retract of $g$}. This corresponds to a retraction (i.e., left-inverse) in the arrow category $\c^I$.

\smallskip

\begin{notation} Let $J$ be a subclass of $\mor(\c)$.
\bi
\item $\ret(J)$ will denote the class of all retracts of elements in $J$.
\item $\rlp(J)$ will denote the class of all maps in $\c$ with the right lifting property against every element of $J$.
\item $\llp(J)$ will denote the class of all maps in $\c$ with the left lifting property against every element of $J$.
\ei
\end{notation}

\smallskip

\begin{lemma}[Retract argument]\label{RA}
Consider any composite $h \equiv f \circ g$ of maps in $\c$.
\be[label=(\alph*)]
\item If $h$ has the left lifting property against $f$, then $h$ is a retract of $g$.
\item Dually, if $h$ has the right lifting property against $g$, then $h$ is a retract of $f$.
\ee
\end{lemma}
\begin{proof}
Suppose that $h$ has the left lifting property against $f$. We have a lifting problem of the form
\[
\begin{tikzcd}
A \arrow[d, "h"'] \arrow[r, "g"] & C \arrow[d, "f"] \\
B \arrow[r, equals]                      & B               
,\end{tikzcd}
\] which has a solution $t: B \to C$ by hypothesis.  As a result, we get a commutative diagram
\[
\begin{tikzcd}
A \arrow[r, equals] \arrow[d, "h"']                        & A \arrow[r, equals] \arrow[d, "g"] & A \arrow[d, "h"] \\
B \arrow[r, "t"] \arrow[rr, "\idd_B"', bend right] & C \arrow[r, "f"]           & B               
\end{tikzcd}
,\] which means that $h$ is a retract of $g$. 
\end{proof}

\begin{lemma}\label{cloret}
Let $\left(\L, \RI\right)$ be a WFS on $\c$. Then both $\L$ and $\RI$ are closed under retracts. 
\end{lemma}
\begin{proof}
For simplicity, let us just show that $\L$ is closed under retracts. To this end, let $f$ and $g$ be as in \eqref{eqn:ret} and suppose that $g\in \L$.  We must show that any commutative square of the form
\[
\begin{tikzcd}
A \arrow[d, "f"'] \arrow[r] & X \arrow[d, "j"] \\
B \arrow[r]                 & Y               
\end{tikzcd}
\] where $j\in \RI$ has a lift $B \to X$. The commutative diagram 
\[
\begin{tikzcd}
C \arrow[d, "g"] \arrow[r] & A \arrow[d, "f"] \arrow[r] & X \arrow[d, "j"] \\
D \arrow[r]                & B \arrow[r]                & Y               
\end{tikzcd}
\] must have a lift $t: D\to X$ since $g\in \L$ by assumption. Therefore, the composite $$A \overset{f}{\longrightarrow} B \longrightarrow D \overset{t}{\longrightarrow} X$$ is a lift for the expanded commutative diagram
\[
\begin{tikzcd}
A \arrow[d, "f"'] \arrow[r] & C \arrow[d, "g"] \arrow[r] & A \arrow[d, "f"] \arrow[r] & X \arrow[d, "j"] \\
B \arrow[r]                 & D \arrow[r]                & B \arrow[r]                & Y               
\end{tikzcd}.
\] It follows easily that the same composite is a lift for our original square. This means that $f\in \L$, as desired.
\end{proof}

\begin{corollary}\label{CR}
Let  $\left(\c, \fib, \cof, \we\right)$ be a model category. The classes $\fib$, $\cof$, $\we \cap \fib$, and $\we \cap \cof$  are closed under retracts.
\end{corollary}

\begin{corollary}
The class $\we$ is closed under retracts.\footnote{\cite[Lemma 2.4]{Riehl2}.}
\end{corollary}
\begin{proof}
 Let $f$ and $g$ be as in \eqref{eqn:ret} and suppose that $g\in \we$. We must show that $f\in \we$. 
 
 \begin{steps}
 \item Assume that $f$  is a fibration. We can factor $g$ as a cofibration $g_1 : C \to D'$ followed by a trivial fibration $g_2: D' \to D$. As $\we$ satisfies two-out-of-three, $g_1$ is actually a trivial cofibration. Thus, the commutative diagram
 \[
\begin{tikzcd}
C \arrow[d, "g_1"'] \arrow[rr] &             & A \arrow[d, "f"] \\
D' \arrow[r, "g_2"']           & D \arrow[r] & B               
\end{tikzcd}
 \]
 admits a lift $\ell : D' \to A$, which fits into the extended commutative diagram
 \[
\begin{tikzcd}
A \arrow[d, equals] \arrow[r]       & C \arrow[d, "g_1"] \arrow[r]          & A \arrow[d, equals]      \\
A \arrow[d, "f"'] \arrow[r] & D' \arrow[d, "g_2"] \arrow[r, "\ell"] & A \arrow[d, "f"] \\
B \arrow[r]                 & D \arrow[r]                           & B               
\end{tikzcd}
 .\] This implies that the composite $A \longrightarrow D' \overset{\ell}{\longrightarrow} A$ equals the identity map $\idd_A$, so that $f$ is a retract of $g_2$. By \cref{CR}, $f$ is a trivial fibration, hence a weak equivalence.
 \item Let $f$ be arbitrary. We can factor $f$ as a trivial cofibration $f_1 : A \to B'$ followed by a fibration $f_2: B' \to B$. By the universal property of pushout squares, we obtain a unique map $d$ such that the diagram
 \[ \label{eqn:pb1}
\begin{tikzcd}[row sep=large]
A \arrow[d, "f_1"'] \arrow[r]   
 \arrow[dr, phantom, "\scalebox{1.5}{\color{black}$\ulcorner$}", very near end, color=black]
        & C \arrow[d, "c_1"'] \arrow[r] \arrow[dd, "g", bend left] & A \arrow[d, "f_1"]  \\
B' \arrow[d, "f_2"'] \arrow[r, "c_2"'] & E \arrow[d, "d"', dashed]                                & B' \arrow[d, "f_2"] \\
B \arrow[r]                            & D \arrow[r]                                              & B                  
\end{tikzcd} \tag{1}
 \] commutes. Likewise, there is a unique map $y : E \to B'$ such that the diagram
 \[ \label{eqn:pb2}
\begin{tikzcd}
A \arrow[r] \arrow[d, "f_1"']     
 \arrow[dr, phantom, "\scalebox{1.5}{\color{black}$\ulcorner$}", very near end, color=black]
                        & C \arrow[r] \arrow[d, "c_1"] & A \arrow[d, "f_1"] \\
B' \arrow[r, "c_2"] \arrow[rr, "\idd_{B'}"', bend right] & E \arrow[r, "y", dashed]      & B'                
\end{tikzcd} \tag{2}
\] commutes. But $f_2 \circ y \circ c_1$ equals the composite $C \longrightarrow A \overset{f_1}{\longrightarrow} B' \overset{f_2}{\longrightarrow} B$, and $f_2 \circ y \circ c_2$ equals the composite $B' \overset{f_2}{\longrightarrow} B \longrightarrow D \longrightarrow B$. Hence $ f_2 \circ y$ is a mediating map for our pullback square. Similarly, one can check that the composite $E \overset{d}{\longrightarrow} D \longrightarrow B$ is a mediating map. But such a map is unique by the universal property of pushout squares.  Therefore, $y$ fits into \eqref{eqn:pb1}, so that $f_1$ is a retract of $d$. 
 
 Note that $c_1$ is a weak equivalence thanks to \cref{CP}. Hence $d$ is one as well by the two-out-of-three property. By Step 1, we see that $f_2$ is also a weak equivalence. Thus, the composite $f= f_2 \circ f_1$ is a weak equivalence  by the two-out-of-three property.
 \end{steps}
\end{proof}

\medskip

It will be useful to extend our ordinary notion of composition to the transfinite case. Let  $\c$ be a cocomplete, locally small category. Let $\left(\alpha, \in\right)$ be any ordinal viewed as an order category and let  $J$ be any subclass of $\mor(\c)$.
\begin{definition}\label{TrC}
An \textit{$\alpha$-sequence of maps in $J$} is an $\alpha$-shaped diagram $F : \alpha \to \c$ such that 
\be[label=(\roman*)]
\item $F$ sends the successor morphism $\beta \overset{\in}{\longrightarrow} \beta+1$ to a map in $J$ for each $\beta +1 \in \alpha$ and 
\item for any limit ordinal $\gamma \in \alpha$, $F_{\gamma}$ together with the family of induced maps $\left\{F_{\beta}\to F_{\gamma} \mid \beta \in \gamma\right\}$ is the colimiting cocone under $F_{\bullet}$ restricted to the full-subdiagram $\left\{\beta \mid \beta \in \gamma\right\}$; in short, we have an isomorphism
\[
\colim_{\beta \in \gamma}F_{\beta} \overset{\cong}{\longrightarrow} F_{\gamma} .
\]
\ee
The \textit{(transfinite) composition} of such a sequence is the induced map $F_0 \to F_{\alpha}\coloneqq \colim_{\beta \in \alpha}F_{\beta}$.
\end{definition}

\begin{note}\label{transf}
Let $\left(\L, \RI\right)$ be a WFS on $\c$. Suppose that $p_1, p_2 \in \L$ and $q\in \RI$. Consider the lifting problem
\[
\begin{tikzcd}
A \arrow[d, "p_1"'] \arrow[r] & X \arrow[dd, "q"] \\
B \arrow[d, "p_2"']           &                   \\
C \arrow[r, "y"']                   & Y                
\end{tikzcd}
.\] We have a lift
\[
\begin{tikzcd}[column sep=large]
A \arrow[d, "p_1"'] \arrow[r]                          & X \arrow[dd, "q"] \\
B \arrow[d, "p_2"'] \arrow[rd, "y\circ p_2"] \arrow[ru, "x", dashed] &                   \\
C \arrow[r, "y"']                                            & Y                
\end{tikzcd}
\] because $p_1 \in \L$. Since $p_2 \in \L$, this yields a solution $C \to X$ to the lifting problem $\left(x,y\right)$. This is a solution to our original lifting problem. This proves that $\L$ (dually, $\RI$) is closed under finite composition. By the universal property of colimits, it follows easily that $\L$ is closed under transfinite compositions as well.
\end{note}

\begin{remark}\label{Satd}
\Cref{transf}, \cref{cloret}, and \cref{CP} together show that for any class $M$ of maps in $\c$, the class of all maps in $\c$ with the left lifting property against $M$ is saturated.
\end{remark}

\begin{prop}\label{coppush}
Consider  a set $\left\{g_s\right\}_{s\in S}$ of objects in the arrow category $\Ar(\c)$. The coproduct $\coprod_{s\in S}g_s$ arises as a transfinite composite  of pushouts of the $g_s$.\footnote{\cite[Proposition 10.2.7]{Hirsch}.}
\end{prop}

\begin{definition}
The class $\cell(J)$ of \textit{relative $J$-cell complexes} consists of all maps in $\c$ arising as transfinite composites of pushouts of elements of $J$.
\end{definition}


\begin{exmp}
Let $\c = \mathbf{Top}$, which is cocomplete because it has all coproducts and coequalizers, and let $J$ consist of all Hurewicz cofibrations, including all inclusions of subcomplexes into CW-complexes.  Let $X_{{-1}} = \emptyset$ and consider any CW-complex $Y\coloneqq \bigcup_{n\geq {-1}}X_n$ with attaching maps $\left\{\varphi_{\alpha} : S^{n-1} \to X_{n-1} \mid \alpha \in A_n\right\}$. Then we have a pushout square
\[
\begin{tikzcd}[column sep=huge]
\coprod_{\alpha \in A_n}S^{n-1} \arrow[d, hook]
 \arrow[dr, phantom, "\scalebox{1.5}{\color{black}$\ulcorner$}", very near end, color=black]
  \arrow[r, "\left(\varphi_{\alpha}\right)_{\alpha \in A_n}"] & X_{n-1} \arrow[d, hook] \\
\coprod_{\alpha \in A_n}D^n \arrow[r]                                                                         & X_n                    
\end{tikzcd}
\] for each $n\geq 0$, and $Y$ is precisely the colimit of the induced diagram
\[
\begin{tikzcd}
X_{{-1}} \arrow[r, hook] & X_0 \arrow[r, hook] & X_1 \arrow[r, hook] & X_2 \arrow[r, hook] & \cdots
\end{tikzcd}.
\] This means that the map $\emptyset \hookrightarrow Y$ is a relative $J$-complex. In this case, we call $Y$ simply a \textit{$J$-cell complex}, thereby recovering our usual notion of a cell complex. 
\end{exmp}

\smallskip

Suppose that $\kappa$ is a cardinal and $F: \kappa \to \c$ is a functor. The colimit of such a functor is called a \textit{$\kappa$-sequential colimit}. Further, if $F$ satisfies condition (i) of \cref{TrC}, then its colimit is called a \textit{$\kappa$-sequential colimit relative to $J$}.

\medskip

More generally, if $\kappa$ is regular and $A$ is a poset such that any subset $B \subset A$ with cardinality $< \kappa$ has an upper bound in $A$, then we call $A$ a \textit{$\kappa$-directed set} and the colimit of any $A$-shaped diagram $D$ a \textit{$\kappa$-directed colimit}. In this case, if every morphism in $D$ belongs to $J$, then we call such a colimit a \textit{$\kappa$-directed colimit relative to $J$}.

\begin{definition} $ $
\be
\item An object $X$ in $\c$ is \textit{$\kappa$-compact relative to $J$} if for any regular cardinal $\lambda \geq \kappa$, the covariant functor $\Hom_{\c}(X, {-})$ preserves $\lambda$-directed colimits relative to $J$, specifically, the set map
\begin{gather*}
\colim_{\beta \in \lambda}\Hom_{\c}(X, F_{\beta}) \overset{\cong}{\longrightarrow} \Hom_{\c}\left(X, \colim_{\beta \in \lambda}F_{\beta}\right)
\\ \left[f_{\beta} : X \to F_{\beta}\right] \mapsto \left( X \overset{f_{\beta}}{\longrightarrow} F_{\beta} \to   \colim_{\beta\in \lambda}F_{\beta}  \right)
\end{gather*} is an isomorphism.

We say that $X$ is \textit{small relative to $J$} if it is $\kappa$-compact relative to $J$ for some cardinal $\kappa$.

We say that $X$ is \textit{small} if it is small relative to $\mor(\c)$.

\item We say that $\c$ is \textit{locally presentable} if there is a regular cardinal $\lambda$ along with a set $S$ of $\lambda$-compact objects of $\c$ such that every object in $\c$ arises as a $\lambda$-directed colimit of a diagram with objects in $S$.
\ee
\end{definition}

\smallskip

Note that if $X$ is $\kappa$-compact relative to $J$, then any map $X\to \colim_{\beta \in \lambda}F_{\beta}$ factors through one of the maps $i_{\beta}$ 
\[
\begin{tikzcd}
X \arrow[d, dashed] \arrow[r] &  \colim_{\beta\in \lambda}F_{\beta} \\
F_{\beta} \arrow[ru, "i_{\beta}"']          &                                    
\end{tikzcd}
\] of the colimiting cocone so that any other such map $X \to F_{\beta'}$ with $\beta < \beta'$ is precisely the composite $X \to F_{\beta} \to F_{\beta'}$.

\smallskip


\begin{lemma}\label{finitesimp} 
 Let $\kappa$ be any infinite regular cardinal. Then any finite simplicial set (\cref{finsimp}) is $\kappa$-compact.\footnote{\cite[Lemma 3.1.2]{Hovey}.}
\end{lemma}
\begin{proof}
Let $K$ be a finite simplicial set and let $D: \kappa \to \sset$ be a $\kappa$-sequence. We must show that the canonical set map
\[
\colim_{\beta \in \kappa}\Hom_{\sset}(K, D_{\beta})  \to  \Hom_{\sset}\left(K, \colim_{\beta \in \kappa}D_{\beta}\right) 
\] is bijective. 

\medskip

To see that it is injective, suppose that we have equal simplicial maps
\begin{align*} 
&K \overset{f_1}{\longrightarrow} D_{\beta} \longrightarrow  \colim_{\beta \in \kappa}D_{\beta} 
\\ &K \overset{f_2}{\longrightarrow} D_{\beta} \longrightarrow  \colim_{\beta \in \kappa}D_{\beta} .
\end{align*}
Since $K$ is finite, there is some ordinal $\beta < \alpha < \kappa$ such that $$D(\beta \to \alpha)\circ f_1\restriction_{\Delta{K}_{\nondeg}} = D(\beta \to \alpha)\circ f_2\restriction_{\Delta{K}_{\nondeg}}$$ where $\Delta{K}_{\nondeg}$ denotes the category of non-degenerate simplices $\Delta[n]\to K$ in $K$. By \cref{E-Z}, it follows easily that 
$$D(\beta \to \alpha)\circ f_1\restriction_{\Delta{K}_{\nondeg}} = D(\beta \to \alpha)\circ f_2\restriction_{\Delta{K}_{\nondeg}},$$ so that $\left[f_1\right] = \left[f_2\right]$ in $\colim_{\beta \in \kappa}\Hom_{\sset}(K, D_{\beta})$, as desired.

\medskip

To see that our canonical function is surjective, consider any simplicial map $g :K \to \colim_{\beta\in \kappa}D_{\beta}$. For every integer $n\geq 0$ and every non-degenerate $n$-simplex $x$ in $K$, there exists an ordinal $\alpha_x \in \kappa$ along with an $n$-simplex $y_x$ in $D_{\alpha_x}$ such that $g_n(x) = \left(i_{\alpha_x}\right)_n(y_x)$. Since $K$ is finite, we may take $$\mu \coloneqq \max\{\alpha_z \mid z \text{ is a non-degenerate simplex in }K\}$$ to form a levelwise \emph{set} map $h : \Delta{K}_{\nondeg} \to D_{\mu}$ such that  $i_{\mu} \circ h =g\restriction_{\Delta{K}_{\nondeg}}$. Using \cref{E-Z}, we can extend $h$ to a levelwise set map $h' : K \to D_{\mu}$ that both commutes with all degeneracy operators and satisfies $i_{\mu} \circ h' =g$.

\medskip

Now, let $x$ be a non-degenerate $n$-simplex $x$ in $K$. For any integer $0\leq i \leq n+1$, we have that
\[
i_{\mu}\left(h'_{n-1}{d_i{x}}\right) = g_{n-1}{d_i{x}} = d_i{g_n{x}} = d_i{i_{\mu}{h'_n}{x}} = i_{\mu}\left(d_i{h'_n{x}}\right)
\]
Hence $\left[d_i{h'_n{x}}\right] = \left[h'_{n-1}{d_i}{x}\right]$ in the quotient set $\colim_{\beta \in \kappa}\left(D_{\beta}\right)_n$. This means that 
\[
D(\mu \to \alpha(x,i))_n(d_i{h'_n{x}}) = D(\mu \to \alpha(x,i))_n(h'_{n-1}{d_i}{x}) 
\]
  for some ordinal $\mu < \alpha(x,i) < \kappa$.  As $K$ is finite, we can take $$\mu' \coloneqq \max\left\{\alpha(x,i) \mid x \text{ is a non-degnerate simplex in }K, \ 0\leq i \leq n+1\right\}$$ to form a levelwise set map $h'' : K \to X_{\mu'}$ such that 
\be[label=(\alph*)]
\item $i_{\mu'} \circ h'' = g$,
\item $h''$ commutes with all degeneracy operators, and 
\item  $h''\restriction_{\Delta{K}_{\nondeg}}$ commutes with all face operators.
\ee
By applying \cref{E-Z} to (c), we see that $h''$ commutes  with all face operators as well. Thanks to the simplicial identities, this means that $h''$ is a simplicial map. By condition (a), this completes our proof.

\end{proof}

\medskip


\begin{lemma}\label{repsmall} Let $\kappa$ be a regular cardinal and $A$ be a $\kappa$-directed set. Let $\d$ be a small category. Let $\b$ be a cocomplete, locally small category and $\e$ be a finite category.
\be[label=(\alph*)]
\item Any representable functor $\Hom_{\d}({-}, C) : \d^{\op} \to \set$ is $\kappa$-compact.
\item Let $F: \e \to \b$ be a diagram where $F(e)$ is a $\kappa$-compact object for each $e\in \ob{\e}$. Then $\colim_eF_e$ is also a $\kappa$-compact object.
\ee
\end{lemma}
\begin{proof} $ $
\be[label=(\alph*)]
\item   Let $D: A \to \left[\d^{\op}, \set\right]$ be any diagram. Using the Yoneda lemma twice, we obtain a chain of natural isomorphsisms
\begin{align*}
\Hom_{\widehat{\d}}\left(\Hom_{\d}({-}, X), \colim_{a}{D_a}\right) & \cong \left(\colim_{a}{D_a}\right)(X)
\\ & \cong \colim_{a}D_a(X)
\\ & \cong \colim_{a}{\Hom_{\widehat{\d}}\left(\Hom_{\d}({-}, X), D_a\right)}.
\end{align*}
\item Let $D' : A \to \b$ be any diagram. It is well-known that $\kappa$-directed colimits commute with  finite limits in $\set$. Since the bifunctor $\Hom_{\b}({-}, {-})$ is continuous in each variable, it follows that 
\begin{align*}
\Hom_{\b}\left(\colim_{e}F_e, \colim_{a}D'_a\right) & \cong \lim_{e}{\Hom_{\b}\left( F_e, \colim_{a}D_a'\right)}
\\ & \cong \lim_e\colim_a{\Hom_{\b}(F_e, D_a')}
\\ & \cong \colim_a\lim_e{\Hom_{\b}(F_e, D_a')}
\\ & \cong \colim_a{\Hom_{\b}\left(\colim_eF_e, D_a'\right)}
.\end{align*}
\ee
\end{proof}

\begin{corollary}\label{PSLP}
If $\d$ is a small category, then the presheaf category $\left[\d^{\op}, \set\right]$ is locally presentable.
\end{corollary}
\begin{proof}[Proof sketch]
Thanks to \cref{psclr}, it suffices to show that any  small colimit of representable presheaves $\d^{\op} \to \set$ arises as an $\aleph_0$-directed colimit of a diagram with objects in a set $S$ of $\aleph_0$-compact objects of $\left[\d^{\op}, \set\right]$. To this end, take $S$ to be the full subcategory $\widehat{\d}_{\fp}$ of $\left[\d^{\op}, \set\right]$ on the class of all $\aleph_0$-compact objects.

\medskip

Let $F : \mathscr{I} \to \left[\d^{\op}, \set\right]$ be any small diagram of representable presheaves. Consider the poset $\left(\mathcal{F}, \subset\right)$ of all finite full subcategories of $\mathscr{I}$. Note that $\mathcal{F}$ is an $\aleph_0$-directed set. It can be shown that 
\[ \label{eqn:filcol}
\colim{F} = \colim_{T\in \mathcal{F}}\colim_{t\in T}F_t .
\] By \cref{repsmall}, $\colim_{t}F_t$ belongs to $\widehat{\d}_{\fp}$ for every $T\in \mathcal{F}$. 

\medskip

It remains to show that $\widehat{\d}_{\fp}$ is small.
Since $\d$ is small by hypothesis, there are only small many representable presheaves $\d^{\op} \to \set$. Therefore, there are only small many finite colimits of representable presheaves. Let $X : \d^{\op} \to \set$ be any $\aleph_0$-compact presheaf. It can be shown that $X$ arises as an $\aleph_0$-directed colimit of finite colimits of representable presheaves: $X \overset{\cong}{\longrightarrow} \colim_{a\in A}G_a$. As $X$ is $\aleph_0$-compact, this yields a commutative triangle
\[
\begin{tikzcd}
X \arrow[r, "\cong"] \arrow[d, dashed] &  \colim_{a}G_a \\
G_a \arrow[ru, hook]                   &                    
\end{tikzcd}
\] for some $a\in A$. Here, the map $X \to G_{a}$ must be monic, so that  $X$ is a \textit{subpresheaf} of $G_a$. This shows that all $\aleph_0$-compact presheaves are  subpresheaves of finite colimits of representable presheaves. This completes our proof.
\end{proof}

\smallskip

\begin{lemma}[Quillen's small object argument]\label{SOA}
Suppose that $J$ \emph{permits the small object argument}, i.e., for any $f\in J$, the object $\dom(f)$ is small relative to $\cell(J)$. Then any map in $\c$ can be factored as a map in $\cell(J)$ followed by a map in $\rlp(J)$.\footnote{\cite[Proposition 10.5.16]{Hirsch}.}
\end{lemma}

\begin{proof}
Let $z:X_0\to Y$ be a map in $\c$. Consider the set 
\[
L_0 \coloneqq \left\{
\begin{tikzcd}
A_f \arrow[d, "f"'] \arrow[r] & X_0 \arrow[d, "z"] \\
B_f \arrow[r]                 & Y               
\end{tikzcd}
\mid f\in J \right\}
\] of lifting problems in $\c$. The universal property of coproducts yields a map $\coprod_{f\in J}f \to z$ in the arrow category $\Ar(\c)$. In particular, we have universal morphisms $\coprod_{f\in J}A_f \to X_0$ and $\coprod_{f\in J}B_f \to Y$. Form the pushout square
\[ \label{eqn:coppushh}
\begin{tikzcd}
\coprod_{f\in J}A_f \arrow[d, "\left(f\right)_{f\in J}"']
 \arrow[dr, phantom, "\scalebox{1.5}{\color{black}$\ulcorner$}", very near end, color=black]
  \arrow[r] & X_0 \arrow[d, "y_1"] \\
\coprod_{f\in J}B_f \arrow[r]                                         & X_1          
\end{tikzcd} \tag{A}
\] together with the unique mediating map
\[ \label{eqn:medi}
\begin{tikzcd}
                                                      & X_0 \arrow[d, "y_1"] \arrow[rdd, "p_0\coloneqq z", bend left] &   \\
\coprod_{f\in J}B_f \arrow[r] \arrow[rrd, bend right] & X_1 \arrow[rd, "p_1", dashed]             &   \\
                                                      &                                           & Y
\end{tikzcd}.
\tag{B} \] Note that $y_1 \in \cell(J)$ by \cref{coppush}. In general, for any successor ordinal $\beta \equiv \alpha+1$, suppose that we have constructed a triple $$\left(y_{\beta} : X_{\alpha} \to X_{\beta}, p_{\alpha} : X_{\alpha} \to Y, p_{\beta}:X_{\beta} \to Y\right)$$ of maps fitting into a diagram like \eqref{eqn:medi}. Then repeat our  construction of $\left(y_1,  p_0, p_1\right)$ with $p_0$ replaced by $p_{\beta}$ to obtain a new triple $\left(y_{\beta+1}, p_{\beta}, p_{\beta+1}\right)$ of maps fitting into \eqref{eqn:medi}. Now, choose the least regular cardinal $\kappa$ such that for any $f\in J$, $\dom(f)$ is $\kappa$-compact relative to $\cell(J)$. For any limit ordinal $\gamma$, suppose that we have constructed a cocone under a $\gamma$-sequence
\[
\begin{tikzcd}
X_0 \arrow[r, "y_1"] \arrow[rrd, "p_0"'] & X_1 \arrow[r] \arrow[rd, "p_1"] & \cdots \arrow[r] & X_{\alpha} \arrow[r, "y_{\gamma+1}"] \arrow[ld, "p_{\alpha}"'] & X_{\alpha+1} \arrow[r] \arrow[lld, "p_{\alpha +1}"] & \cdots \\
                                         &                                 & Y                &                                                                &                                                     &       
\end{tikzcd}
\] of maps in $\cell(J)$. Take the  transfinite composition $y:X_0\to X_{\gamma}$ of this  sequence. By transfinite induction, we  now have defined a $\kappa$-sequence
\[
\begin{tikzcd}
\label{eqn:lseq} X_0 \arrow[r, "y_1"] & X_1 \arrow[r] & \cdots \arrow[r] & X_{\gamma} \arrow[r, "y_{\alpha+1}"] & X_{\gamma+1} \arrow[r] & \cdots
\end{tikzcd} \tag{C}
\] of maps in $\cell(J)$. The universal property of colimits yields a map $p_{\gamma} : X_{\gamma} \to Y$ fitting into the cocone 
\[
\begin{tikzcd}
X_0 \arrow[r, "y_1"] \arrow[rrd, "p_0"'] & X_1 \arrow[r] \arrow[rd, "p_1"] & \cdots \arrow[r] & X_{\gamma} \arrow[r, "y_{\alpha+1}"] \arrow[ld, "p_{\gamma}"'] & X_{\gamma+1} \arrow[r] \arrow[lld, "p_{\gamma +1}"] & \cdots \\
                                         &                                 & Y                &                                                                &                                                     &       
\end{tikzcd}
.\] Now, take the transfinite composition $x : X_0 \to X_{\kappa}$ of \eqref{eqn:lseq}. We see that the diagram
\[
\begin{tikzcd}
X_0 \arrow[d, "y_1"'] \arrow[rd, "x"]          &                                    &   \\
X_{1} \arrow[r] \arrow[rr, "p_1"', bend right] & X_{\kappa} \arrow[r, "p_{\kappa}"] & Y
\end{tikzcd}
\] commutes where $p_{\kappa}$ is induced by the universal property of colimits. Therefore,  $z = p_1 \circ y_1 = p_{\kappa} \circ x$. 

\medskip

It remains to show that  $p_{\kappa} \in \rlp(J)$. To this end, let
\[
\begin{tikzcd}
U \arrow[d, "h"'] \arrow[r] & X_{\kappa} \arrow[d, "p_{\kappa}"] \\
V \arrow[r]                 & Y                                 
\end{tikzcd}
\] be a commutative square with $h\in J$. We must exhibit a lift $V\to X_{\kappa}$. Since $U$ is $\kappa$-compact, the map $U \to X_{\kappa}$ factors as $U \to X_{\epsilon} \to X_{\kappa}$ for some $\epsilon \in \kappa$. In light of \eqref{eqn:coppushh}, we get a commutative diagram
\[
\begin{tikzcd}
U \arrow[r] \arrow[dd, "h"']         & X_{\epsilon} \arrow[r] \arrow[d, "y_{\epsilon+1}"'] & X_{\kappa} \arrow[d, "p_{\kappa}"] \\
                                     & X_{\epsilon+1} \arrow[ru]                         & Y                                  \\
V \arrow[ru] \arrow[rru, bend right] &                                                   &                                   
\end{tikzcd}
\] and thus our desired lift.
\end{proof}

\pagebreak

\begin{definition}[Cofibrantly generated]\label{cofgen}
A model category $\left(\c, \fib, \cof, \we\right)$ is \textit{cofibrantly generated} if it comes equipped with a pair $\left(J, K\right)$ of sets of maps in $\c$ such that
\be[label=(\roman*)]
\item $\ret(\cell(J)) = \cof$,
\item $\ret(\cell(K)) = \we \cap \cof$, and
\item both $J$ and $K$ permit the small object argument.
\ee
In this case, we call elements of $J$ \textit{generating cofibrations} and elements of $K$ \textit{generating trivial cofibrations}.
\end{definition}

\begin{lemma}\label{cfsimp}
Let $\left(\c, J, K\right)$ be a cofibrantly generated model category. Then 
\be[label=(\alph*)]
\item $\ret(\cell(J)) = \llp(\rlp(J))$, and
\item $\ret(\cell(K))= \llp(\rlp(K))$.
\ee
\end{lemma}
\begin{proof}
For simplicity, let us just prove (a). Since $J \subset \llp(\rlp(J))$, the fact that $$\ret(\cell(J)) \subset \llp(\rlp(J))$$ follows from  \cref{cloret} along with \cref{CC} and \cref{transf}.

\smallskip

For the reverse inclusion, let $f\in \llp(\rlp(J))$. By \cref{SOA}, we can factor $f$ as
\[
\begin{tikzcd}
A \arrow[r, "f_1"] & B \arrow[r, "f_2"] & C
\end{tikzcd}
\] such that $f_1 \in \cell(J)$ and $f_2 \in \rlp(J)$. Hence $f$ has the right lifting property against $f_2$. By \cref{RA}, we have that $f$ is a retract of $f_1$. It follows that $f\in \ret(\cell(J))$.
\end{proof}

\Cref{cfsimp} immediately implies that $\cof = \rlp(K)$ and $\we \cap \fib = \rlp(J)$.

\medskip

Our next notion will serve as a noteworthy generalization of $\sset$ equipped with its classical model structure.

\begin{definition}
Let  $\left(\c, \fib, \cof, \we\right)$ be a model category 
\be
\item We say that $\c$ is a \textit{combinatorial} model category if it is locally presentable as a category and cofibrantly generated as a model category.
\item We say that $\c$ is a \textit{type-theoretic} model category if it is  locally cartesian closed as a category and proper as a model category and $\cof$ consists of all monomorphisms in $\c$. 
\ee
\end{definition}


\subsection{Classical model structure on $\sset$}\label{cmodsset}

The \textit{classical model structure} on $\sset$ is due to Quillen and consists of

\bi
\item \underline{Kan fibrations} as fibrations,
\item \underline{monomorphisms} as cofibrations, and
\item \underline{weak homotopy equivalences} as weak equivalences.
\ei

\begin{notation}
We shall write $\sset_{\mathtt{Quillen}}$ for the category $\sset$ equipped with this model structure $\left(\fib_{\mathtt{SS}}, \cof_{\mathtt{SS}}, \we_{\mathtt{SS}}\right)$. 
\end{notation}

\smallskip


\begin{note}\label{trivsurj}
Every object of $\sset_{\mathtt{Quillen}}$ is cofibrant because the empty map $\emptyset \to X$ is trivially levelwise injective for any simplicial set $X$.

\smallskip

In particular, $\Delta[n]$ is cofibrant. Therefore, for any trivial fibration $p : X \to Y$ of simplicial sets, we can find a diagonal fill-in of the form
\[
\begin{tikzcd}
\emptyset \arrow[r] \arrow[d, hook]                 & X \arrow[d, "p"] \\
{\Delta[n]} \arrow[r] \arrow[ru, dashed] & Y               
\end{tikzcd}
.\] In terms of the Yoneda lemma, this means that the preimage $p_n^{-1}(y)$ is nonempty for any $n$-simplex $y$ in $Y$. In other words, every trivial fibration is levelwise surjective.
\end{note}

\smallskip 

In the interest of space, we shall only \emph{partially} verify that $\sset_{\mathtt{Quillen}}$ is, indeed, a model category as well as cofibrantly generated and proper.


\medskip

Recall the class $\mathbb{A}$ of anodyne extensions (\cref{anodyne}) and let $J$ denote the saturated class generated by the set 

\[ \label{classB}
B\coloneqq \left\{\partial{\Delta[n]} \hookrightarrow \Delta[n] \mid n\geq 0\right\}
\] 

of canonical inclusions, i.e., $J = \ret(\cell(B))$.


\begin{lemma}\label{cofJ}
A map in $\sset_{\mathtt{Quillen}}$ is a cofibration if and only if it belongs to $J$.
\end{lemma}
\begin{proof} $ $
\smallskip

($\Longleftarrow$) It is clear that every element of $B$ is a levelwise injection, i.e., a monomorphism. One can readily check that the class of all levelwise injections is closed under pushouts, transfinite compositions, and retracts. For example, suppose that a simplicial map $g : X' \to Y'$ is a levelwise injection and that $f: X \to Y$ is a retract of $g$. Then we have a commutative diagram of the form
\[
\begin{tikzcd}
X \arrow[d, "f"'] \arrow[r, "p_1"'] \arrow[rr, "\idd_X", bend left] & X' \arrow[d, "g"] \arrow[r] & X \arrow[d, "f"] \\
Y \arrow[r, "q_1"] \arrow[rr, "\idd_Y"', bend right]                & Y' \arrow[r]                 & Y               
\end{tikzcd}.
\] Recall that a set map is injective if and only if it has a left inverse. Thus, $g \circ p_1 = q_1 \circ f$ is levelwise injective as the composite of two levelwise injective maps.  Hence $f$ must be levelwise injective, as desired.

\smallskip

It follows that any element of $J$ belongs to $\cof_{\mathtt{SS}}$.

\medskip

($\Longrightarrow$) Suppose that $f_0 : X_0 \to Y$ is a monomorphism. By induction, let us construct an $\omega$-sequence $X_{\bullet}$ of pushouts of coproducts of maps in $B$ along with a sequence $\left(f_n : X_n \to Y\right)_{n \geq 0}$ of monomorphisms such that  $f_n\restriction_{\sk_{n-1}(X_n)}$ is an isomorphism $\sk_{n-1}(X_n) \overset{\cong}{\longrightarrow} \sk_{n-1}(Y)$ for each $n\geq 1$. Let $n\in \Z_{\geq 1}$ and suppose that we have constructed such an $X_n$ and $f_n : X_n \to Y$.  Let $S_n$ denote the category of all $n$-simplices $\Delta[n] \to Y$ in $Y$ that are \emph{not} in $\im(f_n)$. Note that all objects of $S_n$ are non-degenerate.

\begin{claim}
For each $s\in \ob{S_n}$, the map $s\restriction_{\partial{\Delta[n]}} : \partial{\Delta[n]} \to Y$ factors as
\[
\begin{tikzcd}
{\partial{\Delta[n]}} \arrow[r, "\tilde{s}"] & X_n \arrow[r, "f_n"] & Y
\end{tikzcd}
\] for some unique map $\tilde{s}$.
\end{claim}
\begin{proof}
 Notice that the non-degenerate $k$-simplices in $\partial{\Delta[n]}$ are precisely the non-identity monomorphisms in $\varDelta$ of the form $\left[k\right] \to \left[n\right]$. This means that  $\partial{\Delta[n]} \cong \sk_{n-1}(\partial{\Delta[n]})$. Moreover, $f_n\restriction_{\sk_{n-1}(X_n)}$ is an isomorphism $\sk_{n-1}(X_n) \overset{\cong}{\longrightarrow} \sk_{n-1}(Y)$ by assumption. This induces an isomorphism $s' : \partial{\Delta[n]} \overset{\cong}{\longrightarrow} \sk_{n-1}(X_n)$. Now, take $\tilde{s}$ to be the composite
\[  
\partial{\Delta[n]} \overset{s'}{\longrightarrow} \sk_{n-1}(X_n)  \overset{i}{\hooklongrightarrow} X_n,
\] which must be unique since $f_n$ is monic.
\end{proof}

This provides us with a commutative diagram

\[
\begin{tikzcd}[column sep=huge]
{\coprod_{s\in S_n}\partial{\Delta[n]}} \arrow[d] \arrow[r, "\coprod_{s\in S_n}\tilde{s}"] 
 \arrow[dr, phantom, "\scalebox{1.5}{\color{black}$\ulcorner$}", very near end, color=black]
 & X_n \arrow[d] \arrow[rdd, "f_n", bend left] &   \\
{\coprod_{s\in S_n}\Delta[n]} \arrow[r] \arrow[rrd, "\coprod_{s\in S_n}{s}"', bend right]         & X_{n+1} \arrow[rd, "f_{n+1}", dashed]       &   \\
                                                                                                  &                                             & Y
\end{tikzcd}.
\]

Recall that the Yoneda lemma specifies a natural one-to-one correspondence 
\[
y\in Y_n \longleftrightarrow \Delta[n] \overset{y}{\to} Y
\] where the map $y$ sends the unique non-degenerate $n$-simplex in $\Delta[n]$ to the element $y$. From this, we can see that $\coprod_{s\in S_n}{s}$ is monic. Since $f_n$ is also monic and $\im(f) \cap \im(h) = \emptyset$, it follows that $f_{n+1}$ is monic. Further, $f_{n+1}\restriction_{\sk_n(X_{n})}$ is levelwise surjective by construction. Therefore, $f_{n+1}\restriction_{\sk_n(X_{n})}$ is an isomorphism, completing our induction step.

\smallskip

The transfinite composition of $X_{\bullet}$ is precisely $f_0$. In light of \cref{coppush}, $f_0$ belongs to $\cell(B) \subset J$, as desired.
\end{proof}


\smallskip

The following property of the geometric realization functor is a nontrivial consequence of all of \cref{cofJ}, \cref{minfact}, and \cref{Quillen}.

\begin{theorem}[Quillen]\label{fibpres}
If $p : E \to X$ is a Kan fibration, then $\left\lvert{p}\right\rvert : \left\lvert{E}\right\rvert \to \left\lvert{X}\right\rvert$ is a (Serre) fibration of topological spaces.\footnote{\cite[Theorem 10.10]{goerss}.}
\end{theorem}

\begin{comment}
\smallskip

\begin{prop}\label{RLPTB}
A map in $\sset_{\mathtt{Quillen}}$ has the right lifting property against $\cof$ if and only if it is a trivial fibration.
\end{lemma}
\begin{proof}
\textit{To appear.}
\end{proof}

\end{comment}

\medskip

\begin{lemma}\label{fact1}
Any map $f: X \to Y$ in $\sset_{\mathtt{Quillen}}$ can be factored as
\[
\begin{tikzcd}
X \arrow[r, "f"] \arrow[d, "a"'] & Y \\
Z \arrow[ru, "b"']          &  
\end{tikzcd}
\] where $a\in \mathbb{A}$ and $b \in \fib_{\mathtt{SS}}$.\footnote{\autocite[Theorem 3.1.1]{Joyal}.}
\end{lemma}
\begin{proof}
The follows directly from \cref{SOA} applied to the class  $$\left\{\Lambda^k[n] \hookrightarrow \Delta[n] \mid n \geq 1, \ 0\leq k \leq n\right\}$$ together with the fact that $\Lambda^k[n]$ is small by \cref{finitesimp}.
\end{proof}

\begin{comment}
\begin{lemma}\label{fact1}
Any map $f_0: X_0 \to Y$ in $\sset_{\mathtt{Quillen}}$ can be factored as
\[
\begin{tikzcd}
X_0 \arrow[r, "f_0"] \arrow[d, "a"'] & Y \\
Z \arrow[ru, "b"']          &  
\end{tikzcd}
\] where $a\in \mathbb{A}$ and $b \in \fib_{\mathtt{SS}}$.\footnote{\cite[Theorem 3.1.1]{Joyal}.}
\end{lemma}
\begin{proof}
Consider the set 
\[
S\coloneqq \left\{
\begin{tikzcd}
{\Lambda^k[n]} \arrow[d, hook, "\iota"'] \arrow[r] & X_0 \arrow[d, "f"] \\
{\Delta[n]} \arrow[r]                    & Y               
\end{tikzcd} \mid 
n\geq 1, 0\leq k \leq n\right\}
\] of commutative squares in $\sset$. Then the induced square
\[
\begin{tikzcd}[row sep=large]
{\coprod_{s\in S}\Lambda^k[n]} \arrow[d, "\coprod_{s\in S}{\iota}"'] \arrow[r] & X_0 \arrow[d, "f"] \\
{\coprod_{s\in S}\Delta[n]} \arrow[r]                                  & Y               
\end{tikzcd}
\] commutes, and $\coprod_{s\in S}\iota$ is anodyne by \cref{coppush} because $\mathbb{A}$ is saturated. For the same reason, the pushout map
\[
\begin{tikzcd}[row sep=large]
{\coprod_{s\in S}\Lambda^k[n]} \arrow[d, "{\coprod_{s\in S}\iota}"'] \arrow[r] 
 \arrow[dr, phantom, "\scalebox{1.5}{\color{black}$\ulcorner$}", very near end, color=black]
& X_0 \arrow[d, "j_0"] \\
{\coprod_{s\in S}\Delta[n]} \arrow[r]                                            & X_1               
\end{tikzcd}
\] is also anodyne. Further, the universal property of pushouts yields a commutative triangle
\[
\begin{tikzcd}
X_0 \arrow[rd, "f_0"] \arrow[d, "j_0"'] &   \\
X_1 \arrow[r, "f_1"', dashed]       & Y
\end{tikzcd}.
\] By repeating this procedure with $f$ replaced by $f_0$, we obtain another commutative triangle
\[
\begin{tikzcd}
X_1 \arrow[d, "j_1"'] \arrow[r, "f_1"] & Y \\
X_2 \arrow[ru, "f_2"', dashed]         &  
\end{tikzcd}
.\] More generally, by induction, we obtain an $\omega$-sequence
\[
\begin{tikzcd}
X_0 \arrow[r, "j_0"] & X_1 \arrow[r, "j_1"] & X_2 \arrow[r, "j_2"] & X_3 \arrow[r] & \cdots
\end{tikzcd}
\] of anodyne maps in $\sset$. Let $Z= \colim_{n\in \omega}X_n$. Since the $f_n : X_n \to Y$ together form a cocone under $X_{\bullet}$, we have a unique mediating morphism $\psi : Z \to Y$ as well as a commutative triangle
\[
\begin{tikzcd}
X_0 \arrow[r, "i_0"] \arrow[rd, "f"'] & Z \arrow[d, "\psi"] \\
                                      & Y                  
\end{tikzcd}
.\] Note that $i_0$ is anodyne as the transfinite composition of a sequence of anodyne maps.

\smallskip

It remains to show that $\psi$ is a Kan fibration. To this end, consider any lifting problem
\[
\begin{tikzcd}
{\Lambda^k[n]} \arrow[r, "\tau"] \arrow[d, "\iota"', hook] & Z \arrow[d, "\psi"] \\
{\Delta[n]} \arrow[r]                              & Y                  
\end{tikzcd}
\] between $\iota$ and $\psi$.
The non-degenerate $k$-simplices in $\partial{\Delta[n]}$ are precisely the non-identity monomorphisms  in $\varDelta$  of the form $\left[k\right] \to \left[n\right]$. Therefore, $\partial{\Delta[n]}$ and thus $\Lambda^k[n]$ are finite.  By \cref{finitesimp}, $\tau$ admits a factorization of the form

\[
\begin{tikzcd}
{\Lambda^k[n]} \arrow[r, "\tilde{\tau}", dashed] \arrow[rr, "\tau"', bend right] & X_m \arrow[r, "i_m"] & Z
\end{tikzcd}, \ \quad m\in \Z_{\geq 0}.
\] By our construction of $X_{\bullet}$, we have an $n$-simplex  $\Delta[n] \to X_{m+1}$ fitting into a commutative diagram
\[
\begin{tikzcd}
{\Lambda^k[n]} \arrow[d, "\iota"', hook] \arrow[r, "\tilde{\tau}"] & X_m \arrow[d, "j_m"]          &   \\
{\Delta[n]} \arrow[rrd, bend right] \arrow[r, dashed]              & X_{m+1} \arrow[rd, "f_{m+1}"] &   \\
                                                                   &                               & Y
\end{tikzcd}
.\] Then the composite $\Delta[n] \to X_{m+1} \to Z$ is a solution to our lifting problem. This proves that $\psi$ is a Kan fibration.
\end{proof}
\end{comment}


\begin{corollary}\label{anocof}
Any map in $\sset_{\mathtt{Quillen}}$ with the left lifting property against $\fib_{\mathtt{SS}}$ is anodyne.
\end{corollary}
\begin{proof}
Suppose that $f: X \to Y$ has the left lifting property against $\fib$. By \cref{fact1}, we can factor $f$ as an anodyne extension $a$ followed by a Kan fibration $b$. Then we have a diagonal fill-in of the form

\[
\begin{tikzcd}
X \arrow[r, "a"] \arrow[d, "f"']    & Z \arrow[d, "b"] \\
Y \arrow[r, equal] \arrow[ru, "t", dashed] & Y               
\end{tikzcd}.
\] Thus, the diagram

\[
\begin{tikzcd}
X \arrow[r, equal] \arrow[d, "f"'] & X \arrow[r, equal] \arrow[d, "b"] & X \arrow[d, "f"] \\
Y \arrow[r, "t"']           & Z \arrow[r, "b"']          & Y               
\end{tikzcd}
\] 
commutes, which exhibits $f$ as a retract of $b$. Since $\mathbb{A}$ is saturated, it follows that $f\in \mathbb{A}$.
\end{proof}

\smallskip

The converse of \cref{anocof} is clear, and thus  $\mathbb{A}$ equals the class of all $\fib$-projective morphisms.

\smallskip 

\begin{note}
Thanks to \cref{cofJ}, by applying the same argument for \cref{fact1} to the generating set 
\[
\left\{\partial{\Delta[n]} \hookrightarrow \Delta[n] \mid n \geq 0\right\}
\] instead of  $\left\{\Lambda^k[n] \hookrightarrow \Delta[n] \mid n \geq 1, \ 0\leq k \leq n\right\}$, we have that any map $f: X \to Y$ in $\sset_{\mathtt{Quillen}}$ can be factored as a cofibration followed by a $J$-injective morphism.
\end{note}

\smallskip



\begin{theorem}\label{classseq} $ $
\be[label=(\arabic*)]
\item $\underbrace{J{-}\mathsf{inj}}_{\substack{\text{$J$-injective} \\ \text{morphisms}}}= \we_{\mathtt{SS}} \cap \fib_{\mathtt{SS}}$.
\item\ \ $\mathbb{A} = \we_{\mathtt{SS}} \cap \cof_{\mathtt{SS}}$.
\ee 
\end{theorem}
\pagebreak
\begin{proof} $ $
\be[label=(\arabic*)]
\item See \cite[Proposition 3.4.1]{Joyal}, which is based on a different yet equivalent definition of $\we_{\mathtt{SS}}$.
\item For the inclusion $\mathbb{A} \subset \we_{\mathtt{SS}} \cap \cof_{\mathtt{SS}}$, see \cite[Proposition 3.2.3]{Hovey}. For the reverse inclusion, suppose that $f : X \to  Y$ is a trivial cofibration. Apply \cref{fact1} to factor $f$ as
\[
\begin{tikzcd}
X \arrow[r, "f"] \arrow[rd, "a"'] & Y                 \\
                                  & Z \arrow[u, "b"']
\end{tikzcd}
\] where $a\in \mathbb{A}$ and $b\in \fib_{\mathtt{SS}}$. By \cref{SS23} below, we have that $b$ is a weak equivalence because both $a$ and $f$ are weak equivalences. Thanks to part (1), it follows that $b\in J{-}\mathsf{inj}$. Hence there is a lift of the form
\[
\begin{tikzcd}
X \arrow[r, "a"] \arrow[d, "f"']     & Z \arrow[d, "b"] \\
Y \arrow[r, equal] \arrow[ru, dashed, "k"] & Y               
\end{tikzcd}
.\] Then the diagram
\[
\begin{tikzcd}
X \arrow[r, equal] \arrow[d, "f"']                            & X \arrow[r, equal] \arrow[d, "a"] & X \arrow[d, "f"] \\
Y \arrow[r, "k"'] \arrow[rr, "\idd_Y"', bend right=49] & Z \arrow[r, "b"']          & Y               
\end{tikzcd}
\] commutes, so that $f$ is a retract of $a$ and thus is anodyne.
\ee
\end{proof}

\smallskip

In light of \cref{classseq}, we now can see that $\left(\we_{\mathtt{SS}} \cap \cof_{\mathtt{SS}}, \fib_{\mathtt{SS}}\right)$ is a WFS on $\sset$, as required. Likewise, $\left(\cof_{\mathtt{SS}}, \we_{\mathtt{SS}} \cap \fib_{\mathtt{SS}}\right)$ is a WFS on $\sset$.

\smallskip

\begin{lemma}\label{SS23}
$\we_{\mathtt{SS}}$ satisfies two-out-of-three.
\end{lemma}
\begin{proof}
Let $\we_{k\mathbf{Top}}$ denote the class of all weak homotopy equivalences of $k$-spaces.
\begin{claim} 
 $\we_{k\mathbf{Top}}$ satisfies  two-out-of-three.
\end{claim}
\begin{proof} 
Consider any commutative triangle
\[
\begin{tikzcd}
X \arrow[r, "f"] \arrow[rd, "h"'] & Y \arrow[d, "g"] \\
                                  & Z               
\end{tikzcd}
\] in $k\mathbf{Top}$. 
The following two facts are obvious.
\bi
\item If both $f$ and $g$ belong to $\we_{k\mathbf{Top}}$, then so does $h$. 
\item If both $g$ and $h$ belong to $\we_{k\mathbf{Top}}$, then so does $s$. 
\ei
Finally, suppose that both $f$ and $h$ belong to $\we_{k\mathbf{Top}}$. In particular, $\pi_0(f) : \pi_0(X) \to \pi_0(Y)$ is a bijection. Therefore, for any $y\in Y$, there is some $x\in X$ along with a path $p$ from $y$ to $f(x)$. Let $n\in \Z_{\geq 1}$. When $n=1$, the mapping $\left[\gamma\right] \mapsto \left[\bar{p}\ast \gamma \ast p\right]$ defines an isomorphism $\hat{p} : \pi_n(Y, y) \overset{\cong}{\longrightarrow} \pi_n(Y, f(x))$ where $\ast$ denotes concatenation. In this case, we likewise have an isomorphism $\hat{\bar{p}} : \pi_n(Z, g(f(x))) \overset{\cong}{\longrightarrow} \pi_n(Z, g(y))$ given by $\left[\gamma\right] \mapsto \left[\left(g\circ {p}\right)\ast \gamma \ast \left( g \circ \bar{p}\right)\right]$. If $n>1$, then we can define $\hat{p}$ by sending any map $s: \left(I^n, \partial{I^n}\right) \to \left(Y, y\right)$ to a new map $s_p : \left(I^n, \partial{I^n}\right) \to \left(Y, f(x)\right)$ given as follows. Shrink the $n$-cube $I^n$ to  $C\coloneqq \left[\frac{1}{3}, \frac{2}{3}\right]^n$ and draw a radial segment $\ell_t$ from each point $t$ on $\partial{I^n}$ to $\left(\frac{1}{2}, \frac{1}{2}, \ldots, \frac{1}{2}\right)\in C$. Re-parameterize  $s$ so that it has domain $C$ and then extend it to a map $s_p$ on $I^n$ that equals $p$ on each segment $\ell_t$.

\medskip

When $n=1$, it is straightforward to check that the composite
\[
\begin{tikzcd}[column sep=large]
{\pi_n(Y,y)} \arrow[r, "\hat{p}"] & {\pi_n(Y, f(x))} \arrow[r, "{\pi_n(g, f(x))}"] & {\pi_n(Z, g(f(x)))} \arrow[r, "\hat{\bar{p}}"] & {\pi_n(Z, g(y))}
\end{tikzcd}
\] is precisely $\pi_n(g,y)$. This remain true even if $n>1$.
By functoriality of $\pi_n({-}, {-})$, the map $\pi_n(g,f(x))$ equals the composite $\pi_n(h, x) \circ \pi_n(f^{-1}, f(x))$, which is an isomorphism since both $f$ and $h$ are weak homotopy equivalences. Therefore, $\pi_n(g,y)$ is an isomorphism.

\medskip

It is clear that $\pi_0(g)$ is a bijection, and thus $g \in \we_{k\mathbf{Top}}$.
\end{proof}
As $\left\lvert{-}\right\rvert : \sset \to k\mathbf{Top}$ both preserves and reflects weak homotopy equivalences, it follows that $\we_{\mathtt{SS}}$ also satisfies two-out-of-three.
\end{proof}

\smallskip

We have established that $\sset_{\mathtt{Quillen}}$ is, in fact, a model category. 

\medskip

\begin{theorem}\label{CFG}
$\sset_{\mathtt{Quillen}}$ is cofibrantly generated.
\end{theorem}
\begin{proof}
$\sset_{\mathtt{Quillen}}$ satisfies conditions (i) and (ii) of \cref{cofgen} thanks to \cref{cofJ} and \cref{classseq}(2), respectively. It satisfies condition (iii) because every simplicial set is small by an argument similar to that given for \cref{finitesimp}.
\end{proof}

\begin{corollary}
$\sset_{\mathtt{Quillen}}$  is a combinatorial model category.
\end{corollary}
\begin{proof}
This is an immediate consequence of \cref{CFG} together with \cref{PSLP}.
\end{proof}

\medskip

Finally, we want to show that $\sset_{\mathtt{Quillen}}$ is right proper. For this, the following lemma is useful.

\begin{lemma}\label{Toprp}
The pullback of a weak homotopy equivalence of $k$-spaces along a Serre fibration of $k$-spaces is again a weak homotopy equivalence. 
\end{lemma}
\begin{proof}
Consider any pullback square
\[
\begin{tikzcd}
X' \arrow[r, "f'"]
\arrow[dr, phantom, "\scalebox{1.5}{\color{black}$\lrcorner$}" , very near start, color=black]
 \arrow[d, "w'"'] & X \arrow[d, "w"] \\
Y' \arrow[r, "f"']            & Y               
\end{tikzcd}
\] in $k\mathbf{Top}$ where $f$ is a Serre fibration and $w$ is weak homotopy equivalence. We must show that $w'$ is also a weak homotopy equivalence. 

\smallskip

As a pullback of $f$, $f'$ is a fibration with fibers vertically isomorphic to those of $f$. This yields a commutative diagram of the form
\[ \label{eqn:fibdiag}
\begin{tikzcd}
F \arrow[r, hook] \arrow[d, "\cong"'] & X' \arrow[d, "w'"'] \arrow[r, "f'"] & X \arrow[d, "w"] \\
\widetilde{F} \arrow[r, hook]         & Y' \arrow[r, "f"']                  & Y               
\end{tikzcd} \tag{$\ast$}
\] for any two corresponding fibers $F$ and $\widetilde{F}$. By applying the long exact sequence in homotopy for $f'$ and for $f$ to the top and bottom row  of \eqref{eqn:fibdiag}, respectively, we get a commutative diagram
\[
\begin{tikzcd}
{\pi_n(F, {\cdot})} \arrow[r] \arrow[d, "\cong"'] & {\pi_n(X', {\cdot})} \arrow[d, "{\pi_n(w', {\cdot})}"'] \arrow[r, "{\pi_n(f', {\cdot})}"] & {\pi_n(X, {\cdot})} \arrow[d, "{\pi_n(w, {\cdot})}"] \\
{\pi_n(\widetilde{F}, {\cdot})} \arrow[r]         & {\pi_n(Y', {\cdot})} \arrow[r, "{\pi_n(f, {\cdot})}"']                                    & {\pi_n(Y, {\cdot})}                                 
\end{tikzcd}
\] for every integer $n\geq 0$. But $\pi_n(w, {\cdot})$ is an isomorphism because $w$ is a weak homotopy equivalence. The short split five lemma now implies that $\pi_n(w', {\cdot})$ is an isomorphism, as required.
\end{proof}

\begin{theorem}\label{RProp}
$\sset_{\mathtt{Quillen}}$ is right proper.
\end{theorem}
\begin{proof}
Consider any pullback square
\[
\begin{tikzcd}
X' \arrow[r]
\arrow[dr, phantom, "\scalebox{1.5}{\color{black}$\lrcorner$}" , very near start, color=black]
 \arrow[d, "w'"'] & X \arrow[d, "w"] \\
Y' \arrow[r, "f"']            & Y               
\end{tikzcd}
\] in $\sset$ where $f\in \fib_{\mathtt{SS}}$ and $w \in \we_{\mathtt{SS}}$. We must show that $w' \in \we_{\mathtt{SS}}$.
 To this end, let us gather a few properties of $\left\lvert{-}\right\rvert : \sset \to k\mathbf{Top}$ established thus far.

\bi
\item $\left\lvert{-}\right\rvert$ preserves pullbacks by \cref{realpres}(2).
\item $\left\lvert{-}\right\rvert$ sends any Kan fibration to a Serre fibration by \cref{fibpres}.
\item $\left\lvert{-}\right\rvert$ both preserves and reflects all weak homotopy equivalences.
\ei

As a result, we have a pullback square
\[
\begin{tikzcd}
\left\lvert{X'}\right\rvert \arrow[r]
\arrow[dr, phantom, "\scalebox{1.5}{\color{black}$\lrcorner$}" , very near start, color=black] 
 \arrow[d, "\left\lvert{w'}\right\rvert"'] & \left\lvert{X}\right\rvert \arrow[d, "\left\lvert{w}\right\rvert"] \\
\left\lvert{Y'}\right\rvert \arrow[r, "\left\lvert{f}\right\rvert"']            & \left\lvert{Y}\right\rvert                                        
\end{tikzcd}
\]  in $k\mathbf{Top}$ such that $\left\lvert{f}\right\rvert$ is a Serre fibration and $\left\lvert{w}\right\rvert$ is a weak homotopy equivalence. By \cref{Toprp}, $\left\lvert{w'}\right\rvert$ is also a weak homotopy equivalence. Thus, $w' \in  \we_{\mathtt{SS}}$.
\end{proof}

\smallskip

For any model category $\left(\c, \fib, \cof, \we\right)$ and $x\in \ob{\c}$, the over category $\c/x$ inherits a model structure from $\c$ where a  morphism
\[
\begin{tikzcd}
y \arrow[rd] \arrow[r, "\zeta"] & z \arrow[d] \\
                       & x          
\end{tikzcd}
\]
in $\c/x$ is
\bi
\item a fibration if and only if $\zeta \in \fib$,
\item a cofibration if and only if $\zeta \in \cof$, and
\item a weak equivalence if and only if $\zeta \in \we$.
\ei

Note that an object of $\c/x$ is fibrant if and only if it belongs to $\fib$.

\smallskip

\begin{corollary}\label{adjpres}
Let $g : X \to Y$ be a simplicial map. 
\be[label=(\arabic*)]
\item The base change functor $g^{\ast} : \sset_{\mathtt{Quillen}}/Y \to \sset_{\mathtt{Quillen}}/X$ preserves cofibrations. 
\item  The dependent product $\Pi_g : \sset_{\mathtt{Quillen}}/X \to \sset_{\mathtt{Quillen}}/Y$ preserves trivial fibrations in $\sset_{\mathtt{Quillen}}$.


\item If $g$ is a Kan fibration, then $g^{\ast}$ preserves weak equivalences (hence trivial cofibrations).
\item  If $g$ is a Kan fibration, then  $\Pi_g$ preserves Kan fibrations.
\ee
\end{corollary}
\begin{proof} $ $
\be[label=(\arabic*)]
\item Let \begin{tikzcd}[column sep=small, row sep=small]
A \arrow[r, "j"] \arrow[rd] & B \arrow[d] \\
                            & Y          
\end{tikzcd} be a cofibration in $\sset_{\mathtt{Quillen}}/Y$, so that $j$ is a monomorphism of simplicial sets. We have a  commutative diagram
\[
\begin{tikzcd}[row sep=large]
g^{\ast}{A} \arrow[dd, bend right=49] \arrow[r] \arrow[d, dashed] & A \arrow[d, "j"] \arrow[dd, bend left=49] \\
g^{\ast}{B} \arrow[d] \arrow[r]                                & B \arrow[d]                            \\
X \arrow[r, "g"']                                              & Y                                     
\end{tikzcd}
\] obtained by the universal property of pullback squares.  By definition, $g^{\ast}$ sends $j$ to this dotted arrow $g^{\ast}{j}$. Since the upper square must be a pullback and monomorphisms are stable under pullback, it follows that $g^{\ast}{j}$ is a monomorphism. Hence it is a cofibration in $\sset_{\mathtt{Quillen}}/X$.

\begin{note}\label{prestrvifib}
The preceding argument with ``monomorphism'' replaced by ``trivial fibration'' shows that $g^{\ast}$ preserves trivial fibrations as well.
\end{note}
\item Let $f : A \to X$ be a trivial fibration of simplicial sets. We can view this as a trivial fibration
\[
\begin{tikzcd}
A \arrow[rd, "f"'] \arrow[r, "f"] & X \arrow[d, equal] \\
                                  & X          
\end{tikzcd}
\] in $\sset_{\mathtt{Quillen}}/X$. We must show that the \emph{object} $\Pi_g{f}$ of $\sset_{\mathtt{Quillen}}/Y$ is a trivial fibration in $\sset_{\mathtt{Quillen}}$. By part (1) together with \cref{trivcofp}(a), we deduce that the map $\Pi_g{f} : \Pi_g{f} \to \Pi_g{\idd_X}$ is a trivial fibration. Since $\Pi_g({-})$ is right adjoint, it preserves the terminal object $\idd_X$. Hence the map $\Pi_g{f}$ coincides with the object $\Pi_g{f}$, which is thus a trivial fibration in $\sset_{\mathtt{Quillen}}$.
\item Suppose that $g$ is a Kan fibration. Let \begin{tikzcd}[column sep=small, row sep=small]
A \arrow[r, "w"] \arrow[rd] & B \arrow[d] \\
                            & Y          
\end{tikzcd} be a weak equivalence in $\sset_{\mathtt{Quillen}}/Y$, so that $w$ is a weak equivalence of simplicial sets. We again have a pasting
\[
\begin{tikzcd}[row sep=large]
g^{\ast}{A} \arrow[dd, bend right=49] \arrow[r] \arrow[d, "g^{\ast}{w}"'] & A \arrow[d, "w"] \arrow[dd, bend left=49] \\
g^{\ast}{B} \arrow[d] \arrow[r, "g'"']                                    & B \arrow[d]                               \\
X \arrow[r, "g"']                                                         & Y                                        
\end{tikzcd}
\] of pullback squares. \Cref{CP} implies that $g'$ is a Kan fibration, and thus $g^{\ast}{w}$ is a weak equivalence because $\sset_{\mathtt{Quillen}}$ is right proper.
\item This follows from a nearly identical argument to (2).
\ee
\end{proof}

\smallskip

\begin{remark}
It can be shown that any model category where all objects are cofibrant is left proper. Hence $\sset_{\mathtt{Quillen}}$ is also left proper and thus proper.
\end{remark}

\subsection*{Global model structure on $\left[\c^{\op}, \sset\right]$}

Let $\c$ be a small category. \Cref{PSLP} and \cref{CFG} together imply that $\sset_{\mathtt{Quillen}}$ is combinatorial. As it turns out, this ensures that the category $\left[\c^{\op}, \sset\right]$ of simplicial presheaves over $\c$ inherits a model structure from $\sset_{\mathtt{Quillen}}$ in at least two ways:

\be[label=(\alph*)]
\item The \textit{projective model structure  $\left[\c^{\op}, \sset\right]_{\mathtt{proj}}$} consists of
\bi
\item \underline{levelwise weak equivalences} as weak equivalences,
\item \underline{levelwise fibrations} as fibrations, and
\item \underline{$\we$-projective morphisms} as cofibrations. 
\ei
\item The \textit{injective model structure $\left[\c^{\op}, \sset\right]_{\mathtt{inj}}$} consists of
\bi
\item \underline{levelwise weak equivalences} as weak equivalences,
\item \underline{levelwise cofibrations} as cofibrations, and
\item  \underline{$\we$-injective morphisms} as fibrations. 
\ei
\ee

Each of (a) and (b) is called a \textit{global} model structure on $\left[\c^{\op}, \sset\right]$. It is known that $\left[\c^{\op}, \sset\right]_{\mathtt{inj}}$ is both proper and cofibrantly generated, just as $\sset_{\mathtt{Quillen}}$. Therefore, it is also a combinatorial model category.


\begin{comment}
\subsection{Reedy categories}

%https://ncatlab.org/nlab/show/Reedy+model+structure
%Theorem 5.2.5 (Hovey)
%Lemma 6.1 (Shulman)

%https://ncatlab.org/nlab/show/elegant+Reedy+category
	%Examples (direct/inverse categories)
	
	%Lemma 2.4 (prove)
	%Lemma 2.5 (state)
	%Theorem 2.6 (prove)
	%https://ncatlab.org/nlab/show/subobject#LimitsAndColimits
\end{comment}


\subsection{$\infty$-Categories}\label{highcat}

This section sets forth a  generalization of $\sset$ that, in some sense, is the right setting for modeling $\cdtt$, as we shall see 
in \cref{smod}.

\medskip

\begin{comment}
\begin{definition} Let $\c$ be a category and let $J$ be a subclass of $\mor(\c)$.
\be
\item We say that $J$ satisfies \textit{two-out-of-six} if for any three composable maps
\[
A \overset{f}{\longrightarrow} B \overset{g}{\longrightarrow} C \overset{h}{\longrightarrow} D
\] in $J$, whenever both $h \circ g$ and $g\circ f$ are in $J$, so are $f$, $g$, $h$, and $h \circ g \circ f$.
\item We say that $\c$ is a \textit{homotopical category} if it has a distinguished class $\we$ of morphisms (called \textit{weak equivalences}) that both contains all isomorphisms and satisfies two-out-of-six.
\ee
\end{definition}

\begin{prop}
If $\left(\c, \fib, \cof, \we\right)$ is a model category, then $\left(\c, \we\right)$ is a homotopical category.\footnote{See \cite{MSER}.}
\end{prop}

Moreover, it is clear that every homotopical category is a category with weak equivalences. Therefore, the notion of homotopical category  is intermediate between \textit{category with weak equivalences} and \textit{model category}.

\smallskip
\end{comment}

\begin{definition}
An  \textit{$\infty$-category} is a category  enriched over the cartesian monoidal category $\sset$.
\end{definition}

\begin{exmp} $ $
\be
\item $\sset$ is an $\infty$-category. Indeed, since $\sset$ is cartesian closed, we have a sequence of natural isomorphisms
\begin{align*}
Z^Y & \cong \Hom_{\sset}\left(1, Z^Y\right)
\\ & \cong  \Hom_{\sset}\left(1 \times Y, Z\right)
\\ & \cong  \Hom_{\sset}\left(Y, Z\right)
\end{align*} for any simplicial sets $Y$ and $Z$. This shows that every hom-set for $\sset$ is itself a simplicial set, as desired.
\item $\left[\c^{\op}, \sset\right]$ is an $\infty$-category for any small category $\c$. Indeed, this is also cartesian closed as it is isomorphic to the presheaf category $\left[\left(\c \times \varDelta\right)^{\op}, \set\right]$. We thus have another sequence of natural isomorphisms 
\begin{align*}
Z^Y & \cong \Hom_{\left[\c^{\op}, \sset\right]}\left(1, Z^Y\right)
\\ & \cong  \Hom_{\left[\c^{\op}, \sset\right]}\left(1 \times Y, Z\right)
\\ & \cong  \Hom_{\left[\c^{\op}, \sset\right]}\left(Y, Z\right)
\end{align*} for any simplicial presheaves $Y$ and $Z$ over $\c$. Note that any functor of the form $G: \left(\c \times \varDelta\right)^{\op}  \to  \set$ restricts to the simplicial set given by $\left[n\right] \mapsto G(\ast, \left[n\right])$. Hence $Z^Y$ and thus $\Hom_{\left[\c^{\op}, \sset\right]}\left(Y, Z\right)$ may be regarded as simplicial sets, as desired.
\ee
\end{exmp}

\medskip

Next, suppose that $F : K\to \c$ and  $w: K\to \mathscr{V}$ are small diagrams enriched over a closed symmetric monoidal category  $\mathscr{V}$. The \textit{weighted limit $\lim^w{F}$ of $F$ with weight $w$} is the object, if it exists, of $\c$ that represents the $\mathscr{V}$-valued presheaf
\[
\Hom_{\left[K, \mathscr{V}\right]}\left(w, \Hom_{\c}(x, F({-}))\right) :\c^{\op} \to \mathscr{V}
\] naturally in $x\in \ob{\c}$.

\medskip

Let $\mathscr{V} = \sset$, so that $\c$ is an $\infty$-category. Define the weight $w$ by 
\[
k \mapsto N(K/k)
\] where $N({-})$ denotes the nerve of a small category. In this case, we call $\lim^w{F}$ the \textit{homotopy limit of $F$}, denoted by $\holim_K{F}$.

\smallskip

\begin{definition}
Let $\c$ be an $\infty$-category with all finite homotopy limits. We say that $\c$ is a \textit{locally cartesian closed (LCC) $\infty$-category} if for every map $f: x \to y$ in $\c$, the homotopy pullback functor $f^{\ast}: \c/y \to \c/x$ has a right adjoint $\Pi_f$.
\end{definition}

\medskip

We also have a variant of \textit{locally presentable} for $\infty$-categories, but it requires a lot of machinery to state.


\begin{definition}
Let $F : \c \to \d$ be a functor of $\infty$-categories (i.e., a $\sset$-enriched functor). We say that $F$ is a \textit{Dwyer-Kan equivalence} if 
\be[label=(\roman*)]
\item the functor $\pi_0(F) : \pi_0(\c) \to \pi_0(\d)$ (\cref{conncomp}) is essentially surjective and
\item for all $x,y\in \ob{\c}$, the map $F_{x,y} : \Hom_{\c}(x,y) \to \Hom_{\d}(F(x), F(y))$ is a weak equivalence in $\sset_{\mathtt{Quillen}}$.
\ee
In this case, we write $\c \simeq_{\mathtt{DK}} \d$.
\end{definition}

\medskip

We now want to define a way of forming an $\infty$-category out of a given category with weak equivalences $\left(\c, \we\right)$. To this end, consider a  directed graph $G\coloneqq \left(V, E\right)$. The \textit{free category $F{G}$ of $G$} has vertices of $G$ as objects and lists of the form
\begin{gather*}
\left(a_n,f_n,a_{n-1},\ldots,a_{1},f_1,a_0\right)
\\  n\in \N, \ a_i \in V, \ a_0 \equiv a, \ a_n \equiv b
\\ f_i \text{ is an edge from $a_{i-1}$ to $a_i$ for all $0< i \leq n$} 
\end{gather*}
as morphisms $a\to b$. Here, composition is given by
\begin{gather*}
 \left(a_n,f_n,a_{n-1},\ldots,a_{1},f_1,a_0\right)\circ \left(b_m,g_m,a_{m-1},\ldots,b_{1},g_1,b_0\right) 
\\ \vequiv
\\ \left(a_n,f_n,a_{n-1},\ldots,a_{1},f_1,a_0= b_m,g_m,a_{m-1},\ldots,b_{1},g_1,b_0\right). 
\end{gather*}
The identity map $\idd_a$ is precisely the empty path $\emptyset$ from $a$ to itself.

\smallskip

A \textit{reflexive quiver} is a quiver equipped with an edge $i_v$ from $v$ to itself for each vertex $v$. Every small category $\e$ may be regarded as a reflexive quiver $\left(\ob{\e}, \mor(\e)\right)$. This provides us with two functors
\begin{alignat*}
\epsilon_{\e} : F{\e} &\longrightarrow \e , &&\  \quad v \mapsto v
\\ \delta_{\e}: F{\e} &\longrightarrow F{F{\e}}, &&\  \quad v \mapsto v.
\end{alignat*} 

It can be shown that the induced  triple $\mathbb{F}(\e)\coloneqq \left(F({-}), \epsilon, \delta\right)$ is a comonad in $\mathbf{Cat}$. This induces a simplicial category $\mathbb{F}_{\bullet}{\e} : \varDelta^{\op} \to \mathbf{Cat}$ defined on objects by
\[
\mathbb{F}(\e)_n = F^{n+1}{\e}
\]
with face $d_i : \mathbb{F}_k(\e) \to  \mathbb{F}_{k-1}(\e)$ and degeneracy $s_i :  \mathbb{F}_k(\e) \to  \mathbb{F}_{k+1}(\e)$ operators given by
\[
\begin{tikzcd}[column sep=large]
F^{k+1}(\e) \arrow[r, "F^i\epsilon_{F^{k-i}}"] & F^k(\e)     \\
F^{k+1}(\e) \arrow[r, "F^i\delta_{F^{k-i}}"]   & F^{k+2}(\e)
\end{tikzcd}
,\]
respectively.

\medskip

Let $U: \mathbf{Cat} \to \set$ denote the functor given by  $\mathscr{B} \mapsto \ob(\mathscr{B})$. Then the composite functor $U \circ \mathbb{F}_{\bullet}{\e}$ is the constant simplicial set at $\ob{\e}$.

\medskip

Moving to our next piece of machinery, suppose that $\left(\e, \we_{\e}\right)$ is a category with weak equivalences. For any arrow $f$ in $\e$, let $\bar{f}$ denote the reverse of $f$. Let $$\we_{\e}^{\op}(x,y) = \left\{\bar{f} \mid f : x\to y, \ f \in \we_{\e}\right\}.$$ Consider the directed graph $\mathcal{G}$ with objects of $\e$ as vertices, elements of  $\Hom_{\e}(x,y)$ as edges $x \to y$, and elements of $\we_{\e}^{\op}(x,y)$ as edges $y\to x$. Now, define $\sim$ as the smallest equivalence relation on the set $\mor(F{\mathcal{G}})$ such that

\bi
\item for any $x\in \ob{\e}$, $\left(x,\idd_x, x\right) \sim \left(x, \emptyset, x\right)$,
\item for any maps $f: x\to y$ and $g: y \to z$ in $\e$, $\left(z,g,y,f,x\right)\sim \left(z,g\circ f,x\right)$, and 
\item for any map $f: x\to y$ in $\we_{\e}$, 
\begin{align*}
\left(x ,\bar{f} , y,  f, x\right) &\sim \left(x, \idd_x, x \right)
\\  \left(y ,f , x,  \bar{f}, y\right) &\sim \left(y, \idd_y, y \right).
\end{align*}
\ei

The quotient category $\e\left[\we_{\e}^{-1}\right] \coloneqq \faktor{F{\mathcal{G}}}{\sim}$ is called the \textit{localization of $\e$ by $\we_{\e}$}.

\smallskip

\begin{definition}
If $\c$ is small, then the \textit{(standard) simplicial localization of $\c$} is the simplicial category $$\mathbb{F}_{\bullet}{\c}\left[\mathbb{F}_{\bullet}{\we}^{-1}\right] : \varDelta^{\op} \to \mathbf{Cat},$$ where $\we$ may be treated as a subcategory of $\c$.\footnote{\autocite[Section 4.1]{DKan}.}
\end{definition}

The composite $U\circ \mathbb{F}_{\bullet}{\c}\left[\mathbb{F}_{\bullet}{\we}^{-1}\right]$ is again a constant simplicial set.

\begin{prop}
Let $T : \varDelta^{\op} \to \mathbf{Cat}$ be a functor so that $U\circ T$ is a constant simplicial set at, say, $\mathcal{S}$. For any $x,y\in \mathcal{S}$ and any $n\in \N$, let
\[
T(x,y)_n =\left\{ \tau \in \mor(T_n) \mid \tau : x\to y\right\}.
\] 
\be[label=(\alph*)]
\item The family $\left\{T(x,y)_n\right\}_{n\geq 0}$ has the structure of a simplicial set $T(x,y)$.
\item Levelwise composition of $T$ induces a composition operation $T(x,y) \times T(y,z) \to T(x,z)$.
\item The category $\left\langle T \right\rangle$ with $\ob{\left\langle T \right\rangle} \equiv \mathcal{S}$ and $\Hom_{\left\langle T \right\rangle}(x,y) \equiv T(x,y)$ is enriched over $\sset$.
\ee

\end{prop}

It follows at once that $$L_{\we}{\c} \coloneqq \left\langle\mathbb{F}_{\bullet}{\c}\left[\mathbb{F}_{\bullet}{\we}^{-1}\right]\right\rangle$$ is an $\infty$-category.

\begin{definition}
We say that an $\infty$-category $\d$ is \textit{locally presentable} if it has a \textit{presentation} by a combinatorial simplicial model category  $\mathscr{A}$   in the sense that $\d \simeq_{\mathtt{DK}} L_{\we}{\mathscr{A}}$.
\end{definition}

Here, by ``simplicial model category'' we mean a model category $\c$ enriched over $\sset$ such that for every fibration $p: X\to Y$ and cofibration $i:A \to B$ in $\c$, the unique mediating map
\[
\begin{tikzcd}
{\Hom_{\c}(B,X)} \arrow[rd, "i^{\ast}\times p_{\ast}", dashed] \arrow[rdd, "{\Hom_{\c}(\idd_B,p)}"', bend right] \arrow[rrd, "{\Hom_{\c}(i,\idd_X)}", bend left] &                                                                           &                                                     \\
                                                                                                                                                                 & {\Hom_{\c}(B,Y)\times_{\Hom_{\c}(A,Y)}\Hom_{\c}(A,X)} \arrow[d] \arrow[r] & {\Hom_{\c}(A,X)} \arrow[d, "{\Hom_{\c}(\idd_A,p)}"] \\
                                                                                                                                                                 & {\Hom_{\c}(B,Y)} \arrow[r, "{\Hom_{\c}(i,\idd_Y)}"']                      & {\Hom_{\c}(A,Y)}                                   
\end{tikzcd}
\] is a Kan fibration. Moreover, this map must be a weak equivalence whenever $p$ or $i$ is one.

\bigskip

Finally, let us define a certain kind of locally presentable $\infty$-category with good structure for modeling not only $\cdtt$ but also $\univ$. (We shall make this feature precise at the end of \cref{fibunivpf}.) For this, we need a few auxiliary concepts.

\begin{definition}
Let $\c$ be an  $\infty$-category with all homotopy pullbacks and all homotopy colimits of shape $D$. We say that a $D$-shaped homotopy colimit  $\hocolim_{d\in D}F(d)$ in $\c$ is \textit{universal} if for any homotopy pullback square of the form
\[
\begin{tikzcd}
Y\times_Z \left(\hocolim_{d}F(d)\right) \arrow[d] \arrow[r] & \hocolim_dF(d) \arrow[d] \\
Y \arrow[r]                                               & Z                     
\end{tikzcd}
\] in $\c$, we have that $\hocolim_{d}\left(F(d)\times_Z Y\right) \cong Y\times_Z \left(\hocolim_{d}F(d)\right)$.
\end{definition}

\smallskip

\begin{definition}\label{objclass}
  Let $\c$ be an $\infty$-category. Let $J$ be a class of morphisms in $\c$ closed under homotopy pullbacks. We say that a map $\widehat{J\mathtt{Type}} \to J\mathtt{Type}$ is a \textit{$J$-classifier} if for any map $X\to B$ in $J$, there exists a unique homotopy pullback square of the form
\[
\begin{tikzcd}
X \arrow[d] \arrow[r] 
\arrow[dr, phantom, "\scalebox{1.5}{\color{black}$\lrcorner$}" , very near start, color=black]
& \widehat{J\mathtt{Type}} \arrow[d] \\
B \arrow[r, "\ulcorner{X}\urcorner"']           & J\mathtt{Type}                    
\end{tikzcd}.
\]
\end{definition}

\smallskip

  Let $\kappa$ be a cardinal. We say that a map $X \to Y$ in $\c$ is \textit{relatively $\kappa$-compact} if for any $\kappa$-compact object $Y'$ of $\c$ and any homotopy pullback square
\[
\begin{tikzcd}
Z\times_YX \arrow[d] \arrow[r] & X \arrow[d] \\
Z \arrow[r]                    & Y          
\end{tikzcd}
\] in $\c$, 
the object $X\times_{Y}Z$ is also $\kappa$-compact.

\begin{term} $ $
\bi
\item If $J$ denotes the  class of all morphisms in $\c$, then a $J$-classifier is called an \textit{object classifier}.
\item If $J$ denotes the class of all monomorphisms in $\c$, then a $J$-classifier is called a \textit{subobject classifier}.
\item If $\kappa$ is a regular cardinal and $J$ denotes the class of all relatively $\kappa$-compact morphisms in $\c$, then a $J$-classifier is called a \textit{$\kappa$-compact-object classifier}.
\ei
\end{term}

\begin{exmp}
To gain a bit of intuition about $J$-classifiers, let $\c = \set$. Then the function $$T: 1 \to \underbrace{\left\{F,T\right\}}_{1 \coprod 1}$$ picking out the truth value $T$ is a subobject classifier in $\c$. Indeed, for any inclusion function $S \hookrightarrow B$, define $\ulcorner{S}\urcorner : B \to \left\{F,T\right\}$ as the function 
\[
\chi_S(b) \equiv \begin{cases}
T & b\in S
\\ F & b\notin S
\end{cases}.
\]
\end{exmp}

\begin{definition}[Rezk]
We say that an $\infty$-category $\c$ is a \textit{Grothendieck $\infty$-topos}  if 
\be[label=(\roman*)]
\item it is locally presentable, 
\item has all universal colimits, and
\item has a $\kappa$-compact-object classifier for all sufficiently large regular cardinals $\kappa$.\footnote{A precise definition of ``sufficiently large'' is found in the proof of \cite[Proposition 6.1.6.7]{Lurie}.}
\ee
\end{definition}


\section{A simplicial model of HoTT}\label{models}

This section is devoted to examining \cite[Sections 2 and 3]{KL}, which constructs a certain model of $\cdtt  +\univ$ in the category $\sset$ of simplicial sets.
In \cref{fibuniv} and  \cref{smod}, we choose particular universes in our chosen class of presheaf categories and show that they carry all of the logical structure found in our MLDTT without $\univ$, respectively.  Next, turning out attention to the univalence axiom, we define in \cref{suniv} a simplicial notion of univalence for these models that is logically equivalent to our type-theoretic notion of univalence. Finally, in \cref{fibunivpf}, we prove that our chosen universes are univalent  in the simplicial sense and then state a remarkable generalization of this result.

\begin{comment}
This section is devoted to examining both \cite[Sections 2 and 3]{KL}  and \cite{Shul}. The former constructs a certain model of $\cdtt  +\univ$ in the category $\sset$ of simplicial sets. The latter uses this construction to show that $\cdtt  +\univ$ can be interpreted in the category of simplicial presheaves over any small elegant Reedy category $\c$. Since the trivial category $\ast$ is elegant Reedy (as will be evident) and $$\left[\ast^{\op}, \sset\right] \cong \left[\left(\ast \times \varDelta\right)^{\op}, \set\right] \cong \sset,$$  it follows that \cite{Shul} directly generalizes \cite{KL}. Even so, it will be sufficient to take \cite{KL}, on the one hand, as our main source for several key stages of \cite{Shul} as they have exact analogues in the setting of  $\mathbf{sSet}$. Such a setting is also instructive thanks to its familiarity and geometric flavor. On the other hand, we shall take \cite{Shul} as our main source for proving that the internal universe is fibrant since its methodology here is substantially different from that of  \cite{KL}.
\end{comment}

\subsection{Fibrant universes of ``small" fibrations}\label{fibuniv}

This section recounts  \cite[Sections 2.1 and 2.2]{KL}, which defines a class of Kan complexes serving as universes in $\sset$ both in the sense of \cref{univ1} and  in the sense of \textit{internal universe} (p.~\pageref{internaluniv}).

\medskip

For any such Kan complex $U$, we want to find a simplicial map $\widehat{U}\to U$ acting as a classifier for a specific class of Kan fibrations in the sense of \cref{objclass}. To ensure that this class is closed under the categorical versions of our type-forming operations, we shall take the class of all \textit{$\kappa$-small well-ordered} Kan fibrations. Moreover, since dependent types will be interpreted as Kan fibrations,  $\widehat{U}\to U$ must be a Kan fibration to interpret the dependent type ${x: \U \vdash \el(x) \type}$.

\smallskip

\[
\text{Pick any regular cardinal $\kappa$.} 
\]

\begin{definition} Let $f : X \to Y$ be a map of simplicial sets.
\be
\item We say that $f$ is \textit{well-ordered} if it is equipped with  a well-ordering of $Y_x \coloneqq f_n^{-1}(x)$ for each simplex $x\in X_n$.
\item We say that $f$ is \textit{$\kappa$-small} if $\left\lvert{Y_x}\right\rvert < \kappa$ for every simplex $x$. 
\ee
\end{definition}


\smallskip

Let $f: X \to Y$ and $g : Z \to Y$ be well-ordered simplicial maps. A \textit{morphism $f \to g$} is a simplicial map $h : X \to Z$ fitting into a commutative triangle
\[
\begin{tikzcd}
X \arrow[r, "h"] \arrow[rd, "f"'] & Z \arrow[d, "g"] \\
                                  & Y                 
\end{tikzcd}
\]
such that $h_n : f_n^{-1}(y) \to g_n^{-1}(y)$ is order-preserving for every $n\in \N$ and $y\in Y_n$.

\begin{note}\label{uniquewoset}
Recall that for any two well-ordered sets $x$ and $y$, there is exactly one isomorphism of the form $x\overset{\cong}{\longrightarrow} y$. Thus, for any two well-ordered simplicial maps $X$ and $Y$, there is exactly one isomorphism of the form $X\overset{\cong}{\longrightarrow} Y$.
\end{note}

\smallskip


Now, the isomorphism class of any $\kappa$-small well-ordered map is a proper class. We can, however, apply Scott's trick to make this a set. In this case, for any simplicial set $X$, we have a definable class  $\mathcal{W}_{\kappa}(X)$ consisting of all isomorphism classes of $\kappa$-small well-ordered maps $Y \to X$. In fact,  $\mathcal{W}_{\kappa}(X)$ is a set. This gives rise to a presheaf $$\mathcal{W}_{\kappa} : \sset^{\op} \to \set$$ that sends each simplicial map $f: B \to A$ to the pullback action $f^{\ast} : \mathcal{W}_{\kappa}(A) \to \mathcal{W}_{\kappa}(B)$ on equivalence classes.


\begin{lemma}\label{contW}
For any functor $F: J \to \sset$, $\mathcal{W}_{\kappa}(\colim_{j}F_j) \cong \lim_{j}(\mathcal{W}_{\kappa}(F_j))$.
\end{lemma}
\begin{proof}
Applying $\mathcal{W}_{\kappa}({-})$ to the  colimiting cocone $\left\{v_j : F_j \to \colim_{j}F_j \mid j\in \ob{J} \right\}$ induces a cone over $\mathcal{W}_{\kappa}(F_{\bullet})$
 and thus a canonical map $$\psi : \mathcal{W}_{\kappa}\left(\colim_{j}F_j\right) \to  \lim_{j}\left(\mathcal{W}_{\kappa}(F_j)\right)$$ by the universal property of limits. We want to show that $\psi$ is bijective.
 
 \medskip
 
 To see that $\psi$ is surjective, let $\left[f_j : Y_j \to F_j\right]_{j\in \ob{J}}$ be a tuple of equivalence classes in  $\lim_{j}(\mathcal{W}_{\kappa}(F_j))$. For each simplex $x\in \colim_{j}\left(F_j\right)_n$, choose an index $j_x$ along with a simplex $\tilde{x} \in \left(F_{j_x}\right)_n$ such that $\left(v_{j_x}\right)_n(\tilde{x}) =x.$ Define the fiber $Y_x$ over $x$ as the fiber $\left(Y_{j_x}\right)_{\tilde{x}}$.  For any other such $j_x'$ and $\tilde{x}'$, there is some map $j_x\to j_x'$ in $J$ along with a pullback square
 \[
 \begin{tikzcd}
Y_{j_x} \arrow[r, dashed] \arrow[d, "f_{j_x}"']  \arrow[dr, phantom, "\scalebox{1.5}{\color{black}$\lrcorner$}" , very near start, color=black]
& Y_{j_x'} \arrow[d, "f_{j_x'}"] \\
F_{j_x} \arrow[r]                                & F_{j_x'}                       
\end{tikzcd}
 .\] By the pasting law for pullbacks, the total rectangle
 \[
 \begin{tikzcd}
\left(F_{j_x}\right)_{\tilde{x}} \arrow[d] \arrow[r] \arrow[dr, phantom, "\scalebox{1.5}{\color{black}$\lrcorner$}" , very near start, color=black] & Y_{j_x} \arrow[r, dashed] \arrow[d, "f_{j_x}"'] \arrow[dr, phantom, "\scalebox{1.5}{\color{black}$\lrcorner$}" , very near start, color=black]
& Y_{j_x'} \arrow[d, "f_{j_x'}"] \\
{\Delta[n]} \arrow[r, "\tilde{x}"']                  & F_{j_x} \arrow[r]                               & F_{j_x'}                      
\end{tikzcd}
 \] is a pullback. Since $\tilde{x} =\tilde{x}'$ in $\colim_{j}F_j$,  the uniqueness of pullbacks yields an isomorphism $$\left(F_{j_x}\right)_{\tilde{x}}  \cong \left(F_{j_x'}\right)_{\tilde{x}'}$$ of well-ordered sets. \Cref{uniquewoset} now implies that the fiber $Y_x$ is defined up to \emph{canonical} isomorphism. Thus, we may patch the $f_j$  together to form a $\kappa$-small well-ordered map $f: Y \to \colim_{j}F_j$ such that 
 \[
v_j^{\ast}{f} \cong f_j
 \] for each $j\in \ob{J}$, as desired. 
 
 \medskip
 
 By similar reasoning, we can show that $\psi$ is also injective.
\end{proof}

\smallskip

Consider the opposite Yoneda embedding $\mathcal{Y}^{op} : \varDelta^{\op}  \to \sset^{\op}$, from which we can form the simplicial set
\[
\mathrm{W}_{\kappa} \coloneqq \mathcal{W}_{\kappa} \circ \mathcal{Y}^{op} : \varDelta^{\op} \to \set.
\]

\begin{corollary}
The functor $\mathcal{W}_{\kappa}$ is represented by $\mathrm{W}_{\kappa}$.
\end{corollary}
\begin{proof}
For any $n\in \N$, the Yoneda lemma implies that $$\Hom_{\sset}\left(\Delta[n], \mathrm{W}_{\kappa}\right) \cong \left(\mathrm{W}_{\kappa}\right)_n = \mathcal{W}_{\kappa}\left(\Delta[n]\right).$$
By \cref{psclr}, every simplicial set is naturally isomorphic to a small colimit of standard simplices.  Moreover, by \cref{contW}, $ \mathcal{W}_{\kappa}({-})$ is continuous, and the hom-functor of any locally small category is continuous in its first variable. It follows that $\Hom_{\sset}({-}, \mathrm{W}_{\kappa})$ and $ \mathcal{W}_{\kappa}({-})$ are isomorphic functors, as desired. 
\end{proof}

As a result,  we have a natural isomorphism
\[\label{bigiso}
\Psi : \mathcal{W}_{\kappa}({-})     \overset{\cong}{\longrightarrow} \Hom_{\sset}({-}, \mathrm{W}_{\kappa}).
\tag{$\Psi$}
\]

\begin{notation} $ $
\be
\item For any simplicial set $X$ and $\kappa$-small well-ordered map  $f : Y \to X$, let $\ulcorner{f}\urcorner$ refer to the map $\Psi(\left[f\right]): X \to  \mathrm{W}_{\kappa}$.
\item Let $\left[\Omega_{\kappa} : \widehat{\mathrm{W}}_{\kappa} \to  \mathrm{W}_{\kappa} \right]$ denote the element $\Psi^{-1}\left(\idd_{\mathrm{W}_{\kappa}}\right)$.
\ee
\end{notation}

For every $\kappa$-small well-ordered map $f: Y \to X$,  the map $\Psi$ specifies a commutative square
\[
\begin{tikzcd}
\mathcal{W}_{\kappa}(\mathrm{W}_{\kappa}) \arrow[d, "\ulcorner{f}\urcorner^{\ast}{-}"'] \arrow[r, "\cong"] & {\Hom_{\sset}\left(\mathrm{W}_{\kappa}, \mathrm{W}_{\kappa}\right)} \arrow[d, "{-}\circ \ulcorner{f}\urcorner"] \\
\mathcal{W}_{\kappa}(X) \arrow[r, "\cong"']                                                                & {\Hom_{\sset}\left(X, \mathrm{W}_{\kappa}\right)}                                                              
\end{tikzcd}
\] of sets. By evaluating this square at $\Psi^{-1}\left(\idd_{\mathrm{W}_{\kappa}}\right)$, we see  that there exists a unique pullback square of the form
\[
\begin{tikzcd}
Y \arrow[d, "f"'] \arrow[r]    \arrow[dr, phantom, "\scalebox{1.5}{\color{black}$\lrcorner$}" , very near start, color=black]
       & \widehat{\mathrm{W}}_{\kappa} \arrow[d, "\Omega_{\kappa}"] \\
X \arrow[r, "\ulcorner{f}\urcorner"'] & \mathrm{W}_{\kappa}                                       
\end{tikzcd}.
\] This means that $\Omega_{\kappa}$ is a classifier for the class of all $\kappa$-small well-ordered maps.

\begin{note}\label{weakuniv}
By the axiom of choice, we can choose a well-ordering of  each fiber of any $\kappa$-small simplicial map, thereby converting it into a well-ordered map. This map can be expressed as a pullback of $\Omega_{\kappa}$. As a result, any $\kappa$-small simplicial map can be expressed as a pullback of $\Omega_{\kappa}$. Such a pullback square, however, may \emph{not} be unique, as  our choice of well-orderings need not be unique.
\end{note}

\medskip

We want to isolate the Kan fibrations found in $\mathcal{W}_{\kappa}({-})$. Formally, consider the subpresheaf $$\mathcal{U}_{\kappa} \hookrightarrow \mathcal{W}_{\kappa}$$ such that $\mathcal{U}_{\kappa}(X)$ consists of all $\kappa$-small well-ordered \emph{Kan fibrations} for each simplicial set $X$. Also, let
\[
\mathrm{U}_{\kappa} = \mathcal{U}_{\kappa}\circ \mathcal{Y}^{\op} : \varDelta^{\op} \to \set
\]
and consider the pullback square
\[
\begin{tikzcd}
\widehat{\mathrm{U}}_{\kappa} \arrow[r]  \arrow[dr, phantom, "\scalebox{1.5}{\color{black}$\lrcorner$}" , very near start, color=black]
\arrow[d, "p_{\kappa}"'] & \widehat{\mathrm{W}}_{\kappa} \arrow[d, "\Omega_{\kappa}"] \\
\mathrm{U}_{\kappa} \arrow[r, hook]                              & \mathrm{W}_{\kappa}                                       
\end{tikzcd}.
\]

\begin{prop}\label{univfib}
The map $p_{\kappa} : \widehat{\mathrm{U}}_{\kappa} \to \mathrm{U}_{\kappa}$ is a Kan fibration.\footnote{\autocite[Lemma 2.1.10]{KL}.}
\end{prop}
\begin{comment}
\begin{proof}
We need to solve any lifting problem of the form
\[
\begin{tikzcd}
{\Lambda^k[n]} \arrow[r] \arrow[d, hook]        & \widehat{\mathrm{U}}_{\kappa}\arrow[d, "p_{\kappa}"] \\
{\Delta[n]} \arrow[r, "\ulcorner{x}\urcorner"'] & \mathrm{U}_{\kappa}                                  
\end{tikzcd}.
\]
By the universal property of pullback squares, we have a commutative diagram of the form
\[
\begin{tikzcd}[row sep=large]\label{twopulls}
{\Lambda^k[n]} \arrow[d, hook] \arrow[r, dashed] & Z \arrow[r] \arrow[d, "x"]       \arrow[dr, phantom, "\scalebox{1.5}{\color{black}$\lrcorner$}" , very near start, color=black]                &\widehat{\mathrm{U}}_{\kappa} \arrow[d, "p_{\kappa}"] \\
{\Delta[n]} \arrow[r, equal]                            & {\Delta[n]} \arrow[r, "\ulcorner{x}\urcorner"'] &  \mathrm{U}_{\kappa}                                             
\end{tikzcd}. \tag{$\ast$}
\]
\begin{claim}
The map $x: Z \to \Delta[n]$ is a Kan fibration.
\end{claim}
\begin{proof}
\textit{To appear.}
\end{proof}
This means that there is a diagonal fill-in $\Delta[n] \to Z$ for the lefthand square of \eqref{twopulls}. Then the composite $\Delta[n] \to Z \to \widehat{\mathrm{U}}_{\kappa}$ is a solution to our original lifting problem. 
\end{proof}
\end{comment}


\begin{lemma}\label{fibfactors}
Let $f: Y \to X$ be a $\kappa$-small well-ordered map. Then $f$ is a Kan fibration if and only if $\ulcorner{f}\urcorner : X \to  \mathrm{W}_{\kappa}$ factors through the inclusion $\mathrm{U}_{\kappa} \hookrightarrow \mathrm{W}_{\kappa}$. 
\end{lemma}
\begin{proof} $ $
\smallskip

($\Longrightarrow$)  Suppose that $f$ is a Kan fibration.  For any $n$-simplex $x : \Delta[n] \to X$ in $X$, the pullback $$x^{\ast}{f} : \Delta[n] \times_X Y \to \Delta[n]$$ is also a Kan fibration by \cref{CP}. Further, this map is $\kappa$-small and well-ordered because  $\im\left(\Delta[n] \times_X Y \to Y\right)$ is exactly the fiber of $f$ over $x \in X_n$.  Now, by pasting the two pullback squares
\[
\begin{tikzcd}[row sep=large]
{\Delta[n]\times_XY} \arrow[d, "x^{\ast}{f}"'] \arrow[r] & Y \arrow[d, "f"] \arrow[r]    \arrow[dr, phantom, "\scalebox{1.5}{\color{black}$\lrcorner$}" , very near start, color=black]            & \widehat{\mathrm{W}}_{\kappa} \arrow[d, "\Omega_{\kappa}"] \\
{\Delta[n]} \arrow[r, "x"']                              & X \arrow[r, "\ulcorner{f}\urcorner"'] & \mathrm{W}_{\kappa}                                       
\end{tikzcd}, 
\] we see that $ \ulcorner{f}\urcorner \circ x = \ulcorner{x^{\ast}{f}}\urcorner$ as the total rectangle must be a pullback as well. But $ \ulcorner{x^{\ast}{f}}\urcorner \in \left(\mathrm{U}_{\kappa}\right)_n$ because $x^{\ast}{f}$ is a Kan fibration. Hence  $ \ulcorner{f}\urcorner$ sends each simplex in $X$ to a simplex  in $\mathrm{U}_{\kappa}$ and thus factors through $\mathrm{U}_{\kappa} \hookrightarrow \mathrm{W}_{\kappa}$.

\medskip

($\Longleftarrow$) Suppose that $\ulcorner{f}\urcorner$ factors through $\mathrm{U}_{\kappa} \hookrightarrow \mathrm{W}_{\kappa}$. This yields a commutative diagram of the form
\[
\begin{tikzcd}[row sep=large]
Y \arrow[d, "f"'] \arrow[r]                                  & \widehat{\mathrm{U}}_{\kappa} \arrow[d, "p_{\kappa}"] \arrow[r] 
 \arrow[dr, phantom, "\scalebox{1.5}{\color{black}$\lrcorner$}" , very near start, color=black]  
  & \widehat{\mathrm{W}}_{\kappa} \arrow[d, "\Omega_{\kappa}"] \\
X \arrow[r] \arrow[rr, "\ulcorner{f}\urcorner"', bend right] & \mathrm{U}_{\kappa} \arrow[r, hook]                             & \mathrm{W}_{\kappa}                                       
\end{tikzcd}
.\] Since the total rectangle is also a pullback, so is the lefthand square. By \cref{univfib} together with \cref{CP}, we thus have that $\ulcorner{f}\urcorner$ is a Kan fibration. 
\end{proof}

\begin{corollary}\label{Urep}
The functor $\mathcal{U}_{\kappa}$ is represented by $\mathrm{U}_{\kappa}$.
\end{corollary}
\begin{proof}
It follows directly  from \cref{fibfactors} that \eqref{bigiso} restricts to an isomorphism
\[
\mathcal{U}_{\kappa}({-})     \overset{\cong}{\longrightarrow} \Hom_{\sset}({-}, \mathrm{U}_{\kappa}).
\]
\end{proof}

Consider the pullback square
\[
\begin{tikzcd}
Y \arrow[d, "f"'] \arrow[r]    \arrow[dr, phantom, "\scalebox{1.5}{\color{black}$\lrcorner$}" , very near start, color=black]
       & \widehat{\mathrm{W}}_{\kappa} \arrow[d, "\Omega_{\kappa}"] \\
X \arrow[r, "\ulcorner{f}\urcorner"'] & \mathrm{W}_{\kappa}                                       
\end{tikzcd}.
\] 
from above. If $f$ is a Kan fibration, then by \cref{fibfactors} we obtain a pasting of pullback squares
\[
\begin{tikzcd}[row sep=large]
Y \arrow[d, "f"'] \arrow[r, dashed]       \arrow[dr, phantom, "\scalebox{1.5}{\color{black}$\lrcorner$}" , very near start, color=black]
                            & \widehat{\mathrm{U}}_{\kappa} \arrow[r] \arrow[d, "p_{\kappa}"] 
                             \arrow[dr, phantom, "\scalebox{1.5}{\color{black}$\lrcorner$}" , very near start, color=black]
                             & \widehat{\mathrm{W}}_{\kappa} \arrow[d, "\Omega_{\kappa}"] \\
X \arrow[rr, "\ulcorner{f}\urcorner"', bend right] \arrow[r] & \mathrm{U}_{\kappa} \arrow[r, hook]                             & \mathrm{W}_{\kappa}                                       
\end{tikzcd}.
\] This shows that $p_{\kappa}$ is a classifier for the class of all $\kappa$-small well-ordered Kan fibrations. Further, by the axiom of choice, any $\kappa$-small simplicial map can be expressed as a (not necessarily unique) pullback of $p_{\kappa}$. 

\bigskip

In this way, $\mathrm{U}_{\kappa}$ is a universe in the sense of \cref{univ1}. Any closed type in our model will be a fibration of the form $ \boldsymbol{\cdot} \to 1$. Hence if we want $\mathrm{U}_{\kappa}$ (or a smaller copy thereof) to serve also as an internal universe, we must show that it is fibrant.
For this, the following result due to Joyal will be useful.

\begin{lemma}\label{Quillem2}
Let $j : A \to B$ be a cofibration and $p: C \to A$ be a trivial fibration of simplicial sets. 
\be[label=(\alph*)]
\item There exists a pullback square of the form 
\[
\begin{tikzcd}
C \arrow[d, "p"'] \arrow[r] 
\arrow[dr, phantom, "\scalebox{1.5}{\color{black}$\lrcorner$}" , very near start, color=black]
& D \arrow[d, "\tilde{p}"] \\
A \arrow[r, "j"']           & B                       
\end{tikzcd}
\] where $\tilde{p}$ is a trivial fibration.
\item If $p$ is $\kappa$-small, then  $\tilde{p}$ can be made $\kappa$-small.
\ee
\end{lemma}
\begin{proof}  $ $
\be[label=(\alph*)]
\item Let $\tilde{p} = \Pi_j{p}$. This a trivial fibration by \cref{adjpres}(2). It remains to show that $j^{\ast}{\Pi_j{p}}\cong p$. To this end, recall  the adjoint triple
\[
\Sigma_j({-}) \dashv j^{\ast}({-}) \dashv \Pi_j({-}).
 \]This induces an adjunction
 \[
 j^{\ast} \circ \Sigma_j \dashv  j^{\ast} \circ \Pi_j .
 \] Since adjoints are unique up to isomorphism and $\idd_{\sset/A}$ is right adjoint to itself, it suffices to exhibit a natural isomorphism $ j^{\ast} \circ \Sigma_j \overset{\cong}{\longrightarrow} \idd_{\sset/A}$. For any simplicial map $h : E\to A$, it is easy to check that the  square
 \[
 \begin{tikzcd}
E \arrow[d, "h"'] \arrow[r, equal] & E \arrow[d, "j\circ h"] \\
A \arrow[r, "j"']           & B                      
\end{tikzcd}
\] is a pullback because $j$ is monic. Hence $j^{\ast}{\Sigma_j{h}} \cong h$, from which we can define our desired isomorphism.
 
\item Let $x : \Delta[n] \to B$ be any simplex in $B$. Note that 
\[ \label{fiberchain}
\left(\Pi_j{p}\right)_x \cong \Hom_{\sset/B}(x, \Pi_j{p}) \cong \Hom_{\sset/A}(j^{\ast}{x}, p). \tag{$\ast$}
\] Note that $j^{\ast}{\Delta[n]}$ is a simplicial subset of $\Delta[n]$ because monomorphisms are stable under pullback. Now, recall that the non-degenerate $k$-simplices in $\Delta[n]$ are precisely the monomorphisms belonging to $\varDelta\left(\left[k\right], \left[n\right]\right)$. In particular, $\Delta[n]$ is finite (\cref{finsimp}), and thus so is $j^{\ast}{\Delta[n]}$. Moreover, \cref{E-Z} implies that any simplicial map $X\to Y$ is determined by its action on all non-degenerates simplices in $X$. We thus can find an embedding of $\Hom_{\sset/A}(j^{\ast}{x}, p)$ into the finite product $$P\coloneqq \prod_{\substack{z \text{ non-deg.}\\ \text{simplex in} \\ \text{$j^{\ast}{\Delta[n]}$}}}p^{-1}\left(j^{\ast}{x}(z)\right)$$ of fibers of $p$. Assuming that $p$ is $\kappa$-small, we have that $\left\lvert{P}\right\rvert <\kappa$ by basic cardinal arithmetic. In this case, it follows that $\Pi_j{p}$ is $\kappa$-small by \eqref{fiberchain}.
\ee
\end{proof} 


\begin{theorem}\label{UKan}
The simplicial set $\mathrm{U}_{\kappa}$ is a Kan complex.
\end{theorem}
\begin{proof}
Let $n\in \Z_{\geq 1}$. For each integer $0\leq k \leq n$, we must find a filler of the form
\[
\begin{tikzcd}
{\Lambda^k[n]} \arrow[d, hook] \arrow[r, "\ulcorner{q}\urcorner"] & \mathrm{U}_{\kappa} \\
{\Delta[n]} \arrow[ru, dashed]                                    &                    
\end{tikzcd}
.\] Thanks to \cref{Urep}, the map $\ulcorner{q}\urcorner$ naturally corresponds to a $\kappa$-small well-ordered map $q : Z\to \Lambda^k[n]$. It suffices to find a pullback square of the form
\[
\begin{tikzcd}
Z \arrow[r, dashed] \arrow[d, "q"'] 
\arrow[dr, phantom, "\scalebox{1.5}{\color{black}$\lrcorner$}" , very near start, color=black]
& Z' \arrow[d, "q'", dashed] \\
{\Lambda^k[n]} \arrow[r, hook]      & {\Delta[n]}               
\end{tikzcd}
\] such that  $q'$ is a $\kappa$-small well-ordered Kan fibration and the map $Z \to Z'$  induces an order-preserving function  $q^{-1}(x) \to \left(q'\right)^{-1}(x)$ for any simplex $x$ in $\Lambda^k[n]$. For, in this case, we can form a pasting of two pullback squares
\[
\begin{tikzcd}[row sep=large]
Z \arrow[d, "q"'] \arrow[r]    
\arrow[dr, phantom, "\scalebox{1.5}{\color{black}$\lrcorner$}" , very near start, color=black] & Z' \arrow[d, "q'", dashed] \arrow[r]             
\arrow[dr, phantom, "\scalebox{1.5}{\color{black}$\lrcorner$}" , very near start, color=black]
& \mathrm{U}_{\kappa} \arrow[d, "p_{\kappa}"] \\
{\Lambda^k[n]} \arrow[r, hook] & {\Delta[n]} \arrow[r, "\ulcorner{q'}\urcorner"'] & \widehat{\mathrm{U}}_{\kappa}              
\end{tikzcd}
\] where the lower composite must equal $\ulcorner{q}\urcorner$ because $p_{\kappa}$ is a classifier. Hence $\ulcorner{q'}\urcorner$ would serve as our desired filler. 

\medskip

By \cref{Quillen}, we can factor $q$  as a trivial fibration $q_t :Z \to W$ followed by a minimal fibration $q_m: W \to \Lambda^k[n]$.
\begin{claim}
Both  $q_t$ and $q_m$ are $\kappa$-small.
\end{claim}
\begin{proof}
First, to see that $q_t$ is $\kappa$-small, note that
\[
q_{t}^{-1}(w) \subseteq\left(q_{m} \circ q_{t}\right)^{-1}\left(q_{m}(w)\right)=q^{-1}\left(q_{m}(w)\right)
\] for any simplex $w$ in $W$. As $q$ is $\kappa$-small, it follows that $q_t$ is also $\kappa$-small.

\medskip

Next, to see that $q_m$ is $\kappa$-small, note that $q_t$ is levelwise surjective as a trivial fibration. Therefore, for any simplex $\ell$ in $\Lambda^k[n]$, every element of $\left(q_m\right)^{-1}(\ell)$ has the form $q_t(z)$ for some simplex $z$ in $Z$. Then $q$ sends $z$ to $\ell$, which shows that the function $q^{-1}(\ell) \to   \left(q_m\right)^{-1}(\ell)$ given by $x\mapsto q_t(x)$ is surjective. Hence $q_m$ is $\kappa$-small.
\end{proof}
By \cref{trivifibbund}, the map $q_m$ is isomorphic to the trivial bundle $F \times \Lambda^k[n] \overset{\pi_2}{\longrightarrow} \Lambda^k[n]$. This yields a pullback square
\[
\begin{tikzcd}
W \arrow[d, "q_m"'] \arrow[r, hook] 
\arrow[dr, phantom, "\scalebox{1.5}{\color{black}$\lrcorner$}" , very near start, color=black]
& {F\times \Delta[n]} \arrow[d, "\pi_{\Delta[n]}"] \\
{\Lambda^k[n]} \arrow[r, hook]         & {\Delta[n]}                           
\end{tikzcd}
.\] Since $q_m$ is $\kappa$-small, so is the trivial bundle $\pi_{\Delta[n]}$. Further, since $q_t$ is $\kappa$-small,  \cref{Quillem2} provides us with  a $\kappa$-small Kan fibration $\tilde{q}_t$ fitting into a commutative diagram
\[
\begin{tikzcd}[row sep=large]
Z \arrow[d, "q_t"'] \arrow[r] \arrow[dd, "q"', bend right=49] 
\arrow[dr, phantom, "\scalebox{1.5}{\color{black}$\lrcorner$}" , very near start, color=black]
& Z' \arrow[d, "\tilde{q}_t"]            \\
W \arrow[d, "q_m"'] \arrow[r, hook]              
\arrow[dr, phantom, "\scalebox{1.5}{\color{black}$\lrcorner$}" , very near start, color=black]
             & {F\times \Delta[n]} \arrow[d, "\pi_{\Delta[n]}"] \\
{\Lambda^k[n]} \arrow[r, hook]                                & {\Delta[n]}                           
\end{tikzcd}
.\]
Take $q'$ to be $\pi_{\Delta[n]} \circ \tilde{q}_t$, which is a Kan fibration as the composite of two fibrations. To see that $q'$ is $\kappa$-small, observe that for any simplex $z$ in $\Delta[n]$, we have that
\[
\left(q^{\prime}\right)^{-1}\left(z\right)=\left(\pi_{\Delta[n]} \circ \tilde{q}_t\right)^{-1}\left(z\right)=\bigcup_{w \in\left(\pi_{\Delta[n]}\right)^{-1}\left(z\right)}\left(\tilde{q}_{t}\right)^{-1}\left(w\right)
.\] Since $\kappa$ is regular, it follows that $\left\lvert{\left(q^{\prime}\right)^{-1}\left(z\right)}\right\rvert < \kappa$. Thus, $q'$ is $\kappa$-small.

\medskip



Finally, we must extend the well-ordering of $q$ to a well-ordering of $q'$. This is possible because any well-founded binary relation $R$ on a set $Q$ can be extended to a well-ordering of $Q$. Indeed, consider  the rank function $\rank_R: Q\to \alpha$, defined inductively by 
\[
\rank_R(x) = \sup{\left\{\rank_R(y)+1\mid {yRx}\right\}},
\] where $\alpha$ is an ordinal. By induction on $\rank_R$ together with the axiom of choice, we can extend $R$ to a well-ordering of $Q$.
\end{proof} 

\subsection{Modeling MLDTT without $\univ$}\label{smod}

\begin{notation}
Continue to let $\T$ denote our MLDTT without $\univ$.
\end{notation}

\[
\text{Assume now that $\kappa$ is inaccessible.}
\]

\smallskip

In this section, we verify that $\mathrm{U}_{\kappa}$ has sufficient logical structure to induce a model of $\T$ via \cref{univlog}. In the interest of space, we shall describe just the $\Pi$- and $\id$-structure on $\mathrm{U}_{\kappa}$ along with smaller copies of  $\mathrm{U}_{\kappa}$ serving as nested universes in  $\mathrm{U}_{\kappa}$. See \cite[Theorem 2.3.4]{KL} for sketches of the remaining cases. 


\bigskip

Recall from page~\pageref{depprods} that  an $\Pi$-structure on $\mathrm{U}_{\kappa}$ is precisely a map $$\bar{\Pi} : \Pi\left(\mathrm{U}_{\kappa}\right) \to \mathrm{U}_{\kappa}$$ together with an isomorphism $\bar{\Pi}^{\ast}{p_{\kappa}}\cong \Pi_{\alpha_g}\beta_g$. By \cref{Urep}, both $\alpha_g$ and $\beta_g$ are  $\kappa$-small Kan fibrations.

\begin{prop}\label{depprodsmall}
Let $h : X \to Y$ and $f: Y \to Z$ be $\kappa$-small Kan fibrations. Then the dependent product $\Pi_f{h}$ is a $\kappa$-small Kan fibration.  
\end{prop}
\begin{proof}
 \Cref{adjpres}(4) immediately implies that $\Pi_f{h}$ is a Kan fibration. To see that it is $\kappa$-small,  let $n\in \N$ and notice from \eqref{fiberchain} that
 \[
\left(\Pi_f{h}\right)_x \cong \Hom_{\sset/Y}(f^{\ast}{x}, h)
 \] for any simplex $x: \Delta[n] \to Z$ in $Z$. The cardinality of $\Hom_{\sset/Y}(f^{\ast}{x}, h)$ is at most the cardinality of the set $S$ of all functions
 \[
f_n^{-1}(x) \to \left(f\circ h\right)_n^{-1}(x)
 .\] As $\kappa$ is regular, both $f_n^{-1}(x)$ and $\left(f\circ h\right)_n^{-1}(x) $ have cardinality $<\kappa$. Hence $$\left\lvert{\left(\Pi_f{h}\right)_x}\right\rvert \leq \left\lvert{S}\right\rvert <\kappa$$ because $\kappa$ is inaccessible.
 \end{proof}

We have seen that any $\kappa$-small simplicial map can be expressed as a pullback of $p_{\kappa}$. Thus, by \cref{depprodsmall}, the dependent product $\Pi_{\alpha_g}\beta_g$ can be expressed as a pullback $\bar{\Pi}^{\ast}{p_{\kappa}}$, as required.

\bigskip

Next, we want to define an $\id$-structure on $\mathrm{U}_{\kappa}$. For any Kan fibration $p:E \to B$, the \textit{fibered path space object} of $p$ is the pullback
\[
\begin{tikzcd}
\rho_B(E) \arrow[d] \arrow[r] 
\arrow[dr, phantom, "\scalebox{1.5}{\color{black}$\lrcorner$}" , very near start, color=black]
& 
{E^{\Delta[1]}} \arrow[d, "{p^{\Delta[1]}}"] \\
B \arrow[r, "c_B"']             & {B^{\Delta[1]}}                             
\end{tikzcd}
\] of the exponential object $E^{\Delta[1]}$ along the constant path map. We have a unique mediating map
\[
\begin{tikzcd}
E \arrow[rdd, "p"', bend right] \arrow[rrd, "c_E", bend left] \arrow[rd, "r_p", dashed]  &                               &                                              \\
                                                                                        & \rho_B(E) \arrow[d] \arrow[r] & {E^{\Delta[1]}} \arrow[d, "{p^{\Delta[1]}}"] \\
                                                                                        & B \arrow[r, "c_B"']           & {B^{\Delta[1]}}                             
\end{tikzcd}.
\] We also have composites 
\begin{align*}
s_p &: \rho_B(E) \to E^{\Delta[1]} \xrightarrow{\ev_0} E
\\ t_p &: \rho_B(E) \to E^{\Delta[1]} \xrightarrow{\ev_1} E.
\end{align*}

\begin{prop}\label{fibpathspace}
The diagonal map $\Delta_p : E \to E\times_B E$ factors as
\[
\begin{tikzcd}[column sep=large]
E \arrow[r, "r_p"] & \rho_B(E) \arrow[r, "{\left(s_p,t_p\right)}"] & E\times_BE
\end{tikzcd}
\] over $B$ such that
\bi
\item $r_p$ is a trivial cofibration, 
\item $\left(s_p,t_p\right)$ is a Kan fibration, and
\item $r_p$ is stably orthogonal to $\left(s_p,t_p\right)$ over $B$.\footnote{\autocite[Proposition 2.3.3]{KL}.}
\ei
\end{prop}


\smallskip

If $p: E \to B$ is $\kappa$-small, then so is $\left(s_p, t_p\right)$ because $r_p$ is monic. In this case, there is a pullback square of the form
\[
\begin{tikzcd}
\rho_B(E) \arrow[r] \arrow[d, "{\left(s_p, t_p\right)}"'] \arrow[dr, phantom, "\scalebox{1.5}{\color{black}$\lrcorner$}" , very near start, color=black]
& E \arrow[d, "p"] \\
E\times_BE \arrow[r, "\overline{\id}_p"']                                      & B               
\end{tikzcd}.
\] 
Specifically,  the Kan fibration $p_{\kappa}$ is $\kappa$-small. Thus, we may take $\left(\overline{\id}_{p_{\kappa}}, r_{p_{\kappa}}\right)$ as our $\id$-structure. 

\bigskip

Finally, to define a $\U$-structure on $\mathrm{U}_{\kappa}$, suppose that $\lambda < \kappa$ is another inaccessible cardinal. Then $\mathrm{U}_{\lambda}$ is again a Kan complex. Also, the unique map $\mathrm{U}_{\lambda} \to 1$ is $\kappa$-small, i.e., the set
\[
\left(\mathrm{U}_{\beta}\right)_n  = \mathcal{U}_{\lambda}(\Delta[n])
\] has cardinality $<\kappa$ for each $n\in \N$. Indeed, each isomorphism class $\left[f: X\to \Delta[n]\right]\in  \mathcal{U}_{\lambda}(\Delta[n])$ is determined by 
\be[label=(\roman*)]
\item a family of isomorphism classes $\left[f^{-1}(z)\right]$ over $\Delta[n]$ indexed by the countable set of all simplices $z$ in $\Delta[n]$ together with
\item  all of the face and degeneracy operators between the total spaces of $k$-simplices ($k\in \N$) induced by the fibers of $f$.
\ee
Since every well-ordered set is isomorphic to exactly one ordinal, there are exactly $\lambda$ many isomorphism classes of well-ordered sets of size $< \lambda$. Hence there are exactly $\aleph_0 \cdot \lambda = \lambda$ many possible families of isomorphism classes as in (i). Moreover, the domain and codomain of any  face or degeneracy operator as in (ii) are of size $<\lambda$. Thus, there are exactly $\lambda$ many ways of defining such an operator because $\lambda$ is inaccessible. Also, there are countably many face and degeneracy operators to define in total. Therefore, there are exactly $\aleph_0 \cdot \lambda = \lambda$ many possible face and degeneracy operators as in (ii). We can conclude that 
\[
 \left\lvert{ \mathcal{U}_{\lambda}(\Delta[n])}\right\rvert \leq \lambda \cdot \lambda = \lambda <\kappa
.\] 

\smallskip

We thus have a unique pullback square of the form
\[
\begin{tikzcd}
\mathrm{U}_{\lambda} \arrow[r] \arrow[d] 
\arrow[dr, phantom, "\scalebox{1.5}{\color{black}$\lrcorner$}" , very near start, color=black]
& \widehat{\mathrm{U}}_{\kappa} \arrow[d, "p_{\kappa}"] \\
1 \arrow[r, "u_{\lambda}"']                              & \mathrm{U}_{\kappa}                    
\end{tikzcd}
\] as well as an inclusion map 
\[
\iota : \mathrm{U}_{\lambda} \to \mathrm{U}_{\kappa}, \ \quad \left[f\right] \mapsto \left[f\right]
.\] Finally, we take the pair $\left(u_{\lambda}, \iota\right)$ as  our $\U$-structure.

\subsection*{Generic structure for modeling $\cdtt$}

We can check that any LCCC $\c$ carries all data of a $\T$-structure (forgetting its contextual-categorical structure) aside from an $\id$-type structure (p.~\pageref{idtype}). It is possible, however, to \emph{almost} correctly interpret \emph{extensional} identity types (\cref{exttypes}) in $\c$. Indeed, suppose that we can derive 
\begin{gather*}
\Gamma \vdash A \type
\\ \Gamma \vdash a :A
\\ \Gamma \vdash b:A
\end{gather*} 
in $\T$. Then the well-formed identity type $\Gamma \vdash \id_A(a,b)\type $ is interpreted as the equalizer in the diagram
\[
\left\llbracket{\Gamma, z:\id_A(a,b)}\right\rrbracket   \overset{p_{\id_A(a,b)}}{ \xdasharrow{\hspace*{2cm}}}  \left\llbracket{\Gamma}\right\rrbracket   \doublerightarrow{\left\llbracket{a}\right\rrbracket}{\left\llbracket{b}\right\rrbracket}    \left\llbracket{\Gamma, x:A}\right\rrbracket
.\] As a result, the canonical projection $p_{\id_A(a,b)}$ must be monic. Further, we interpret the canonical term $\Gamma \vdash \refl_a:\id_A(a,a)$ as the morphism $\left\llbracket{\Gamma}\right\rrbracket \to \left\llbracket{\Gamma, z:\id_A(a,b)}\right\rrbracket$  induced by the universal property of equalizers, i.e., fitting into a commutative triangle
\[
\begin{tikzcd}[column sep=large, row sep=large]
{\left\llbracket{\Gamma, z:\id_A(a,a)}\right\rrbracket} \arrow[r, "{p_{\id_A(a,a)}}"]                                                        & \left\llbracket{\Gamma}\right\rrbracket \\
\left\llbracket{\Gamma}\right\rrbracket \arrow[ru, "\idd_{\left\llbracket{\Gamma}\right\rrbracket}"'] \arrow[u, "\left\llbracket{\refl_a}\right\rrbracket", dashed] &                             
\end{tikzcd}.
\] In general, our interpretation function $\left\llbracket{-}\right\rrbracket$ sends any well-formed term of type $\id_A(a,b)$ to a section of $p_{\id_A(a,b)}$. Since  $p_{\id_A(a,b)}$ is monic, it follows that $\c$ satisfies $\uip$. To see that it satisfies $\err$, recall that any morphism that is both a monomorphism and a split epimorphism is an isomorphism. Therefore, $p_{\id_A(a,b)}$ is an isomorphism, so that $\left\llbracket{a}\right\rrbracket = \left\llbracket{b}\right\rrbracket$, as desired.

\medskip

The reason that such an interpretation is \emph{almost} correct is that endowing $\c$ with a strictly functorial pullback operation may be impossible. One can, however, convert $\c$ into an  equivalent category with attributes, which has such an operation \autocite{Hofmann}. In this sense, every LCCC admits a model of extensional $\cdtt$. 

\medskip

Moreover, thanks to \autocite[Theorem 7.10]{GK}, we know that every locally presentable LCC $\infty$-category $\c$ has a presentation by a type-theoretic model category $\underline{\c}$.\footnote{It is hoped that, eventually, we can drop the hypothesis that $\c$ is locally presentable.} Consider the full subcategory $\underline{\c}_{\mathbf{f}}$ of $\underline{\c}$ on all fibrant objects.

\begin{prop}\label{genmodel}
The class $\fib$ of all fibrations in $\underline{\c}_{\mathbf{f}}$ is both closed and factorizing. (See \cref{PrUn}.)
\end{prop}
\begin{proof}
Note that $\fib$ is factorizing because $\underline{\c}$ is a model category. For the same reason, $\fib$ satisfies condition (b) of \cref{closedmaps}. It also satisfies condition (c) because all objects of $\underline{\c}_{\mathbf{f}}$ are fibrant.

\medskip

To see that $\fib$ satisfies condition (d) of \cref{closedmaps}, note that $\fib$ is closed under pullbacks and that the pullback of a fibrant object in any model category is again fibrant. Thus, for every map $f: A \to B$ in $\underline{\c}_{\mathbf{f}}$, we have a base change functor $f^{\ast} : \underline{\c}_{\mathbf{f}}/B \to \underline{\c}_{\mathbf{f}}/A$ preserving all fibrations.  Since  $\underline{\c}$ is locally cartesian closed by definition of \textit{type-theoretic model category}, it follows that $f^{\ast}$ has a right adjoint $\Pi_f$. We must show that $\Pi_f$ preserves fibrations. But $f^{\ast}$ preserves trivial cofibrations by \cref{trivcofp}.

\medskip

It remains to show that $\fib$ satisfies condition (e) of \cref{closedmaps}. By \cref{adjpres}(4) together with  \cref{inthom} and the fact that fibrations are stable under pullback, we have that the exponential of two fibrations over $C\in \ob{\underline{\c}_{\mathbf{f}}}$ is again a fibration. It is easy to see that this remains the exponential in $\fib(C)$.  Hence the inclusion functor $\fib(C) \hookrightarrow \underline{\c}_{\mathbf{f}}/C$ preserves exponentials.
\end{proof}

Assuming \cref{initial}, \cref{Awthm} directly implies that $\underline{\c}_{\mathbf{f}}$ models $\T$ (without the universe type). In this sense, every locally presentable LCC $\infty$-category $\c$ can be presented by a model of intensional $\cdtt$. In particular, $\left[\d, \sset\right]$ has such a presentation for any small category $\d$.


\subsection{The simplicial notion of univalence}\label{suniv}

At this point, let us turn to proving that the induced contextual category $\sset_{\mathrm{U}_{\kappa}}$ (p.~\pageref{indcxtcat}) satisfies the univalence axiom. To begin with, we define a simplicial notion of univalence that will be equivalent to our type-theoretic one.

\bigskip

Let $p_1 : E_1\to B$ and $p_2 : E_2 \to B$ be Kan fibrations. The over category $\sset/B$ is cartesian closed by \cref{slicecc}. Thus, we may form the exponential $$p_2^{p_1}: \mathbf{hom}_B(E_1, E_2) \to B$$ of $p_1$ and $p_2$ in $\sset/B$. 

\begin{note}
The map $p_2^{p_1}$ is a Kan fibration.
\end{note}

\smallskip

By adjunction, any map 
\[
\begin{tikzcd}[row sep=large]
X \arrow[rd, "f"'] \arrow[r] & {\mathbf{hom}_B(E_1, E_2)} \arrow[d, "p_2^{p_1}"] \\
                             & B                                                
\end{tikzcd}
\] over $B$ naturally corresponds to a map $f^{\ast}{E_1} \to E_2$ over $B$. This, in turn, naturally corresponds to a map $f^{\ast}{E_1} \to f^{\ast}{E_2}$ fitting into a commutative diagram
\[
\begin{tikzcd}
f^{\ast}{E_1} \arrow[rrd, bend left] \arrow[rdd, "\pi"', bend right] \arrow[rd, dashed] &                                   &                      \\
                                                                                        & f^{\ast}{E_2} \arrow[d] \arrow[r] & E_2 \arrow[d, "p_2"] \\
                                                                                        & X \arrow[r, "f"']                 & B                   
\end{tikzcd}
.\] By the Yoneda lemma, it follows that an $n$-simplex $x$ in $\mathbf{hom}_B(E_1, E_2)$ is precisely a pair  of maps $$\left(x: \Delta[n]\to B, s_x : x^{\ast}{E_1} \to x^{\ast}{E_2}\right).$$

\begin{lemma}\label{alsowe}
Let $g : E_1 \to E_2$ be a weak equivalence in $\sset/B$ and let $h: B' \to B$ be a simplicial map. Then the map $ h^{\ast}{g}: h^{\ast}{E_1} \to h^{\ast}{E_2}$ is a weak equivalence in $\sset/B'$.
\end{lemma}
\begin{proof}
By \cref{prestrvifib}, the base change functor $h^{\ast}$ preserves trivial fibrations. Therefore, \cref{KB} implies that $h^{\ast}$ preserves weak equivalences of fibrant objects. In particular, $h^{\ast}{g}$ is a weak equivalence.
\end{proof}

\begin{prop}\label{thenweq}
Let  $g : E_1 \to E_2$ be a map over $B$. If every connected component of $B$ has a vertex $v: \Delta[0] \to B$ such that the induced map $v^{\ast}{g}: v^{\ast}{E_1} \to v^{\ast}{E_2}$ of fibers is a weak equivalence, then $g$ is also a weak equivalence.\footnote{\autocite[Lemma 3.2.7]{KL}.}
\end{prop}

\smallskip

Consider any map $f: \left[n\right] \to \left[m\right]$ in $\varDelta$ along with induced map $\mathcal{Y}(f) : \Delta[n] \to \Delta[m]$. Then the function $$\mathbf{hom}_B(E_1, E_2)(f) : \mathbf{hom}_B(E_1, E_2)_m \to \mathbf{hom}_B(E_1, E_2)_n$$ is given by 

\[
\left(x: \Delta[m] \to B, s_x: x^{\ast}{E_{1}} \to x^{\ast}{E_{2}}\right) \ \mapsto \ \left(x \circ \mathcal{Y}(f), s_{x \circ \mathcal{Y}(f)} : \left(x \circ \mathcal{Y}(f)\right)^{\ast}{E_{1}} \to \left(x \circ \mathcal{Y}(f)\right)^{\ast}{E_{2}}\right).
\]
If $s_x$ is a weak equivalence, then so is
\[
s_{x \circ \mathcal{Y}(f)} : \mathcal{Y}(f)^{\ast}\left(x^{\ast}{E_{1}}\right) \to  \mathcal{Y}(f)^{\ast}\left(x^{\ast}{E_{2}}\right).
\] by \cref{alsowe}. Thus, we have a simplicial subset $\mathbf{eq}_B(E_1, E_2) \subset \mathbf{hom}_B(E_1, E_2)$ whose $n$-simplices are exactly pairs of maps
\begin{gather*}
\left(x: \Delta[n]\to B, s_x : x^{\ast}{E_1} \to x^{\ast}{E_2}\right)
\\ s_x \text{ is a weak equivalence}.
\end{gather*}

\smallskip

Now, consider any map $$\left(f:X \to B, s_f : f^{\ast}{E_1} \to f^{\ast}{E_2}\right) : X \to  \mathbf{hom}_B(E_1, E_2)$$ over $B$. This sends any $x\in X_n$ to the $n$-simplex $\left(f\circ x, x^{\ast}{s_f}\right)$ in $\mathbf{hom}_B(E_1, E_2)$. If $x^{\ast}{s_f}$ is a weak equivalence for every simplex $x$ in $X$, then $s_f$ must be a weak equivalence by \cref{thenweq}. Conversely, if $s_f$ is a weak equivalence, then any such  map $x^{\ast}{s_f}$ must be a weak equivalence because the base change functor $x^{\ast}$ preserves weak equivalences of fibrant objects. In conclusion, the map $s_f$ is a weak equivalence if and only if $\left(f,s_f\right)$ factors through $\mathbf{eq}_B(E_1, E_2) \hookrightarrow  \mathbf{hom}_B(E_1, E_2)$. As a result, any map $X \to \mathbf{eq}_B(E_1, E_2)$ corresponds naturally to a pair of maps
\begin{gather*}
\label{equniv} \left(f: X\to B, s_f : f^{\ast}{E_1} \to f^{\ast}{E_2}\right) \tag{$\bigstar$}
\\ s_f \text{ is a weak equivalence}.
\end{gather*}



\begin{lemma}\label{restKfib}
The restriction $p_2^{p_1}: \mathbf{eq}_B(E_1, E_2)\to B$ is a Kan fibration.
\end{lemma}
\begin{proof}
We must exhibit a lift of the form
\[
\begin{tikzcd}
{\Lambda^k[n]} \arrow[d, "i"', hook] \arrow[r] & {\mathbf{eq}_B(E_1, E_2)} \arrow[d, "p_2^{p_1}"] \\
{\Delta[n]} \arrow[r, "x"'] \arrow[ru, dashed] & B                                  
\end{tikzcd}.
\] Since $p_2^{p_1}: \mathbf{hom}_B(E_1, E_2)\to B$ is a Kan fibration, we can find a diagonal fill-in of the form $$\left(x: \Delta[n] \to B, s_x: x^{\ast}{E_1}\to x^{\ast}{E_2}\right) : \Delta[n] \to \mathbf{hom}_B(E_1, E_2).$$ In light of \eqref{equniv}, it suffices to show that $s_x$ is a weak equivalence. We already have a weak equivalence $s_{x\circ i} : i^{\ast}{x^{\ast}{E_1}}\to i^{\ast}{x^{\ast}{E_2}}$ from our original square. Therefore, \cref{thenweq} implies that $s_x$ is a weak equivalence because $\Delta[n]$ is connected.
\end{proof}

\medskip

We are now in position to formulate our simplicial notion of univalence. Let $p: E \to B$ be a Kan fibration and let
\[
\mathbf{eq}(E) = \mathbf{eq}_{B\times B}\left(\pi_1^{\ast}{E}, \pi_2^{\ast}{E}\right).
\]
As the product of two simplicial sets is computed levelwise, the $n$-simplices of $\mathbf{eq}(E)$ are precisely triples of the form 
\begin{gather*}
\left(b_1, b_2, s_{b_1, b_2}\right)
\\ b_1, b_2\in B_n
\\ s_{b_1, b_2} : b_1^{\ast}{E} \to b_2^{\ast}{E}.
\end{gather*}
From \eqref{equniv}, we see that any map $X\to \mathbf{eq}(E)$ corresponds naturally to a triple of maps
\[ \label{trips}
\left(f_1: X\to B, f_2: X\to B, s_{f_1, f_2}:f_1^{\ast}{E} \to f_2^{\ast}{E}\right). \tag{\RomanNumeralCaps{1}}
\] In particular, we have a map $\delta_E : B\to \mathbf{eq}(E)$ corresponding to the triple $\left(\idd_B, \idd_B, \idd_E\right)$, with 
\[ \label{deltaE}
\left(\delta_E\right)_n(b) = \left(b,b,\idd_{b^{\ast}{E}}\right), \ \quad b\in B_n. \tag{\RomanNumeralCaps{2}}
\]
Then $\delta_E$ has two retractions defined by the composites
\[
\begin{tikzcd}[column sep=large] \label{splitmono}
\mathbf{eq}(E) \arrow[r, "\pi_2^{\ast}{p}^{\pi_1^{\ast}{p}}"] & B\times B \arrow[r, "\pi_i"] & {B},\ \quad {i=1,2}
\end{tikzcd}. \tag{\RomanNumeralCaps{3}}
\] Therefore, $\delta_E$ is a split monomorphism.  

\begin{definition}[Simplicial univalence]
A Kan fibration $p:E\to B$ is \textit{univalent} if $\delta_E$ is a weak equivalence in $\sset_{\mathtt{Quillen}}$.
\end{definition}

Note that the diagram
\[
\begin{tikzcd}[column sep=large]
B \arrow[r, "\delta_{E}"'] \arrow[rr, "\Delta_{B}", bend left] & \mathbf{eq}(E) \arrow[r, "\pi_2^{\ast}{p}^{\pi_1^{\ast}{p}}"'] & B\times B
\end{tikzcd}
\] commutes. Thus, since $\delta_{E}$ is monic and $\pi_2^{\ast}{p}^{\pi_1^{\ast}{p}}$ is a Kan fibration by \cref{restKfib}, we have that $\delta_{E}$ is univalent if and only if $\mathbf{eq}(E)$ is a path space object of $B$.

\begin{exmp}
For any Kan complex $X$, the unique map $X \to 1$ is univalent if and only if the space of homotopy autoequivalences of $X$ is contractible.
\end{exmp}

\subsection{Proof of univalence}\label{fibunivpf}

This section first unifies our type-theoretic and simplicial definitions of univalence and then shows that the Kan fibration $p_{\kappa} : \widehat{\mathrm{U}}_{\kappa} \to \mathrm{U}_{\kappa}$ is simplicially univalent. This implies that the contextual category $\sset_{\mathrm{U}_{\kappa}}$ satisfies Voevodsky's univalence axiom (\cref{uaxiom}).

\smallskip

\begin{remark}
Recall that we interpret  dependent types as pullbacks of  $p_{\kappa}$. For convenience, we may write such pullbacks as  dependent types \emph{within} our MLDTT, thereby abusing notation.
\end{remark}

\bigskip

Let $p_1:E_1 \to B$ and $p_2 : E_2 \to B$ be pullbacks of $p_{\kappa}$. Suppose that $B$ is a Kan complex (i.e., a closed type). Consider both the function type $$\mathbf{\left[E_1, E_2\right]}\coloneqq  \left\llbracket{x: B \vdash {E_1 \to E_2}\ \type}\right\rrbracket$$ and the type $$\mathbf{E_1 \simeq E_2} \coloneqq  \left\llbracket{x: B \vdash {E_1\simeq E_2}\ \type}\right\rrbracket$$ of equivalences from $E_1$ to $E_2$ interpreted as Kan fibrations over $B$.

\begin{prop}\label{funciso} $ $
\be[label=(\arabic*)]
\item There is an isomorphism $\mathbf{\left[E_1, E_2\right]} \overset{\cong}{\longrightarrow} \mathbf{hom}_B(E_1, E_2)$ over $B$.\footnote{\autocite[Corollary 3.3.3]{KL}.}
\item The induced map $\mathbf{E_1 \simeq E_2} \to  \mathbf{hom}_B(E_1, E_2)$ factors through $\mathbf{eq}_B(E_1,E_2) \hookrightarrow   \mathbf{hom}_B(E_1, E_2)$, and $\mathbf{E_1 \simeq E_2} \to \mathbf{eq}_B(E_1,E_2)$ is a trivial fibration.\footnote{\autocite[Lemma 3.3.4]{KL}.}
\ee
\end{prop}

\smallskip


\begin{theorem}
Suppose that $p:E \to B$ is a Kan fibration.  Then $p$ is simplicially univalent if and only if it is type-theoretically univalent in the sense that the Kan fibration $$\left\llbracket{x,y:B \vdash \isequiv\left(\equiveq_{{x:B}; E(x)}(x,y)\right)\ \type}\right\rrbracket$$ over $B\times B$ has a section.
\end{theorem}
\begin{proof}[Proof sketch]
Consider the map 
\[
w_E\coloneqq \left\llbracket{x,y:B, z: x\leadsto_B y \vdash  \equiveq_{{x:B}; E(x)}(x,y)(z) : E(x) \simeq E(y)}\right\rrbracket
.\] It follows from \cite[Lemma 3.3.2]{KL} that $p$ is type-theoretically univalent if and only if
\[
\begin{tikzcd}
{\left\llbracket{x,y:B, z: x\leadsto_B y}\right\rrbracket} \arrow[rd] \arrow[rr, "w_E"] &           & {\left\llbracket{x,y:B, f: \isequiv\left(E(x), E(y)\right)}\right\rrbracket} \arrow[ld] \\
                                                                                                                                                                                                                      & B\times B &                                                                                        
\end{tikzcd}
\]  is a weak equivalence over $B\times B$. The same Lemma together with \cref{funciso} provides us with a commutative diagram
\[
\begin{tikzcd}[column sep=large]
B \arrow[r, "r_B"] \arrow[rrddd, "\Delta_B"', bend right=49] & \rho_1(B) \arrow[rddd, "{\left(s_B, t_B\right)}"', bend right] \arrow[r, "w_E"] & \mathbf{\pi_1^{\ast}{E}\simeq \pi_2^{\ast}(E)} \arrow[d]                          \\
                                                             &                                                                                 & {\mathbf{eq}_{B\times B}\left(\pi_1^{\ast}{E}, \pi_2^{\ast}{E}\right)} \arrow[d]  \\
                                                             &                                                                                 & {\mathbf{hom}_{B\times B}\left(\pi_1^{\ast}{E}, \pi_2^{\ast}{E}\right)} \arrow[d] \\
                                                             &                                                                                 & B\times B                                                                        
\end{tikzcd}.
\]
Recall that the map $r_B$ is precisely the interpretation $\left\llbracket{x:B \vdash \refl(B,x) : x\leadsto_B x}\right\rrbracket$ of reflexivity. By applying the inference rule $\id$-\textsc{comp}  to our construction of $\equiveq_{{x:B}; E(x)}(x,y)$  (\cref{considequiv}), we deduce that the composite
\[
B\longrightarrow \mathbf{hom}_{B\times B}\left(\pi_1^{\ast}{E}, \pi_2^{\ast}{E}\right) 
\] is precisely the interpretation $\left\llbracket{  x: B\vdash  \lambda(y:E(x) ).y: E(x) \to E(x) }\right\rrbracket$ of the identity map on $E(x)$. Therefore, the composite $B\longrightarrow \mathbf{eq}_{B\times B}\left(\pi_1^{\ast}{E}, \pi_2^{\ast}{E}\right) $  is precisely $\delta_E$, defined by \eqref{deltaE}. 

\medskip

Both $r_B$ and $\mathbf{E_1 \simeq E_2} \to \mathbf{eq}_B(E_1,E_2)$ are weak equivalences by \cref{fibpathspace} and \cref{funciso}(2), respectively. Hence the two-out-of-three property implies that $\delta_E$ is a weak equivalence if and only if $w_E$ is one. 
\end{proof}

\medskip

\begin{theorem}
The Kan fibration $p_{\kappa} : \widehat{\mathrm{U}}_{\kappa} \to \mathrm{U}_{\kappa}$ is simplicially univalent. %Theorem 3.4.1 (KL)
\end{theorem}
\begin{proof}
We must show that $\delta_{\widehat{\mathrm{U}}_{\kappa}}$ is a weak equivalence. Recall from \eqref{splitmono} that the composite 
$$\tau \coloneqq \pi_2\circ \pi_2^{\ast}{p_{\kappa}}^{\pi_1^{\ast}{p_{\kappa}}}$$ is a retraction of $\delta_{\widehat{\mathrm{U}}_{\kappa}}$. Therefore, by two-out-of-three, it suffices to show that $\tau$ is a weak equivalence. In fact, we shall show that it is a trivial fibration. To this end, consider any lifting problem of the form
\[
\begin{tikzcd}[row sep=large, column sep=large]
A \arrow[r] \arrow[d, "j"', hook] & \mathbf{eq}\left(\widehat{\mathrm{U}}_{\kappa}\right) \arrow[d, "\delta_{\widehat{\mathrm{U}}_{\kappa}}"] \\
B \arrow[r]                       & \mathrm{U}_{\kappa}                                                                                      
\end{tikzcd}
.\]
By \cref{Urep} and our characterization \eqref{trips} of maps into $\mathbf{eq}\left(\widehat{\mathrm{U}}_{\kappa}\right)$, this square naturally corresponds  to a commutative diagram of the form 
\[
\begin{tikzcd}[row sep=large]
E_1 \arrow[r, "w"] \arrow[rd, "p_1"'] & E_2 \arrow[d, "p_2"] \arrow[r] \arrow[dr, phantom, "\scalebox{1.5}{\color{black}$\lrcorner$}" , very near start, color=black]
& \underline{E}_2 \arrow[d, "q_2"] \\
                                      & A \arrow[r, hook]               & B                   
\end{tikzcd}
\] where
\bi
\item $w$ is a weak equivalence and 
\item $p_1$, $p_2$, and $q_2$ are $\kappa$-small well-ordered Kan fibrations.
\ei Our pullback square here comes from our lifting problem, which exhibits $p_2$ as the pullback of $p_{\kappa}$ along the composite $A \hookrightarrow B \to \mathrm{U}_{\kappa}$.

\smallskip

Now, a solution to our lifting problem naturally corresponds  to a  commutative diagram of the form
\[ \label{uglydiag}
\begin{tikzcd}[row sep=large]
                                                                     &                                & \underline{E}_1 \arrow[d, "\underline{w}"', dashed] \arrow[dd, "q_1", dashed, bend left=49] \\
E_1 \arrow[r, "w"] \arrow[rd, "p_1"'] \arrow[rru, dashed, bend left] & E_2 \arrow[d, "p_2"] \arrow[r] \arrow[dr, phantom, "\scalebox{1.5}{\color{black}$\lrcorner$}" , very near start, color=black]
& \underline{E}_2 \arrow[d, "q_2"']                                               \\
                                                                     & A \arrow[r, hook]              & B                                                                  
\end{tikzcd} \tag{$\bullet$}
\] where
\bi
\item $\underline{w}$ is a weak equivalence,
\item $q_1$ is a $\kappa$-small well-ordered Kan fibration, and
\item the square
\[
\begin{tikzcd}
E_1 \arrow[d, "p_1"'] \arrow[r, dashed] & \underline{E}_1 \arrow[d, "q_1", dashed] \\
A \arrow[r, hook]                       & B                           
\end{tikzcd}
\] is a pullback.
\ei

\medskip



Let us define $\underline{w}$ as the pullback
\[ \label{circ1}
\begin{tikzcd}[row sep=large, column sep=large]
\underline{E}_1 \arrow[r] \arrow[d, "\underline{w}"'] \arrow[dr, phantom, "\scalebox{1.5}{\color{black}$\lrcorner$}" , very near start, color=black]
& \Pi_j{E_1} \arrow[d, "\Pi_j{w}"]             \\
\underline{E}_2 \arrow[r, "\eta_{\underline{E}_2}"']                 & \underbrace{\Pi_j{E_2}}_{\Pi_jj^{\ast}{\underline{E}_2}}
\end{tikzcd}  \tag{$\bigcirc$}
\] where $\eta$  denotes the unit of the adjunction $j^{\ast} \dashv \Pi_j$.
First, we want to show that $E_1 \cong j^{\ast}{\underline{E}_1}$ and that $j^{\ast}{\underline{w}} \cong w$ in $\sset/A$ so that \eqref{uglydiag} commutes. To this end, recall from our proof of \cref{Quillem2}(a) that $j^{\ast}{\Pi_j}\cong \idd_{\sset/A}$. Thus, using the triangle identities for $\left(j^{\ast}, \Pi_j\right)$, we can apply the functor $j^{\ast}({-})$ to \eqref{circ1} to get
\[
\begin{tikzcd}[row sep=large, column sep=large]
j^{\ast}{\underline{E}_1} \arrow[d, "j^{\ast}{\underline{w}}"'] \arrow[r]  \arrow[dr, phantom, "\scalebox{1.5}{\color{black}$\lrcorner$}" , very near start, color=black]
& E_1 \arrow[d, "w"] \\
\underbrace{j^{\ast}{\underline{E}_2}}_{E_2} \arrow[r, "\idd_{E_2}"']                      & E_2               
\end{tikzcd}
.\] This square is a pullback because $j^{\ast}({-})$ preserves products as a right adjoint to $\Sigma_j$. Since the pullback of an isomorphism is again an isomorphism, it follows that $E_1 \cong j^{\ast}{\underline{E}_1}$ and  $j^{\ast}{\underline{w}} \cong w$, as desired.

\medskip

Next, we want to show that  $q_1$ is $\kappa$-small and well-ordered. In light of our proof of \cref{Quillem2}(b), both $\Pi_j{p_1}$ and $\Pi_j{p_1}$ are $\kappa$-small. Since $q_2 = \Pi_j{p_2}\circ \eta_{\underline{E}_2}$, it follows that  $\eta_{\underline{E}_2}$ is also  $\kappa$-small. Hence the map $\underline{E}_1 \to \Pi_{j}{E_1}$ is $\kappa$-small as the pullback of  $\eta_{\underline{E}_2}$. As $\kappa$ is regular, this implies that $q_1$ is $\kappa$-small as the composite of two $\kappa$-small maps. Moreover, by the axiom of choice, we may assume that $q_1$ is a well-ordered extension of $p_1$ as in \cref{UKan}.

\medskip

It remains to show that $q_1$ is a Kan fibration and that $\underline{w}$ is a weak equivalence. As a weak equivalence, $w$ can be factored as trivial cofibration $w_c$ followed by a trivial fibration $w_f$. Further, the partial mapping $\Ar(\sset/A)\to \Ar(\sset/B)$ given by $w\mapsto \underline{w}$  respects composition, so that $$\underline{w} = \underline{w_f}\circ \underline{w_c}.$$ Hence we may assume that $w$ is either a trivial cofibration or a trivial fibration. First, suppose that $w$ is a trivial fibration. Then $\Pi_j{w}$ is also a trivial fibration by \cref{adjpres}(2). Hence so is $\underline{w}$ as the pullback of a trivial fibration, and now $q_1$ is a Kan fibration as the composite of two Kan fibrations.

\medskip

Suppose, instead, that $w$ is a trivial cofibration. In this case, $\Pi_j{w}$ must be a monic in $\sset$ because $\Pi_j({-})$ preserves monomorphisms as a right adjoint. Hence $\underline{w}$ is also monic in light of \eqref{circ1}. Without loss of generality, we may assume that $\underline{w}$ is an inclusion map into $\underline{E}_2$. 
\begin{claim}
There exists a strong deformation retraction $H: E_2 \times \Delta[1] \to E_1$ of  $w$.
\end{claim}
\begin{proof}
Since $w$ is a trivial cofibration, we have a diagonal fill-in
\[
\begin{tikzcd}
E_1 \arrow[d, "w"', hook] \arrow[r, equal]           & E_1 \arrow[d, "p_1"] \\
E_2 \arrow[r, "p_2"'] \arrow[ru, "r", dashed] & A                   
\end{tikzcd}.
\] Thanks to \cref{chep}, this yields another diagonal fill-in
\[
\begin{tikzcd}[column sep=huge, row sep=large]
\left(E_2\times \left\{0\right\}\right)\cup \left(E_1\times \Delta[1]\right)\cup \left(E_2\times \left\{1\right\}\right) \arrow[d, hook] \arrow[r, "\idd_{E_2}\cup w\cup w\circ r"] & E_2 \arrow[d, "p_2"] \\
{E_2\times \Delta[1]} \arrow[r, "p_2\circ \pi_1"'] \arrow[ru, "H", dashed]                                                                                                  & A                   
\end{tikzcd}
.\]
\end{proof}
Now, we may apply \cref{chep} to obtain a diagonal fill-in
\[
\begin{tikzcd}[column sep=huge, row sep=large]
{\left(E_2\times \Delta[1]\right)\cup \left(\underline{E}_1\cup \Delta[1]\right)\cup \left(\underline{E}_2\times \left\{0\right\}\right)} \arrow[d, hook] \arrow[r, "H\cup \underline{w}\cup\idd_{\underline{E}_2}"] & \underline{E}_2 \arrow[d, "q_2"] \\
{\underline{E}_2\times \Delta[1]} \arrow[r, "q_2\circ \pi_1"'] \arrow[ru, "H'", dashed]                                                                                          & B                   
\end{tikzcd}
.\]
The induced map $H_1'$ factors through $\underline{E}_1$ because $H_1$ factors through $E_1$ as a retraction of $q$. Therefore, $H_1'$ is a strong deformation retraction of $\underline{w}$. This means that $\left\lvert{\underline{w}}\right\rvert$ is a homotopy equivalence, i.e., that $\underline{w}$ is a weak equivalence in $\mathbf{Top}$. Moreover, $H_1'$ exhibits $q_1$ as a retract of $q_2$ in the sense of \eqref{eqn:ret}, and thus  $q_1$ is a Kan fibration by \cref{cloret}.


\begin{comment}
we have two diagonal fill-ins
\[
\begin{tikzcd}[row sep=large]
E_1 \arrow[d, "w"', hook] \arrow[r, equal]           & E_1 \arrow[d, "p_1"] &  & {E_2\cup \left(E_1\times \Delta[1]\right)\cup E_2} \arrow[d, hook] \arrow[r] & E_2 \arrow[d] \\
E_2 \arrow[r, "p_2"'] \arrow[ru, "r", dashed] & A                    &  & {E_2\times \Delta[1]} \arrow[r] \arrow[ru, "H", dashed]                      & A            
\end{tikzcd}
,\] so that $r$ is a strong deformation retraction of $w$ (see \cref{defret}).
\end{comment}
\end{proof}

\smallskip

\begin{corollary}
For any inaccessible cardinal $\lambda < \kappa$, the contextual category $\sset_{\mathrm{U}_{\lambda}}$ models $\cdtt+\univ$.
\end{corollary}

\subsection*{Awodey's conjecture}

In closing, it is worth mentioning some subsequent generalizations of \cite{KL}.
To this end, let us introduce a new kind of category.

\begin{definition}[Reedy category]
A \textit{Reedy category} is a category  $\rr$ equipped with two wide subcategories $R_+$ and $R_{-}$ along with a \textit{degree function} $d: \ob{\rr} \to \alpha$ where $\alpha$ is an ordinal such that
\be[label=(\roman*)]
\item every morphism in $\rr$ factors as a map in $R_{-}$ followed by a map in $R_+$,
\item every non-identity map $a \to b$ in $R_+$ satisfies $d(a) <d(b)$, and
\item every non-identity map $a \to b$ in $R_{-}$ satisfies $d(b) <d(a)$.
\ee
\end{definition}

\begin{exmp} $ $
\be
\item Any ordinal $\alpha$ is a Reedy category with $R_+ \equiv \alpha$, $R_{-}$ the discrete category on $\ob{\alpha}$, and $d\equiv \idd_{\alpha}$.
\item By \cref{mordelt}(1), the simplex category $\varDelta$ is Reedy with $R_+$ the wide subcategory on all monomorphisms, $R_{-}$ the subcategory on all epimorphisms, and $d$ the inclusion map $\ob{\varDelta} \hookrightarrow \omega$.
\ee
\end{exmp}

Also, note that the opposite of a Reedy category is a Reedy category with $R_{+}$ and $R_{-}$ switched. In particular, $\varDelta^{\op}$ is a Reedy category.

\medskip

Let $\rr$ be a Reedy category and let $G$ be a simplicial presheaf $\rr^{\op} \to \sset$ over $\rr$. For any $r\in \ob{\rr}$, the \textit{latching object of $G$ over  $r$} is the simplicial set
\[
L_r{G} \coloneqq \colim_{f\in \left(\partial{R_{+}/r}\right)^{\op}}G_{\dom(f)}
\]  where $\partial{R_{+}/r}$ denotes the full subcategory of the over category $R_{+}/r$ on all objects except the identity map $\idd_r$. The universal property of colimits yields a unique map $L_r{G} \to G_r$.

\medskip

Now, we say that $\rr$ is an \textit{elegant Reedy category} if for any monomorphism $A \hookrightarrow B$ in $\left[\rr^{\op}, \sset\right]$ and any $x\in \ob{\rr}$, the unique mediating map

\[
\begin{tikzcd}
L_x{A} \arrow[d] \arrow[r]                  & L_x{B} \arrow[d] \arrow[rdd, bend left]   &     \\
A_x \arrow[r] \arrow[rrd, hook, bend right] & A_x\cup_{L_x{A}}L_x{B} \arrow[rd, dashed] &     \\
                                            &                                           & B_x
\end{tikzcd}
\] is monic in $\sset$. (This notion is due to \cite{Rezk}.)

\begin{exmp}
The simplex category $\varDelta$ is elegant Reedy.
\end{exmp}

\smallskip

Thanks to \cref{genmodel}, we know that $\left[\rr^{\op}, \sset\right]$ can be presented by a type-theoretic model category for  any small elegant Reedy category $\rr$. Specifically, in \cite{Shul},  Michael Shulman modifies the construction of \cite{KL} to show that $\cdtt  +\univ$ can be interpreted in the model category  $\left[\rr^{\op}, \sset\right]_{\mathtt{inj}}$. Since the trivial category $\ast$ is elegant Reedy  and $$\left[\ast^{\op}, \sset\right] \cong \left[\left(\ast \times \varDelta\right)^{\op}, \set\right] \cong \sset,$$  it follows that \cite{Shul} directly generalizes \cite{KL}.

\medskip

The general belief that compact-object classifiers model univalent universes leads to  ``Awodey's conjecture":
\[
\textit{Every Grothendieck $\infty$-topos is presentable by a model category that models $\cdtt + \univ$}.
\] Assuming \cref{initial}, Shulman released a proof of Awodey's conjecture in April of 2019 {\autocite{Shul2}}.
Therefore, every theorem of homotopy type theory is true in any $\infty$-topos.  In other words, a result in synthetic homotopy theory (including $\univ$) holds in any $\infty$-topos. In this sense, homotopy type theory serves as a formal language for reasoning within a number of general settings for algebraic topology at once.

\newpage

\appendix

\section{Deductive systems}\label{natded}

\begin{definition}
A \textit{deductive system} consists  of the following data:
\be[label=(\alph*)]
\item a  countable set $\mathcal{A}$ of symbols,
\item a countable set $\mathcal{S}$ of (finite) strings over $\mathcal{A}$  called \textit{expressions} or \textit{raw terms},
\item a finite set $\mathcal{B}$ of positive integers,
\item a finite set $ \left\{\sigma_i\right\}_{i\in \mathcal{B}}$ where each $\sigma_i$ is a subset of $\mathcal{S}^i$, and
\item a finite set of ordered pairs called \textit{inference rules}.
\ee
The set $\mathcal{S}$ is known as the \textit{object language} of the deductive system (whereas the language of set theory is chosen as the metalanguage).
\end{definition}
\smallskip
For any $n\in \mathcal{B}$, we say that an element of $\sigma_n$ is a \textit{judgment (of order $n$)}.
By definition, every inference rule is a pair $\left(\left\{J_1, \ldots, J_n\right\}, J\right)$ where 
\bi
\item $n\in \N$ and, 
\item for all $1\leq i \leq n$, $J_i$ and $J$ denote judgments.
\ei  This is represented graphically as
\[
\inferrule*[Left = Name]{J_1 \\ \ldots \\ J_n}{J}.
\]

\smallskip

We call $J$ the \textit{conclusion} and each $J_i$ a \textit{premise} of the rule. 
When $n=0$, the inference rule is called an \textit{axiom}. 

\medskip

The set of inference rules generates, via mutual recursion, a finite set $\left\{R_i\right\}_{i\in \mathcal{B}}$ of $i$-place relations on $\mathcal{S}$ as follows.

\be
\item If $J$ is the conclusion of an axiom and has order $i$, then $J\in R_i$.
\item If $\left(\left\{J_1, \ldots, J_n\right\}, J\right)$ is an inference rule, each $J_i$ has order $n_i$, $J$ has order $k$, and $J_i \in R_{n_i}$, then $J \in R_k$. 
\ee  

That is, $\left\{R_i\right\}$ is precisely the smallest family of relations closed under the inference rules.

\begin{definition}
A \textit{theorem} is an element of $\bigcup_{i\in \mathcal{B}}R_i$.
\end{definition}

\begin{definition}
A \textit{derivation} (or a \textit{derived rule}) is a finite rooted tree with the following properties.
\bi
\item Each node is a judgment.
\item For any non-leaf $F$ with children $F_1, \ldots, F_k$, there is some inference rule $\left(\left\{F_1, \ldots, F_k\right\}, F\right)$.
\ei
\end{definition}
Given a derivation, we say that the root  is \textit{derivable from} the leaves.

\medskip

A theorem can be characterized recursively as either the conclusion of an axiom or a judgment derivable from a theorem.  

\begin{note}
Since each inference rule has only finitely many premises, each node of a proof tree has at most $k$ children for some fixed $k \in \N$. Hence any proof tree can be viewed as a complete $k$-ary tree with nodes marked by inference rules. We can define the set of such trees inductively and thus perform so-called structural induction to prove properties about proof trees.
\end{note}

\section{Univalent group theory}\label{groups}

In this appendix, we shall define some basic notions of group theory within our MLDTT. Our goal is to state, in a precise way, the fact that any two isomorphic groups are propositionally equal provided that $\univ$ is true.  In ordinary group theory, identifying two isomorphic groups  is common practice  but is technically an abuse of notation when they are not equal as sets. A virtue of homotopy type theory is that in it, such an abuse of notation becomes formally true.

\medskip
We shall follow \cite[Section 11.2]{Rijke} and adopt  our informal notation from \cref{synth}.

\bigskip

\begin{definition}
We say that a (well-formed) type $A$ is an \textit{h-set} (or \textit{set}) if there is some term of type $$ \isset(A) \coloneqq \prod_{x,y:A}\isprop(x\leadsto y)    .$$ 
\end{definition}

\begin{remark}
The axiom $\uip$ asserts that every type is a set.
\end{remark}

Now, we can encode the set-theoretic notion  of a group  in our MLDTT as follows. 

\begin{definition}
We say that a small type $G$ is a \textit{group} if there is some term of type 
\begin{align*}
\isgrp(G) \coloneqq & \sum_{p : \isset(G)} \sum_{e : G}\sum_{i: G \to G}\sum_{\mu : G \to (G \to G)}
\\ & \left(\prod_{(x, y, z : G)} \mu(\mu(x, y), z) \leadsto \mu(x, \mu(y, z))\right) \times
\\ & \left(\prod_{(x : G)} \mu(e, x) \leadsto x\right) \times\left(\prod_{(x : G)} \mu(x, e) \leadsto x\right) \times
\\ & \left(\prod_{(x : G)} \mu(i : x, x)\leadsto e\right) \times\left(\prod_{(x : G)} \mu(x, i(x))\leadsto x\right)
.
\end{align*} We may write $xy$ for $\mu(x,y)$. 
\end{definition}

We require $G$ to be a set so that it is invariant (up to propositional equality) under a different proof of, say, associativity of $\mu$. Moreover, Voevodsky has proven that from $\univ$ we can derive that $\isset(X)$ is a mere proposition for any small type $X$. Hence the definition of $G$ is invariant under a different proof of $\isset(G)$.

\begin{exmp}
We can turn the groupoid from \cref{groupoid} into a group just as we turn the fundamental groupoid into the fundamental group in classical topology.  
\smallskip

Specifically, for any small type $X$ and element $x:X$ such that $\isset(x\leadsto x)$, define the \textit{loop space of $X$ at $x$} as the type $ x \leadsto_X x. $
\end{exmp}

\begin{definition}
 If $G$ and $H$ are groups, then the type of \textit{homomorphisms from $G$ to $H$} is $$ \homm(G, H) \coloneqq
 \sum_{f: G\to H}\prod_{x,y:G}f(xy) \leadsto_{H} f(x)f(y)
     .$$ 
\end{definition} 

\begin{exmp}
For any group $G$, the \textit{identity homomorphism} is $\underline{\idmap}_G \coloneqq (\idmap_G, p)$ where $p(x,y) \coloneqq \refl_{xy}$.
\end{exmp}

\begin{definition} $ $
\be
\item Let $\left(h, p\right) : \homm(G, H)$ and $\left(k, q\right): \homm(H, K)$. The \textit{composition of $k$ with $h$} is the term
 $$k \underline{\circ} h \coloneqq (k \circ h, p \ast q) : \homm(G, K)$$
\item For any groups $G$ and $H$, the type of \textit{group isomorphisms from $G$ to $H$} is 
\[
G\cong H \coloneqq 
\sum_{h : \homm(G, H)} \sum_{k : \homm(H, G)}\left(k \underline{\circ} h \leadsto \underline{\idmap}_{G}\right) \times\left(h \underline{\circ} k \leadsto \underline{\idmap}_{H}\right)
. \]
\ee 
\end{definition} 

\begin{theorem}
Assume $\univ$. Let $A, B :\U$ and suppose that $\el(A)$ and $\el(B)$ are groups. Define the function $$\isoeq_{A,B} : (A \leadsto_{\U} B) \to (\el(A) \cong \el(B))$$ inductively by $$\isoeq_{A,B}(\refl_A) \coloneqq \left(\underline{\idmap}_{\el(A)}, \left(\underline{\idmap}_{\el(A)}, \left(\refl_{\underline{\idmap}_{\el(A)}}, \refl_{\underline{\idmap}_{\el(A)}}\right)\right)\right).$$ Then $\isoeq_A$ is an equivalence. 
\end{theorem}

\section{Locally cartesian closed categories}\label{LCC}

\begin{remark}
All categories in this section are assumed to be locally small.
\end{remark}

\begin{definition}\label{tens-hom}
A symmetric monoidal category $\left(\c, \otimes\right)$ is \textit{closed} if for every $X\in \ob{\c}$, the functor ${-}\otimes X: \c \to \c$ has a right adjoint, denoted by
\[
\left[X, {-}\right] : \c \to \c.
\] In this case, an object of the form $\left[X, Y\right]$ is called the \textit{internal hom from $X$ to $Y$}.
\end{definition}

This generalizes the tensor-hom adjunction found in the category of $R$-modules.

\smallskip

\begin{definition}[Cartesian closed]
A category $\c$ is \textit{cartesian closed} if it has all finite products (hence a terminal object) and for any object $X$ in $\c$, the functor $ {-} \times X : \c \to \c   $ has a  right adjoint,  denoted by $${-}^X : \c \to \c.$$ In this case, an object of the form $Y^X$ is called the \textit{exponential of $Y$ by $X$}.
\end{definition}

This means that  $\c$ is cartesian closed exactly when it  is closed as a cartesian monoidal category, with $Y^X$ being the internal hom of $X$ and $Y$.

\medskip

Suppose that $\c$ is cartesian closed. For any object $X$ of $\c$, there is some natural isomorphism $\left(\varphi_{Z}\right)_{Z\in \ob\left(\c^{\op} \times \c\right)}$ between the bifunctors 
\begin{align*}
\Hom_{\c}\left({-}, {-}^X\right) &: \c^{\op} \times \c  \to \set
\\ \Hom_{\c}\left({-}\times X, {-}\right) &: \c^{\op} \times \c  \to \set.
\end{align*}   For any $Y \in \ob{\c}$, we call $$\ev_{X, Y} \coloneqq \varphi_{\left(Y^X, Y\right)}\left(\idd_{Y^X}\right)$$ the \textit{evaluation} morphism for $Y^X$. Note that $\left(\ev_{X, Y}\right)_{Y \in \ob{\c}}$ is precisely the counit of our chosen adjunction. This satisfies the following universal property.


\begin{prop}
For any object $Z$ and any morphism $f: Z \times X \to Y$, there is a unique morphism $\tilde{f} :Z \to Y^X$ such that $ f= \ev_{X,Y} \circ \left(\tilde{f} \times \idd_X\right)$.
\[
\begin{tikzcd}
Z \arrow[d, "\tilde{f}"', dashed] &  & Z \times X \arrow[d, "\tilde{f} \times \idd_X"'] \arrow[rd, "f"] &   \\
Y^X                                &  & Y^X \times X \arrow[r, "{\ev_{X,Y}}"']                           & Y
\end{tikzcd}
\]

\end{prop}
\begin{proof}
It suffices to show that the bijection $ \varphi_{(Z, Y)} : \Hom_{\c}({Z}, Y^X) \to \Hom_{\c}(Z\times X, {Y})$ is given by
\[
\left(Z \overset{\psi}{\longrightarrow} Y^X\right) \ \mapsto \  \left(Z \times X \xrightarrow{\psi \times \idd_X } Y^X \times X \xrightarrow{\ev_{X,Y} } Y\right)
.\] 

By naturality, we have that
\[
\begin{tikzcd}[column sep =19ex]
{\Hom_{\c}\left(Y^X, Y^X\right)} \arrow[d, "{\varphi_{\left(Y^X, Y\right)}}"'] \arrow[r, "{\Hom_{\c}\left(\psi^{\op} , \idd_{Y^X}\right)}"] & {\Hom_{\c}\left(Z, Y^X\right)} \arrow[d, "{\varphi_{\left(Z, Y\right)}}"] \\
{\Hom_{\c}\left(Y^X \times X, Y\right)} \arrow[r, "{\Hom_{\c}\left(\psi^{\op} \times \idd_X, \idd_Y\right)}"']                    & {\Hom_{\c}\left(Z \times X, Y\right)}                         
\end{tikzcd}
.\]
In particular,
\begin{align*}
\ev_{X, Y} \circ \left(\psi \times \idd_X\right) & = \idd_Y \circ \varphi_{\left(Y^X, Y\right)}\left(\idd_{Y^X}\right) \circ \left(\psi \times \idd_X\right)
\\ & = \varphi_{\left(Z,Y\right)}\left(\idd_{Y^X} \circ \idd_{Y^X} \circ \psi\right) 
\\ &= \varphi_{\left(Z,Y\right)}
.
\end{align*}
\end{proof}

\begin{term}
The morphism $\tilde{f}$ is known as the \textit{(exponential) transpose of $f$ relative to $\varphi$}.
\end{term}

This is precisely the ordinary adjunct of $f$ under $\varphi$.

\begin{note}\label{trans}
Conversely, for any morphism $g: Z \to Y^X$, let $$\bar{g} \coloneqq \ev_{X, Y} \circ \left(g \times \id_X\right).$$ By uniqueness of the transpose, we have that $\tilde{\bar{g}} = g$ and $\bar{\tilde{f}} = f$. Thus, we get a new adjunction $\Hom_{\c}\left(Z \times X, Y\right) \cong \Hom_{\c}\left(Z, Y^X\right)$ given by $f \mapsto  \tilde{f}$ and $\bar{g} \mapsfrom g$. We call such an adjunction $\trans$.
\end{note}

\smallskip

\begin{exmp} We can think of the exponential $Y^X$  as a generalized version of the set of functions from $X$ to $Y$. In particular, both
 $\set$ and $\mathbf{FinSet}$ are cartesian closed, where $Y^X$ is taken to be $\left\{f \mid f: X \to Y\right\}$.
\end{exmp}

\smallskip

Let us review now the Yoneda lemma, through which we can  establish a key class of additional examples of cartesian closed categories.

\begin{notation}
If $\c$ is a category, then $\widehat{\c}$ will denote the presheaf category $ \left[\c^{\op}, \set\right]$.
\end{notation}

\begin{lemma}[Yoneda]\label{Yoneda}
Let $\c$ be a category and let $\mathcal{Y}: \c \to \widehat{\c}$ denote the Yoneda embedding. In particular, for any $C\in \ob{\c}$, $\mathcal{Y}_C\coloneqq \mathcal{Y}(C)$ denotes the unique (set-valued) presheaf on $\c$ represented by $C$. If $F \in \ob{\widehat{\c}}$, then the set map 
\begin{align*}
& \varphi:  \Hom_{\widehat{\c}}(\mathcal{Y}_C, F) \to F(C)
\\ & \varphi(f) = f_C\left(\idd_C\right)
\end{align*}
 is a natural bijection in both $C$ and $F$. In particular, $\mathcal{Y}$ is fully faithful.
\end{lemma}

For any $x\in F(C)$, the natural transformation $\varphi^{-1}(x)$ is given componentwise by
\begin{align*}
& \left(\varphi^{-1}(x)\right)_D : \Hom_{\c}(D, C) \to F(D)
\\ & \left(\varphi^{-1}(x)\right)_D(g) = F(g)(x).
\end{align*}

Therefore, from the natural one-to-one correspondence
\[
{x\in F(B)} \longleftrightarrow {\mathcal{Y}_B \xrightarrow{x} F},
\]
we obtain another such correspondence 
\[
F(f)(x) \in F(A) \longleftrightarrow x \circ \mathcal{Y}(f)
\] for any map $f: A \to B$ in $\c$.

\medskip

Turning now to a well-known consequence of the Yoneda lemma, consider any presheaf $F : \c^{\op} \to \set$. In order to prove our second corollary, define the \textit{category $\int_{\c}{F}$ of elements  of $F$} as follows.  Its objects are precisely pairs $\left(C, x\right)$ with $C\in \ob{\c}$ and $x\in F(C)$, and its morphisms $\left(C, x\right) \to \left(C', x'\right)$ are precisely morphisms $g: C\to C'$ in $\c$ such that $$F(g)(x') = x.$$ Note that $\int_{\c}{F}$ is small whenever $\c$ is small. 

\begin{theorem}[Density theorem]\label{psclr}
Let $\c$ be a small category. For any $F\in \ob{\widehat{\c}}$, there is a functor $G: J \to \c$ such that $J$ is small and $\colim_{j\in J} \mathcal{Y}_{G_j} \cong F$.\footnote{\cite[Proposition 8.10]{Awodey}.}
\end{theorem}
\begin{proof}[Proof sketch]
Take $\int_{\c}{F}$ as $J$. Consider the projection functor $\pi: \int_{\c}{F} \to \c$ given by $\pi(C,x) = C$ and $\pi(g) = g$ on objects and morphisms, respectively. For each element $\left(C,x\right)$ of $F$, define $\alpha_{\left(C,x\right)} : \mathcal{Y}_{\pi(C,x)} \to F$ as the unique map $x: \mathcal{Y}_C \to F$ corresponding to $x$ via the Yoneda lemma. This determines a cocone $\left\{\alpha_{\left(C,x\right)}\right\}$ under $\mathcal{Y} \circ \pi$. Further, it can be shown that this is colimiting. 
\end{proof}

\begin{lemma}\label{ppcl}
Let $\c$ be any category and $J$ be a small category. For any functors $A: J \to \left[  \c^{\op}, \set\right]$ and $B: \c^{\op} \to \set$, there is a natural isomorphism $$ \colim_j\left(A_j \times B\right) \cong \left(\colim_j{A_j}\right) \times B.$$
\end{lemma}
\begin{proof}
 By definition of a colimit, we have a canonical cocone $\left\{\alpha_j : A_j \to \colim_j{A_j}\right\}_j$. Applying the functor ${-}\times B$ to this, we get another cocone $$\left\{\alpha_j \times \idd_B : A_j \times B \to \left(\colim_j{A_j}\right) \times B\right\}_j.$$ By the universal property of colimits, this induces a unique morphism $$\alpha :\colim_j\left(A_j \times B\right) \to \left(\colim_j{A_j}\right) \times B.$$ We want to show that $\alpha$ is a natural isomorphism. It suffices to show that each component $\alpha_C$ is a bijection. Since colimits in $\widehat{\c}$ are computed pointwise, we thus may assume wlog that $A_j, B\in \ob{\set}$ for any $j\in J$. Applying the fact that $\set$ is cartesian closed,  we obtain the following chain of bijections natural in $X\in \ob{\set}$:
 \begin{align*}
 \Hom_{\set}\left(\colim_j(A_j \times B), X\right) & \cong \lim_j{\Hom_{\set}}\left(A_j \times B, X\right)
 \\ & \cong \lim_j{\Hom_{\set}}\left(A_j, X^B\right)
 \\ & \cong \Hom_{\set}\left(\colim_j{A_j}, X^B\right)
 \\ & \cong \Hom_{\set}\left(\left(\colim_j{A_j}\right) \times B, X\right).
 \end{align*}
 Since the Yoneda embedding is fully faithful and thus reflects isomorphisms, it follows that  \linebreak $\colim_j\left(A_j \times B\right) \cong \left(\colim_j{A_j}\right) \times B,$ as desired. 
\end{proof}


\begin{lemma}\label{pcc}
If $\c$ is small, then $\widehat{\c}$ is cartesian closed.
\end{lemma}
\begin{proof}
First of all, recall that  $\widehat{\c}$ has all binary products where $P\times Q$ is computed pointwise for any $P,Q \in \ob{\widehat{\c}}$, i.e., $\left(P\times Q\right)(C) = P(C) \times P(Q)$. Recall also that $\widehat{\c}$ has a terminal object, namely the constant functor at any singleton set. Hence it has all finite products.

\medskip
It remains to show that ${-} \times P$ always has a right adjoint. 
For any presheaf $Q$ on $\c$, define $Q^P$ to be the presheaf $\Hom_{\widehat{\c}}(\mathcal{Y}_{-} \times P, Q)$ on $\c$. Given any $F \in \ob{\widehat{\c}}$, apply \cref{psclr} to get a natural isomorphism $ \colim_{j\in J} \mathcal{Y}_{G_j} \cong   F .$ Using the Yoneda lemma together with \cref{ppcl}, we thus get the following chain of bijections natural in $\left(F, Q\right)$:
\begin{align*}
\Hom_{\widehat{\c}}(F\times P, Q) & \cong \Hom_{\widehat{\c}}\left(\left(\colim_{j\in J} \mathcal{Y}_{G_j}\right) \times P, Q\right)
\\ & \cong  \Hom_{\widehat{\c}}\left(\colim_{j\in J}\left( \mathcal{Y}_{G_j} \times P\right), Q\right)
\\ & \cong \lim_j \Hom_{\widehat{\c}}\left( \mathcal{Y}_{G_j} \times P, Q\right)
\\ & \cong \lim_j \Hom_{\widehat{\c}}\left( \mathcal{Y}_{G_j}, Q^P\right)
\\ & \cong \Hom_{\widehat{\c}}\left(\colim_j \mathcal{Y}_{G_j}, Q^p\right)
\\ & \cong \Hom_{\widehat{\c}}\left(F, Q^p\right).
\end{align*} 
Hence $\left({-}\times P, {-}^P\right)$ is an adjoint pair, as required. 
\end{proof}

If $\c$ has all pullbacks, then any morphism $f:X \to Y$ in $\c$ induces a \textit{base change} functor \linebreak $f^{\ast} : \c/Y \to \c/X$  defined on objects and morphisms, respectively, by 
\begin{align*}
\left(p : K \to Y\right)
 & \mapsto
  \left(
  \begin{array}{ccc}
    X \times_Y K &\longrightarrow & K
    \\
    {}^{\mathllap{f^{\ast}(p)}}\Big\downarrow && \Big\downarrow
    \\
    X &\overset{f}{\longrightarrow}& Y 
  \end{array}
  \right)
  \,
 \\  \left( 
     \begin{array}{ccc}
       K \ \ &\overset{r}{\longrightarrow} & \ \ K' \\
        {}_p \searrow & & \swarrow_{q}
       \\   
       & Y &
    \end{array}
  \right)   & \mapsto   \left( 
     \begin{array}{ccc}
        X \times_Y K \ \ &\overset{r^{\ast}}{\longrightarrow} & \ \  X \times_Y K' \\
        {}_{f^{\ast}(p)} \searrow & & \swarrow_{f^{\ast}(q)}
       \\   
       & X &
    \end{array}
  \right)
\end{align*}
 where $f^{\ast}(r)$ is the unique morphism such that
 \[
\begin{tikzcd}
X \times_Y K \arrow[r] \arrow[d, "f^{\ast}(r)", dashed] \arrow[dd, "f^{\ast}(p)"', bend right=59] & K \arrow[d, "r"'] \arrow[dd, "p", bend left=49] \\
X \times_Y K' \arrow[d, "f^{\ast}(q)"] \arrow[r]                                               & K' \arrow[d, "q"']                             \\
X \arrow[r, "f"']                                                                                    & Y                                             
\end{tikzcd}
 \]
 commutes.

\begin{lemma}
Let $\c$ be a category with pullbacks and $f : X \to Y$ be any morphism in $\c$. Define the \textit{dependent sum} functor $\Sigma_f : \c/X \to \c/Y$ by post-composition with $f$. Then $\left(\Sigma_f, f^{\ast}\right)$ is an adjoint pair.
\end{lemma}
\begin{proof}
We must show that there is a natural isomorphism of bifunctors
\[
 \Hom_{\c/Y}\left(\Sigma_f({-}), {-}\right) \to \Hom_{\c/X}({-}, f^{\ast}({-}))
.\]
For each $\left(a, b\right) \in \ob\left(\left(\c/X\right)^{\op} \times \left(\c/Y\right)\right)$, define $\varphi_{(a, b)}$ by $\left(\mathbin{\textcolor{blue}{h}} : \Sigma_f(a) \to b\right) \mapsto \left(\mathbin{\textcolor{red}{\hat{h}}} : a \to f^{\ast}(b)\right)$.
\[
\begin{tikzcd}
A \arrow[rd, "\hat{h}"', red] \arrow[rrd, "h", bend left, blue] \arrow[rdd, "a"', bend right] &                                  &                  \\
                                                                                        & X \times_Y B \arrow[r, "t"] \arrow[d, "f^{\ast}(b)"] & B \arrow[d, "b"] \\
                                                                                        & X \arrow[r, "f"']                & Y               
\end{tikzcd}
\] The universal property of pullbacks implies that this is a bijection natural in $\left(a,b\right)$. To see explicitly that it is natural, let  $\left(a: A \to X, b: B \to Y\right)$ and $\left(a' : A' \to X, b': B' \to Y\right)$ be objects of $\left(\c/X\right)^{\op} \times \left(\c/Y\right)$ and consider any morphism $\left(v, u\right) : \left(a,b\right) \to \left(a',b'\right)$. We must prove that $$\varphi_{\left(a',b'\right)}\left(u \circ h \circ \Sigma_f(v)\right) = f^{\ast}(u) \circ \varphi_{(a,b)}(h) \circ v$$ for any $h : \Sigma_f(a) \to b$. Note that $\Sigma_f(v) = v$. By the universal property of pullbacks, it suffices to prove that 
\[
\begin{tikzcd}[row sep=huge, column sep=large]
                                                                                                          & A \arrow[r, "h"]                                       & B \arrow[d, "g"]   \\
A' \arrow[r, "f^{\ast}(u) \circ \hat{h}\circ v"] \arrow[ru, "v", bend left] \arrow[rd, "a'"', bend right] & X\times_Y B' \arrow[d, "f^{\ast}(b')"] \arrow[r, "t'"] & B' \arrow[d, "b'"] \\
                                                                                                          & X \arrow[r, "f"]                                       & Y                 
\end{tikzcd}
\]
commutes. We have that $ h =  t \circ \hat{h}$ by definition of $\hat{h}$. Further, by definition of $f^{\ast}(u)$, we have that $u \circ t = t' \circ f^{\ast}(u)$. Therefore, $$t' \circ f^{\ast}(u) \circ \hat{h} \circ v = u \circ t \circ \hat{h}\circ v = u\circ h \circ v,$$ i.e., the top trapezoid commutes. 

Next, note that 
\begin{align*}
& f^{\ast}(b') \circ  f^{\ast}(u) \circ \hat{h} \circ v
\\ & = f^{\ast}(b) \circ \hat{h} \circ v
\\ & = a \circ v
\\ &  = a'.
\end{align*}
Thus, the bottom left triangle commutes as well, so that the whole diagram commutes. 
\end{proof}

\begin{exmp}
Any  morphism $g: A \to X$ in $\set$ corresponds uniquely to an $X$-indexed family of disjoint sets $\left(g^{-1}(x)\right)_{x\in X}$. Under this identification, the functor $\Sigma_f$ is given by $$g \mapsto \left(\coprod_{x\in f^{-1}(y)} g^{-1}(x)\right)_{y\in Y} $$ for any set function $f: X \to Y$.
\end{exmp}

\medskip
We have established that $f^{\ast}$ always has a left adjoint. If $f^{\ast}$ always has a right adjoint and $\c$ has all finite limits (equivalently, $\c$ has a terminal object in addition to all pullbacks), then we obtain the following notion.  


\begin{definition}[Locally cartesian closed category (LCCC)]
A category $\c$ with finite limits is \textit{locally cartesian closed (LCC)} if for each morphism $f: X \to Y$ in $\c$, the base change functor $f^{\ast}: \c/Y \to \c/X$ has a right adjoint, called the \textit{dependent product $\Pi_f$}.
\end{definition}

\begin{exmp}
The category $\set$ of sets is LCC. Indeed, for any set functions $f: X \to Y$ and $g: A \to X$, we have $$\Pi_f(g)  = \left(\prod_{x \in f^{-1}(y)} g^{-1}(x)\right)_{y \in Y},$$ which is precisely a $Y$-indexed family of sets of choice functions $f^{-1}(y) \to \bigcup_{x \in f^{-1}(y)}g^{-1}(x)$.
\end{exmp}

\begin{prop}\label{inthom}
Let $f: X\to Y$ be any map in a LCCC $\c$. There is a unique isomorphism $$\left({-}\right)^f \cong \Pi_f \circ f^{\ast}$$ of right adjoint functors $\c/Y \to \c/Y$.
\end{prop}
\begin{proof}
Note that the binary product functor $\left({-}\right)\times f : \c/Y\to \c/Y$ is precisely the composite $\Sigma_f \circ f^{\ast}$. Hence  both $\Pi_f \circ f^{\ast}$ and $\left({-}\right)^f$ are right adjoint to $\left({-}\right)\times f$. But right adjoints are unique up to unique adjunction-compatible isomorphism, and thus our proof is complete. 
\end{proof}

\smallskip

\begin{prop}[Beck-Chevalley condition]\label{C-B} Suppose that
\[
\begin{tikzcd}
D \arrow[d, "k"'] \arrow[r, "h"] & C \arrow[d, "g"] \\
A \arrow[r, "f"']                & B               
\end{tikzcd}
\] is a pullback square in a LCCC $\c$. Then there are isomorphisms 
\begin{align*}
\Sigma_k{h^{\ast}{\varphi}} &  \cong f^{\ast}{\Sigma_g{\varphi}}
\\  \Pi_k{{h^{\ast}{\varphi}}} & \cong f^{\ast}{\Pi_g{\varphi}}     
\end{align*}
 natural in $\varphi \in \ob{\c/C}$.
\end{prop}
\begin{proof} 
The first isomorphism follows immediately from the pasting law for pullbacks. From this, we obtain the second isomorphism. Indeed, we have a chain of isomorphisms
\begin{align*}
\Hom\left(\psi, f^{\ast}{\Pi_g{\varphi}}\right) & \cong \Hom\left(g^{\ast}{\Sigma_f{\psi}}, \varphi\right) 
\\ & \cong \Hom\left(\Sigma_h{k^{\ast}{\psi}}, \varphi\right) 
\\ & \cong \Hom\left(\psi , \Pi_k{h^{\ast}{\varphi}}  \right) 
\end{align*}
natural  in $\left(\psi, \varphi\right) \in \ob\left(\c/A \times \c/C\right)$. Our desired isomorphism now follows from the fact that  the Yoneda embedding is conservative.
\end{proof}

\medskip

\begin{notation}
The terminal object of a category will always be denoted by $1$.
\end{notation}

\begin{theorem}\label{slicecc}
Suppose that $\c$ has a terminal object. Then $\c$ is LCC if and only if $\c/X$ is cartesian closed for every object $X$ of $\c$.
\end{theorem}
\begin{proof} $ $\\
($\Longrightarrow$) Let $X\in \ob{\c}$. We must show that $\c/X$ is cartesian closed. Let $a: A \to X$ and $b: B \to X$ be morphisms in $\c$. Recall that $\c/X$ has all finite products, with binary products being pullbacks
\[
\begin{tikzcd}
A\times_X B \arrow[d, "\pi_1"'] \arrow[r, "\pi_2"] \arrow[rd, dashed] & B \arrow[d, "b"] \\
A \arrow[r, "a"']                                                     & X               
\end{tikzcd}
\] and the terminal object being $\idd_X : X \to X$. To see that ${-} \times a$ has a right adjoint, note that it is the same as the composite functor $$ \c/X \overset{a^{\ast}}{\longrightarrow} \c/A \overset{\Sigma_a}{\longrightarrow} \c/X    .$$ Since $\c$ is LCC, we have an adjoint triple $\Sigma_a \dashv a^{\ast} \dashv \Pi_a$. This yields an adjunction $$\Sigma_a \circ a^{\ast} \dashv \Pi_a \circ a^{\ast},$$ so that ${-} \times a$ has a right adjoint.

\medskip
($\Longleftarrow$) Let us show that $\c$ has all finite limits. It suffices to show that $\c$ has a terminal object and all pullbacks. By assumption, it has the former. To see that is has the latter, note that each slice of $\c$ is cartesian closed and thus has all binary products, which are precisely pullback squares in $\c$. It follows that $\c$ has all pullbacks. 

\smallskip

Next, let us show that $\Pi_f$ exists for every morphism $f: X \to Y$ in $\c$. Define $\Pi_f : \c/X \to \c/Y$ on objects by mapping $a : A \to X$ to the pullback
\[
\begin{tikzcd}
\Pi_f(a) \arrow[d] \arrow[r]        
\arrow[dr, phantom, "\scalebox{1.5}{\color{black}$\lrcorner$}" , very near start, color=black]
& (f\circ a)^f \arrow[d, "a^f"] \\
\idd_Y \arrow[r, "\tilde{\tau}"'] & f^f                          
\end{tikzcd}
\] in $ \c/Y   $   where $\tilde{\tau}$ denotes the transpose of  the canonical isomorphism $$\tau: \idd_Y \times f \overset{\cong}{\longrightarrow}  f$$ relative to $\trans$ (defined in \cref{trans}). We see that $\Pi_f$ is functorial as the composite $U_f \circ \tilde{\tau}^{\ast} \circ {-}^f$ where $U_f$ denotes a suitable forgetful functor that outputs just the object of the pullback. 

\smallskip
It remains to exhibit a bijection $$\Hom_{\c/Y}\left(b, \Pi_f(a)\right) \cong \Hom_{\c/X}\left(f^{\ast}(b), a\right)$$ natural in  $\left(b: B \to Y, a:A \to Y\right) \in \left(\c/Y\right)^{\op} \times \c/X$. By the universal property of pullbacks, every $g\in \Hom_{\c/Y}(b, \Pi_f(a))$ naturally corresponds  to a pair $\left(g_1, g_2\right)$ of morphisms in $\c/Y$ such that
\[
\begin{tikzcd}
b \arrow[rd, "g"] \arrow[rdd, "g_1"', bend right] \arrow[rrd, "g_2", bend left] &                                         &                               \\
                                                                           & \Pi_f(a) \arrow[d] \arrow[r]  
\arrow[dr, phantom, "\scalebox{1.5}{\color{black}$\lrcorner$}" , very near start, color=black]
          & (f\circ a)^f \arrow[d, "a^f"] \\
                                                                           & \idd_Y \arrow[r, "\tilde{\tau}"'] & f^f                          
\end{tikzcd}
\]
commutes. But $g_1$ must equal $b$, so that $g$ naturally corresponds to a morphism $g' : b \to \left(f\circ a\right)^f$ such that $a^f \circ g' = \widetilde{\idd_Y} \circ b$.
Furthermore, under the equivalence $\left(\c/X\right) \simeq \left(\c/Y\right)/f$, every $g \in \Hom_{\c/X}(f^{\ast}(b), a)$ is precisely a morphism
\[
\begin{tikzcd}
b\times f \arrow[r, "g"] \arrow[rd, "\pi_X"'] & f\circ a \arrow[d, "a"] \\
                                              & f                      
\end{tikzcd}
\]
in $\left(\c/Y\right)/f$ where $\pi_X$ is as in
\[
\begin{tikzcd}
B\times_Y X \arrow[d, "\pi_B"'] \arrow[r, "\pi_X"] & X\arrow[d, "f"] \\
B \arrow[r, "b"']                         & Y               
\end{tikzcd}
.\]  Now, consider the product-exponential adjunction given by $\trans$ in $\c/Y$. We want to show that $\trans$ restricts to a bijection $$\left\{g' : b \to \left(f\circ a\right)^f \mid a^f \circ g' = \widetilde{\idd_Y} \circ b\right\} \overset{\cong}{\longrightarrow} \left\{g : b \times f \to f \circ a \mid a \circ g = \pi_X\right\},$$ which must be natural. That is, we want to show that $a \circ \trans_{b, f\circ a}(g') = \pi_X$. Since we have assumed that 
\[
\begin{tikzcd}
b \arrow[d, "b"'] \arrow[r, "g'"]      & (f\circ a)^f \arrow[d, "a^f"] \\
\idd_Y \arrow[r, "\widetilde{\tau}"'] & f^f                          
\end{tikzcd}
\] 
commutes, it follows from adjunction that
\[
\begin{tikzcd}[column sep =huge]
b \times f \arrow[d, "b \times \idd_f"'] \arrow[r, "{\trans_{b, f\circ a}(g')}"] & (f\circ a) \arrow[d, "a"] \\
\idd_Y \times f \arrow[r, "{\trans_{\idd_{Y, f}}(\tilde{\tau})}"']            & f                        
\end{tikzcd}
\] 
does as well. But $\trans_{\idd_{Y, f}}(\widetilde{\tau}) = \bar{\tilde{\tau}} = \tau$, and $\tau \circ \left(b \times \idd_f\right)$ equals the projection $b \times f \overset{\pi_f}{\longrightarrow} f$. Since $\pi_f$ is precisely $\pi_X$, we have that  $\pi_X = \tau \circ \left(b \times \idd_f\right) = a \circ \trans_{b, f\circ a}(g'), $ as desired. 
\end{proof}

\pagebreak

\begin{corollary} Let $\c$ be LCC.
\be
\item $\c$ is cartesian closed.
\item Every slice $\c/A$ of $\c$ is LCC.
\ee
\end{corollary}
\begin{proof} $ $
\be
\item  Simply observe that $\c \cong \c/1$.
\item It is clear that $\c/X \simeq \left(\c/A\right)/a$ for every $X\in \ob{\c}$ and every morphism $a: X\to A$ in $\c$. Hence every slice of $\c/A$ is cartesian closed, so that $\c/A$ is LCC.
\ee
\end{proof}

\smallskip

\begin{corollary}
If $\c$ is small, then $\widehat{\c}$ is LCC.
\end{corollary}
\begin{proof}
Suppose that $\c$ is small. Then $\int_{\c}{P}$ is also small for any presheaf $P$ on $\c$. By \cref{pcc}, it suffices to show that $\widehat{\c}/P \simeq \widehat{\int_{\c}{P}}$. Let us construct such an equivalence.

\medskip

Define the functor $\varphi : \widehat{\c}/P \to \widehat{\int_{\c}{P}}$ as follows. For each morphism $f : F \to P$ in $\widehat{\c}$, define the presheaf $\varphi(f)$ on $\int_{\c}{P}$ by mapping each object $\left(C, x\right)$ to the fiber product  
\[
\begin{tikzcd}
\overbrace{F(C) \times_{P(C)} \{x\}}^{f_C^{-1}(x) \times \{x\}} \arrow[d] \arrow[r] & \{x\} \arrow[d, hook] \\
F(C) \arrow[r, "f_C"']                       & P(C)                 
\end{tikzcd}
\] and each morphism $a: \left(C, x\right) \to \left(C', x'\right)$ to the set function given by $\left(y, x'\right) \mapsto \left(F(a)(y), x\right)$. 

\smallskip

Also, for each morphism 
\[
\begin{tikzcd}
F \arrow[r, "h"] \arrow[rd, "f"'] & G \arrow[d, "g"] \\
                                  & P               
\end{tikzcd}
\] in $\widehat{\c}/P$, define the natural transformation $\varphi(h) : \varphi(f) \to \varphi(g)$ componentwise by 
\begin{align*}
\varphi(h)_{C,x} :  f_C^{-1}(x) \times \{x\} & \to g_C^{-1}(x) \times \{x\}
\\   \left(y,x\right)& \mapsto \left(h_C(y), x\right)
.\end{align*}

\medskip

For the reverse direction, define the functor $\pi :  \widehat{\int_{\c}{P}} \to \widehat{\c}/P$ as follows. For each presheaf $f$ on $\int_{\c}{P}$, first define the presheaf $R_f : \c^{\op} \to \set$ on objects by $C \mapsto \coprod_{x\in P(C)}f(C, x)$ and on morphisms by $$\left(a : C \to C'\right) \mapsto  \left(\left(x, y\right) \mapsto \left(P(a)(x), f(a)(y)\right) \right),$$ where $a$ is also a morphism $\left(C,  P(a)(x)\right) \to \left(C', x\right)$ in $\int_{\c}{P}$. Now, define the natural transformation $\pi(f) : R_f \to P$ componentwise by $\pi(f)_C(x,y) = x$. 

\smallskip

Also, for each morphism $K \coloneqq \left(k_{(C,x)}\right) : f \to g$ in $ \widehat{\int_{\c}{P}}$, define the natural transformation \linebreak $\pi(K) : R_f \to R_g$ componentwise by  $$\pi(K)_C(x,y) = \left(x, k_{\left(C,x\right)}(y)\right).$$ Clearly, $\pi(g) \circ \pi(K) = \pi(f)$, so that $\pi(K)$ is indeed a morphism in $\widehat{\c}/P$.

\medskip

Finally, it is easy yet tedious to check that $\varphi \circ \pi \cong \idd_{ \widehat{\int_{\c}{P}} }$ and $\pi \circ \varphi \cong \idd_{ \widehat{\c}/P}$, and we won't do so here.
\end{proof}

\begin{exmp} $ $
\be
\item $\sset$, the category of simplicial sets.
\item $\left[\c^{\op}, \sset\right] \cong \left[\left(\c \times \varDelta\right)^{\op}, \set\right]$, the category of \textit{simplicial presheaves} over a small category $\c$.
\ee
\end{exmp}

\printbibliography[
heading=bibintoc,
title={References}
]


\end{document}
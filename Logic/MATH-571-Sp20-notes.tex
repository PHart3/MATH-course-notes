\documentclass[10pt,letterpaper,cm]{nupset}
\usepackage[margin=1.2in]{geometry}
\usepackage{graphicx}
 \usepackage{enumerate}
  \usepackage{enumitem}
 \usepackage{stmaryrd}
 \usepackage{bm}
\usepackage{amsfonts}
\usepackage{amssymb}
\usepackage{pgfplots}
\pgfplotsset{compat=1.13}
\usepackage{amsmath,amsthm}
\usepackage{lmodern}
\usepackage{tikz-cd}
\usetikzlibrary{quotes}
\usetikzlibrary{decorations.markings}
\usepackage{faktor}
\usepackage{xcolor}
\usepackage{soul}
\usepackage{xfrac}
\usepackage{mathtools}
\usepackage{bm}
\usepackage{ dsfont }
\usepackage{mathrsfs}
\usepackage{hyperref}
\usepackage{siunitx}
\usepackage[utf8]{inputenc}
\sisetup{group-separator={,}}
\hypersetup{colorlinks=true, linkcolor=red,          % color of internal links (change box color with linkbordercolor)
    citecolor=green,        % color of links to bibliography
    filecolor=magenta,      % color of file links
    urlcolor=cyan           }

\usepackage{thmtools}
\usepackage[capitalise]{cleveref} 
    
\theoremstyle{definition}
\newtheorem{definition}{Definition}[subsection]
\newtheorem{exmp}[definition]{Example}
\newtheorem{non-exmp}[definition]{Non-example}
\newtheorem{note}[definition]{Note}

\theoremstyle{theorem}
\newtheorem{theorem}[definition]{Theorem}
\newtheorem{lemma}[definition]{Lemma}
\newtheorem{prop}[definition]{Proposition}
\newtheorem{corollary}[definition]{Corollary}
\newtheorem*{claim}{Claim}
\newtheorem{exercise}[definition]{Exercise}

\theoremstyle{remark}
\newtheorem{remark}[definition]{Remark}
\newtheorem*{todo}{To do}
\newtheorem*{question}{Question}
\newtheorem*{conv}{Convention}
\newtheorem*{aside}{Aside}
\newtheorem*{notation}{Notation}
\newtheorem*{term}{Terminology}
\newtheorem*{background}{Background}
\newtheorem*{further}{Further reading}
\newtheorem*{sources}{Sources}

\makeatletter
\def\th@plain{%
  \thm@notefont{}% same as heading font
  \itshape % body font
}
\def\th@definition{%
  \thm@notefont{}% same as heading font
  \normalfont % body font
}
\makeatother

\makeatletter
\renewcommand*\env@matrix[1][*\c@MaxMatrixCols c]{%
  \hskip -\arraycolsep
  \let\@ifnextchar\new@ifnextchar
  \array{#1}}
\makeatother
\pgfplotsset{unit circle/.style={width=4cm,height=4cm,axis lines=middle,xtick=\empty,ytick=\empty,axis equal,enlargelimits,xmax=1,ymax=1,xmin=-1,ymin=-1,domain=0:pi/2}}
\DeclareMathOperator{\Ima}{Im}
\newcommand{\A}{\mathbb A}
\newcommand{\C}{\mathbb C}
\newcommand{\D}{\mathcal{D}}
\newcommand{\E}{\vec E}
\newcommand{\CP}{\mathbb CP}
\newcommand{\F}{\mathbb F}
\newcommand{\G}{\mathbb G}
\renewcommand{\H}{\mathbb H}
\newcommand{\HP}{\mathbb HP}
\newcommand{\K}{\mathbb K}
\renewcommand{\L}{\mathcal L}
\newcommand{\M}{\mathbb M}
\newcommand{\N}{\mathbb N}
\renewcommand{\O}{\mathbf O}
\newcommand{\OP}{\mathbb OP}
\renewcommand{\P}{\mathcal P}
\newcommand{\Q}{\mathbb Q}
\newcommand{\I}{\mathbb I}
\newcommand{\R}{\mathbb R}
\newcommand{\RP}{\mathbb RP}
\renewcommand{\S}{\mathtt S}
\newcommand{\T}{\mathbf T}
\newcommand{\Z}{\mathbb Z}
\newcommand{\B}{\mathbb{B}}
\newcommand{\1}{\mathbf{1}}
\newcommand{\ds}{\displaystyle}
\newcommand{\ran}{\right>}
\newcommand{\lan}{\left<}
\newcommand{\bmat}[1]{\begin{bmatrix} #1 \end{bmatrix}}

\renewcommand{\a}{\mathscr{A}}
\renewcommand{\b}{\mathscr{B}}
\renewcommand{\c}{\mathscr{C}}
\renewcommand{\d}{\mathscr{D}}
\newcommand{\e}{\mathscr{E}}
\newcommand{\y}{\mathscr{Y}}
\renewcommand{\j}{\mathscr{J}}
\newcommand{\X}{\mathscr X}

\newcommand{\h}{\vec h}
\newcommand{\f}{\vec f}
\newcommand{\g}{\vec g}
\renewcommand{\i}{\mathscr{I}}
\renewcommand{\k}{\vec k}
\newcommand{\n}{\vec n}
\newcommand{\p}{\frak{p}}
\newcommand{\q}{\mathscr{Q}}
\renewcommand{\r}{\vec r}
\newcommand{\s}{\vec s}
\renewcommand{\t}{\vec t}
\renewcommand{\u}{\mathscr{U}}
\renewcommand{\v}{\mathscr{V}}
\newcommand{\w}{\vec w}
\newcommand{\x}{\vec x}
\newcommand{\z}{\vec z}
\newcommand{\0}{\mathsf 0}

\newcommand{\arrowcircle}[1][]{%
  \begin{tikzpicture}[#1]
    \draw[->] (0,0ex) -- (2em,0ex);
    \draw (1em,0ex) circle (0.7ex);
  \end{tikzpicture}%
}

\DeclareMathOperator{\card}{\text{card}}

\newcommand{\Rho}{\mathrm{P}}

\newcommand{\properideal}{%
  \mathrel{\ooalign{$\lneq$\cr\raise.22ex\hbox{$\lhd$}\cr}}}

\makeatletter
\newcommand*\bigcdot{\mathpalette\bigcdot@{.5}}
\newcommand*\bigcdot@[2]{\mathbin{\vcenter{\hbox{\scalebox{#2}{$\m@th#1\bullet$}}}}}
\makeatother

\DeclareMathOperator*{\Span}{span}
\DeclareMathOperator{\lcm}{lcm}
\DeclareMathOperator*{\GL}{GL}
\DeclareMathOperator*{\SL}{SL}
\DeclareMathOperator{\rng}{range}
\DeclareMathOperator{\gemu}{gemu}
\DeclareMathOperator{\almu}{almu}
\DeclareMathOperator{\ann}{ann}
\DeclareMathOperator{\Char}{\mathsf{char}}
\DeclareMathOperator{\id}{id}
\DeclareMathOperator{\graph}{Graph}
\DeclareMathOperator{\gal}{Gal}
\DeclareMathOperator{\tr}{tr}
\DeclareMathOperator{\trdeg}{trdeg}
\DeclareMathOperator{\ev}{ev}
\DeclareMathOperator{\norm}{N}
\DeclareMathOperator{\aut}{Aut}
\DeclareMathOperator{\Int}{Int}
\DeclareMathOperator{\ext}{Ext}
\DeclareMathOperator{\stab}{Stab}
\DeclareMathOperator{\orb}{Orb}
\DeclareMathOperator{\inn}{Inn}
\DeclareMathOperator{\out}{Out}
\DeclareMathOperator{\op}{op}
\DeclareMathOperator{\fix}{Fix}
\DeclareMathOperator{\ab}{ab}
\DeclareMathOperator{\sgn}{sgn}
\DeclareMathOperator{\syl}{syl}
\DeclareMathOperator{\Syl}{Syl}
\DeclareMathOperator{\conj}{conj}
\DeclareMathOperator{\im}{im}
\DeclareMathOperator{\ed}{End}
\DeclareMathOperator{\gr}{\mathsf{gr}}
\DeclareMathOperator{\map}{Map}
\DeclareMathOperator{\mor}{mor}
\DeclareMathOperator{\ob}{ob}
\DeclareMathOperator{\pr}{pr}
\DeclareMathOperator{\fs}{fs}
\DeclareMathOperator{\fo}{\mathsf{FO}}
\DeclareMathOperator{\cl}{cl}
\DeclareMathOperator{\thh}{Th}
\DeclareMathOperator{\Sing}{Sing}
\DeclareMathOperator{\Mat}{Mat}
\DeclareMathOperator{\Hom}{Hom}
\DeclareMathOperator{\Fun}{Fun}
\DeclareMathOperator{\Fr}{Fr}
\DeclareMathOperator{\cone}{cone}
\DeclareMathOperator{\colim}{colim}
\DeclareMathOperator{\nilp}{Nilp}
\DeclareMathOperator{\tot}{Tot}
\DeclareMathOperator{\kor}{Kor}
\DeclareMathOperator{\Jac}{Jac}
\DeclareMathOperator{\Max}{Max}
\DeclareMathOperator{\ho}{Ho}
\DeclareMathOperator{\cn}{Cn}
\DeclareMathOperator{\mult}{mult}
\DeclareMathOperator{\Frac}{Frac}
\DeclareMathOperator{\disc}{Discr}
\DeclareMathOperator{\supp}{supp}
\DeclareMathOperator{\acyc}{acyclic}

\pagestyle{headings}

\linespread{1.3}

% info for header block in upper right hand corner
\name{Perry Hart}
\class{MATH 571}
\assignment{Spring 2020}

\begin{document}
\thispagestyle{empty}
\begin{abstract}
These notes are based on Scott Weinstein's ``Model Theory'' lectures at UPenn along with David Marker's \textit{Model Theory: An Introduction}. Any mistake in what follows is my own.
\end{abstract}

\tableofcontents
\newpage

\section{Introduction}
\subsection{Lecture 1}

Recall the structure $\N \coloneqq \langle \omega, \S, \mathtt{0}\rangle$ where 
\begin{itemize}
\item $\omega$ denotes the set of natural numbers $\left\{0,1,2,\ldots,\right\}$, 
\item $\S$ is interpreted as the successor function $\omega \to \omega$, and 
\item the constant symbol $\mathtt{0}$ is interpreted as the natural number $0$.
\end{itemize}

The formal language $\mathcal{L}$ for which $\N$ is a structure consists of the first-order (FO) logical symbols
\[
\forall, \ \exists, \ \land,\ \neg,\ \vee,\ \rightarrow, \ =
\]
along with non-logical symbols such as $\mathtt{0}$, $\S^n{\mathtt{0}}\coloneqq \underbrace{\S\cdots \S}_{n\text{ copies}}{\mathtt{0}}$, and $\S^n{x}$. Let $\fo$ denote the set of all (first-order) $\mathcal{L}$-sentences. 

\smallskip

The \textit{theory of $\N$} is 
\[
\thh(\N) \coloneqq \left\{\varphi \in \fo \mid \N \models \varphi \right\},
\] which consists of all sentences satisfied by $\N$. Further, for any subset $\Sigma \subset \fo$, consider the set $$\cn(\Sigma)  \coloneqq \left\{\varphi \in \fo \mid \Sigma \models \varphi\right\}$$ of consequences of $\Sigma$.

\begin{question}
Can we find a theory $\Sigma$ (other than $\thh(\N)$) such that $\cn(\Sigma) = \thh(\N)$?
\end{question}

Let $\Delta = \left\{\forall{x}\left(\S{x}\ne \0\right), \ \forall{x}\forall{y}\left(\S{x} = \S{y} \rightarrow x =y\right), \ \forall{x}\left(x\ne \0 \rightarrow \exists{y}\left(\S{y}=x\right)\right)\right\}$. Each element of $\Delta$ is clearly true in $\N$, i.e., $\A \models \Delta$. But is it the case that $\cn(\Delta) = \thh(\N)$? No, provided that we allow ourselves access to monadic second-order sentences. Specifically, consider the \textit{induction axiom $\mathsf{IA}$}:
\[ \label{eqn:IA}
\forall{P}\left(\left(P(0) \land \forall{x}\left(P(x) \rightarrow P(\S{x})\right) \rightarrow \forall{x}\left(P(x)\right)\right)\right). \tag{$\ast$}
\] This is clearly true in $\N$. Consider, however, a new structure $\A \coloneqq \langle \omega \cup \Z, \S, \0\rangle$. Then $\Delta \subset \thh(\A)$, and we have a \textit{$\Z$-chain} in $\A$ (pretending, for the moment, that the universe $\lvert{\A}\rvert$ has the usual order $<$):
\[
\begin{tikzcd}
\cdots \arrow[r, bend left] & {-\left(n+1\right)}^{\A} \arrow[r, bend left] & {-n}^{\A} \arrow[r, bend left] & \cdots \arrow[r, bend left] & {-1}^{\A} \arrow[r, bend left] & \0^{\A} \arrow[r, bend left] & 1^{\A} \arrow[r, bend left] & \cdots
\end{tikzcd}
.\] The second-order sentence \eqref{eqn:IA} with $P$ instantiated by the ``initial segment" $\Z_{\geq {-1}}$ is not true in $\A$, so that  $\A \not\models \mathsf{IA}$. In this case, $\mathsf{IA} \in \thh(\N)\setminus \cn(\Delta)$.

Nevertheless, we want to restrict ourselves to $\fo$. Recall that two structures $\B$ and $\C$ are \textit{elementarily equivalent} if $\thh(\B) = \thh(\C)$. 

\begin{question}
Are $\A$ and $\N$ elementarily equivalent?
\end{question}

If we can find some sentence belonging to $\thh(\N)\setminus \thh(\A)$, then $\cn(\Delta) \ne \thh(\N)$.

\begin{definition}
A theory $\Sigma$ is \textit{categorical} if for any structures $\B$ and $\C$, if $\B \models \Sigma$ and $\C \models \Sigma$, then $\B \cong \C$.
\end{definition}

\begin{exmp}
$\Delta' \coloneqq \Delta +\mathsf{IA}$ is categorical.
\end{exmp}

Perhaps exhibiting that the usual order $<$ on $\omega$ is definable in $\N$ would reveal that $\A \not \equiv \N$. For this, we must find a (well-formed) formula $\theta(x,y)$ such that for every $n,m\in \omega$, $$m<n \iff \N \models \theta[n,m].$$
Thanks to Lagrange's four square theorem, we could define $<$ on the positive integers. But it's unclear how to proceed further.

\smallskip

\begin{theorem}
If $\B$ is infinite, then for every infinite  cardinal $\kappa$, there is some $\C$ such that $\C \equiv \B$ and $\card(\C) = \kappa$.
\end{theorem}

\begin{corollary}
If $\B$ is infinite, then there exists a $\C$ such that $\C \equiv \B$ and $\C \not\cong \B$.
\end{corollary}

Therefore, $\Delta$ does \emph{not} categorically describe $\N$. Now, consider the structure $\widetilde{\N}$ obtained from $\N$ by adding a single point $\bullet$ fixed by $\S$. Then the sentence $\forall{x}\left(S{x} \ne x\right)$ is true in $\N$ but not in $\widetilde{\N}$. Moreover, $\Delta \subset \thh(\widetilde{\N})$, which proves that $$\cn(\Delta) \ne \thh(\N).$$ With this in mind, let $\Sigma = \Delta \cup \left\{ \forall{x}\left(\S^n{x} \ne x\right) \mid n \in \omega \right\}$. To show that $\cn(\Sigma) = \thh(\N)$, it suffices to show that for any $\B$, if $\B \models \Sigma$, then $\B \equiv \N$. To this end, for any cardinal $\kappa$, let $\A_{\kappa} = \N \cup \left(\kappa \times \Z\right)$, which is precisely the structure obtained form $\N$ by adding $\kappa$ many disjoint $\Z$-chains.

\begin{question}
How many structures are there up to isomorphism that 
\begin{enumerate}[label=(\alph*)]
\item satisfy $\Sigma$ and 
\item are of cardinality $\kappa$?
\end{enumerate}
\end{question}

If $\kappa < \omega$, then $\card(\A_{\kappa}) = \aleph_0$. Also, if  $\kappa > \omega$, then $\card(\A_{\kappa}) = \kappa$, so that $\Sigma$ is \textit{$\kappa$-categorical}, i.e., every structure satisfying both (a) and (b) is isomorphic to $\A_{\kappa}$.  

\subsection{Lecture 2}

\end{document}

\documentclass[10pt,letterpaper,cm]{nupset}
\usepackage[margin=1in]{geometry}
\usepackage{graphicx}
 \usepackage{enumitem}
 \usepackage{stmaryrd}
 \usepackage{bm}
\usepackage{amsfonts}
\usepackage{amssymb}
\usepackage{pgfplots}
\usepackage{amsmath,amsthm}
\usepackage{lmodern}
\usepackage{tikz-cd}
\usepackage{faktor}
\usepackage{xfrac}
\usepackage{mathtools}
\usepackage{bm}
\usepackage{ dsfont }
\usepackage{mathrsfs}
\usepackage{comment}
\usepackage{hyperref}
\usepackage{scrextend}
\hypersetup{colorlinks=true, linkcolor=red,          % color of internal links (change box color with linkbordercolor)
    citecolor=green,        % color of links to bibliography
    filecolor=magenta,      % color of file links
    urlcolor=cyan           }
\usepackage{adjustbox}
\usepackage{media9}
\usepackage{thmtools}
\usepackage[capitalise]{cleveref} 
\usepackage{quiver}
    
\theoremstyle{definition}
\newtheorem{definition}{Definition}%[section]
\newtheorem{exmp}[definition]{Example}
\newtheorem{non-exmp}[definition]{Non-example}
\newtheorem{note}[definition]{Note}

\theoremstyle{theorem}
\newtheorem{theorem}[definition]{Theorem}
\newtheorem{lemma}[definition]{Lemma}
\newtheorem{corollary}[definition]{Corollary}
\newtheorem{prop}[definition]{Proposition}
\newtheorem{conj}[definition]{Conjecture}
\newtheorem*{claim}{Claim}
\newtheorem{exercise}[definition]{Exercise}

\theoremstyle{remark}
\newtheorem{remark}[definition]{Remark}
\newtheorem*{todo}{To do}
\newtheorem*{conv}{Convention}
\newtheorem*{aside}{Aside}
\newtheorem*{notation}{Notation}
\newtheorem*{term}{Terminology}
\newtheorem*{background}{Background}
\newtheorem*{further}{Further reading}
\newtheorem*{sources}{Sources}

\makeatletter
\def\th@plain{%
  \thm@notefont{}% same as heading font
  \itshape % body font
}
\def\th@definition{%
  \thm@notefont{}% same as heading font
  \normalfont % body font
}
\makeatother

\makeatletter
\renewcommand*\env@matrix[1][*\c@MaxMatrixCols c]{%
  \hskip -\arraycolsep
  \let\@ifnextchar\new@ifnextchar
  \array{#1}}
\makeatother
\pgfplotsset{unit circle/.style={width=4cm,height=4cm,axis lines=middle,xtick=\empty,ytick=\empty,axis equal,enlargelimits,xmax=1,ymax=1,xmin=-1,ymin=-1,domain=0:pi/2}}
\DeclareMathOperator{\Ima}{Im}
\newcommand{\A}{\mathcal A}
\newcommand{\C}{\mathbb C}
\newcommand{\E}{\mathcal E}
\newcommand{\CP}{\mathbb CP}
\newcommand{\F}{\mathcal F}
\newcommand{\G}{\vec G}
\renewcommand{\H}{\vec H}
\newcommand{\HP}{\mathbb HP}
\newcommand{\K}{\mathbb K}
\renewcommand{\L}{\mathcal L}
\newcommand{\M}{\mathbb M}
\newcommand{\N}{\mathbb N}
\renewcommand{\O}{\mathbf O}
\newcommand{\OP}{\mathbb OP}
\renewcommand{\P}{\mathcal P}
\newcommand{\Q}{\mathbb Q}
\newcommand{\I}{\mathbb I}
\newcommand{\R}{\mathbb R}
\newcommand{\RP}{\mathbb RP}
\renewcommand{\S}{\mathbf S}
\newcommand{\T}{\mathbf T}
\newcommand{\X}{\mathbf X}
\newcommand{\Z}{\mathbb Z}
\newcommand{\B}{\mathcal{B}}
\newcommand{\1}{\mathbf{1}}
\newcommand{\ds}{\displaystyle}
\newcommand{\ran}{\right>}
\newcommand{\lan}{\left<}
\newcommand{\bmat}[1]{\begin{bmatrix} #1 \end{bmatrix}}
\renewcommand{\a}{\vec{a}}
\renewcommand{\b}{\vec b}

\renewcommand{\c}{\mathscr{C}}
\renewcommand{\d}{\mathscr{D}}
\newcommand{\e}{\mathscr{E}}
\newcommand{\y}{\mathcal{Y}}

\newcommand{\h}{\vec h}
\newcommand{\f}{\vec f}
\newcommand{\g}{\vec g}
\renewcommand{\i}{\vec i}
\renewcommand{\j}{\vec j}
\renewcommand{\k}{\vec k}
\newcommand{\n}{\vec n}
\newcommand{\p}{\vec p}
\newcommand{\q}{\vec q}
\renewcommand{\r}{\vec r}
\newcommand{\s}{\vec s}
\renewcommand{\t}{\vec t}
\renewcommand{\u}{\vec u}
\renewcommand{\v}{\vec v}
\newcommand{\w}{\vec w}
\newcommand{\x}{\vec x}
\newcommand{\z}{\vec z}
\newcommand{\0}{\vec 0}
\DeclareMathOperator*{\Span}{span}
\DeclareMathOperator*{\GL}{GL}
\DeclareMathOperator*{\ev}{ev}
\DeclareMathOperator{\rng}{range}
\DeclareMathOperator{\gemu}{gemu}
\DeclareMathOperator{\almu}{almu}
\newcommand{\Char}{\mathsf{char}}
\DeclareMathOperator{\id}{id}
\DeclareMathOperator{\true}{\mathtt{true}}
\DeclareMathOperator{\im}{Im}
\DeclareMathOperator{\graph}{Graph}
\DeclareMathOperator{\gal}{Gal}
\DeclareMathOperator{\tr}{Tr}
\DeclareMathOperator{\norm}{N}
\DeclareMathOperator{\aut}{Aut}
\DeclareMathOperator{\Int}{Int}
\DeclareMathOperator{\ext}{Ext}
\DeclareMathOperator{\stab}{Stab}
\DeclareMathOperator{\orb}{Orb}
\DeclareMathOperator{\sieves}{\mathtt{sieves}}
\DeclareMathOperator{\inn}{Inn}
\DeclareMathOperator{\out}{Out}
\DeclareMathOperator{\op}{op}
\DeclareMathOperator{\fix}{Fix}
\DeclareMathOperator{\ab}{ab}
\DeclareMathOperator{\sgn}{sgn}
\DeclareMathOperator{\syl}{syl}
\DeclareMathOperator{\Syl}{Syl}
\DeclareMathOperator{\ob}{ob}
\DeclareMathOperator{\el}{\mathtt{El}}
\DeclareMathOperator{\sub}{\mathtt{Sub}}
\DeclareMathOperator{\mor}{mor}
\DeclareMathOperator{\ar}{Ar}
\DeclareMathOperator{\dom}{dom}
\DeclareMathOperator{\ZFC}{ZFC}

\newcommand{\bi}{\begin{itemize}}
\newcommand{\ei}{\end{itemize}}

\newcommand{\be}{\begin{enumerate}}
\newcommand{\ee}{\end{enumerate}}

\linespread{1}

% info for header block in upper right hand corner
\name{Perry Hart}
\class{April 8, 2022}

\setlength\parindent{0pt}

\begin{document}

\begin{abstract}
This is a brief introduction to elementary toposes. These play a central role in categorical semantics of dependent type theory (along with other areas of categorical logic). We assume knowledge of basic category theory, and our main source for this material is the $n$Lab.
\end{abstract}

\bigskip

Let $\c$ be a category with finite limits. For any object $A \in \ob{\c}$, a \textit{power object} of $A$ is  an object $\P(A)$ of $\c$ together with a monomorphism $\in_A  \to A \times \P(A)$ such that for every monomorphism $ f: C \to A \times D$ in $\c$, there is a unique pullback square of the form
\[
\begin{tikzcd}[column sep=huge, row sep=large, ampersand replacement=\&]
	C \& \in_A \\
	A \times D \& A \times \P(A)
	\arrow[from=1-2, to=2-2]
	\arrow["f"', from=1-1, to=2-1]
	\arrow["{\id_A \times \chi_f}"', from=2-1, to=2-2]
	\arrow[from=1-1, to=1-2]
	\arrow["\lrcorner"{anchor=center, pos=0.125}, draw=none, from=1-1, to=2-2]
\end{tikzcd}
.\] We call $\chi_f$ the \textit{classifying map} of $f$. If $A = 1$, then a power object of $A$ is called a \textit{subobject classifier}.

\medskip

A category $\e$ is an  \textit{elementary topos} if it
\bi
\item has finite limits,
\item is cartesian closed, and
\item has a subobject classifier $\true : 1 \to \Omega$.
\ei
In this case, any global element $1 \to \Omega$  is called a \textit{truth value}. 

\medskip

Let $\c$ be a category with finite limits. The mapping $X \in \ob{\c} \ \mapsto \ \sub(X)$, the subobject poset of $X$, induces a functor $\c^{\op} \to \mathbf{Set}$ sending a map $A \xrightarrow{f} B$ in $\c$ to the function $\sub(B) \xrightarrow{f^{\ast}({-})} \sub(A)$ of sets. The functor $\sub({-})$ is represented by $\Gamma$ if and only if $\Gamma$ is a subobject classifier of $\c$. By uniqueness of representing objects, it follows that a subobject classifier is unique up to isomorphism.

\begin{theorem}[Fundamental theorem of topos theory]
If $\e$ is a topos and $X \in \ob{\e}$, then the overcategory ${\e}/{X}$ is a topos.
\end{theorem}

\begin{corollary}\label{tlcc}
Every topos is locally cartesian closed.
\end{corollary}

\begin{prop}\label{power}
A category $\c$ with finite limits is a topos if and only if every object of $\c$ has a power object. 
\end{prop}

In particular, for any topos $\e$ and $A\in \ob{\e}$, the exponential object $\Omega^A$ is a power object of $A$. In this case, the power object functor $\Omega^{\left({-}\right)} : \e^{\op} \to \e$ sends a map  $X \xrightarrow{f} Y$ in $\e$ to the transpose of the composite 
\[
\Omega^Y \times X \xrightarrow{\id_{\Omega^B} \times f}  \Omega^{Y} \times Y \xrightarrow{\ev_{Y, \Omega}} \Omega
\] under the adjunction ${-} \times X \vdash {-}^X$. We have a chain of natural isomorphisms
\[
\e(X, \Omega^Y) \cong \e(X \times Y, \Omega) \cong \e(Y \times X, \Omega) \cong \e(Y, \Omega^X) \cong \e^{\op}(\Omega^X, Y)
,\] which gives us an adjunction $\left(\Omega^{\left({-}\right)}\right)^{\op} \vdash \Omega^{\left({-}\right)}$. 
By an argument due to Par\'e, this adjunction is \textit{monadic} in the sense that $\Omega^{\left({-}\right)}$ reflects isomorphisms and preserves reflexive coequalizers, which implies that $\Omega^{\left({-}\right)}$ creates limits. Since $\e$ has finite limits as a topos, it follows that $\e^{\op}$ has finite limits, i.e., $\e$ has finite \emph{colimits}. In particular, $\e$ has an initial object $0$. 

\begin{lemma}
The initial object of $\e$ is strict.
\end{lemma} 
\begin{proof}
Let $X \in \ob{\e}$ and $f : X \to 0$. We must show that $f$ is an isomorphism. Notice that the map $0 \xrightarrow{\id_0} 0$ is both initial and terminal in the overcategory ${\e}/{0}$. The pullback functor $f^{\ast} : {\e}/{0} \to {\e}/{X}$ has a left adjoint and thus preserves limits. Therefore, $g \coloneqq f^{\ast}(\id_0)$ is terminal in ${\e}/{X}$. We thus have an isomorphism $h : g \xrightarrow{\cong} \id_X$. Moreover, $f^{\ast}$ has a right adjoint by \cref{tlcc} and thus preserves colimits. Hence $g$ is also initial in ${\e}/X$. This means that $\dom(g) =0$, and $f \circ g = \id_0$. The map $h$ gives us an isomorphism $ h: 0 \xrightarrow{\cong} X$ in $\e$ such that $g = \id_A \circ h$. This implies that  $f = h^{-1}$, so that $f$ is an isomorphism. 
\end{proof}

\begin{corollary}
For any $X \in \ob{\e}$, the unique map $0 \to X$ is monic.
\end{corollary}

Notably, the classifying map of $0 \to 1$ is called \textit{false}.

\smallskip

\begin{prop}
Suppose that $ \Omega \xrightarrow{f} \Omega$ is monic. Then $f \circ f = \id_{\Omega}$ (so that $f$ is an automorphism). 
\end{prop}

\smallskip

\begin{exmp} $ $
\be
\item The category $\mathbf{Set}$ is a \textit{Boolean} topos, i.e., $\Omega \cong 1 \coprod 1$. 
\item For any small category $\c$, the presheaf category $\widehat{\c} \coloneqq \left[ \c^{\op}, \mathbf{Set}\right]$ is a topos where the functor $\Omega$ sends $U \in \ob{\c}$ to the set $\sieves(U)$ of \textit{sieves on $U$}, i.e., sets $\sigma$ of morphisms over $U$ such that for any morphisms $f : X \to Y$ and $g : Y \to U$ in $\c$,
\[
Y \xrightarrow{g} U \  \in \ \sigma \ \ \implies \ \  X \xrightarrow{f} Y \xrightarrow{g} U \ \in \ \sigma 
.\] 
The action of $\Omega$ on morphisms in $\c$ is defined by
\[
V \xrightarrow{h} U  \ \  \mapsto \ \  \sigma \mapsto \left\{f : X \to V \mid h \circ f \in \sigma, \ X \in \ob{\c}    \right\}   
.\]

The sieve on $U$ generated by  $\id_U$ is the top element $\mathtt{sieve}_{\mathtt{top}}(U)$ of $\sieves(U)$. We define $\true : 1 \to \Omega$ as  the natural transformation with components
\begin{align*}
\true(U) & : \left\{\ast\right\} \to \sieves(U)
\\ \ast & \mapsto \mathtt{sieve}_{\mathtt{top}}(U).
\end{align*}
For any monomorphism $\varphi: F \hookrightarrow G$ in $\widehat{\c}$, the classifying map of $\varphi$ has components
\begin{align*}
& \chi_{\varphi}(U)  : G(U) \to \Omega(U)
\\ & x  \mapsto  \left\{f : X \to U \mid G(f)(x) \in  F(X),\ X \in \ob{\c}   \right\}.
\end{align*}

The subobject $\Omega_{\mathtt{dec}} \hookrightarrow \Omega$ consisting of decidable sieves classifies all monomorphisms $F \xrightarrow{\psi} G$ in $\widehat{\c}$ such that $\psi_A : F(A) \to G(A)$ has decidable image for every $A \in \ob{\c}$. Here, for any set $T$, a subset $S \subset T$ is decidable if and only if for any $x \in T$, the disjunction $x \in S \vee x \notin S$ is provable. If our metatheory includes $\mathsf{LEM}$, then $\Omega_{\mathtt{dec}} = \Omega$.
\ee
\end{exmp}

\begin{comment}  
\begin{note} 
Let $\c$ be a small category and let $\y : \c \to \widehat{\c}$ denote the Yoneda embedding.
Let $U \in \ob{\c}$. For any sieve $\sigma$, define the subfunctor $F_{\sigma} \hookrightarrow \y_{U}$ by 
\[
A  \mapsto  \y_U(A) \cap \sigma
\] for  all $A \in \ob{\c}$. Conversely, for every subfunctor $F$ of $\y_U$, define the sieve 
\[
\sigma_F \equiv \coprod_{X \in \ob{\c}}F(X)
\] on $U$. Then $F_{-} : \sieves(U) \to \sub(\y_U)$ is a bijection with inverse $\sigma_{-}$.
\end{note}
\end{comment}

\medskip

\begin{definition}[Heyting algebra]
Let $L$ be a bounded lattice. We say that $L$ is a \textit{Heyting algebra} if it has a binary operation $\Rightarrow : L \times L \to L$, called \textit{implication}, such that 
\begin{align*}
p \Rightarrow p &  = 1
\\ p \land \left(p \Rightarrow q\right) & = p \land q
\\ q \land \left(p \Rightarrow q\right) & = q
\\ p \Rightarrow \left(q \land r\right) & = \left(p \Rightarrow q\right) \land \left(p \Rightarrow r\right).
\end{align*}
\end{definition}

\smallskip

For any topos $\e$ and $A \in \ob{\e}$, the poset $\sub(A)$ is a Heyting algebra. As a result, $\sub(A)$ is a model of intuitionistic  propositional calculus.  For example, the meet $\cap$ and join $\cup$ operation for $\sub(A)$ are precisely the binary product and binary coproduct in $\sub(A)$, respectively.  

\begin{prop}
Let $U_1$ and $U_2$ be subobjects of $A$. 
\be
\item We have a pullback square 
\[
\begin{tikzcd}[ampersand replacement=\&]
	{U_1 \cap U_2} \& {U_2} \\
	{U_1} \& A
	\arrow[from=1-1, to=2-1]
	\arrow[from=1-1, to=1-2]
	\arrow[from=2-1, to=2-2]
	\arrow[from=1-2, to=2-2]
	\arrow["\lrcorner"{anchor=center, pos=0.125}, draw=none, from=1-1, to=2-2]
\end{tikzcd}
\] in $\e$ consisting of monomorphisms.
\item We have a pushout square
\[
\begin{tikzcd}[ampersand replacement=\&]
	{U_1 \cap U_2} \& {U_2} \\
	{U_1} \& {U_1 \cup U_2} \\
	\&\& A
	\arrow[from=1-1, to=1-2]
	\arrow[from=2-1, to=2-2]
	\arrow[from=1-2, to=2-2]
	\arrow[from=1-1, to=2-1]
	\arrow["\lrcorner"{anchor=center, pos=0.125, rotate=180}, draw=none, from=2-2, to=1-1]
	\arrow[curve={height=18pt}, from=2-1, to=3-3]
	\arrow[curve={height=-18pt}, from=1-2, to=3-3]
	\arrow["\alpha", dashed, from=2-2, to=3-3]
\end{tikzcd}
\] in $\e$ where $\alpha$ is a monomorphism.
\ee
\end{prop}

Moreover, implication $\sub(A) \times \sub(A) \xrightarrow{\Rightarrow} \sub(A)$ is defined by 
\[
U_1 \Rightarrow U_2 \ = \ \Pi_{U_1}(U_1 \cap U_2) 
,\]  
where $\Pi$ denotes the dependent product.

\smallskip

\begin{remark}
A \textit{Boolean algebra} is a Heyting algebra $L$ where every $x \in L$ has a complement, i.e., an element $c_x \in L$ such that $x \vee c_x = 1$ and $x \land c_x = 0$. A topos $\e$ is Boolean if and only if $\sub(A)$ is a Boolean algebra for all $A \in \ob{\e}$. In this case, $\sub(A)$ satisfies $\mathsf{LEM}$.
\end{remark}

\medskip

Let $\e$ be a topos and consider a map $\el : \widehat{U} \to U$ in $\e$. We say that a map $f: X \to Y$ in $\e$ is \textit{$U$-small} if there exists a pullback square  of the form  
\[ \label{pbu}
\begin{tikzcd}[ampersand replacement=\&]
	X \& {\widehat{U}} \\
	Y \& U
	\arrow["\el", from=1-2, to=2-2]
	\arrow[from=2-1, to=2-2]
	\arrow["f"', from=1-1, to=2-1]
	\arrow[from=1-1, to=1-2]
	\arrow["\lrcorner"{anchor=center, pos=0.125}, draw=none, from=1-1, to=2-2]
\end{tikzcd} \tag{$\ast$}
.\] Note that the class of $U$-small maps is closed under pullbacks.
We say that $\el$ is a \textit{universe  in $\e$} if the class of $U$-small maps 
\be[label=(\alph*)]
\item is closed under  
\bi
\item products,  
\item dependent sums,
\item dependent products, and 
\item pullbacks of $1 \xrightarrow{\true} \Omega$ and
\ei
\item contains the unique map $\Omega \to 1$.
\ee 
Condition (b) expresses that $U$ is \textit{impredicative}.

\medskip

Although the square \eqref{pbu} need not be unique, it is when $U$ has the structure of a univalent type universe.

\begin{exmp}
The subobject classifier is a \emph{predicative} universe as long as $\Omega \ne 1$, and the $\Omega$-small maps are precisely the monomorphisms.
\end{exmp}

\begin{remark}
Closure under dependent sums is sometimes used as an alternative definition of \textit{impredicative}, in which case $\Omega$ is impredicative. Unfortunately, both definitions appear in the type theory literature.
\end{remark}

\end{document} 
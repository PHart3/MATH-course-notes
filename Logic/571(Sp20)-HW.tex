  %  \nonstopmode
  % !TEX encoding = UTF-8 Unicode
\documentclass[10pt,letterpaper,cm]{nupset}
\usepackage[margin=1in]{geometry}
\usepackage[utf8]{inputenc}
\usepackage[T1]{fontenc}
\usepackage{graphicx}
\usepackage{enumerate}
\usepackage{enumitem}
\usepackage{float}
\usepackage{stmaryrd}
\usepackage{amsfonts}
\usepackage{amssymb}
\usepackage{mathtools}
\usepackage{upgreek}
\usepackage{pgfplots}
\pgfplotsset{compat=1.13}
\usepackage{amsmath,amsthm}
\usepackage{tikz-cd}
\usetikzlibrary{knots,calc}
\usepackage{xcolor}
\usepackage{soul}
\usetikzlibrary{decorations.markings}
\usepackage{faktor}
\usepackage{xfrac}
\usepackage{ mathrsfs }
\usepackage{hyperref}
\usepackage{comment}
\hypersetup{colorlinks=true, linkcolor=red,          % color of internal links (change box color with linkbordercolor)
    citecolor=green,        % color of links to bibliography
    filecolor=magenta,      % color of file links
    urlcolor=cyan           }
\usepackage{adjustbox}
\usepackage{media9}


\usepackage{thmtools}
\usepackage[capitalise]{cleveref} 
    
\theoremstyle{definition}
\newtheorem{definition}{Definition}
\newtheorem{exmp}[definition]{Example}
\newtheorem{non-exmp}[definition]{Non-example}
\newtheorem{note}[definition]{Note}

\theoremstyle{theorem}
\newtheorem{theorem}[definition]{Theorem}
\newtheorem{lemma}[definition]{Lemma}
\newtheorem{prop}[definition]{Proposition}
\newtheorem{corollary}[definition]{Corollary}
\newtheorem{claim}{Claim}
\newtheorem{exercise}[definition]{Exercise}

\theoremstyle{remark}
\newtheorem*{remark}{Remark}
\newtheorem*{todo}{To do}
\newtheorem*{question}{Question}
\newtheorem*{conv}{Convention}
\newtheorem*{aside}{Aside}
\newtheorem*{notation}{Notation}
\newtheorem*{term}{Terminology}
\newtheorem*{background}{Background}
\newtheorem*{further}{Further reading}
\newtheorem*{sources}{Sources}

\makeatletter
\def\th@plain{%
  \thm@notefont{}% same as heading font
  \itshape % body font
}
\def\th@definition{%
  \thm@notefont{}% same as heading font
  \normalfont % body font
}
\makeatother


\makeatletter
\renewcommand*\env@matrix[1][*\c@MaxMatrixCols c]{%
  \hskip -\arraycolsep
  \let\@ifnextchar\new@ifnextchar
  \array{#1}}
\makeatother
\pgfplotsset{unit circle/.style={width=4cm,height=4cm,axis lines=middle,xtick=\empty,ytick=\empty,axis equal,enlargelimits,xmax=1,ymax=1,xmin=-1,ymin=-1,domain=0:pi/2}}
\DeclareMathOperator{\Ima}{Im}
\newcommand{\A}{\mathbb A}
\newcommand{\C}{\mathbb C}
\newcommand{\D}{\mathbb D}
\newcommand{\E}{\mathbb E}
\newcommand{\CP}{\mathbb{CP}}
\newcommand{\F}{\mathbb F}
\newcommand{\G}{\vec G}
\renewcommand{\H}{\mathbb H}
\newcommand{\HP}{\mathbb HP}
\newcommand{\K}{\mathbb K}
\renewcommand{\L}{\mathcal L}
\newcommand{\N}{\mathbb N}
\renewcommand{\O}{\mathbf O}
\newcommand{\OP}{\mathbb OP}
\renewcommand{\P}{\mathcal P}
\newcommand{\Q}{\mathbb Q}
\newcommand{\U}{\mathcal U}
\newcommand{\I}{\mathbb I}
\newcommand{\R}{\mathbb{R}}
\newcommand{\RP}{\mathbb{RP}}
\renewcommand{\S}{\mathbb S}
\newcommand{\T}{\mathcal T}
\newcommand{\X}{\mathbf X}
\newcommand{\Z}{\mathbb Z}
\newcommand{\B}{\mathbb{B}}
\newcommand{\1}{\mathbb{1}}
\newcommand{\ds}{\displaystyle}
\newcommand{\ran}{\right>}
\newcommand{\lan}{\left<}
\newcommand{\bmat}[1]{\begin{bmatrix} #1 \end{bmatrix}}
\renewcommand{\a}{\vec{a}}
\renewcommand{\b}{\vec b}
\renewcommand{\c}{\vec c}
\renewcommand{\d}{\vec d}
\newcommand{\e}{\vec e}
\newcommand{\h}{\vec h}
\newcommand{\f}{\vec f}
\newcommand{\g}{\vec g}
\renewcommand{\i}{\vec i}
\renewcommand{\j}{\vec j}
\renewcommand{\k}{\vec k}
\newcommand{\n}{\vec n}
\newcommand{\p}{\vec p}
\newcommand{\q}{\vec q}
\renewcommand{\r}{\vec r}
\newcommand{\s}{\vec s}
\renewcommand{\t}{\vec t}
\renewcommand{\u}{\vec u}
\newcommand{\w}{\vec w}
\newcommand{\x}{\vec x}
\newcommand{\y}{\vec y}
\newcommand{\z}{\vec z}
\newcommand{\0}{\vec 0}
\newcommand{\pt}{\mathsf{pt}}
\newcommand{\from}{\longleftarrow}
\newcommand{\intprodl}{%
    \mathbin{\scalebox{1.5}{$\lrcorner$}}%
}
\newcommand{\intprodr}{%
    \mathbin{\scalebox{1.5}{$\llcorner$}}%
}
\DeclareMathOperator*{\Span}{span}
\DeclareMathOperator{\rng}{range}
\DeclareMathOperator{\gemu}{gemu}
\DeclareMathOperator{\almu}{almu}
\newcommand{\Char}{\mathsf{char}}
\DeclareMathOperator{\id}{id}
\DeclareMathOperator{\tr}{Tr}
\DeclareMathOperator{\tor}{Tor}
\DeclareMathOperator{\im}{im}
\DeclareMathOperator{\homeo}{Homeo}
\DeclareMathOperator{\GL}{GL}
\DeclareMathOperator{\SL}{SL}
\DeclareMathOperator{\norm}{N}
\DeclareMathOperator{\aut}{Aut}
\DeclareMathOperator{\Int}{Int}
\DeclareMathOperator{\ext}{Ext}
\DeclareMathOperator{\supp}{supp}
\DeclareMathOperator{\cl}{cl}
\DeclareMathOperator{\dom}{dom}
\DeclareMathOperator{\rnk}{rank}
\DeclareMathOperator{\Hom}{Hom}
\DeclareMathOperator{\Alt}{Alt}
\DeclareMathOperator{\dr}{dR}
\DeclareMathOperator{\ed}{End}
\DeclareMathOperator{\BM}{BM}
\DeclareMathOperator{\ob}{ob}
\DeclareMathOperator{\clength}{cup{-}length}
\DeclareMathOperator{\sgn}{sgn}
\DeclareMathOperator{\orb}{Orb}
\DeclareMathOperator{\cyl}{Cyl}
\DeclareMathOperator{\rel}{rel}
\DeclareMathOperator{\res}{res}
\DeclareMathOperator{\cat}{cat}
\DeclareMathOperator{\op}{op}
\DeclareMathOperator{\Gd}{Gd}
\DeclareMathOperator{\coker}{coker}
\DeclareMathOperator{\map}{Map}
\DeclareMathOperator{\sing}{Sing}
\DeclareMathOperator{\Op}{\mathbf{Op}}
\DeclareMathOperator{\colim}{colim}
\DeclareMathOperator{\tot}{Tot}
\DeclareMathOperator{\ef}{EF}
\DeclareMathOperator{\thh}{\mathsf{Th}}
\DeclareMathOperator{\cn}{\mathsf{Cn}}
\DeclareMathOperator{\pgl}{PGL}
\DeclareMathOperator{\fo}{\mathsf{FO}}
\DeclareMathOperator{\Et}{\acute{E}t}
\DeclareMathOperator{\ch}{\mathbf{Ch}}
\DeclareMathOperator{\vf}{\mathscr{X}}

\newcommand{\bi}{\begin{itemize}}
\newcommand{\ei}{\end{itemize}}

\newcommand{\be}{\begin{enumerate}}
\newcommand{\ee}{\end{enumerate}}

\newcommand{\bmp}{\begin{mathpar}}
\newcommand{\emp}{\end{mathpar}}

\setlength{\parindent}{0pt}

\newcommand{\mathcolorbox}[2]{\colorbox{#1}{$\displaystyle #2$}}

\newlist{steps}{enumerate}{1}
\setlist[steps, 1]{label = Step \arabic*:}

% info for header block in upper right hand corner
\name{Perry Hart}
\class{MATH 571}
\assignment{Exercises}
\duedate{Spring 2020}

\begin{document}


\begin{problem}[1.] 
Let $I$ be a countably infinite set. Let $\D \coloneqq \langle I, E\rangle$ be a structure where $E$ is an equivalence relation for which  there is exactly one equivalence class of size $k$ for each $k\in \Z_{\geq 1}$.
\be[label=(\arabic*)]
\item Show that the set $\Lambda$ of (first-order) sentences expressing that $E$ is an equivalence relation with exactly one equivalence class of size $k$ for each $k\in \Z_{\geq 1}$ axiomatizes $\D$, i.e., $\thh(\D) = \cn(\Lambda)$ where 
\[
\cn(\Lambda) \coloneqq  \left\{\varphi \in \fo_{\D} \mid \Lambda \models \varphi\right\}
.\]
\item Show that for every (first-order) formula $\theta(y, \overline{w})$ and every $\bar{a}\in I$, the set $$\theta\left[\D, \bar{a}\right] \coloneqq \left\{x\in \dom(\D) \mid \D\models \theta\left[x, \bar{a}\right]\right\}$$ is either finite or cofinite.
\ee
\end{problem}
\begin{solution} $ $
\be[label=(\arabic*)]
\item It suffices to prove that $\Lambda$ is complete. For, in this case, any two models of $\Lambda$ must be elementarily equivalent.
\begin{claim}\label{C1}
Let $\E$ be any model of $\Lambda$ of size $\kappa \geq \omega$. There exists an elementary extension $\E_{\kappa} \succeq \E$ of size $\kappa$ such that $\E_{\kappa}$ has exactly $\kappa$ equivalence classes each of size $\kappa$.
\end{claim} 
\begin{proof}
Let $\lambda$ denote the cardinality of the set of all equivalence classes in $\dom(\E)$. Note that $\lambda \leq \kappa$. For every $\alpha,\beta \in \kappa$, adjoin to the language of $\E$ a new constant symbol $c(\alpha, \beta)$. Consider the theory
\[
\Delta \coloneqq \Lambda   \cup \left\{E{c(x,y)}{c(x,z)} \mid x,y,z \in \kappa \right\} \cup \left\{ \neg{E{c(x,0)}{c(y,0)}} \mid x,y\in \kappa, \ x\ne y\right\} .
\]
Any finite subset $F$ of $\Delta$ is satisfiable by a suitable expansion $\E_F$ of $\E$. Then there exists an ultrafilter on the family of finite subsets of $\Delta$ such that the ultraproduct $$\prod_{\underset{\text{finite}}{F\subset \Delta}}\E_F/\mathcal{U}$$ satisfies $\Delta$. Moreover, its reduct $\A$ to the language of $\E$ is an elementary extension of $\E$. By the downward L\"owenheim-Skolem theorem, there exists a structure $\E_0$ of size $\kappa$ such that $\A \succeq \E_0 \succeq \E$.

Now, repeat our preceding construction $\omega$ times to get an increasing chain
\[
\E \preceq \E_0 \preceq \E_1 \preceq  \E_2 \preceq \cdots
\]
of structures such that each $\dom(\E_i)$ has cardinality $\kappa$. Note that $\E_{\kappa}$ is an elementary extension of $\E$. Further, the universe of $\E_{\kappa} \coloneqq \bigcup_{i\in \omega}\E_i$ also has cardinality $\kappa$, so that $\E_{\kappa}$ has exactly $\kappa$ equivalence classes. Finally, for any $x\in \E_{\kappa}$, $x$ belongs to some $\E_n$. Hence the equivalence class $\left[x\right]$ has size $\kappa$ in $\E_{n+1}$ and thus in $\E_{\kappa}$. It follows that every equivalence class in $\E_{\kappa}$ has size $\kappa$.
\end{proof}

Suppose, toward a contradiction, that there is a sentence $\varphi$ in the language of $\D$ such that neither $\varphi$ nor $\neg{\varphi}$ belongs to  $\cn(\Lambda)$. Then there are models $\A^1$ and $\A^2$ of $\Lambda$ a such that $\A^1 \models \neg{\varphi}$ and $\A^2 \models \varphi$. By the L\"owenheim-Skolem theorem, we may assume that both of these have size $\kappa \geq \omega$. By \cref{C1}, we thus have two  structures $\A^1_{\kappa}$ and $\A^2_{\kappa}$ such that $\A^1_{\kappa} \models \neg{\varphi}$ and $\A^2_{\kappa}\models \varphi$. But it's easy to see that $\A^1_{\kappa}$ and $\A^2_{\kappa}$ must be isomorphic, which yields a contradiction.
\item Suppose, toward a contradiction, that there exist a formula $\theta(y, w_1, \ldots, w_n)$ and an element $\bar{a}\in I$ such that $\theta\left[\D, \bar{a}\right]$ is both infinite and coinfinite. Adjoin to the language of $\D$ new constant symbols $\bar{e} \coloneqq \left(e_1, \ldots, e_n\right)$, $c$, and $d$. For each $k\in \Z_{\geq 1}$, let $\lambda_k(x)$ denote the formula expressing that the equivalence class of $x$ has cardinality $>k$. Now, consider the theory
\begin{align*}
\Gamma \coloneqq \Lambda  \cup \left\{\lambda_k(c) \mid k \geq 1\right\} & \cup \left\{\lambda_k(d) \mid k \geq 1\right\} 
\\ & \cup \left\{\neg{E{e_i}{c}} \mid 1\leq i \leq n\right\} 
\\ & \cup \left\{\neg{E{e_i}{d}} \mid 1\leq i \leq n\right\} 
\\ & \cup \left\{\theta(c, \bar{e}), \neg{\theta(d, \bar{e})}\right\}
\end{align*}
in our new language.

Let $F$ be any finite subset of $\Gamma$. Since both $\theta\left[\D, \bar{a}\right]$ and $\neg{\theta\left[\D, \bar{a}\right]}$ are infinite by assumption, we can find an expansion of $\D$ that satisfies $F$ by interpreting $\bar{e}$ as $\bar{a}$ and both $c$ and $d$ as members of large enough equivalence classes. By the compactness theorem, it follows that there is some model $\C$ of $\Gamma$, which must be infinite. Let $\C'$ denote the reduct of $\C$ to the language of $\D$. Thanks to the L\"owenheim-Skolem theorem, we may assume that $\dom(\C')$ is countable. Thus, the equivalence classes $\left[c^{\C}\right]$ and $\left[d^{\C}\right]$ are countable. Note that $e_i^{\C} \notin \left[c^{\C}\right] \cup \left[d^{\C}\right]$ for each $1\leq i \leq n$. Therefore, there is an automorphism of $\C'$ sending $c^{\C}$ to $d^{\C}$ and fixing each $e_i^{\C}$. But this contradicts the fact that $\C' \models \theta\left[c^{\C}, \bar{e}^{\C}\right] \land \neg{\theta\left[d^{\C}, \bar{e}^{\C}\right]}$.
\ee
\end{solution}


\begin{problem}[2.]
Show that a $\L$-structure $\A$ is finite if and only if for any $\L$-structure $\B$,
\[
\A \equiv \B \iff \A \cong \B
.\]
\end{problem}
\begin{solution}

$(\Longrightarrow)$

\smallskip

It is always true that any two isomorphic structures are elementarily equivalent. Thus, it remains to show that $\A \equiv \B \implies \A \cong \B$.

\smallskip

First, assume that $\L$ is finite. Consider the \textit{atomic diagram} of $\A$, i.e., the set $$\mathtt{D}(\A)\coloneqq \left\{\varphi \mid \underline{\A} \models \varphi, \  \varphi \text{ is either atomic or the negation of an atomic formula}\right\}$$ where $\underline{\A}$ denotes the expansion of $\A$ obtained by adjoining a constant symbol $c_a$ for each $a\in \dom(\A)$. Since both $\L$ and $\dom(\A)$ are finite, we can encode $\mathtt{D}(\A)$ with a single sentence $\psi$. Therefore, the sentence $$\psi_{\A} \coloneqq \forall{x}\left(\bigvee_{a\in \dom(\A)} x=c_a\right) \land \psi$$ has the property that $\underline{\B} \models \psi_{\A} \implies \B \cong \A$ for any other $\L$-structure $\B$. Now, if $\A \equiv \B$, then clearly both $\underline{\A}$ and $\underline{\B}$ satisfy $\psi_{\A}$, so that $\B \cong \A$.

\smallskip

Next, let $\L$ be arbitrary and let $\A \equiv \B$. Suppose, toward a contradiction, that $\A \not\cong \B$. Then for any bijection $f: \dom(\A) \to \dom(\B)$, there is some finite sublanguage $\L_f$ of $\L$ such that $f$ is \emph{not} an isomorphism $\A^{\L_f}\to \B^{\L_f}$ of reducts to $\L_f$. Consider the language $$\L' \coloneqq \bigcup_{\underset{\text{bijection}}{f:\dom(\A)\to \dom(\B)}}\L_f,$$ which is finite as the finite union of finite sets. Thanks to our preceding discussion, we obtain an isomorphism $g: \A^{\L'}\overset{\cong}{\longrightarrow} \B^{\L'}$. But $\L_g \subset \L'$ by our construction of $\L'$, and thus $g$ induces an isomorphism $\A^{\L_g}\overset{\cong}{\longrightarrow} \B^{\L_g}$, contrary to our choice of $\L_g$.

\medskip

$(\Longleftarrow)$

\smallskip

Suppose that $\A$ is infinite. We must find a structure $\B$ such that $\A \equiv \B$ but $\A \not\cong \B$. But this follows at once from the the L\"owenheim-Skolem theorem, which implies that $\thh(\A)$ has a model of any infinite size. 

\end{solution}


\begin{problem}[3.]
Let $\N^{\ast} = \langle \omega, <\rangle$. Show that for any infinite cardinal $\kappa$, $\thh(\N^{\ast})$ is not $\kappa$-categorical.
\end{problem}
\begin{solution}
Expand the language of $\N^{\ast}$ by adjoining countably many constants $\left\{c_i\right\}_{i\in \Z}$. Consider the theory
\[ \label{eqn:st1}
T \coloneqq \thh(\N^{\ast}) \cup \left\{ c_i > c_{i+1} \mid i\in \Z \right\}. \tag{$\star$}
\] in our new language.  Any finite subset of $T$ is satisfied by an expansion of $\N^{\ast}$ suitably interpreting the $c_i$ since $\N^{\ast}$ has descending chains of all finite lengths. By the compactness theorem, it follows that there is some model $\A$ of $T$, which must be infinite. If $\left\lvert{\A}\right\rvert >\aleph_0$, then apply the L\"owenheim-Skolem theorem to get a model $\B$ of $T$ such that $\left\lvert{\B}\right\rvert = \aleph_0$.  Let 
\[
\A' = \begin{cases} \B & \left\lvert{\A}\right\rvert >\aleph_0
\\ \A & \left\lvert{\A}\right\rvert =\aleph_0
\end{cases}.
\] Note that $\A' \models T$. Since the property of being a linearly ordered set is expressible by a first-order sentence, we see that $\A'$ is linearly ordered by $<$. Further, we see that $\A'$ has an infinite descending chain, which means that $\A'$ is not well-ordered by $<$.  But $\left(\omega, <\right)$ is a well-ordered set, and thus the reduct of $\A'$ to the language of $\N^{\ast}$ is not isomorphic to $\N^{\ast}$. It does, however, satisfy $\thh(\N^{\ast})$. This shows that $\thh(\N^{\ast})$ is not $\aleph_0$-categorical.

\medskip

Unfortunately, it's unclear that this method can be adapted to show that $\thh(\N^{\ast})$ is not $\kappa$-categorical when $\kappa$ is uncountable. In this case, we instead shall employ two binary  operations on the class of all linear orderings.
Let $L_1$ and $L_2$ be linearly ordered sets.
\begin{itemize}
\item $L_1 \cdot L_2$ refers to $L_1 \times L_2$ equipped with the lexicographic order.
\item ${L_1 + L_2}$ refers to $L_1$ with its ordering followed by $L_2$ with its ordering.
\end{itemize}
Now, consider the following linearly ordered structures:
\begin{gather*}
 {\N^{\ast} + \left(\Z \cdot \kappa\right)}
\\ {\N^{\ast} +  \left(\Z \cdot \left(\Q +\kappa\right)\right)},
\end{gather*} both of which have cardinality $\kappa$.
These orderings possess minimal elements and are \textit{discrete} in the sense that both structures satisfy the sentences
\begin{align}
\begin{split}
& \forall{x}\exists{y}\left(x<y \land \neg{\exists{z}\left(x<z \land z<y\right)}\right) \\
 \forall{x}&\left(\exists{w}\left(w < x\right)\rightarrow \exists{y}\left(y<x \land \neg{\exists{z}\left(y<z \land z<x\right)}\right)\right).
\end{split} \label{eqn:disc}
\end{align}
Note that, on the one hand, $ {\N^{\ast} + \left(\Z \cdot \kappa\right)}$ cannot possess an descending chain of length $\omega^2$, for otherwise $\kappa$, which is well-ordered, would possess an infinite descending chain. On the other hand, ${\N^{\ast} +  \left(\Z \cdot \left(\Q +\kappa\right)\right)}$ does possess such a chain since $\omega^{\ast}$ (the order type of $\Z_{<0}$) can be embedded in $\Q$. Therefore, \[
{\N^{\ast} + \left(\Z \cdot \kappa\right)} \not\cong {\N^{\ast} +  \left(\Z \cdot \left(\Q +\kappa\right)\right)}.
\]
\begin{definition}
Suppose that $\L$ is a finite language without function symbols. Let $\D$ and $\E$ be two $\L$-structures. Let $n\in \omega$. The \textit{Ehrenfeucht-Fra\"{\i}ss\'e  game $\ef_n(\D, \E)$ of length $n$ on $\D$ and $\E$} is a game of perfect information played as follows.
\begin{enumerate}[label=(\alph*)]
\item There are exactly two players, the \textit{spoiler} and the \textit{duplicator}.
\item There are exactly $n$ rounds.
\item The spoiler begins round $k\leq n$ by picking an element of either $\dom(\D)$ or $\dom(\E)$. Next, the duplicator picks an element of the other domain.
\item This yields two sequences $\left(d_1, \ldots, d_n\right)$ and $\left(e_1, \ldots, e_n\right)$ such that $d_i\in \dom(\D)$ and $e_i\in \dom(\E)$ for each $i=1, \ldots, n$.  If the mapping $d_i \mapsto e_i$ defines an isomorphism of finite substructures, then we say that the duplicator has won $\ef_n(\D, \E)$. Otherwise, we say that the spoiler has won.
\end{enumerate}
\end{definition}

\begin{theorem}[Fra\"{\i}ss\'e]\label{EF}
The duplicator has a winning strategy in $\ef_n(\D, \E)$ for each $n\in \omega$ if and only if $\D \equiv \E$.
\end{theorem}

\begin{claim}
Suppose that $\left(\E, <\right)$ is a discrete linear ordering with a minimal element but no maximal element. Then $\E \equiv \N^{\ast}$.
\end{claim}
\begin{proof}[Proof sketch]
Consider the Ehrenfeucht-Fra\"{\i}ss\'e  game $\ef_n(\E, \N^{\ast})$. The duplicator has a winning strategy in $\ef_n(\E, \N^{\ast})$ by adhering to the following rules.
\begin{enumerate}[label=(\roman*)]
\item If, in round $m$, the spoiler chooses an element of one of the structures that is connected to a previously chosen element or the minimal element by a path of successors of length $k<\infty$, then choose the corresponding element of the other structure in round $m$.
\item Otherwise, make sure that any chosen element of $\dom(\N^{\ast})$ is always separated by at least $n+1$ elements from any previously chosen element of $\dom(\N^{\ast})$ while preserving the required order of your choices. 

In this case, choose first a natural number separated by more than $3^n$ elements from the greatest previously chosen clement of $\dom(\N^{\ast})$.
\end{enumerate}
\end{proof}
Thanks to \cref{EF}, it follows that both $ {\N^{\ast} + \left(\Z \cdot \kappa\right)}$ and ${\N^{\ast} +  \left(\Z \cdot \left(\Q +\kappa\right)\right)}$ are elementarily equivalent to $\N^{\ast}$ and thus models of $\thh(\N^{\ast})$. Hence $\thh(\N^{\ast})$ is not $\kappa$-categorical. 
\end{solution}

\begin{problem}[4.]
Show that any set definable over $\N^{\ast}$ is either finite or cofinite. 
\begin{remark}
This shows that $\N^{\ast}$ is \textit{o-minimal} in the sense that every definable set over $\N^{\ast}$ is a finite union of points and intervals in $\omega$. 
\end{remark}
\end{problem}
\begin{solution}
Note that any set definable over $\N^{\ast}$ is $0$-definable  because any natural number $n$ is uniquely determined by the first-order property
\[
\begin{cases}
 ``n \text{ is less than any other element}" & n=0
\\``\text{there are exactly }n-1\text{ elements between $0$ (the minimal element) and }n"  & n>1
\end{cases}.
\]
Suppose, toward a contradiction, that there exist a formula $\theta(y)$ such that $\theta\left[\N^{\ast}\right]$ is both infinite and coinfinite. Consider, again, the theory \eqref{eqn:st1}. Let $$T' = T \cup \left\{\theta(c_0), \neg{\theta(c_1)}\right\}.$$ Since both $\theta\left[\N^{\ast}\right]$ and $\neg{\theta\left[\N^{\ast}\right]}$ are infinite by assumption, we can find an expansion of $\N^{\ast}$ that satisfies any finite subset of $T'$, By the compactness theorem together with the L\"owenheim-Skolem theorem, we thus can find a countable model $\D$ of $T'$ and take its reduct $\C$ to the language of $\N^{\ast}$. Note that $\left(\dom(\C), < \right)$ is a countable linear ordering with an  infinite descending and ascending chain $\chi$ on which both $c_0^{\D}$ and $c_1^{\D}$ lie. Moreover, this ordering is discrete in the sense of \eqref{eqn:disc}.
Therefore, we may assume that $\chi$ has the form
\[
 \cdots < x_{m-1} < x_m < x_{m+1} < \cdots
\] where $x_{m+1}$ denotes the immediate successor of $x_m$. There is an automorphism of $\C$ mapping $c_0^{\D}$ to $c_1^{\D}$ by suitably shifting $\chi$ finitely many places to the left and fixing all elements outside $\chi$. But this contradicts the fact that $\C \models \theta\left[c_0^{\D}\right] \land \neg{\theta\left[c_1^{\D}\right]}$.
\end{solution}

\end{document}
  %  \nonstopmode
\documentclass[10pt,letterpaper,cm]{nupset}
\usepackage[margin=1in]{geometry}
\usepackage{graphicx}
\usepackage{enumerate}
\usepackage{enumitem}
\usepackage{float}
\usepackage{stmaryrd}
\usepackage{amsfonts}
\usepackage{amssymb}
\usepackage{mathtools}
\usepackage{upgreek}
\usepackage{pgfplots}
\pgfplotsset{compat=1.13}
\usepackage{amsmath,amsthm}
\usepackage{tikz-cd}
\usetikzlibrary{knots,calc}
\usepackage{xcolor}
\usepackage{soul}
\usetikzlibrary{decorations.markings}
\usepackage{faktor}
\usepackage{xfrac}
\usepackage{ mathrsfs }
\usepackage{hyperref}
\hypersetup{colorlinks=true, linkcolor=red,          % color of internal links (change box color with linkbordercolor)
    citecolor=green,        % color of links to bibliography
    filecolor=magenta,      % color of file links
    urlcolor=cyan           }
\usepackage{adjustbox}
\usepackage{media9}


\usepackage{thmtools}
\usepackage[capitalise]{cleveref} 
    
\theoremstyle{definition}
\newtheorem{defn}{Definition}[subsection]
\newtheorem{exmp}[defn]{Example}
\newtheorem{non-exmp}[defn]{Non-example}
\newtheorem{note}[defn]{Note}

\theoremstyle{theorem}
\newtheorem{theorem}[defn]{Theorem}
\newtheorem{lemma}[defn]{Lemma}
\newtheorem{prop}[defn]{Proposition}
\newtheorem{fact}[defn]{Fact}
\newtheorem{corollary}[defn]{Corollary}
\newtheorem*{claim}{Claim}
\newtheorem{exercise}[defn]{Exercise}

\theoremstyle{remark}
\newtheorem{remark}[defn]{Remark}
\newtheorem*{todo}{To do}
\newtheorem*{question}{Question}
\newtheorem*{conv}{Convention}
\newtheorem*{aside}{Aside}
\newtheorem*{notation}{Notation}
\newtheorem*{term}{Terminology}
\newtheorem*{background}{Background}
\newtheorem*{further}{Further reading}
\newtheorem*{sources}{Sources}

\makeatletter
\def\th@plain{%
  \thm@notefont{}% same as heading font
  \itshape % body font
}
\def\th@definition{%
  \thm@notefont{}% same as heading font
  \normalfont % body font
}
\makeatother


\makeatletter
\renewcommand*\env@matrix[1][*\c@MaxMatrixCols c]{%
  \hskip -\arraycolsep
  \let\@ifnextchar\new@ifnextchar
  \array{#1}}
\makeatother
\pgfplotsset{unit circle/.style={width=4cm,height=4cm,axis lines=middle,xtick=\empty,ytick=\empty,axis equal,enlargelimits,xmax=1,ymax=1,xmin=-1,ymin=-1,domain=0:pi/2}}
\DeclareMathOperator{\Ima}{Im}
\newcommand{\A}{\mathbb A}
\newcommand{\C}{\mathbb C}
\newcommand{\E}{\vec E}
\newcommand{\CP}{\mathbb{CP}}
\newcommand{\F}{\mathbb F}
\newcommand{\G}{\vec G}
\newcommand{\J}{\mathcal J}
\renewcommand{\H}{\mathbb H}
\newcommand{\HP}{\mathbb HP}
\newcommand{\K}{\mathbb K}
\renewcommand{\L}{\mathcal L}
\newcommand{\N}{\mathbb N}
\renewcommand{\O}{\mathcal O}
\newcommand{\OP}{\mathbb OP}
\renewcommand{\P}{\mathbb P}
\newcommand{\Q}{\mathbb Q}
\newcommand{\U}{\mathcal U}
\newcommand{\I}{\mathbb I}
\newcommand{\R}{\mathbb{R}}
\newcommand{\RP}{\mathbb{RP}}
\renewcommand{\S}{\mathbb S}
\newcommand{\T}{\mathcal T}
\newcommand{\X}{\mathbf X}
\newcommand{\Z}{\mathbb Z}
\newcommand{\B}{\mathbb{B}}
\newcommand{\1}{\mathbb{1}}
\newcommand{\ds}{\displaystyle}
\newcommand{\ran}{\right>}
\newcommand{\lan}{\left<}
\newcommand{\bmat}[1]{\begin{bmatrix} #1 \end{bmatrix}}
\renewcommand{\a}{\vec{a}}
\renewcommand{\b}{\vec b}
\renewcommand{\c}{\vec c}
\renewcommand{\d}{\vec d}
\newcommand{\e}{\vec e}
\newcommand{\h}{\vec h}
\newcommand{\f}{\vec f}
\newcommand{\g}{\vec g}
\renewcommand{\i}{\vec i}
\renewcommand{\j}{\vec j}
\renewcommand{\k}{\vec k}
\newcommand{\n}{\vec n}
\newcommand{\p}{\vec p}
\newcommand{\q}{\vec q}
\renewcommand{\r}{\vec r}
\newcommand{\s}{\vec s}
\renewcommand{\t}{\vec t}
\renewcommand{\u}{\vec u}
\newcommand{\w}{\vec w}
\newcommand{\x}{\vec x}
\newcommand{\y}{\vec y}
\newcommand{\z}{\vec z}
\newcommand{\0}{\vec 0}
\newcommand{\pt}{\mathsf{pt}}
\newcommand{\from}{\longleftarrow}
\newcommand{\intprodl}{%
    \mathbin{\scalebox{1.5}{$\lrcorner$}}%
}
\newcommand{\intprodr}{%
    \mathbin{\scalebox{1.5}{$\llcorner$}}%
}

\DeclareMathOperator*{\Span}{span}
\DeclareMathOperator{\rng}{range}
\DeclareMathOperator{\gemu}{gemu}
\DeclareMathOperator{\almu}{almu}
\newcommand{\Char}{\mathsf{char}}
\DeclareMathOperator{\id}{id}
\DeclareMathOperator{\tr}{Tr}
\DeclareMathOperator{\tor}{Tor}
\DeclareMathOperator{\im}{im}
\DeclareMathOperator{\homeo}{Homeo}
\DeclareMathOperator{\GL}{GL}
\DeclareMathOperator{\SL}{SL}
\DeclareMathOperator{\norm}{N}
\DeclareMathOperator{\aut}{Aut}
\DeclareMathOperator{\Int}{Int}
\DeclareMathOperator{\ext}{Ext}
\DeclareMathOperator{\M}{M}
\DeclareMathOperator{\supp}{supp}
\DeclareMathOperator{\cl}{cl}
\DeclareMathOperator{\dom}{dom}
\DeclareMathOperator{\rnk}{rank}
\DeclareMathOperator{\Hom}{Hom}
\DeclareMathOperator{\Alt}{Alt}
\DeclareMathOperator{\dr}{dR}
\DeclareMathOperator{\ed}{End}
\DeclareMathOperator{\BM}{BM}
\DeclareMathOperator{\ob}{ob}
\DeclareMathOperator{\clength}{cup{-}length}
\DeclareMathOperator{\sgn}{sgn}
\DeclareMathOperator{\orb}{Orb}
\DeclareMathOperator{\cyl}{Cyl}
\DeclareMathOperator{\rel}{rel}
\DeclareMathOperator{\cat}{cat}
\DeclareMathOperator{\op}{op}
\DeclareMathOperator{\Gd}{Gd}
\DeclareMathOperator{\coker}{coker}
\DeclareMathOperator{\map}{Map}
\DeclareMathOperator{\sing}{Sing}
\DeclareMathOperator{\Op}{\mathbf{Op}}
\DeclareMathOperator{\colim}{colim}
\DeclareMathOperator{\tot}{Tot}
\DeclareMathOperator{\ev}{eval}
\DeclareMathOperator{\BL}{\mathcal{BL}}
\DeclareMathOperator{\Et}{\acute{E}t}
\DeclareMathOperator{\ch}{\mathbf{Ch}}
\DeclareMathOperator{\vf}{\mathscr{X}}

\newcommand{\vertneq}{\rotatebox{90}{$\,\neq$}}
\newcommand{\net}[2]{\underset{\scriptstyle\overset{\mkern4mu\vertneq}{#2}}{#1}}

\newcommand{\bi}{\begin{itemize}}
\newcommand{\ei}{\end{itemize}}

\newcommand{\be}{\begin{enumerate}}
\newcommand{\ee}{\end{enumerate}}

\newcommand{\bmp}{\begin{mathpar}}
\newcommand{\emp}{\end{mathpar}}

\setlength{\parindent}{0pt}


\newcommand{\mathcolorbox}[2]{\colorbox{#1}{$\displaystyle #2$}}

\newlist{steps}{enumerate}{1}
\setlist[steps, 1]{label = Step \arabic*:}

% info for header block in upper right hand corner
\name{Perry Hart}
\class{MATH 622}
\assignment{Fall 2019}

\begin{document}

\begin{abstract}
These notes are based on Ron Donagi's lectures for the course ``Complex Algebraic Geometry'' given at UPenn along with Daniel Huybrechts's \textit{Complex Geometry}. Any mistake in what follows is my own.
\end{abstract}

\tableofcontents
\newpage

\section{A quick overview of algebraic geometry} 

\subsection{Lectures 1-4}

These lectures consisted of informal surveys of certain fundamental concepts of algebraic geometry. They were meant as previews of various topics that we will cover rigorously.

\section{Complex analysis}

\subsection{Lecture 5}

First, let's review some basic concepts about functions of a single complex variable.

\begin{defn}
Let $z_0\in \C$. A function $f = u+iv : U\subset \C \to \C$ is \textit{holomorphic} or \textit{analytic} if at least one of  the following equivalent conditions holds.
\bi
\item Both $u$ and $v$ are $C^1$, and $f$ satisfies the Cauchy-Riemann equations, i.e., 
\begin{align*}
u_x & = v_y
\\ u_y & = {-}v_x.
\end{align*}
\item $\frac{\partial{f}}{\partial{\bar{z}}} =0$, where $\frac{\partial}{\partial{\bar{z}}} \coloneqq \frac{1}{2}\left(\frac{\partial}{\partial{x}} + i\frac{\partial}{\partial{y}}\right ) .$
\item The Cauchy integral formula holds, i.e., 
\[f(w) = \frac{1}{2\pi i}\int_{\gamma}\frac{f(w)}{\eta -w}d{\eta}
\] for any closed circular path $\gamma$ centered at $w$ in $U$. 
\item $f$ has a power series representation on $U$.
\ei
\end{defn}

\smallskip

\begin{defn}
A bijective function $f: U \subset \C \to V \subset \C$ is \textit{biholomorphic} if it is holomorphic and its inverse is holomorphic. In this case, we say that $U$ is \textit{biholomorphic to} $V$, written as $U \approx V$.
\end{defn}

\begin{fact}\label{singfacts} $ $
\be[label = (\alph*)]
\item (The maximum modulus principle) If $U\subset \C$ is a domain, $f: U \to \C$ is holomorphic, and $\lvert{f}\rvert$ has a local maximum, then $f$ is constant. 
\item (Liouville's theorem) Any bounded entire function is constant. 
\item (The Riemann extension theorem) If $\epsilon >0$ and $f: B_{\epsilon}(z) \setminus \{z\} \subset \C \to \C$ is bounded and holomorphic, then $f$ can be extended to a holomorphic function on $B_{\epsilon}(z)$.
\item (The Riemann mapping theorem) If $U\subsetneq \C$ is simply connected and open, then $U\approx B_1(0)$. 
\item (The residue theorem) If $f: B_{\epsilon}(0)\setminus \{0\}$ is holomorphic, then $f$ can be expanded in a Laurent series
$\sum_{n={-}\infty}^{\infty}a_nz^n$ such that $a_{{-}1} = \frac{1}{2\pi i}\oint f(z)d{z}$.
\ee
\end{fact}

\medskip

Next, let's look at some basic concepts about functions of several complex variables.

\begin{defn}
A function $f = u +iv: U\subset \C^n \to \C$ is \textit{holomorphic} if at least one of  the following equivalent conditions holds.
\bi
\item $f$ is holomorphic in each variable individually.
\item  Both $u$ and $v$ are $C^1$, and $f$ satisfies the Cauchy-Riemann equations,
\begin{align*}
u_{x_i} & = v_{y_i}
\\ u_{y_i} & = {-}v_{x_i}
\end{align*}
for each $i=1, \ldots, n$.
\item $\sum_{i=1}^n\frac{\partial{f}}{\partial{\bar{z}_i}} =0$.
\item $f$ has a power series representation on $U$,
\[
\sum_{k_1=0}^{\infty}\cdots \sum_{k_n=0}^{\infty} a_{k_1, \ldots, k_n}z_1^{k_1}\cdots z_n^{k_n}.
\]
\ei
\end{defn}

\begin{note}
Statements (a), (b), and (c) from \cref{singfacts} generalize to functions of several variables, as does the Cauchy integral formula:
\[
f(z) = \frac{1}{(2\pi i)^n} \int_{\lvert{\eta_i - z_i}\rvert = \epsilon_i} \frac{f(\eta_1, \ldots, \eta_n)}{(\eta_1 - z_1)\cdots (\eta_n - z_n)}d{\eta_1}\cdots d{\eta_n}
\] where $\eta_i >0$ for each $i=1, \ldots, n$.
\end{note}

\begin{theorem}[Hartog]
If $n>1$, then any holomorphic function $f: B_{\epsilon}(0)\setminus \{0\} \subset \C^n \to \C$ extends to a holomorphic function on $B_{\epsilon}(0)$.
\end{theorem}

\begin{defn} Let $X$ be a (topological) space. A \text{sheaf $F$ on $X$} is a presheaf on $X$ such that for any open $U\subset X$ and any open cover $\{U_i\}_{i\in J}$ of $U$, there is an exact sequence
\[
\begin{tikzcd}
0 \arrow[r] & F(U) \arrow[r] & F(U_i) \arrow[r] & F(U_{ij})
\end{tikzcd}
\] where $U_{ij} \coloneqq U_i \cap U_j$.
\end{defn}

\begin{defn}
A \textit{ringed space} is a pair $(X, \J)$ where $X$ is a space and $\J$ is a sheaf of rings on $X$.
\end{defn}

\begin{remark} Given any standard object $(X, \J)$, we can define a \textit{geometric object} as a ringed space locally isomorphic to $(X, \J)$.
\end{remark}

\begin{defn}[Vector bundle]
Let $X$ and $V$ be complex manifolds. Let $\pi : V \to X$ be holomorphic. We say that $\pi$ is a \textit{(holomorphic) vector bundle of rank $n$} if for any $x\in X$, there exist an open set $U\ni x$ in $X$ and an isomorphism $\pi^{-1}(U) \overset{\cong}{\longrightarrow} U \times \C^n$ such that the \textit{transition maps} $U_{ij} \times \C^n \to U_{ij} \times \C^n$ are holomorphic and fiber linear.
\end{defn}

\smallskip

Any vector bundle $\pi : V \to X$ induces a sheaf on $X$ given by
\[
F(U) = \Gamma\left(U, \pi^{-1}(U)\right).
\]

\begin{exmp} $ $
\be
\item The sheaf induced by the trivial bundle $\mathbf{1} \coloneqq X \times \C$ is denoted by $\O_X$.
\item The tangent bundle $T{X}$ of a smooth manifold $X$ induces the sheaf of vector fields on $X$.
\item The cotangent bundle $T^{\ast}{X}$ induces the sheaf $\Omega^1(X)$ of one-forms on $X$.
\item The alternating bundle $\bigwedge^p{X}$ of rank $p$ induces the sheaf $\Omega^p(X)$ of $p$-forms on $X$.
\ee
\end{exmp}

\section{Line bundles}

\subsection{Lecture 6}

\begin{defn}
A \textit{line bundle} is a vector bundle of rank $1$.
\end{defn}

\begin{defn}
Let $X$ be a complex manifold.  A \textit{sheaf $F$ of $\O_X$-modules} is a sheaf on $X$ such that for any open set $U$ in $X$,
\bi
\item $F(U)$ is a module over $\O_X(U)$ and
\item if $U \subset V \subset X$, then $\left(f\cdot a\right)\restriction_U = f\restriction_U\cdot a\restriction_U$.
\ei
\end{defn}

\begin{exmp}[Sheaf of sections]
Let $X$ be a complex manifold and $J$ be a vector bundle over $X$. For any open $U\subset X$, let $$\L_J(U) = \Gamma\left(U, L\right).$$ This inherits a vector space structure from the family of fibers of $V$. Also, any relation of the form $U_1 \subset U_2 \subset U$ induces a linear map $\Gamma\left(U_2, L\right) \to \Gamma\left(U_1, L\right)$ given by $\sigma \mapsto \sigma\restriction_{U_1}$. Thus, $\L_J\left({-}\right)$ is a sheaf of vector spaces. Moreover, it is easily seen to be a sheaf of $\O_X$-modules. 
\end{exmp}

Since any vector bundle is locally trivial, we see that $\L_J$ is \textit{locally free}, i.e., for any $x\in X$, there exist an (open) neighborhood $U$ of $x$ in $X$ and an isomorphism $\varphi : \L_J(U) \to \bigoplus_{i=1}^{\rnk(J)}\O_X(U)$ such that for any open set $V\subset U$, the square
\[
\begin{tikzcd}
\L_J(U) \arrow[r, "\cong"] \arrow[d] &  \bigoplus_{i=1}^{\rnk(J)}{\O_X(U)} \arrow[d] \\
\L_J(V) \arrow[r, "\cong"']          &  \bigoplus_{i=1}^{\rnk(J)}{\O_X(V)}          
\end{tikzcd}
\]
commutes. In other words, $\L_J$ is locally isomorphic to $\left(\O_X\right)^{\oplus{\rnk(J)}}$.

\begin{defn} 
A sheaf $F$ on a complex manifold $X$ is \textit{invertible} if there exist an open cover $\{U_i\}$ of $X$ and a family of holomorphic isomorphisms $\varphi_i : \O_{U_i} \to \L_J\restriction_{U_i}$.
\end{defn}

\begin{exmp} 
If $J$ is a line bundle, then $\L_J$ is invertible.
\end{exmp}

Consider the composition
\[
\begin{tikzcd}
\O_{U_i \cap U_j} \arrow[r, "\varphi_i"] & \L_J\restriction_{U_i \cap U_j} \arrow[r, "\varphi_j^{{-}1}"] & \O_{U_i \cap U_j}
\end{tikzcd}
, \ \quad \quad 1 \longmapsto g_{ij}.
\] From this, we can construct a line bundle $L$ over $X$ by defining the total space as 
\[
\faktor{\coprod_i\left(U_i \times \C\right)}{\sim}
\] where $\left(x, \lambda \right)_i \sim \left(y, \mu\right)$ if $x=y$ and $\mu = g_{ij}\lambda$. 

\medskip

\begin{defn}[Divisor]
A \textit{divisor on a complex manifold $X$} is a locally finite $\Z$-combination of irreducible holomorphic hypersurfaces of $X$. Equivalently, it is a subset of $X$ locally defined  by the vanishing of a holomorphic function.
\end{defn}

\begin{exmp}
If $X = \A^1$, then any divisor $D$ on $X$ is of the form $$D = \sum m_ip_i, \ \quad p_i \in \A^1, \ m_i \in \Z.$$ 
\end{exmp}

\begin{term}
Each $m_i$ is known as the \textit{multiplicity of $p_i$}.
\end{term}

Any divisor $D$ defines a line bundle $\O_X(D)$ on $X$ and a holomorphic map $X \dashrightarrow \P\left(V^{\vee}\right)$ where $V\equiv \Gamma\left(X, \O_X(D)\right)$. It is also true that any line bundle defines a divisor. It follows that
\[
\begin{tikzcd} 
\label{eqn:corr}
\left(\text{line bundles}\right) \arrow[r] & \left(\text{invertible sheaves}\right) \arrow[l, "\sim"'] \arrow[r] & \left(\text{divisors module linear equiv.}\right) \arrow[l, "\sim"'] \tag{$\dagger$}
\end{tikzcd}.
\]

\medskip

Consider the case where $D = \pt$. Let $f \in \Gamma\left(U, \O_U\right)$ and let $U_i = X \setminus D$, which is a tubular neighborhood of $D$. Note that $U_i = f^{{-}1}\left (\C\setminus\text{hyperplane}\right)$. Define $\O_X(D)$ as the line bundle with transition functions of the form $f \restriction_{U_i \cap U_j}$.

\medskip

Alternatively, let $$\left(\O_X\left(D\right)\right)\left(U\right) = \{g : U \to \C  \mid g \text{ is meromorphic}, \ \overbrace{fg}^{\text{\tiny product}} \text{ is holomorphic}\}.$$
For example,
let $X = \P^1$ and $D$ be a point $p$. Let $(x_0, x_1)$ denote local coordinates on $X$ near $p$. Let $g$ be meromorphic in these coordinates and let $f\left(x_0, x_1\right) = \frac{x_1}{x_0}$. Then $f{g}$ is holomorphic, i.e., $g$ has a pole of order at most one at $p$. 
\begin{question} $ $
\be
\item What is $\Gamma\left(\P^1, \O_X\right)$?
\item What is $\Gamma\left(\P^1, \O_X\left(D\right)\right)$?
\ee
\end{question}

In fact, it can be shown that
\[
\Gamma\left(\P^1, \O_X\left(m,p\right)\right) = \begin{cases}
\C\langle 1, x, \ldots, x^m\rangle & m \geq 0 
\\ 0 & \text{otherwise}
\end{cases}
\]

\bigskip

In general, $D$ is defined locally, and thus so is $\O_U(D)$. Specifically,
$\Gamma\left(U, \O_U\left(D\right)\right)$ consists of all holomorphic functions $f : U \setminus \supp\left(D\right) \to \C$ such that if $D = \sum m_iY_i$ and $Y_i \cap U = \{f_i =0\}$, then $g\prod_if_i^{m_i}$ is holomorphic in $U$.

\begin{exmp}[Veronese embedding]
Let $X = \P^1$ and $p$ be as before. 
\be
\item Let $D = \O(2p)$. Consider the space $V \coloneqq \Gamma\left(\P^1, \O\left(2p\right)\right) = \C\langle 1, x, x^2\rangle.$ Define the map $\varphi_{\O(2p)} : \P^1 \to \P^2$ by $$\varphi_{\O(2p)}(x) = \underbrace{\left(1, x, x^2\right)}_{\left(x,y,z\right)}.$$ This is an embedding. Its image is precisely the smooth curve given by $y^2 = xz$.
\item Let $D = \O(3p)$. Then the image of the map $\varphi_{\O(3p)} : \P^1 \hookrightarrow \P^3$ given by $x \mapsto \left(1, x, x^2, x^3\right)$ is a so-called twisted cubic.
\ee
\end{exmp}


The line bundle $L$ on $X$ determines the map $X \dashrightarrow \P\left(\Gamma\left(X, L\right)^{\vee}\right)$ directly, as follows.
\[
x\mapsto \ker\left(\Gamma\left(X, L\right) \overset{\ev_x}{\longrightarrow} L_p\right)
\]

\begin{defn}
The \textit{base locus of $L$} is $\BL\left(L\right) \equiv \{x  \in X \mid s(x) =0 \text{ for each } s\in \Gamma\left(X, L\right)\}$.
\end{defn} 

Note that we get a map $X \setminus \BL\left(L\right) \to \P\left(\Gamma\left(X, L\right)^{\vee}\right)$.

\bigskip

Now, let's consider a slight generalization of our preceding discussion. Let $V \subset \Gamma\left(X, L\right)$. This induces a map 
\[
\begin{tikzcd}
X \arrow[r, dashed]                                    & \P\left(V^{\vee}\right) \\
X\setminus\BL\left(V\right) \arrow[u, hook] \arrow[ru] &                        
\end{tikzcd}.
\]

Let $X = \P^1$ and $p = \{x=0\}$. Then $V \subset \Gamma\left(\P^1, \O(2)\right) = \C\langle 1, x, x^2\rangle$, and
\[
\begin{tikzcd}
\P^1 \arrow[rd, "\varphi_V"', dashed] \arrow[r, "\varphi_{\O(2)}"] & \P^2 \arrow[d, "\rho", dashed] \\
                                                                   & \P^1                       
\end{tikzcd}
\] commutes where $\rho$ denotes the linear projection. Note that $\varphi_V$ is a morphism so long as the center of $\rho$ is not in the image of $\varphi_{\O(2)}$. In this case, we have that
\begin{align*}
\varphi_{\O(2)}(x) & = \frac{a +by + cx^2}{d + ex +fx^2}
\\  \rho(x) & = \frac{a+bx}{c+dx}.
\end{align*}

\subsection{Lecture 7}

Let $L_1$ and $L_2 $ be line bundles over $X$ with transition functions $\{g_1^{k{l}} : U_{k{l}} \to \C^{\ast}\}$ and $\{g_2^{i{j}} : U_{i{j}} \to \C^{\ast}\}$, respectively. We can take a refinement $\{U_{i} \cap U_k\}$ where both $L_1$ and $L_2$ are trivial.  Define $L^1 \otimes L^2$ as the line bundle with transition functions $\{g_1^{k{l}}g_2^{i{j}} : U_{ij} \cap U_{k{l}} \to \C^{\ast}\}$. Further, define $\left(L^1\right)^{{-}1}$ as the line bundle with transition functions $\{\left(g_1^{k{l}}\right)^{{-}1} : U_{k{l}} \to \C^{\ast}\}$. Note that, locally, $L^1 \otimes \left(L^1\right)^{{-}1} \cong \O_X$.

\begin{defn}
We say that a divisor $D = \sum_im_iY_i$ is \text{effective} if $m_i\geq 0$ for each $i$.
\end{defn}

Let $V = \Gamma\left(X, \O_X(D)\right)$ and let $D$ be effective. Note that $\C\langle D \rangle \subset V$. We have that $\supp(D) = \varphi^{{-}1}\left(\text{hyperplane}\right)$ where $\left(\C \langle 0\rangle \right)^{\perp}$ is precisely the hyperplane in $\P\left(V^{\vee}\right)$.

\begin{exmp} Let $X = \P^1$.
\be
\item Let $x= \frac{x_1}{x_0}$ and $D =p \coloneqq \{x=0\}$. Then $V = \C\langle 1, x \rangle$, and the map $\varphi_V : \P^1 \to \P\left(V^{\vee}\right)$ is given by $c\mapsto y\coloneqq \frac{x}{1}$.
\item Let $D = m\left(\infty\right)$ with $m>0$. Then $V= \C\langle x_1, \ldots, x_m^m\rangle$, and the map $\varphi_{m{\infty}} : \P^1 \to \P^m$ is given by
\begin{align*}
\left(x_0, x_1\right) & \mapsto \left(x_0^m, x_0^{m-1}x_1, \ldots, x_0x_1^{m-1}, x_1^m\right)
\\ x \ & \mapsto \  \left(1, x, \ldots, x^m\right).
\end{align*}
\item Let $D = p_1 + \cdots + p_m$ where $p_i = \left[1:t_i\right]$. Let $x = \frac{x_1}{x_0}$, so that $\infty$ is given by $x_0 =0$. Then $V = \C\langle \underbrace{1, \frac{1}{x-t_1}, \ldots, \frac{1}{x-t_m}}_{a_0, \frac{a_1}{x-t_1}, \ldots, \frac{a_m}{x-t_m}}\rangle$. This can be viewed as the space of all regular meromorphic functions on open subsets of $\P^1$ having poles of order at most $m$. The image of $\varphi : \P^1 \to \P^m$ is precisely the hyperplane $\{a_0 =0\}$.
\ee
\end{exmp}

\begin{exmp}
Let $X$ be an elliptic curve, i.e., a space of the form $\C/\Lambda$. Let $p$ be the image of $0$ and let $D = mp$. 
\be
\item Let $m =1$. Then $V = \Gamma\left(X, \O_X(D)\right)$, which consists of all maps $f: X \to \P^1$ such that $f^{{-}1}\left(\infty\right) = \{0\}$. These are precisely the constant maps, so that $V \cong \C\langle s\rangle$ where $s$ is a holomorphic section of $\O_X(D)$ vanishing at $p$ and is meromorphic on $\O_X$.
\[
\begin{tikzcd}
X \arrow[r, dashed]                     & \P^0 \\
X\setminus p \arrow[u, hook] \arrow[ru] &     
\end{tikzcd}
\]
It follows that  $\BL\left(\O_X(D)\right) =p$.
\item Let $m=2$. Then $V= \C\langle 1, p\rangle$, and $\varphi_{2p} : X \to \P^1$ is precisely the $D$-th Weierstrass function. Moreover, we have an exact sequence
\[
\begin{tikzcd}
0 \arrow[r] & \O_X(p) \arrow[r] & \O_X(2p) \arrow[r] & \O_p \arrow[r] & \cdots
\end{tikzcd}.
\]
\item Let $m=3$. Then $V = \langle 1, p, p'\rangle$, and the image of $\varphi_{3p} :X \to \P^2$ is given by $y^2 = x^3+ax+b$. 
\ee
\end{exmp}

\begin{exmp} Let $X = \P^2$. Let $D= m\underbrace{\left(\text{line at }\infty\right)}_{\{z=0\}}$.
\be
\item Let $m=0$. Then $V = \C\langle 1\rangle$, and $\BL= \emptyset$.
\item Let $m=1$. Then $C = \C\langle \frac{x}{z}, \frac{y}{z}, \frac{z}{z}\rangle \cong \C\langle 1, X, Y\rangle$, and $\BL = \emptyset$. The map $\varphi_D : \P^2 \to \P^2$ is precisely the identity. 
\item Let $m=2$. Then $V = \langle \frac{x^2}{z^2}, \frac{x^4}{z^2}, \frac{y^2}{z^2}. \frac{x}{z}, \frac{y}{z}, \frac{z}{z}\rangle$, and the map $\varphi_D : \P^2 \to \P^5$ is an embedding given by $(x,y,z) \mapsto \langle x^2, xy, y^2, xz, yz, z^2\rangle$.
\ee
\end{exmp}

In general, if $H\subset \P^n$ is a hyperplane, then $\varphi_{\O\left(d{H}\right)} : \P^n \to \P^{{{d+n}\choose n} -1}$ is given by $$\left(x_0, \ldots, x_n\right) \mapsto \left(d\text{-th order homogenous polynomials}\right),$$  known as the $d$-th order Veronese embedding on $\P^n$.

\begin{exmp}
Let $X = \P^2$ with coordinates $\left(x,y,z\right)$. Let $H$ denote the hyperplane given by $z=0$ and let $D = 2{H}$. Then $ V= \{s\in \Gamma  \left(\O\left(2{H}\right)\right) \mid s\left(0,0,1\right) =0\}$, and
\[
\begin{tikzcd}
V \arrow[r, hook]                                                         & \Gamma\left(\O\left(2{H}\right)\right)                           \\
{\C\langle x^2, xy, y^2, xz,yz\rangle} \arrow[r, hook] \arrow[u, "\cong"] & {\C\langle x^2, xy, y^2, xz, yz, z^2\rangle} \arrow[u, "\cong"']
\end{tikzcd}
\] commutes. Further, $\BL(V) = \{0\} = \left[0,0,1\right]$, and $\varphi_V$ is a map $\P^2 \setminus \{0\} \to \P^4$ but does not extend to $\P^2$. Indeed, we have that
\begin{align*}
\lim_{\substack{\left(0, y, 1\right) \\ y\to 0}} \varphi_V  & = \lim_{y\to0}\left(0,0,y^2,0, y\right)= \net{\left(0,0,0,0,1\right)}{}
\\ \lim_{\substack{\left(x, 0, 1\right) \\ x\to 0}} \varphi_V  & = \lim_{x\to0}\left(x^2,0,0,x, 0\right)= \left(0,0,0,1,0\right).
\end{align*} Note that for any $p\in X$, there exist $\widetilde{X}$ and $\pi : \widetilde{X}\to X$ such that $\pi$ restricted to $\pi^{{-}1}\left(X\setminus p\right)$ is an isomorphism and $\pi^{{-}1}(p)$ is a divisor on $\widetilde{X}$ that is isomorphic to $\P^1$.
\end{exmp}

\begin{prop}
Let $Y \subset X$ be a submanifold of codimension $k\geq 2$. Let $\varphi : X \setminus Y \to Z$. Then there exist $\widetilde{X}$ and $\pi : \widetilde{X}\to X$ such that $\pi$ restricted to $\pi^{{-}1}\left(X\setminus Y\right)$ is an isomorphism and restricted to $\underbrace{\pi^{{-}1}(Y)}_{\text{divisor on }X}$ is a bundle with each fiber isomorphic to $\P^{k-1}$.
\end{prop}

\subsection{Lecture 8}



\subsection{Lecture 9}

\end{document}
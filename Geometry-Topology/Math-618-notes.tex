  %  \nonstopmode
\documentclass[10pt,letterpaper,cm]{nupset}
\usepackage[margin=1in]{geometry}
\usepackage{graphicx}
\usepackage{enumerate}
\usepackage{enumitem}
\usepackage{float}
\usepackage{stmaryrd}
\usepackage{amsfonts}
\usepackage{amssymb}
\usepackage{mathtools}
\usepackage{upgreek}
\usepackage{pgfplots}
\pgfplotsset{compat=1.13}
\usepackage{amsmath,amsthm}
\usepackage{tikz-cd}
\usetikzlibrary{knots,calc}
\usepackage{xcolor}
\usepackage{soul}
\usetikzlibrary{decorations.markings}
\usepackage{faktor}
\usepackage{xfrac}
\usepackage{ mathrsfs }
\usepackage{hyperref}
\usepackage{scrextend}
\hypersetup{colorlinks=true, linkcolor=red,          % color of internal links (change box color with linkbordercolor)
    citecolor=green,        % color of links to bibliography
    filecolor=magenta,      % color of file links
    urlcolor=cyan           }
\usepackage{adjustbox}
\usepackage{media9}


\usepackage{thmtools}
\usepackage[capitalise]{cleveref} 
    
\theoremstyle{definition}
\newtheorem{defn}{Definition}[subsection]
\newtheorem{exmp}[defn]{Example}
\newtheorem{non-exmp}[defn]{Non-example}
\newtheorem{note}[defn]{Note}

\theoremstyle{theorem}
\newtheorem{theorem}[defn]{Theorem}
\newtheorem{lemma}[defn]{Lemma}
\newtheorem{prop}[defn]{Proposition}
\newtheorem{corollary}[defn]{Corollary}
\newtheorem*{claim}{Claim}
\newtheorem{exercise}[defn]{Exercise}

\theoremstyle{remark}
\newtheorem{remark}[defn]{Remark}
\newtheorem*{todo}{To do}
\newtheorem*{question}{Question}
\newtheorem*{conv}{Convention}
\newtheorem*{aside}{Aside}
\newtheorem*{notation}{Notation}
\newtheorem*{term}{Terminology}
\newtheorem*{background}{Background}
\newtheorem*{further}{Further reading}
\newtheorem*{sources}{Sources}

\makeatletter
\def\th@plain{%
  \thm@notefont{}% same as heading font
  \itshape % body font
}
\def\th@definition{%
  \thm@notefont{}% same as heading font
  \normalfont % body font
}
\makeatother


\makeatletter
\renewcommand*\env@matrix[1][*\c@MaxMatrixCols c]{%
  \hskip -\arraycolsep
  \let\@ifnextchar\new@ifnextchar
  \array{#1}}
\makeatother
\pgfplotsset{unit circle/.style={width=4cm,height=4cm,axis lines=middle,xtick=\empty,ytick=\empty,axis equal,enlargelimits,xmax=1,ymax=1,xmin=-1,ymin=-1,domain=0:pi/2}}
\DeclareMathOperator{\Ima}{Im}
\newcommand{\A}{\mathcal A}
\newcommand{\C}{\mathbb C}
\newcommand{\E}{\vec E}
\newcommand{\CP}{\mathbb{CP}}
\newcommand{\F}{\mathbb F}
\newcommand{\G}{\vec G}
\renewcommand{\H}{\mathbb H}
\newcommand{\HP}{\mathbb HP}
\newcommand{\K}{\mathbb K}
\renewcommand{\L}{\mathscr L}
\newcommand{\N}{\mathbb N}
\newcommand{\OP}{\mathbb OP}
\renewcommand{\P}{\mathcal P}
\newcommand{\Q}{\mathbb Q}
\newcommand{\U}{\mathcal U}
\newcommand{\I}{\mathbb I}
\newcommand{\R}{\mathbb{R}}
\newcommand{\RP}{\mathbb{RP}}
\renewcommand{\S}{\mathbb S}
\newcommand{\T}{\mathcal T}
\newcommand{\X}{\mathbf X}
\newcommand{\Z}{\mathbb Z}
\newcommand{\1}{\mathbb{1}}
\newcommand{\ds}{\displaystyle}
\newcommand{\ran}{\right>}
\newcommand{\lan}{\left<}
\newcommand{\bmat}[1]{\begin{bmatrix} #1 \end{bmatrix}}
\renewcommand{\a}{\vec{a}}
\renewcommand{\b}{\vec b}
\renewcommand{\c}{\vec c}
\renewcommand{\d}{\vec d}
\newcommand{\e}{\vec e}
\newcommand{\h}{\vec h}
\newcommand{\f}{\vec f}
\newcommand{\g}{\vec g}
\renewcommand{\i}{\vec i}
\renewcommand{\j}{\vec j}
\renewcommand{\k}{\vec k}
\newcommand{\n}{\vec n}
\newcommand{\p}{\vec p}
\newcommand{\q}{\vec q}
\renewcommand{\r}{\vec r}
\newcommand{\s}{\vec s}
\renewcommand{\t}{\vec t}
\renewcommand{\u}{\vec u}
\newcommand{\w}{\vec w}
\newcommand{\x}{\vec x}
\newcommand{\y}{\vec y}
\newcommand{\z}{\vec z}
\newcommand{\0}{\vec 0}
\newcommand{\pt}{\mathsf{pt}}
\newcommand{\from}{\longleftarrow}
\newcommand{\intprodl}{%
    \mathbin{\scalebox{1.5}{$\lrcorner$}}%
}
\newcommand{\intprodr}{%
    \mathbin{\scalebox{1.5}{$\llcorner$}}%
}
\DeclareMathOperator*{\Span}{span}
\DeclareMathOperator{\rng}{range}
\DeclareMathOperator{\gemu}{gemu}
\DeclareMathOperator{\almu}{almu}
\DeclareMathOperator{\id}{id}
\DeclareMathOperator{\tr}{Tr}
\DeclareMathOperator{\tor}{Tor}
\DeclareMathOperator{\im}{im}
\DeclareMathOperator{\homeo}{Homeo}
\DeclareMathOperator{\GL}{GL}
\DeclareMathOperator{\SL}{SL}
\DeclareMathOperator{\norm}{N}
\DeclareMathOperator{\aut}{Aut}
\DeclareMathOperator{\Int}{Int}
\DeclareMathOperator{\ext}{Ext}
\DeclareMathOperator{\Or}{O}
\DeclareMathOperator{\Un}{U}
\DeclareMathOperator{\SO}{SO}
\DeclareMathOperator{\M}{M}
\DeclareMathOperator{\supp}{supp}
\DeclareMathOperator{\cl}{cl}
\DeclareMathOperator{\dom}{dom}
\DeclareMathOperator{\rnk}{rank}
\DeclareMathOperator{\Hom}{Hom}
\DeclareMathOperator{\Alt}{Alt}
\DeclareMathOperator{\dr}{dR}
\DeclareMathOperator{\ed}{End}
\DeclareMathOperator{\BM}{BM}
\DeclareMathOperator{\ob}{ob}
\DeclareMathOperator{\clength}{cup{-}length}
\DeclareMathOperator{\sgn}{sgn}
\DeclareMathOperator{\orb}{Orb}
\DeclareMathOperator{\cyl}{Cyl}
\DeclareMathOperator{\rel}{rel}
\DeclareMathOperator{\cat}{cat}
\DeclareMathOperator{\op}{op}
\DeclareMathOperator{\Gd}{Gd}
\DeclareMathOperator{\coker}{coker}
\DeclareMathOperator{\map}{Map}
\DeclareMathOperator{\sing}{Sing}
\DeclareMathOperator{\Op}{\mathbf{Op}}
\DeclareMathOperator{\colim}{colim}
\DeclareMathOperator{\tot}{Tot}
\DeclareMathOperator{\B}{\mathcal{B}}
\DeclareMathOperator{\Et}{\acute{E}t}
\DeclareMathOperator{\ch}{\mathbf{Ch}}
\DeclareMathOperator{\vf}{\mathscr{X}}


\tikzset{commutative diagrams/.cd,
mysymbol/.style={start anchor=center,end anchor=center,draw=none}
}
\newcommand\MySymb[2][\alpha]{%
  \arrow[mysymbol]{#2}[description]{#1}}

\newcommand{\bi}{\begin{itemize}}
\newcommand{\ei}{\end{itemize}}

\newcommand{\be}{\begin{enumerate}}
\newcommand{\ee}{\end{enumerate}}

\newcommand{\bmp}{\begin{mathpar}}
\newcommand{\emp}{\end{mathpar}}

\setlength{\parindent}{0pt}


\newcommand{\mathcolorbox}[2]{\colorbox{#1}{$\displaystyle #2$}}

\newlist{steps}{enumerate}{1}
\setlist[steps, 1]{label = Step \arabic*:}

% info for header block in upper right hand corner
\name{Perry Hart}
\class{MATH 618}
\assignment{Fall 2019}

\begin{document}

\begin{abstract}
These notes are based on Julius Shaneson's lectures for the course ``Algebraic Topology, Part I'' at UPenn. Any mistake in what follows is my own.
\end{abstract}

\tableofcontents
\newpage

\section{Background material} 

\subsection{Lecture 1}

Here are the topics for the course:
\bi
\item fiber bundles over cell complexes,
\item spectral sequences,
\item characteristic classes, and
\item cobordism theory.
\ei

To start, let's review some basic concepts from homology theory.

\begin{defn}
A \textit{(finite) cell complex} is a (topological) space $X$ that can be written as $\bigcup_{n=0}^K{X^n}$ for some $K\in \N$ (called the \textit{dimension of $X$}) 
where 
\bi
\item$X^0$ is chosen to be finite, 
\item $X^n = \frac{X^{n-1}  \coprod D_1^n \coprod \cdots \coprod D_{k_n}^n}{x\sim \varphi_i(x)}$,
\item $D_i^n \cong D^n \coloneqq \{x \in \R^n : |x| \leq 1\}$ for each $i\in \{1, \ldots, k_n\}$, and 
\item $\varphi_i : \partial{D_i^n} = S^{n-1} \to X^{n-1}$, called an \textit{attaching map}.
\ei
\end{defn}

\begin{term}
Each $D_i^n$ is called an \textit{$n$-cell of $X$}.
\end{term}

Every attaching map $\varphi_i : \partial{D_i^n} \to X^{n-1}$ can be extended to a \textit{characteristic map} given by the composition  $$ D^n_i \hookrightarrow  X^{n-1}  \coprod D_1^n \coprod \cdots \coprod D_{k_n}^n \twoheadrightarrow X^n \hookrightarrow X     .$$

\begin{exmp} There are at least two ways of endowing $S^2$ with a cell structure.
\be
\item $X^0 \equiv \{N, S\}$, $X^1 \equiv X^0 \cup_{\varphi_1} D_1^1 \cup_{\varphi_2} D_2^1$ where each $\varphi_i$ is an embedding, and $X^2 \equiv X^1  \cup_{\varphi'_1} D_1^2 \cup_{\varphi'_2} D_2^2$ where each $\varphi'_i$ is an embedding. 
\item $\pt \cup_{\varphi} D^2$ where $\varphi$ identifies the equator of the upper half-sphere with $\pt$.
\ee
\end{exmp}

\begin{defn}
A cell complex $X$ is \textit{regular} if every characteristic map $D_i^n \to X$ is an embedding. 
\end{defn}

\begin{defn}
Given a family of functors $\{H_n : \mathbf{Top}^2 \to \mathbf{Ab}\}_{n\in \N}$ where $\mathbf{Top}^2$ denotes the category of (topological) pairs, we say that $H_i$ is a \textit{homology functor} if  each of the following properties holds.
\be
\item (LES) For any pair $(X, A)$ of space, there is a natural long exact sequence
\[
\begin{tikzcd}
\cdots \arrow[r] & H_i(A)  \arrow[r] & H_i(X) \arrow[r] & {H_i(X, A)} \arrow[r, "\partial"] & H_{i-1}(A) \arrow[r] & \cdots
\end{tikzcd},
\] where $H_i(Z) \coloneqq H_i(Z, \emptyset)$ for any space $Z$.
\item (Excision) If $\cl(A) \subset \underset{open}{U} \subset X$, then $H_i(X\setminus A, U \setminus A) \cong H_i(X, U)$.
\item (Dimension) $H_i(\pt) = \begin{cases} 0 & i \ne 0 \\ \Z & i =0 \end{cases}$.
\item (Homotopy) If $f$ and $g$ are homotopic, then $f_{\ast} = g_{\ast}$, where $h_{\ast} \coloneqq H_i(h)$ for any map $h: (X,A) \to (Y, B)$. 
\ee
\end{defn}

\begin{theorem}
There exists a family of homology functors.
\end{theorem}

\begin{exmp}
In singular homology theory, we have that $H_i(S^n) = \begin{cases} \Z & i = 0, n \\ 0 & \text{otherwise} \end{cases}.$ 
\end{exmp}

Let $X$ be a cell complex. Let $C_n(X)$ denote the free abelian group on the set of all $n$-cells of $X$. Define $\partial : C_n(X) \to C_{n-1}(X)$ by $\partial[D_i^n] = \sum_{j=1}^{k_{n-1}} \lambda_{ij}[D_j^{n-1}]$ where $\lambda_{ij}$ is defined, up to sign, as follows. Consider the map
\[
\begin{tikzcd}
S^{n-1} \arrow[r, equals] \arrow[rrrrr, "\omega"', bend right] & \partial{D_i^n} \arrow[r, "\varphi_i"] & X^{n-1} \arrow[r, two heads] & \frac{X^{n-1}}{X^{n-2} \cup (\text{all cells of dim. } n-1 \text{ except } D_j^{n-1})} \arrow[r, equals] & \faktor{D^{n-1}}{\partial{D_j^{n-1}}} \arrow[r, equals] & S^{n-1}
\end{tikzcd}
.\]
 Then let $\lambda_{ij}$ satsify $\omega_{\ast}(x) = \lambda_{ij}x$ with $x$ a chosen generator (i.e., orientation) of $H_{n-1}(S^{n-1}) \cong \Z$. 

\begin{term}
The integer $\lambda_{ij}$ is called the \textit{degree of $\omega$}, denoted by $\deg(\omega)$.
\end{term}

\begin{theorem}
$\partial_n{\partial_{n+1}} = 0$, and $H_n(X) \cong \faktor{\ker{\partial_n}}{\im{\partial_{n+1}}}$, which is independent of our choice of generator $x$.
\end{theorem}

\begin{exmp}
Suppose that $f: S^n \to S^n$ is smooth. By Sard's theorem, we can find a regular value $x \in S^n$. There is some neighborhood $U$ of $x$ such that $f^{-1}(U) = U_1 \cup \cdots \cup U_n$ for some $n$. Using the inverse function theorem and the compactness of $S^n$, it follows that $f^{-1}$ is of the form $\{x_1, \ldots, x_n\}$. Note that the differential $(d{f})_{x_i} : S^n_{x_i} \to S^n_x$ satisfies $\det{(d{f})_{x_i}} - \pm 1$. In fact, $$\deg(f) = \sum_{i=1}^n \det{(d{f})_{x_i}}.$$
\end{exmp}

\begin{exercise}\label{reg}
Prove that any cell complex $X= \bigcup_{n=0}^KX^n$ is homotopy equivalent to a regular cell complex.\footnote{Hint: Consider the map $S^{n-1} \to X^{n-1} \times D^n$ given by $x \mapsto (\varphi(x), x)$.}
\end{exercise}
\begin{proof}
??
\end{proof}

\subsection{Lecture 2}

\begin{exmp}[Real projective space]
Recall that $\RP^n = \faktor{S^n}{x \sim {-}x}$. Then $\RP^n = \RP^{n-1} \cup_{\pi_{n-1}} D^n$ where $\pi_{n-1} : S^{n-1} \to \RP^{n-1}$ denotes the canonical projection. Thus, $\RP^n$ is an $n$-dimension cell complex with $(\RP^n)^m = \RP^m$ for each integer $0\leq m\leq n$.

\medskip

Now, for each $0\leq m \leq n$, we have that $C_m(\RP^n) \cong \Z$ with generator $[D^m]$. To determine $\partial[D^m]\in C_{m-1}(\RP^m)$, we must find the degree of the map
\[
\begin{tikzcd}
S^{m-1} \arrow[rrrr, "\varphi"', bend right] \arrow[r] & \RP^{m-1} \arrow[r, two heads] & \faktor{\RP^{m-1}}{\RP^{m-1}} \arrow[r, equals] & \faktor{D^{m-1}}{\partial{D^{m-1}}} \arrow[r, equals] & S^{m-1}
\end{tikzcd}
\]
Assume, for convenience, that $m=2$. Choose a regular value $p\in S^{1}$ so that $\varphi^{-1}(p) = \{N, S\}$. Let $\varphi_T$ and $\varphi_B$ denote the restrictions of $\varphi$ to the top and bottom components of $S^1 \setminus \{({-}1, 0), (1,0)\}$, respectively. Note that both of these are homeomorphisms and thus have degrees equal to $\pm 1$. If $a: S^{m-1} \to S^{m-1}$ denotes the antipodal map, we have that $\varphi_B \circ a= \varphi_T$. Hence $(d{\varphi})_S \circ (d{a})_N = (d{\varphi})_N$.  Since $\deg(a) = \det(d{a}) = ({-}1)^m$, it follows that $$\deg(\varphi) = \begin{cases} \pm 2 & m  \text{ even} \\ 0 & m \text{ odd} \end{cases}.$$
Thus, we get a chain complex
\[
\begin{tikzcd}
0 \arrow[r] & C_n(\RP^n) \arrow[r, "\kappa_1"] & C_{n-1}(\RP^n) \arrow[r, "\kappa_2"] & \cdots \arrow[r, "0"] & C_2(\RP^n) \arrow[r, "\pm 2"] & C_1(\RP^n) \arrow[r, "0"] & C_0(\RP^n) \arrow[r] & 0
\end{tikzcd}
\] where $\kappa_1 = \begin{cases} 0 & n \text{ odd} \\ \pm 2 & n \text{ even} \end{cases}$ and $\kappa_2 = \begin{cases} \pm 2 & n \text{ odd} \\ 0 & n \text{ even} \end{cases}$.

This proves that $$ H_i( \RP^n) = \begin{cases} \Z & i =0 \\ \Z/2 & i =1 \\ 0 & i =2 \\ \Z/2 & \underset{odd}{i} <n \\ 0 & \underset{even}{i} <n \\ 0 & i >n \\ \Z & i =n \text{ odd} \\ 0 & i= n \text{ even}    \end{cases}  .$$
\end{exmp}

\bigskip

Next, let's introduce some fundamental concepts from homotopy theory. 

\begin{defn} Let $M(X, Y)$ denote the set of maps $X \to Y$. 
\be
\item For any compact $C \subset X$ and open $U \subset Y$, let $$N(C, U) = \{f : X \to Y \mid f(C) \subset U \}  .$$ The \textit{compact-open topology on $M(X, Y)$} consists of all unions of finite intersections of subsets of the form $N(C, U)$. Under this topology, $M(X,Y)$ is called a \textit{mapping space}. 
\item The \textit{$n$-th loop space} of a pointed space $(X, x)$ is $$\Omega^{n-1}(X, x) \coloneqq M((D^{n-1}, \partial{D^{n-1}}), (X, x)),$$  which is a subset of $M(D^{n-1}, X)$.
\ee 
\end{defn}


\begin{defn}[Higher homotopy groups]
If $n \geq 2$, then the \textit{$n$-th homotopy group} of $(X, x)$ is $$\pi_n(X,x) \coloneqq \pi_1(\Omega^{n-1}{X}, e_x).$$
\end{defn}

Recall that $\pi_1({-})$ is a functor $\mathbf{Top}_{\ast} \to \mathbf{Grp}$. Also, $\Omega^{n-1}({-})$ is a functor $\mathbf{Top}_{\ast}  \to \mathbf{Top} $ defined on morphisms $f: (X, x) \to (Y, y)$ by post-composition with $f$.
 Therefore, it's easy to see that $\pi_n({-})$ is a functor $\mathbf{Top}_{\ast} \to \mathbf{Grp}$ as well.

\begin{notation} 
Let $f_{\ast} =  \pi_n(f)$ for any $f: (X, x) \to (Y, y)$.
\end{notation}

\begin{prop}
There is a homeomorphism $M(X \times Y, Z) \cong M(X, M(Y, Z))$ so long as $Y$ is locally compact and Hausdorff. 
\end{prop}

In particular, we have a composite 

\[
M(([0,1], \{0,1\}), (M((D^{n-1}, \partial), (X,x)), e_x)) \hookrightarrow M([0,1], M(D^{n-1}, X)) \overset{\cong}{\longrightarrow} M([0,1] \times D^{n-1}, X),
\] whose image is precisely $M((D^n, \partial), (X,x)) \cong M((S^n, \pt), (X,x))$. This proves that $\pi_n(X,x)$ consists of all homotopy classes of maps $(I^n, \partial) \to (X,x)$ under the operation $[f] \ast [g] = [f \ast g]$ where $$ f \ast g(t_1, \ldots, t_n) = \begin{cases}
      f(2t_1, t_2, \ldots, t_n) & 0\leq t_1 \leq \frac{1}{2}
      \\ f(2t_1-1, t_2, \ldots, t_n) & \frac{1}{2} \leq t\leq 1
\end{cases}    .$$

\begin{prop}
If $n\geq 2$, then $\pi_n(X, x)$ is abelian.
\end{prop}


\begin{remark}
A map $f: S^{n-1} \to X$ is homotopic to the constant map if and only if there is some $g$ such that 
\[
\begin{tikzcd}
D^n \arrow[rd, "g"]                     &   \\
S^{n-1} \arrow[r, "f"'] \arrow[u, hook] & X
\end{tikzcd}
\] commutes.
\end{remark}

\begin{theorem}[Whitehead]\label{WH}
If $f: X \to Y$ is a map of connected cell complexes, then $f$ is a homotopy equivalence if and only if $f_{\ast} : \pi_n(X,x) \to \pi_n(Y, y)$ is an isomorphism for each $n \in \N$.
\end{theorem}

\subsection{Lecture 3}

\begin{defn}
If $x\in A \subset X$, then the \textit{$n$-th relative homotopy group $\pi_n(X, A, x)$} consists of all homotopy classes of maps $(D^n, S^{n-1}, x_0) \to (X, A, x)$. 
\end{defn}

We see that  $$M((D^n, S^{n-1}, x), (X, A, x_0)) \cong M((I^n, I^{n-1} \times \{1\}, \underbrace{\partial{I^n} \setminus \Int(I^{n-1} \times \{1\})}_{\partial_0{I^n}}), (X, A, x_0))$$ by considering the homeomorphism $(I^n/\partial_0{I^n}, \partial{I^n}/\partial_0{I^n}) \cong (D^n, S^{n-1})$. Therefore, $\pi_n(X, A, x)$ can be viewed as consisting of all homotopy classes of maps $(I^n, \partial{I^n}, \partial_0{I^n}) \to (X, A, x)$.

\begin{defn}
In order to interpret an exact sequence involving objects in the category of pointed sets, we define the \textit{kernel of a function $f: (X,x) \to (Y,y)$ of pointed sets} as $\ker{f} \equiv f^{-1}(y)$.   
\end{defn}

\begin{prop} $ $
\be
\item If $n\geq 2$, then $\pi_n(X, A, x)$ is, in fact, a group.
\item If $n\geq 3$, then $\pi_n(X, A, x)$ is abelian.
\item We have a long exact sequence 
\[
\begin{tikzcd}
\cdots \arrow[r] & {\pi_n(A,x)} \arrow[r]     & {\pi_n(X, x)} \arrow[r] & {\pi_n(X, A, x)} \arrow[r, "\partial"] & {\pi_{n-1}(A,x)} \arrow[llld] &   \\
                 & {\pi_{n-1}(X,x)} \arrow[r] & \cdots \arrow[r]        & {\pi_0(A, x)} \arrow[r]                & {\pi_0(X,x)} \arrow[r]        & 0
\end{tikzcd}
\] with $\partial{[f]} = \left[f\restriction_{I^{n-1}}\right]$.
\ee 
\end{prop}

\begin{theorem}[Hurewicz] 
Let $n\in \N_{\geq 2}$. If $\pi_i(X) =0$ for each $i < n$, then $\pi_n(X) \cong H_n(X)$. 
\end{theorem}

\begin{note}
This result can't be improved in general. For example, $\pi_3(S^2) \cong \Z$, whereas $H_3(S^2) =0$.
\end{note}

Let $A \subset X$ be a subcomplex. Recall that $H_i(X, A) \cong H_i(X/A. \ast)$ for each $i\geq 1$. But it is \emph{not} the case that $\pi_i(X, A) \cong \pi_i(X/A. \ast)$, for otherwise  $\pi_i(S^n) \cong \pi_i(D^n, S^{n-1}) \cong \pi_i(S^{n-1})$, which is known to be false exactly when $i > 2n-2$. 

\begin{exmp}
$\pi_4(S^3) \cong \Z/2 \not \cong  \pi_4(S^4)$.
\end{exmp}

\bigskip

Finally, let's review the notion of a fibration of spaces.

\medskip

Recall that if $p: E \to B$ is a covering projection, then TFAE.
\be
\item For any $f: Z \to B$, there exists a unique $\hat{f} :Z \to E$ such that $p \circ \hat{f} = f$.
\item $f_{\ast}(\pi_1(Z)) \subset p_{\ast}(\pi_1(E))$.
\ee

The existence of $\hat{f}$ follows from the fact that any covering space satisfies the homotopy lifting property.

\begin{defn}[Fibration]
Suppose that $p: E \to B$ is any map. We say that $p$ is a \textit{fibration} if it satisfies the homotopy lifting property, i.e., given a commutative square
\[
\begin{tikzcd}
X\times \{0\} \arrow[d, hook] \arrow[r, "\widehat{f_0}"] & E \arrow[d, "p"] \\
{X\times [0,1]} \arrow[r, "f"']                      & B               
\end{tikzcd}
,\]  where $X$ is a cell complex, there is some $G$ such that
\[
\begin{tikzcd}
X\times \{0\} \arrow[d, hook] \arrow[r, "\widehat{f_0}"] & E \arrow[d, "p"] \\
{X\times [0,1]} \arrow[r, "f"'] \arrow[ru, "G"]      & B               
\end{tikzcd}
\] commutes. 
\end{defn}

\begin{theorem}
If $p: E \to B$ is a fibration with $e \in F \coloneqq p^{-1}(b)$, then $$p_{\ast} : \pi_i(E, F, e) \overset{\cong}{\longrightarrow} \pi_i(B, b, b) = \pi_i(B, b).$$
\end{theorem}
\begin{proof}
Let $f: (I^n, \partial{I^n}) \to (B, b)$. To prove that $p_{\ast}$ is surjective, it suffices to find some $G : (I^n, \partial{I^n}) \to (E, F)$ such that 
\[
\begin{tikzcd}
\partial_0{I^n} \arrow[r, two heads] \arrow[d, hook]       & \{e\} \arrow[r, hook] & F \arrow[r, hook] & E \arrow[d, "p"] \\
{I^{n-1} \times [0,1]} \arrow[rrr, "f"'] \arrow[rrru, "G"] &                       &                   & B               
\end{tikzcd}
\] commutes, for in this case $[p \circ G'] = [f]$. Since $p$ is a fibration, there is some $G$ such that
\[
\begin{tikzcd}
I^{n-1} \times \{0\} \arrow[r, two heads] \arrow[d, hook]       & \{e\} \arrow[r, hook] & F \arrow[r, hook] & E \arrow[d, "p"] \\
{I^{n-1} \times [0,1]} \arrow[rrr, "f"'] \arrow[rrru, "G'"] &                       &                   & B               
\end{tikzcd}
\] 
commutes. But $(I^n, \partial_0{I^n}) \cong (I^n, I^{n-1} \times \{0\})$, and thus such a $G'$ is enough. 
\end{proof}

\begin{corollary}\label{exhtpy}
We have a long exact sequence 
\[
\begin{tikzcd}
\cdots \arrow[r] & {\pi_i(F, e)} \arrow[r] & {\pi_i(E, e)} \arrow[r] & {\pi_i(B, b)} \arrow[r, "\partial"] & {\pi_{i-1}(F, e)} \arrow[r] & \cdots
\end{tikzcd}
.\]
\end{corollary}

\begin{exmp} $ $
\be
\item Suppose that
\[
\begin{tikzcd}
X\times \{0\} \arrow[d, hook] \arrow[r, "\hat{f}"] & B \times F \arrow[d, "\pi_B"] \\
{X\times [0,1]} \arrow[r, "f"]                           & B                            
\end{tikzcd}
\] commutes. Then $\hat{f}(x,0) = \left(\hat{f}_1(x,0), \hat{f}_2(x,0)\right)$ where $\hat{f}_1(x,0) = f(x,0)$. Let $G(X,t) =  (f(x,t), \hat{f}_2(x,0))$. Then \[
\begin{tikzcd}
X\times \{0\} \arrow[d, hook] \arrow[r, "\widehat{f_0}"] & B \times F \arrow[d, "\pi_B"] \\
{X\times [0,1]} \arrow[r, "f"] \arrow[ru, "G"]           & B                            
\end{tikzcd}
\] commutes, so that $\pi_B$ is a fibration. (Moreover, $\pi_n(B \times F) \cong \pi_n(B) \times \pi_n(F)$.)
\item Let $A \subset X$ be a subcomplex. The map $\varphi : M(X, Y) \to M(A, Y)$ defined by $f \mapsto f\restriction_A$ is a fibration. 
\item Define the \textit{Hopf fibration} as the quotient map
\[
S^3 = \{(z_1, z_2) \in \C^2 \mid z_1\overline{z_1} + z_2\overline{z_2} =1\} \twoheadrightarrow \faktor{S^3}{x \sim {-}x} = \CP^1 =S^2.
\]
\ee
\end{exmp}

\begin{corollary}
$\pi_3(S^3) \cong \pi_3(S^2)$.
\end{corollary}
\begin{proof}
Since we have an exact sequence 
\[
\begin{tikzcd}
\pi_3(S^1) \arrow[r] & \pi_3(S^3) \arrow[r] & \pi_3(S^2) \arrow[r] & \pi_2(S^1)
\end{tikzcd},
\] it suffices to show that both $\pi_3(S^1)$ and $\pi_2(S^1)$ are trivial. To this end, note that since $\pi_1(S^k) =0$ for every $k>1$, we can always find,  for any $f: S^k \to S^1$,  a map $\hat{f}$ such that
\[
\begin{tikzcd}
                                                  & \R \arrow[d, "e^{2\pi i x}"] \\
S^k \arrow[r, "f"'] \arrow[ru, "\hat{f}"] & S^1                         
\end{tikzcd}
\] commutes. Thus, $f$ is homotopic to the constant map. Since $f$ was arbitrary, our proof is complete.
\end{proof}

\begin{defn}\label{ltriv}
A map $p : E \to B$ is \textit{locally trivial} if for any $b\in B$, there exist a neighborhood $U\ni b$ in $B$, a space $F$, and a homeomorphism $\varphi: p^{-1}(U) \overset{\cong}{\longrightarrow} U \times F$ such that $\pi_U \circ \varphi = p \restriction_{p^{-1}(U)}$.
\end{defn}

\begin{theorem}
Any locally trivial map $p: E \to B$ is a fibration whenever $B$ is a cell complex. 
\end{theorem}

\begin{exercise}
Prove that the Hopf fibration is locally trivial.
\end{exercise}
\begin{proof}
For each $k \in \{0,1\}$, let $U_k = \{[z_0, z_1] \in \CP^1 \mid z_k \ne 0\}$. Then $U_0$ and $U_1$ form an open cover of $\CP^1$. Note that the preimage of $U_k$ under the Hopf fibration $q$ is precisely $\{(z_0, z_1) \in S^3 \mid z_k \ne 0\}$. Define $f: q^{-1}(U_k) \to  U_k \times S^1$ by  $$(z_0, z_1) \mapsto \left([z_0, z_1], \frac{z_k}{|z_k|}\right).$$ This is clearly continuous. Further, define the map $g: U_k \times S^1 \to q^{-1}(U_k)$ by $$\left([z_0, z_1], e^{i{\theta}}\right) \mapsto \frac{e^{i\theta}|z_k|}{z_k \vert{(z_0, z_1)}\rvert}\left(z_0, z_1 \right).$$ Since $U_k$ is a saturated open set, we have that the restriction of $q$ to $q^{-1}(U_k)$ is a quotient map. But $g \circ q\restriction_{q^{-1}(U_k)}$ is continuous, so that $g$ is also continuous by the characteristic property of quotient maps. Finally, it is easy to verify that $g$ and $f$ are inverses of each other and that $\pi_{U_I} \circ f = p\restriction_{q^{-1}(U_k)}$.
\end{proof}

\subsection{Lecture 4}

\begin{theorem}
Let $A\subset X$ be a subcomplex. Define $r: M(X, Y) \to M(A, Y)$ by $r(f) = f\restriction_A$. Then $r$ is a fibration. 
\end{theorem}
\begin{proof}
We must fill any diagram of the form
\[
\begin{tikzcd}
Z\times \{0\} \arrow[r, "\hat{f}"] \arrow[d, hook]                 & {M(X,Y)} \arrow[d, "r"] \\
{Z\times [0,1]} \arrow[ru, "F", dashed] \arrow[r, "f"'] & {M(A,Y)}               
\end{tikzcd}
.\] It suffices to find a map $\overline{F}$ such that
\[
\begin{tikzcd}
Z\times \{0\}\times X \arrow[r, "\hat{\bar{f}}"] \arrow[d, hook] & Y \arrow[d, equals] \\
{Z\times [0,1]\times X} \arrow[ru, "\overline{F}"]               & Y           \\
{Z\times [0,1]\times A} \arrow[u, hook] \arrow[ru, "\bar{f}"']   &            
\end{tikzcd}
\] commutes for, in this case, we can set $F(z,t)(x) = \overline{F}(z, t, x)$.

\begin{note}
Suppose that such an $\overline{F}$ exists. Define $g: Z \times X \to Y$ by $g(z,x) = \hat{\bar{f}}(z,0, x)$. Define $h: Z \times X \times [0,1] \to Y$ by $H(z,x,t) = \overline{F}(z,t,x)$. Then
\[
\begin{tikzcd}
Z\times X\times \{0\} \arrow[rd, "g"] \arrow[d, hook]     &   \\
{Z\times X \times [0,1]} \arrow[r, "H"]                   & Y \\
{Z\times A \times [0,1]} \arrow[u, hook] \arrow[ru, "K"'] &  
\end{tikzcd}
\]
commutes where $K(z,a,t) = \bar{f}(z,t,a)$. In the case where $Z = \pt$, this means that if $K: A \times [0,1] \to Y$ is a homotopy from a map $f: A \to Y$ and $g$ extends $f$ to $X$, then there exists a homotopy $H : X \times [0,1]\to Y$ such that $H\restriction_{A \times [0,1]} = K$. In other words, the extension problem for cell complexes is a homotopy problem. 
\end{note}
Let's return to proving our theorem. By induction, it suffices to consider just the case where $X = A \cup_{\varphi} D^n$, with characteristic map $\chi: D^n \to X$.   Thus, it suffices to find a map $w$ such that
\[
\begin{tikzcd}[column sep = huge]
Z\times D^n \times \{0\} \arrow[d, hook] \arrow[rrd, "\id_Z \times (g \circ \chi)"]                    &                                          &   \\
{Z\times D^n \times [0,1]} \arrow[rr, "w"]                                                                  &                                          & Y \\
{Z\times S^{n-1} \times [0,1]} \arrow[u, hook] \arrow[r, "{\id_Z \times \varphi \times \id_{[0,1]}}"'] & {Z\times A\times [0,1]} \arrow[ru, "K"'] &  
\end{tikzcd}
\] commutes for, in this case, we can set $H(z, x, t) = g \cup_{\varphi} w$, thereby making
\[
\begin{tikzcd}[column sep = huge]
Z\times D^n \times \{0\} \arrow[d, hook] \arrow[rrd, "\id_Z \times (g \circ \chi)"]                          &                                           &   \\
{Z\times D^n \times [0,1]} \arrow[r, "{\id_Z \times \chi \times \id_{[0,1]}}"'] \arrow[rr, "w"', bend right] & {Z\times X\times [0,1]} \arrow[r, "H"']   & Y \\
{Z\times S^{n-1} \times [0,1]} \arrow[u, hook] \arrow[rd, "{\id_Z \times \varphi \times \id_{[0,1]}}"']      &                                           &   \\
                                                                                                             & {Z\times A\times [0,1]} \arrow[ruu, "K"'] &  
\end{tikzcd}
\] commute. To this end, define the retraction $u : D^n \times [0,1] \to D^n \times \{0\} \cup S^{n-1} \times [0,1]$ by  picking a point $\ast$ directly above the cylinder $D^n \times [0,1]$ and then sending any point $x$ in the cylinder to the unique point where $D^n \times \{0\} \cup S^{n-1} \times [0,1]$ intersects the line containing $\ast$ and $x$. Now, define $w$ so that
\[
\begin{tikzcd}[column sep = huge]
{Z \times (D^n \times [0,1])} \arrow[d, "\id_Z \times u"'] \arrow[r, "w"]                                                                     & Y \\
{Z \times (D^n \times \{0\} \cup S^{n-1} \times [0,1])} \arrow[ru, "{\id_Z \times \left(g \circ \chi \cup K \circ (\varphi \times \id_{[0,1]})\right)}"'] &  
\end{tikzcd}
\] commutes. 
\end{proof}

\begin{exercise}
Let $x\in X$. Consider the  loop space $\Omega(X, x) \equiv M((S^1, \pt), (X, x))$. Prove that $\pi_n(\Omega{X})\cong \pi_{n+1}(X)$.
\end{exercise}
\begin{proof}
Consider the \textit{path space $P{X} \equiv \{\gamma : [0,1] \to X \mid \gamma(0) =x\}$ of $(X,x)$}, equipped with the compact-open topology. We claim that $P{X}$ is contractible. Indeed, define $K: P{X} \times [0,1] \to P{X}$ by $$(\gamma, t) \mapsto \left(s \mapsto \gamma(s(1-t))\right).$$ Then $K$ is a homotopy from $\id_{P{X}}$ to the constant map at the constant path at $x$.

\medskip

Define the map $p : P{X} \to X$ by $\gamma \mapsto \gamma(1)$. Then $p^{-1}(x) = \Omega(X)$. By \cref{exhtpy}, it suffices to show that $p$ is a fibration. To this end, suppose that the square
\[
\begin{tikzcd}
Y\times \{0\} \arrow[d, hook] \arrow[r, "\hat{f}"] & P{X} \arrow[d, "p"] \\
{Y\times [0,1]} \arrow[r, "f"']                    & X                  
\end{tikzcd}
\] commutes. Define $H: Y \times [0,1] \to P{X}$ by $(y, t) \mapsto  H(y,t)$ where 
\[
H(y, t)(s) = \begin{cases} 
\hat{f}(y)\left((1+t)s\right) & 0\leq s\leq \frac{1}{1+t}
\\ f(y, (1+t)s -1) & \frac{1}{1+t}\leq s \leq 1
\end{cases}.
\] We see that $H$ is continuous when viewed as a function of $(y,t,s)$ and thus is continuous. It is easy to check that 
\[
\begin{tikzcd}
Y\times \{0\} \arrow[d, hook] \arrow[r, "\hat{f}"] & P{X} \arrow[d, "p"] \\
{Y\times [0,1]} \arrow[r, "f"'] \arrow[ru, "H"]    & X                  
\end{tikzcd}
\] commutes, as desired.
\end{proof}

Let $p : E \to B$ be a map. Recall that the pullback of $p$ along $f : X \to B$ is given explicitly as   $$f^{\ast}{E} \equiv \{(x, e) \in X \times E \mid f(x) = p(e)\}.$$  Let  $f^{\ast}{p}$ denote the map $\pi_X\restriction_{f^{\ast}{E}}$.

\begin{prop}
If $p$ is a fibration, then so is $f^{\ast}{p}$.
\end{prop}

\begin{lemma}\label{pbtriv}
If $p$ is locally trivial, then so is $f^{\ast}{p}$. 
\end{lemma}
\begin{proof}
Let $a \in X$. Since $p$ is locally trivial by assumption, we can find a neighborhood $U$ of $f(a)$ in $B$ and a homeomorphism $\varphi : p^{-1}(U) \to U \times F$. Observe that 
\[
(f^{\ast}{p})^{-1}(f^{-1}(U)) = \{(x,e) \mid f(x) = p(e), \ f(x) \in U\} \subset f^{-1}(U) \times p^{-1}(U).
\] Further, we have a map $\psi : f^{-1}(U) \to p^{-1}(U) \to f^{-1}(U) \times F$ given by $(x,e) \mapsto (x, \pi_F(\varphi(e)))$. Define $\lambda : f^{-1}(U) \times F \to (f^{\ast}{p})^{-1}(f^{-1}(U))$ by $(x,y) \mapsto (x, \varphi^{-1}(f(x), y))$. Using the fact that 
\[
\begin{tikzcd}
p^{-1}(U) \arrow[r, "\varphi"] \arrow[rd, "p"'] & U\times F \arrow[d, "\pi_U"] \\
                                                    & U                           
\end{tikzcd}
\] commutes, it is easy to check that $\psi$ and $\lambda$ are inverses of each other. 
\end{proof}



\subsection{Lecture 5}

\begin{theorem}
Let $B$ be a cell complex and let $p : E \to B$ be locally trivial. Then $p$ is a fibration. 
\end{theorem}
\begin{proof}
It suffices to prove the following claim: 
\begin{addmargin}[1em]{2em}

\smallskip

If $h : Z \to X \times [0,1]$ is locally trivial, $X = \bigcup_{i=0}^n X^i$ is a cell complex, and $\sigma_0 : X \times \{0\} \to Z$ satisfies $h \circ \sigma_0 = \id_{X \times \{0\}}$, then there is some map $\sigma : X \times [0,1] \to Z$ such that $\sigma_{X \times \{0\}} = \sigma_0$ and $h \circ \sigma = \id_{X \times [0,1]}$. 
\end{addmargin}

\smallskip

For, in this case, \cref{pbtriv} implies that given any commutative square
\[
\begin{tikzcd}
X \times \{0\} \arrow[d, hook] \arrow[r, "\hat{f}"] & E \arrow[d, "p"] \\
{X \times [0,1]} \arrow[r, "f"']                    & B               
\end{tikzcd}
,\] we can find some $\sigma$ such that
\[
\begin{tikzcd}
                                                      & f^{\ast}{E} \arrow[r] \arrow[d]                                   & E \arrow[d, "p"] \\
X \times \{0\} \arrow[ru, "\sigma_0"] \arrow[r, hook] & {X \times [0,1]} \arrow[r, "f"'] \arrow[u, "\sigma"', bend right] & B               
\end{tikzcd}
\] commutes where $\sigma_0(x,0) = (x,0, \hat{f}(x,0))$.

\medskip

For induction, we will assume that our claim is true for each $X^0, X^1, \ldots, X^{n-1}$. We may assume, wlog, that $X = D^n$. It suffices to find a map $\tau : S^{n-1} \times [0,1] \to Z$ such that $h \circ \tau = \id_{S^{n-1}\times [0,1]}$ and
\[
\begin{tikzcd}
                                                        & Z \arrow[d, "h"]                                       &                                                           \\
D^n \times \{0\} \arrow[r, hook] \arrow[ru, "\sigma_0"] & {D^n \times [0,1]}                                     & {S^{n-1}\times [0,1]} \arrow[l, hook] \arrow[lu, "\tau"'] \\
                                                        & S^{n-1} \times \{0\} \arrow[lu, hook] \arrow[ru, hook] &                                                          
\end{tikzcd}
\] commutes since there is a retraction $D^n \times [0,1] \to D^{n} \times \{0\} \cup S^{n-1}\times [0,1]$. Fix a positive integer $m$. For any $i\in \N$, let $a_i = \frac{i}{m}$ and let $I_j = [a_j, a_{j+1}]$. By making $m$ large enough, we can ensure that $p\restriction_{p^{-1}(I_{j_1} \times \cdots I_{j_{n+1}})}$ is trivial. 
\begin{claim}
$p\restriction_{p^{-1}(I_{j_1} \times I_{j_n} \times \cdots [0,1])}$ is also trivial. 
\end{claim}
\begin{proof}
??
\end{proof}
?? 
\end{proof}


\section{Fiber bundles}

\begin{defn}
A \textit{topological group} $G$ is a group such that both multiplication $G \times G \overset{\mu}{\longrightarrow} G$ and inversion $G \overset{({-})^{-1}}{\longrightarrow} G$ are continuous.
\end{defn}

\begin{defn}[Fiber bundle] Let $G$ be a topological group.
\be 
\item A \textit{fiber $F$ of $G$} is a space equipped with a faithful (i.e., injective) group action $\rho : G \to \homeo(F) \subset M(F, F)$. 
\item An \textit{atlas for the structure of a (fiber) bundle with group $G$ and fiber $F$ on a map $p: E \to B$} consists of 
\be
\item a family $(U_{\alpha}, h_{\alpha})_{\alpha \in A}$ where each $U_{\alpha}$ is open and each $h_{\alpha}$ is a homeomorphism $p^{-1}(U_{\alpha}) \to U_{\alpha} \times F$ and
\item  a family of continuous \textit{transition functions} $\{h_{\beta{\alpha}} : U_{\alpha} \cap U_{\beta} \to G\}_{\alpha, \beta \in A}$ 
\ee such that
\be[label= \roman*]
\item $B = \bigcup_{\alpha \in A} U_{\alpha}$,
\item $\pi_{U_{\alpha}} \circ h_{\alpha} = p\restriction_{p^{-1}(U_{\alpha})}$, and
\item  $x\in U_{\alpha} \cap U_{\beta} \implies h_{\beta} \circ h_{\alpha}^{-1}(x,f) = (x, h_{\beta{\alpha}}(x)\cdot f)$
\ee
\item Two atlases are \textit{compatible} if their union is an atlas. 
\item A \textit{bundle structure on $B$} is a maximal atlas on $p$. 
\ee
\end{defn}

\begin{term}
If $B$ is equipped with a bundle structure, then we say that $p$ is a (fiber) bundle.
\end{term}

\begin{exmp} $ $
\be
\item The tangent bundle $\pi : TM \to M$ of a smooth $n$-manifold $M$ is a bundle with group $\GL(n, \R)$.
\begin{proof}
Let $(U, \varphi)$ be any coordinate chart for $M$ with coordinate functions $(x^i)$. Define $h: \pi^{-1}(U) \to U \times \R^n$ by $$v^i\frac{\partial}{\partial{x^i}}\left(p\right) \mapsto \left(p, \left(v^1, \ldots, v^n\right)\right).$$ It is clear that $\pi_{U}(h(p)) = \pi(c)$ for any $c\in \pi^{-1}(U)$. To see that $h$ is a homeomorphism, note that the composite $\left(\varphi \times \id_{\R^n}\right) \circ h : \pi^{-1}(U) \to \varphi(U) \times \R^n$ is given by $$v^i\frac{\partial}{\partial{x^i}}\left(p\right) \mapsto \left(x^1(p), \ldots, x^n(p), v^1, \ldots, v^n\right),$$ the inverse of which is given by $\left(x^1, \ldots, x^n, v^1, \ldots, v^n\right) \mapsto v^i\frac{\partial}{\partial{x^i}}\left(\varphi^{-1}(x)\right)$. Therefore,  $\left(\varphi \times \id_{\R^n}\right) \circ h$ is given locally by 
\[
\left(x^1, \ldots, x^n, v^1, \ldots, v^n\right) \mapsto \left(\tilde{x}^1(x), \ldots, \tilde{x}^n(x), \frac{\partial{\tilde{x}^1}}{\partial{x^j}}(x)v^j, \ldots, \frac{\partial{\tilde{x}^n}}{\partial{x^j}}(x)v^j\right),
\] which is smooth. Thus, $h$ is a diffeomorphism as the composition of two diffeomorphisms. In particular, $h$ is a homeomorphism. 

\smallskip

It remains to describe the transition functions $\{h_{\beta{\alpha}} : U_{\alpha} \cap U_{\beta} \to \GL(n, \R)\}$ for $T{M}$. Note that 
\[
\begin{tikzcd}
U_{\alpha{\beta}}\times \R^n \arrow[rd, "\pi_1"'] & \pi^{-1}(U_{\alpha{\beta}}) \arrow[l, "h_{\alpha}"'] \arrow[r, "h_{\beta}"] \arrow[d, "\pi"] & U_{\beta{\alpha}}\times \R^n \arrow[ld, "\pi_1"] \\
                                                  & U_{\alpha{\beta}}                                                                              &                                                 
\end{tikzcd}
\] 
commutes. In particular, $\pi_1 \circ h_{\beta} \circ h_{\alpha}^{-1} = \pi_1$, which implies that $ h_{\beta} \circ h_{\alpha}^{-1}(u,v) =\left(u, f(u,v)\right) $ for some smooth map $f: U_{\alpha{\beta}} \times \R^n \to \R^n$. This must be a linear isomorphism when restricted to $\{u\}\times \R^n$ for any $u\in U_{\alpha{\beta}}$, which is uniquely determined by an element $h_{\beta{\alpha}}(u)$ of $\GL(n, \R)$ (provided that we have fixed a basis of $\R^n$). Hence $$ h_{\beta} \circ h_{\alpha}^{-1}(u,v) =\left(u, h_{\beta{\alpha}}(u) v\right) .$$ Since the map $h_{\beta{\alpha}} :U_{\alpha{\beta}} \to \GL(n, \R)$ is continuous, our proof is complete.
\end{proof}

\item Let $p: E \to B$ be any bundle with group $\{e\}$. Then $p$ is the trivial bundle, i.e., is isomorphic to the projection map. \begin{proof}
We have that $h_{\beta} = h_{\alpha}$ on $U_{\alpha} \cap U_{\beta}$, so that $h \equiv \bigcup_{\alpha \in A}h_{\alpha}$ is a well-defined homeomorphism $E \cong B \times F$.
\end{proof}
\ee 
\end{exmp}

\subsection{Lecture 6}

Let $\left\{\left(U_{\alpha}, h_{\alpha}\right)\right\}$ be a bundle structure with group $G$ and fiber $F$ on $p: E \to B$. Let $U = U_{\alpha} \cap U_{\beta} \cap U_{\gamma}$. 
Consider the commutative diagram
\[
\begin{tikzcd}
                                                                           &                                     & p^{-1}(U) \arrow[rrd, "h_{\gamma}"]    &                                      &           \\
U\times F \arrow[r, "h_{\alpha}^{-1}"'] \arrow[rru, "h_{\alpha}^{-1}"] & p^{-1}(U) \arrow[r, "h_{\beta}"'] & U\times F \arrow[r, "h_{\beta}^{-1}"'] & p^{-1}(U) \arrow[r, "h_{\gamma}"'] & U\times F
\end{tikzcd}.
\]
The bottom row is given by $\left(u, f\right) \mapsto \left(u, h_{\beta{\alpha}}(u) \cdot f\right) \mapsto \left(u, h_{\gamma{\beta}}(u) \cdot  \left( h_{\beta{\alpha}}(u) \cdot f\right)\right) =   \left(u, \left( h_{\gamma{\beta}}(u)  h_{\beta{\alpha}}(u)\right) \cdot f\right)$, and the top composite is given by $\left(u, f\right) \mapsto \left(u, h_{\gamma{\alpha}}(u)\cdot f\right)$.
 It follows that $$h_{\gamma{\beta}}(u)  h_{\beta{\alpha}}(u) = h_{\gamma{\alpha}}(u)$$ for each $u\in U$. This property is known as the \textit{cocycle condition}.

\begin{theorem}
Let $G$ be a topological group acting on a space $F$. Suppose that $\{U_{\alpha}\}$ is an open cover of $B$ and $\{h_{\beta{\alpha}} : U_{\alpha}\cap U_{\beta} \to G\}$ is a family of continuous functions satisfying the cocycle condition. Then there exists a bundle $p: E \to B$ with group $G$, fiber $F$, and transition functions  $h_{\beta{\alpha}}$.
\end{theorem}
\begin{proof}[Proof sketch]
Let $E = \faktor{\coprod_{\alpha}{U_{\alpha} \times F}}{\sim}$ where $\left(u, f\right)_{\alpha} \sim \left(u, h_{\beta{\alpha}}
(u) \cdot f\right)_{\beta}$. Define $p: E \to B$ by $\left(u, f\right) \mapsto  u$.
\end{proof}

\begin{defn}[Bundle map]
A \textit{morphism of bundles $p_1$ and $p_2$ with group $G$ and fiber $F$} is a commutative square of the form
\[
\begin{tikzcd}
E_1 \arrow[r, "\hat{g}"] \arrow[d, "p_1"'] & E_2 \arrow[d, "p_2"] \\
B_1 \arrow[r, "g"']                        & B_2                 
\end{tikzcd}
.\]
\end{defn}

Suppose that $\left(\hat{g}, g\right)$ is a bundle map $p_1 \to p_2$.  Let $\left\{\left(U_{\alpha}, h_{\alpha}\right)\right\}$ and $\left\{\left(V_{\beta}, k_{\beta}\right)\right\}$ be bundle structures on $B_2$ and $B_1$, respectively.  We have a commutative diagram
\[
\begin{tikzcd}
\left(g^{-1}(U_{\alpha}) \cap V_{\beta}\right) \times F \arrow[rd, "\pi_1"'] \arrow[rrr, "d_{\alpha{\beta}}", bend left] \arrow[r, "k_{\beta}^{-1}"] & p^{-1}_1\left(g^{-1}(U_{\alpha}) \cap V_{\beta}\right) \arrow[d] \arrow[r, "\hat{g}"] & p^{-1}_2(U_{\alpha}) \arrow[d] \arrow[r, "h_{\alpha}"] & U_{\alpha}\times F \arrow[ld, "\pi_1"] \\
                                                                                                                                                         & g^{-1}(U_{\alpha})\cap V_{\beta} \arrow[r, "g"']                                        & U_{\alpha}                                               &                                       
\end{tikzcd}
,\] so that $d_{\alpha{\beta}}(x,f)  = \left(g(x), \lambda_{\alpha{\beta}}(x) \cdot f\right)$ for some continuous map $\lambda_{\alpha{\beta}} : g^{-1}(U_{\alpha})\cap V_{\beta} \to G$. Letting $W = g^{-1}(U_{\alpha} \cap U_{\alpha'}) \cap (V_{\beta} \cap V_{\beta'})$, we have that 
\[
h_{\alpha'{\alpha}}(w)\lambda_{\alpha{\beta}}(w)k_{\beta{\beta'}}(w) = \lambda_{\alpha'{\beta'}}(w) \label{eq:mapcycle} \tag{$\dagger$}
\] for every $w\in W$.

\begin{exercise}[Pullback bundle]
Let  $\left\{\left(U_{\alpha}, h_{\alpha}\right)\right\}$ be a bundle structure on $p: E \to B$ with group $G$ and consider the pullback diagram \[
\begin{tikzcd}
g^{\ast}{E} \arrow[d, "g^{\ast}{p}"'] \arrow[r] & E \arrow[d, "p"] \\
X \arrow[r, "g"']                               & B               
\end{tikzcd}
.\] Define $h'_{\beta{\alpha}}: g^{-1}(U_{\alpha}) \cap g^{-1}(U_{\beta}) \to G$ as the composite $h_{\beta{\alpha}} \circ g$ restricted to  $g^{-1}(U_{\alpha} \cap U_{\beta})$. Show that the family $\{h'_{\beta{\alpha}}\}$ induces a bundle structure on $g^{\ast}{p}$.
\end{exercise}

\begin{theorem}\label{factors}
Every bundle map
\[
\begin{tikzcd}
E_1 \arrow[d, "p_1"'] \arrow[r, "\hat{g}"] & E_2 \arrow[d, "p_2"] \\
B_1 \arrow[r, "g"']                        & B_2                 
\end{tikzcd}
\]
factors as
\[
\begin{tikzcd}
E_1 \arrow[r, "\tau"] \arrow[d, "p_1"'] & g^{\ast}{E_2} \arrow[r, "\bar{g}"] \arrow[d, "g^{\ast}{p_2}"] & E_2 \arrow[d, "p_2"] \\
B_1 \arrow[r, "\id_{B_1}"']            & B_1 \arrow[r, "g"']                                           & B_2                 
\end{tikzcd}
\]
where $\tau(e) = \left(p_1(e), \hat{g}(e)\right)$ for any $e\in E_1$.
\end{theorem}

\subsection{Lecture 7}

\begin{note}\label{trnote}
If $\{h_{\beta{\alpha}}:U_{\alpha}\cap U_{\beta}\to G\}$ is a family of transition functions, then
\[
h_{\alpha{\beta}}(x) = \left(h_{\beta{\alpha}}(x)\right)^{-1}
\] for any $x\in U_{\alpha}\cap U_{\beta}$. In particular, $h_{\alpha{\alpha}}(x) = \left(h_{\alpha{\alpha}}(x)\right)^{-1}$.
\end{note}

\begin{theorem}
Any bundle map of the form
\[
\begin{tikzcd}
E_1 \arrow[rd, "p_1"'] \arrow[r, "\hat{g}"] & E_2 \arrow[d, "p_2"] \\
                                            & B                   
\end{tikzcd}
\]
is an isomorphism.
\end{theorem}
\begin{proof}
Note that
\[
\begin{tikzcd}
                                                         & p_2^{-1}(U_{\alpha}\cap U_{\beta}) \arrow[ld, "h_{\beta}"'] \arrow[rd, "h_{\alpha}"]      &                                                                                          \\
\left(U_{\alpha}\cap U_{\beta}\right)\times F \arrow[rd] & p_1^{-1}(U_{\alpha}\cap U_{\beta}) \arrow[u, "\hat{g}"] \arrow[d] \arrow[l, "k_{\beta}"'] & \left(U_{\alpha}\cap U_{\beta}\right)\times F \arrow[l, "k_{\alpha}^{-1}"'] \arrow[ld] \\
                                                         & U_{\alpha}\cap U_{\beta}                                                                    &                                                                                         
\end{tikzcd}
\] commutes.  We have that $h_{\beta} \circ \hat{g} \circ k_{\alpha}^{-1}(x,f) = (x, \lambda_{\beta{\alpha}}(x)\cdot f)$. Thus, if  $h_{\alpha}(e) = (x,f)$, then $h_{\alpha}\left(\hat{g}(e)\right) = (x, \lambda_{\alpha{\alpha}}(x) \cdot d).$ Let $$\left(\hat{g}\right)^{-1}(e) = k_{\alpha}^{-1}\left(x, \lambda_{\alpha{\alpha}}(x)^{-1}\cdot f\right)$$ where $(x,f) = h_{\alpha}(e)$. If this is well-defined on $E_2$ (??), then it indeed equals the inverse of $\hat{g}$.  Moreover, by \cref{trnote}, it is easy to check that $d_{\alpha'{\beta'}}(x)^{-1}$ satisfies \eqref{eq:mapcycle}, and thus it can be shown that $\left(\hat{g}\right)^{-1}$ is a bundle map.

\end{proof}

\begin{corollary}
Every bundle $E\to X$ is isomorphic to the pullback of $E$ by $\id_X$.
\end{corollary}

Let $\left\{\left(U_{\alpha}, h_{\alpha}\right)\right\}$ be a bundle structure with group $G$ and fiber $G$ on $p: E \to X$. In particular,
\[
\begin{tikzcd}
U_{\alpha}\times G \arrow[d, "\pi_1"'] & p^{-1}\left(U_{\alpha}\right) \arrow[l, "h_{\alpha}"'] \arrow[ld, "p"] \\
U_{\alpha}                             &                                                                                
\end{tikzcd}
\]
commutes. Define the free action $E \times G \to E$ by
\[
e\cdot g = h_{\alpha}^{-1}\left(h_{\alpha}(e)\cdot g\right).
\] where $p(e) \in U_{\alpha}$ and $(u,h) \cdot g \equiv \left(u, hg\right)$. This is well-defined because it does not depend on our choice of $\alpha$. Indeed, suppose that $p(e)$ also belongs to $U_{\beta}$. We have that $h_{\alpha}(e) =\left(p(e), h\right)$ and $h_{\beta}(e) = \left(p(e), h'\right)$ for some $h, h' \in G$. Then $e\cdot g = h_{\alpha}^{-1}\left(p(e), hg\right)$, and we must show that this equals $h_{\beta}^{-1}\left(p(e), h'g\right)$.  Note that $h_{\beta}\left(e\cdot g\right) = \left(p(e), h_{\beta{\alpha}}(p(e))hg\right)$.  But $$\left(p(e), h_{\beta{\alpha}}(p(e))h\right)  =  h_{\beta}\left(h_{\alpha}^{-1}\left(p(e), h\right)\right) = \left(p(e), h'\right),$$ 
so that $h_{\beta{\alpha}}(p(e))h = h'$, and thus $h_{\beta}\left(e\cdot g\right) = \left(p(e), h'g\right)$, as desired.
\begin{note}
$\faktor{E}{G} \cong \{p^{-1}(x) \mid x \in X \} \cong X$.
\end{note}

\begin{defn}[Balanced product]
Let $F$ be a space. The \textit{balanced product $E\times_G F$ of $E$ and $F$} is the quotient space $ \faktor{E\times F}{\sim}$ where $$\left(e, f\right)  \sim \left(eg, g^{-1}f\right)$$
for any $e\in E$ and $f\in F$.
\end{defn}

By the universal property of the quotient space, there is a unique map $\bar{p}$ such that
\[
\begin{tikzcd}
E\times F \arrow[d, "p\circ \pi_E"'] \arrow[r, two heads] & E\times_GF \arrow[ld, "\bar{p}"] \\
X                                                         &                                 
\end{tikzcd}. \label{eq:indmap}  \tag{$\star$}
\]

\begin{notation}
Let $\B\left(X, G, \rho, F\right)$ denote the set of all isomorphism classes of bundles over $X$ with group $G$ and fiber $F$. 
\end{notation}

\begin{lemma}
$\bar{p}$ is a bundle with group $G$ and fiber $F$.
\end{lemma}
\begin{proof}
As $\left(g,f\right)\sim \left(e_G, gf\right)$, we see that $\left( U \times G\right) \times_G F\cong U\times F$. Thus, we can endow $\bar{p}$ with local trivializations and transition functions that are exactly similar to those for $p$.
\end{proof}

\begin{prop}\label{iso}
 The function $p\mapsto \bar{p}$ defines a set isomorphism $\B\left(X, G, \rho, G\right) \overset{\cong}{\longrightarrow} \B\left(X, G, \rho, F\right)$. 
\end{prop}

Let $p_1 : E \to B_1$ and $p_2 : E \to B_2$ be bundles. Let $e_1 \in E_1$, $e_2\in E_2$, and $b_1\in B_1$.

\begin{question}
Can we find a bundle map
\[
\begin{tikzcd}
E_1 \arrow[r, dashed] \arrow[d, "p_1"'] & E_2 \arrow[d, "p_2"] \\
B_1 \arrow[r, dashed]                   & B_2                 
\end{tikzcd}
\]
such that $e_1\mapsto e_2$ and $e_1\mapsto b_1$?
\end{question}

Define the action $G\times E_2 \to E_2$ by $g\ast e_2 = e_2\cdot g^{-1}$. From this, we obtain a bundle $$\psi: \underbrace{E_1 \times_G E_2}_{\left(E_1 \times E_2\right)\mathbin{/}{G}} \to E_1 \times_G \pt \cong B_1$$ with fiber $E_2$.

\begin{lemma}\label{corr}
There is a one-to-one correspondence between bundle maps $p_1\to p_2$ and sections of $\psi$.
\end{lemma}
\begin{proof}
Suppose that $\sigma$ is a section of $\psi$.  As $G$ acts freely on $E_1 \times E_2$, we see that for any $e\in E_1$, there exists a unique $\tilde{e}$ such that $\sigma\left(p(e)\right)=\left[\left(e, \tilde{e}\right)\right]$. Define $\hat{g} : E_1\to E_2$ by $e\mapsto \tilde{e}$. This respects the action of $G$ and thus must be a bundle map.
\end{proof}

Now, let $A\subset B_1$ and suppose that
\[
\begin{tikzcd}
p_1^{-1}(A) \MySymb{dr} \arrow[r] \arrow[d] & E_2 \arrow[d, "p_2"] \\
A \arrow[r]                       & B_2                 
\end{tikzcd}
\]
is a bundle map.  Then $\alpha$ extends when ??. Also, the corresponding section $$\sigma : A \to p^{{-1}}(A) \times_G E_2 \subset E_1 \times_G E_2$$ extends.

\begin{defn}[Principal bundle]
Let $G$ be a topological group. A \textit{principal $G$-bundle} is a fiber bundle with group $G$ and fiber $G$ with $G$ acting on itself by left translation.
\end{defn} 


\begin{theorem}\label{class}
Let $f$ and $g$ be homotopic maps $X \to Y$. Let $p: E \to Y$ be any bundle with group $G$ and fiber $F$. Then $f^{\ast}{p} \cong g^{\ast}{p}$.
\end{theorem}



\subsection{Lecture 8}

Before proving this, we wish to determine when, given any two bundles $p_1 : E_1 \to B_1$ and $p_2 : E_2 \to B_2$ and any map $g: B_1 \to B_2$, we can find a map $\hat{g}$ such that  
\[
\begin{tikzcd}
E_1 \arrow[d, "p_1"'] \arrow[r, "\hat{g}"] & E_2 \arrow[d, "p_2"] \\
B_1 \arrow[r, "g"']                        & B_2                 
\end{tikzcd}
\] commutes.

Define the \textit{diagonal action $\Delta{G}$} of $G$ on $E_1 \times E_2$ by $$\left(e_1, e_2\right)\cdot h = \left(e_1\cdot h, e_2 \cdot h\right),$$ so that $E_1 \times_G E_2 = \faktor{E_1 \times E_2}{\Delta{G}}$. By \eqref{eq:indmap}, we can find a unique map $\tau$ such that
\[
\begin{tikzcd}
E_1\times_G E_2 \arrow[rd, "\tau"] \arrow[d] &                                  \\
B_1                                          & B_1\times B_2 \arrow[l, "\pi_1"]
\end{tikzcd}
\] commutes.

\begin{exercise}
Show that $\hat{g}$ exists if and only if there is some $\lambda : B_1 \to E_1 \times_G E_2$ such that $\tau\left(\lambda\left(b_1\right)\right) = \left(b_1, g(b_1)\right)$.
\end{exercise}
\begin{proof} $ $ \\
$\left(\Longleftarrow\right)$ As $G$ acts freely on $E_1 \times E_2$, we see that $\left(e, e' \right) \sim \left(e, e''\right) \implies e'=e''$ for any $e', e'' \in E_2$. Hence for any $e \in E_1$, there exists a unique $\hat{e} \in E_2$ such that $\lambda\left(p_1(e)\right) = \left[\left(e, \hat{e}\right) \right]$. Let $\hat{g}(e) = \hat{e}$. Then $\hat{g}$  is clearly continuous and $G$-equivariant, and thus $\left(\hat{g}, g\right)$ is a bundle map.

\medskip

$\left(\Longrightarrow\right)$ Consider the homeomorphism $\varphi : B_1 \overset{\cong}{\longrightarrow} \faktor{E_1}{G}$ with $\varphi(b) = p_1^{-1}(b)$. Let $b\in B_1$. Let $\varphi(b) = \left[e\right]$. Define $\lambda : B_1 \to E_1 \times_G E_2$ by $\lambda(b) = \left[\left(e, \hat{g}(e)\right)\right]$. Since $\hat{g}$ is $G$-equivariant, we see that $\lambda$ is well-defined. Further, $\lambda$ is continuous as the quotient of the map 
\[ f : E_1 \to E_1 \times E_2, \quad f(x) = \left(x, \hat{g}(x)\right)
\] by $G$.  Finally, it is easy to check that $\tau\left(\lambda\left(b_1\right)\right) = \left(b_1, g(b_1)\right)$ for any $b_1 \in B_1$.
\end{proof}


\begin{lemma}\label{ttriv}
$\tau$ is locally trivial, hence a fibration.
\end{lemma}
\begin{proof}
Locally, we have that $E_1 \cong U \times G$ and $E_2 \cong V \times G$, so that $E_1 \times E_2 \cong U \times V \times G \times G$. It follows that, locally, $E_1\times_G E_2 \cong U_1  \times U_2 \times \faktor{G\times G}{\Delta{G}}$ where $\Delta{G} \equiv \{\left(g, g\right) \mid g\in G\}$.
\end{proof}

\begin{remark}
In fact, $\tau$ is a bundle with fiber $\faktor{G\times G}{\Delta{G}} \cong G$. 
\end{remark}

\begin{proof}[Proof of \cref{class}.]
Due to \cref{iso},  we may assume that $p$ is a principal $G$-bundle.  By assumption, there is some homotopy $H : X \times I \to Y$ from $f$ to $g$.
Let $\omega=  H^{\ast}{p}$. Then 
\begin{align*}
f^{\ast}{p} &  = \omega\restriction_{\omega^{-1}\left(X\times \{0\}\right)} : \omega^{-1}\left(X\times \{0\}\right) \to X\times \{0\} \cong X
\\ g^{\ast}{p}  & = \omega\restriction_{ \omega^{-1}\left(X\times \{1\}\right) } : \omega^{-1}\left(X\times \{1\}\right) \to X\times \{1\} \cong X.
\end{align*}
Therefore, it suffices to show that $f^{\ast}{p}\times \id_I \cong \omega $ such that the diagram
\[
\begin{tikzcd}
f^{\ast}{E}\times I \arrow[r, "\cong"]  \arrow[d, "f^{\ast}{p}\times \id_I"']                        & H^{\ast}{E} \arrow[r] \arrow[d, "\omega"] & E \arrow[d, "p"] \\
X\times I \arrow[r, equals] & X\times I \arrow[r, "H"']                      & Y               
\end{tikzcd}
\] commutes. For, in this case, our isomorphism restricts over $X \times \{1\}$, i.e., $g^{\ast}{p} = \omega\restriction_{X\times \{1\}} \cong f^{\ast}{p}$. It thus suffices to exhibit a bundle map $f^{\ast}{p} \times I\to \omega $ over $\id_{X \times I}$ that equals the identity over $\omega\restriction_{X\times \{0\}} = f^{\ast}{p}$.

\begin{remark}
It is easy to show that there is some bundle map $f^{\ast}{p}\times \id_I \to \omega$.  Indeed, by the homotopy lifting property, we obtain a section $\sigma$ fitting into the commutative diagram
\[
\begin{tikzcd}
                                   & \left(f^{\ast}{E}\times I\right) \times_G H^{\ast}{E} \arrow[d] \\
X\times \{0\} \arrow[r] \arrow[ru, "\lambda_0"] & X\times I \arrow[u, "\sigma"', dashed, bend right]             
\end{tikzcd}
,\] in which case we obtain our desired map by \cref{corr}. As mentioned, however, we want a bundle map that  equals the identity over $f^{\ast}{p}$.
\end{remark}

To get such a map, we must find a section $\lambda$ such that
\[
\begin{tikzcd}
                                                      & \left(f^{\ast}{E} \times I\right) \times_G H^{\ast}{E} \arrow[rd, "\tau"] \arrow[d] &                                                       \\
X\times \{0\} \arrow[ru, "\lambda_0"] \arrow[r, hook] & X\times I \arrow[r, "\Delta"'] \arrow[u, "\lambda"', dashed, bend right]            & \left(X\times I\right) \times \left(X \times I\right)
\end{tikzcd}
\] commutes. But  $\lambda$ must exist since $\tau$ is a fibration by virtue of \cref{ttriv}. 

\end{proof}


\begin{corollary}
Any bundle over a contractible space $B$ is trivial.
\end{corollary}
\begin{proof}
Let $i : \pt \to B$ and $\pi : B \to \pt$ denote inclusion and projection, respectively. Then 
\begin{align*}
p &  \cong \left(\id\right)^{\ast}{p}
\\ & \cong \left(i{\pi}\right)^{\ast}{p}
\\ & \cong \pi^{\ast}\underbrace{{i^{\ast}}{p}}_{\text{trivial}},
\end{align*}
which is trivial since the pullback of a trivial bundle is trivial.
\end{proof}

\begin{corollary}\label{restiso}
Every bundle $p$ over $X\times I$ is isomorphic to $\left(p\restriction_{p^{-1}\left(X\times \{0\}\right)}\right) \times \id_I$.
\end{corollary}

\begin{exmp}
Consider $S^1 \subset \R^2$ with center the origin. Let $p: E \to S^1$ be a bundle with group $G$ and fiber $F$. Cover $S^1$ with the open intervals $I_1 \coloneqq S^1\setminus \{{-}1\}$ and $I_2 \coloneqq S^1\setminus \{1\}$. We may assume that $F= p^{-1}\left({-}1\right)$. Then $E = E_1 \cup E_2$ where $ E_i \cong I_i \times F$ via, say, $\varphi_i$ for each $i=1,2$. By \cref{restiso}, we see that $$\varphi_1\restriction_{\varphi_1^{-1}\left(\{1\} \times F\right)} = \varphi_2\restriction_{\varphi_2^{-1}\left(\{{-}1\} \times F\right)}= \id_F.$$ Moreover, the transition function $\varphi_2^{-1} \circ \varphi_1\restriction_{p^{-1}\left(1\right)} : F \to F$ is given by multiplication by some $g\in G$. Hence the map $G \to \B\left(S^1, G, F\right)$ is surjective. In fact, it can be shown that this maps descends to an isomorphism
$$\pi_0\left(G\right) \cong \faktor{G}{G_0} \overset{\cong}{\longrightarrow} \B\left(S^1, G, F\right)$$ 
where $G_0$ denotes the connected component of $e_G$. 

\smallskip

For example, if $G=F = \GL\left(n, \R\right)$, then $\pi_0(G)$ consists of the set of matrices with positive determinant and the set of matrices with negative determinant, so that $\B\left(S^1, G, F\right) \cong \Z/2$.
\end{exmp}

\begin{exmp}
The set $\B\left(S^2, G, F \right)$ is isomorphic to the set of homotopy classes of maps $S^1 \to G$, As it turns out, we can ignore base points, so that  $\B\left(S^2, G, F \right) \cong \pi_1\left(G\right)$. 

\smallskip

For example, if $G = F = \SO(2)$, then $G\cong S^1$, so that $\B\left(S^2, G, F \right) \cong \Z$.
\end{exmp}

\subsection{Lecture 9}

\begin{theorem}\label{ext}
Let $X$ be a cell complex with $\dim{X} \leq n$. Let $A \subset X$ be a subcomplex. Let $p: E \to X$ be a bundle with fiber $F$ such that $\pi_i\left(F, f\right) =0$ for each $i \leq n-1$. Suppose that $\sigma_0 : A \to E$ satisfies $p\circ \sigma_0(a) = a$ for each $a\in A$. Then $\sigma_0$ extends to a section $\sigma : X \to E$ of $p$.
\[
\begin{tikzcd}
                                         & E \arrow[d, "p"']                          \\
A \arrow[ru, "\sigma_0"] \arrow[r, hook] & X \arrow[u, "\sigma"', dashed, bend right]
\end{tikzcd}
\]
\end{theorem}
\begin{proof}
First, assume that $X$ is a regular complex. Since $X$ is finite, we may assume that $X= A \cup_{S^{k-1}}D^k$ where $k\leq n$. Further, we may assume, wlog, that $X = D^k$. Thus, we must find a section $\sigma$ such that
\[
\begin{tikzcd}
                                                                     & E \arrow[d, "p"']                            \\
S^{k-1} \arrow[ru, "\sigma_0\restriction_{S^{k-1}}"] \arrow[r, hook] & D^k \arrow[u, "\sigma"', dashed, bend right]
\end{tikzcd}
\] commutes. Since $D^k$ is contractible, we have that $E \cong D^k \times F$. Then $\sigma_0(x) = \left(x, \tilde{\sigma}_0(x)\right)$ for each $x\in S^{k-1}$.  But $ \tilde{\sigma}_0(x) : S^{k-1} \to F$ extends to a map $\tilde{\sigma} :D^k\to F$ because $\pi_{k-1}\left(F\right)=0$. Hence we can take $\sigma$ to be the map defined by $x\mapsto  \left(x, \tilde{\sigma}(x)\right)$.

\medskip

Next, drop the assumption that $X$ is regular. Using \cref{reg}, we get a homotopy equivalence
\[
\begin{tikzcd}
{\left(X,A\right)} \arrow[r, "h", bend left] & {\underbrace{\left(\overline{X}, \overline{A}\right)}_{\text{regular}}} \arrow[l, "g", bend left]
\end{tikzcd}
\] of pairs. Define $\overline{A}\to g^{\ast}{E}$ by $\bar{\sigma}_0(a) = \left(a, \sigma_0\left(g(a)\right)\right)$. By our preceding discussion, this extends to a section $\bar{\sigma}$ on $\overline{X}$.\footnote{As $\dim{\overline{X}} >\dim{X}$, we tacitly rely on the fact that $\pi_i\left(F\right)$ is trivial for large enough $i$.}  We wish to find $\sigma$ such that
\[
\begin{tikzcd}
g^{\ast}{E} \arrow[r] \arrow[d]                                                    & E \arrow[d, "p"]                                      \\
\overline{X} \arrow[r, "g"'] \arrow[u, "\bar{\sigma}"', bend right]                  & X \arrow[u, "\sigma", dashed, bend left]              \\
\overline{A} \arrow[r, "g"'] \arrow[u, hook] \arrow[uu, "\bar{\sigma}_0", bend left] & A \arrow[u, hook] \arrow[uu, "\sigma_0"', bend right]
\end{tikzcd}
\] commutes. But since $p \cong h^{\ast}{g^{\ast}{p}}$, we have a commutative diagram
\[
\begin{tikzcd}
g^{\ast}{E} \arrow[d, "g^{\ast}{p}"]              & h^{\ast}{g^{\ast}{E}} \arrow[d, "h^{\ast}{g^{\ast}{p}}"'] \arrow[l] \arrow[r, "\cong"] & E \arrow[ld, "p"] \\
\overline{X} \arrow[u, "\bar{\sigma}", bend left] & X \arrow[l, "h"]                                                                      &                  
\end{tikzcd}
,\] from which we obtain our desired section $\sigma$.
\end{proof}

\begin{notation}
$\left[X, Y\right] \coloneqq \left(\text{homotopy classes of maps } X \to Y\right)$.
\end{notation}

\begin{corollary}\label{classif}
Let $p: E \to B$ be a principal $G$-bundle and suppose that $\pi_i(E) =0$ for any $i\leq n-1$. The function $\chi_X : \left[X, B\right] \to \B\left(X, G, G \right)$ given by $f \mapsto f^{\ast}{p}$ is bijective. 
\end{corollary}
\begin{proof} $ $ 

\smallskip


\underline{Surjective:} Let $p_1 : E_1 \to X$ be a bundle. Due to \cref{factors}, it suffices to find a bundle map $\left(\hat{f}, f\right)$ such that
\[
\begin{tikzcd}
E_1 \arrow[d, "p_1"'] \arrow[r, "\hat{f}", dashed] & E \arrow[d] \\
X \arrow[r, "f"', dashed]                          & B          
\end{tikzcd}
\] commutes. Such a map can be found precisely when there exists a section of the bundle $E_1 \times_G E \to X$, which holds by applying \cref{ext} to the case where $A = \emptyset$.

\medskip

\underline{Injective:} Suppose that $\chi_X(f) = \chi_X(g)$. We must show that $f\simeq g$, i.e., that there is some bundle map $\left(\hat{H}, H\right)$ such that
\[
\begin{tikzcd}
                                       & {f^{\ast}{p}\times \{0,1\}} \arrow[r, hook] \arrow[rr, bend left] \arrow[ld, "\cong"'] & f^{\ast}{p}\times I \arrow[r, "\hat{H}"', dashed] \arrow[d] & E \arrow[d, "p"] \\
f^{\ast}{p}\cup g^{\ast}{p} \arrow[rd] &                                                                                        & X\times I \arrow[r, "H"', dashed]                           & B                \\
                                       & {X\times \{0,1\}} \arrow[ru]                                                           &                                                             &                 
\end{tikzcd}
\] commutes. This is equivalent to finding a section $\lambda$ such that
\[
\begin{tikzcd}
{\left(X\times \{0,1\}\right) \times B}                                       & \left(f^{\ast}{p} \times I\right)\times_G E \arrow[d] \arrow[l, "\tau"'] \\
{X\times \{0,1\}} \arrow[ru, "\lambda_0"] \arrow[r, hook] \arrow[u, "\gamma"] & X\times I \arrow[u, "\lambda"', dashed, bend right]                  
\end{tikzcd}
\] commutes where $$\gamma\left(x,t\right) = 
\begin{cases} \left(x,t, f(x)\right) & t= 0 \\ \left(x,t, g(x)\right) & t= 1 
\end{cases}.$$ But this exists by \cref{ext} because $\pi_i(E)=0$ by assumption.
\end{proof}

\begin{defn}[Classifying space]
  A \textit{classifying space for principal $G$-bundles} is a space $B$ such that $\chi_X$ is bijective for every cell complex $X$.
\end{defn}

\begin{exmp}
Let $G = \{\pm 1\}$.  Then any principal $G$-bundle over $X$ is a two-fold covering space of $X$, i.e., a subgroup of index two in $\pi\left(X\right)$, i.e., a nontrivial homomorphism $\pi_1{X} \to G$.

\smallskip

 For example, let $\{U_i\}$ denote the usual open covering of $\RP^n = \faktor{S^n}{G}$. Let $\pi : S^n \to \RP^n$ denote the projection map. We have that $\pi^{-1} \left(U_i\right) \cong U_i \times G$. Indeed,  define $h_i : \pi^{-1} \left(U_i\right) \to  U_i \times G$ by $$\left(x_0, \ldots, x_n\right) \mapsto \left(\left[x_0, \ldots, x_n\right], \frac{x_i}{\lvert{x_i}\rvert}\right),$$ the inverse of which is  given by 
 \begin{align*}
 \left(y_0, \ldots y_n\right) & \mapsfrom \left(\left[x_0, \ldots, x_n\right], \epsilon \right) 
 \\ y_k & \equiv \epsilon x_k\cdot \frac{\lvert{x_i}\rvert}{x_i}.
 \end{align*}
 Note that any transition function $h_{ji} : U_i \cap U_j \to G$ is given by $h_{ji}(x) = {-}1$.
\end{exmp}

Using the fact that $\pi_1$ is the abelianization of $H_1$ along with the universal coefficient theorem for cohomology, one can prove the following.

\begin{prop}
$\B\left(X, \Z_2, F \right) \cong \left[X, \RP^n\right] \cong \Hom\left(\pi_1\left(X\right), \Z/2\right) \cong H^1\left(X, \Z/2\right)$. 
\end{prop}

Let $w_1 \in H^1\left(\RP^n, \Z/2\right) \cong \Z_2$ be nonzero. Let $p_1 : E \to X$ be a $\Z/2$-bundle. We call $w_1\left(p_1\right) \coloneqq f^{\ast}{w_1} \in H^1\left(X, \Z/2\right)$ the \textit{first Stiefel-Whitney class of p}. 

\subsection{Lecture 10}

\begin{exmp}
Let $n\in \N$. Recall that $\CP^n$, by definition, consists of all the complex lines in $\C^{n+1}$. Let $G = S^1$. Then $G$ acts on $\C^{n+1}$ by $g\cdot \left(z_0, \ldots, z_n\right) = \left(gz_0, \ldots, gz_n\right)$. We have that $\CP^n \cong \faktor{S^{2n+1}}{\sim}$ where $z\sim \zeta\cdot z$ for any $\zeta \in S^1$. Consider the projection map $\pi : S^{2n+1} \twoheadrightarrow \CP^n$. For each $i\in \{0, \ldots, n\}$, let $H_i = \{z\in \CP^n \mid z_i =0\} \cong \CP^{n-1}$ and let $U_i = \CP^n \setminus H_i$. Then the $U_i$ form an open cover of $\CP^n$.  Define $h_i : \pi^{-1}\left(U_i\right) \to U_i \times S^1$ by $\left(z_0, \ldots, z_n\right) \mapsto \left(\left[z_0, \ldots, z_n\right], \frac{z_i}{\lvert{z_i}\rvert}\right)$. 

\begin{exercise} $ $
\be
\item Prove that $h_i$ is a homeomorphism.
\item Find the transition functions $h_{ij} : U_j \cap U_i \to S^1$.
\ee
\end{exercise}
\begin{proof} $ $
\be
\item It is obvious that $h_i$ is continuous. Define $g_i : U_i \times S^1 \to \pi^{-1}\left(U_i\right)$ by 
\begin{align*}
\left(\left[z_0, \ldots, z_n\right], \epsilon \right) &  \mapsto \left(y_0, \ldots, y_n\right)
\\ y_k & \equiv \epsilon{z_k}\cdot \frac{\lvert{z_i}\rvert}{z_i}, \ k =0, \ldots, n.
\end{align*}
It is easy to check that this is well-defined and that $g_i$ is the inverse of $h_i$. It remains to show that $g_i$ is continuous. Consider the quotient map  $q\coloneqq \pi \times \id_{S^1}: S^{2n+1} \times S^1 \to \CP^n \times S^1$. Let $\widetilde{U}_i = \{z \in S^{2n+1} \mid z_i \ne 0\}$. Note that $g_i \circ q\restriction_{\widetilde{U}_i \times S^1}$ is clearly continuous. But $\widetilde{U}_i \times S^1$ is both open in $S^{2n+1} \times S^1$ and saturated with respect to $q$. Hence $\restriction_{\widetilde{U}_i \times S^1}$ is a quotient map, so that $g_i$ is continuous.  
\item Note that 
\[
h_i \circ h_j^{-1}\left(\left[z_0, \ldots, z_n\right], \epsilon\right) = \left(\left[z_0, \ldots, z_n\right], \epsilon  \frac{\lvert{z_j}\rvert}{z_j} \cdot \frac{z_i}{\lvert{z_i}\rvert}\right)
\] for any $\left[z_0, \ldots, z_n\right] \in U_i \cap U_j$. This implies that $$h_{ij}\left(\left[z_0, \ldots, z_n\right]\right) = \frac{\lvert{z_j}\rvert}{z_j} \cdot \frac{z_i}{\lvert{z_i}\rvert}.$$
\ee
\end{proof}

It follows that $\pi$ is a principal $S^1$-bundle. Since each homotopy group $\pi_i\left(S^{2n+1}\right)$ is trivial,  \cref{classif} implies that $$\B\left(X, S^1, F\right) \cong \left[X, \CP^n\right],$$ which for large enough $n$, is isomorphic to $\left[X, \CP^{\infty}\right]$ where $X$ denotes and any cell complex and $$\CP^{\infty} \equiv \bigcup_{k\in \N}\CP^k$$ equipped with the weak topology.
\end{exmp}

\begin{defn}
An \textit{Eilenberg-MacLane space of type $K\left(G, n\right)$} is a space satisfying 
\[
\begin{cases}
\pi_i{K} = 0 & i \ne n
\\ \pi_i{K} \cong G & i =n
\end{cases}.
\]
\end{defn}

\begin{theorem}\label{EM}
If $X$ is a cell complex, then $\left[X, K\left(G, n\right)\right] \cong H^n\left(X, G\right)$.
\end{theorem}

\begin{exmp}
By inspecting the long exact sequence
\[
\begin{tikzcd}
\cdots \arrow[r] & \pi_2\left(S^{2n+1}\right) \arrow[r]              & \pi_2\left(\CP^n\right) \arrow[ld]   &        \\
                 & \underbrace{\pi_1\left(S^1\right)}_{\Z} \arrow[r] & \pi_1\left(S^{2n+1}\right) \arrow[r] & \cdots
\end{tikzcd}
,\] we see that $\CP^n$ is an Eilenberg-MacLane space of type $K\left(\Z, 2\right)$. Moreover, there is a commutative triangle
\[
\begin{tikzcd}
\CP^{\infty}             & \CP^n \arrow[l, hook] \\
S^i \arrow[u] \arrow[ru] &                      
\end{tikzcd}
\] for any $i\in \N$. Thus, $\pi_i\left(\CP^{\infty}\right) = \pi_i\left(\CP^n\right)$ when $n$ is large enough. This means that $\CP^{\infty}$ is also 
an Eilenberg-MacLane space of type $K\left(\Z, 2\right)$. By \cref{EM}, we have that $$\B\left(X, S^1, F\right) \cong H^2\left(X, \Z\right)$$ whenever $X$ is a cell complex. 
\end{exmp}

For us, a CW complex refers to a cell complex $X$ for which there may be infinitely many attaching maps of any dimension. In this name, ``C" stands for the property \textit{closure-finite}, i.e., every open cell $e^i$ is contained in a finite subcomplex of $X$. Further, ``W" stands for the weak topology, with which $X$ is equipped.

\begin{remark}
Each of our results holds even if we assume that a certain space is merely a CW complex rather than a cell complex. 
\end{remark}

\begin{note}[Milnor construction]
There exists a functor $\mathbf{TopGrp} \to \mathbf{PrinBund}$ that maps each topological group $G$ to a principal $G$-bundle $$E_G \overset{p_G}{\longrightarrow} B_G$$ such that $B_G$ is a CW complex and $\pi_i\left(E_G\right) =0$. This means that $B_G$ is a classifying space for principal $G$-bundles.

\smallskip

By applying our LES on homotopy groups to $p_G$, we see that  $\pi_i\left(B_G\right) \cong \pi_{i-1}\left(G\right)$.
\end{note}


Alternatively, one can use the Brown representability theorem (\href{https://ncatlab.org/nlab/show/Brown+representability+theorem}{nLab article}) to obtain a classifying space $B'_G$ (not necessarily a CW complex) because the pullback functor satisfies
\bi
\item homotopy invariance,
\item excision, and
\item Mayer-Vietoris.
\ei

\begin{lemma}\label{uniq}
Let $p_1: E_1 \to B_1$ and $p_2 : E_2 \to B_2$ be classifying spaces for principal $G$-bundles. Then $B_1 \simeq B_2$. 
\end{lemma}
\begin{proof}
By \cref{classif}, there is some map $f : B_1 \to B_2$ such that $f^{\ast}{p_2} \cong  p_1$. Likewise, there is some map $g: B_2 \to B_1$ such that $g^{\ast}{p_1}\cong p_2$. Therefore, 
\begin{align*}
\left(f \circ g\right)^{\ast}{p_2} & \cong g^{\ast}{f^{\ast}{p_2}} 
\\ & \cong g^{\ast}{p_1}
\\ & \cong p_2
\\ & \cong \id_{B_2}^{\ast}{p_2}.
\end{align*}
Therefore, $f \circ g \simeq \id_{B_2}$. Similarly, $g\circ f \simeq \id_{B_1}$.
\end{proof}

In particular, $B_G \simeq B'_G$.

\begin{exmp}
$B_{S^1} = \CP^{\infty}$.
\end{exmp}


Let $H\leq G$. Consider the commutative square
\[
\begin{tikzcd}
E_G \arrow[d, "p_G"'] \arrow[r, "q", two heads] & \faktor{E_G}{H} \arrow[d, "r"] \\
B_G \arrow[r, equals]                                   & \faktor{E_G}{G}               
\end{tikzcd}.
\]  Note that, locally, $r$ looks like the trivial map with fiber $\faktor{G}{H}$. Thus, $q$ locally looks like the map $$U \times G \to U \times \faktor{G}{H}.$$ This shows that if the natural projection $G \to \faktor{G}{H}$ is a principal $H$-bundle, then so is $q$. In this case, we have that $B_H \simeq \faktor{E_G}{H}$ by \cref{classif} together with \cref{uniq}.


\begin{theorem}\label{H-bund}
If $G$ is a Lie group and $H$ is a closed subgroup of $G$, then the natural projection $G \to \faktor{G}{H}$ is a principal $H$-bundle. 
\end{theorem}

\begin{defn}
The \textit{orthogonal group $\Or\left(n, \R\right)$} is the group of $n\times n$ real matrices $A$ such that $AA^t = A^tA = I_n$, equivalently, $Av \bullet Aw = v\bullet w$ for any $v,w\in \R^n$. We call such an $A$ \textit{orthogonal}.
\end{defn}

In particular, if $A$ is orthogonal, then $\|Av\| = \|v\|$ for any $v\in \R^n$.

\begin{exmp}
The orthogonal group $\Or\left(n, \R\right)$ is a closed subgroup of $\GL\left(n, \R\right)$ because  $\Or\left(n, \R\right)= f^{-1}\left(I_n\right)$ where $f: \GL\left(n, \R\right) \to \GL\left(n, \R\right)$ is given by $X \mapsto XX^t$. Let $\gamma : \GL\left(n, \R\right) \to \Or\left(n, \R\right)$ denote the map given by the Gram-Schmidt procedure. Let $i : \Or\left(n, \R\right) \to \GL\left(n, \R\right)$ denote the inclusion map. Then $\gamma$ and $i$ are homotopy inverses of each other, so that 
\[
\GL\left(n, \R\right) \simeq \Or\left(n, \R\right).
\]  Since $\pi : \GL\left(n, \R\right) \to \underbrace{\faktor{\GL\left(n, \R\right) }{\Or\left(n, \R\right)}}_{M}$ is an $\Or\left(n, \R\right)$-bundle by \cref{H-bund}, our LES on homotopy groups applied to $\pi$ shows that $\pi_i\left(M\right) =0$ for each $i\in \N$. Further, our LES  applied to the $M$-bundle $r: B_{\Or\left(n, \R\right)} \to B_{\GL\left(n, \R\right)}$ shows that $$\pi_i\left(B_{\Or\left(n, \R\right)}\right) \cong \pi_i\left(B_{\GL\left(n, \R\right)}\right)$$ for each $i$. By \cref{WH}, it follows that 
\[
B_{\Or\left(n, \R\right)} \simeq B_{\GL\left(n, \R\right)}
.\] An exactly similar argument proves that $B_{\Un\left(n, \C\right)} \simeq B_{\GL\left(n, \C\right)}$.
\end{exmp}

Our next goal is to describe $H^{\ast}\left(B_G\right)$. This leads us to the notion of a spectral sequence. 

\section{Spectral sequences}


\end{document}

\documentclass[10pt,letterpaper,cm]{nupset}
\usepackage[margin=1in]{geometry}
\usepackage{graphicx}
 \usepackage{enumitem}
 \usepackage{stmaryrd}
 \usepackage{bm}
\usepackage{amsfonts}
\usepackage{amssymb}
\usepackage{pgfplots}
\usepackage{amsmath,amsthm}
\usepackage{lmodern}
\usepackage{tikz-cd}
\usepackage{faktor}
\usepackage{xfrac}
\usepackage{mathtools}
\usepackage{bm}
\usepackage{ dsfont }
\usepackage{mathrsfs}
\usepackage{hyperref}
\usepackage{scrextend}
\hypersetup{colorlinks=true, linkcolor=red,          % color of internal links (change box color with linkbordercolor)
    citecolor=green,        % color of links to bibliography
    filecolor=magenta,      % color of file links
    urlcolor=cyan           }
\usepackage{adjustbox}
\usepackage{media9}
\usepackage{thmtools}
\usepackage[capitalise]{cleveref} 
    
\theoremstyle{definition}
\newtheorem{definition}{Definition}[section]
\newtheorem{exmp}[definition]{Example}
\newtheorem{non-exmp}[definition]{Non-example}
\newtheorem{note}[definition]{Note}

\theoremstyle{theorem}
\newtheorem{theorem}[definition]{Theorem}
\newtheorem{lemma}[definition]{Lemma}
\newtheorem{corollary}[definition]{Corollary}
\newtheorem{prop}[definition]{Proposition}
\newtheorem{conj}[definition]{Conjecture}
\newtheorem*{claim}{Claim}
\newtheorem{exercise}[definition]{Exercise}

\theoremstyle{remark}
\newtheorem{remark}[definition]{Remark}
\newtheorem*{todo}{To do}
\newtheorem*{conv}{Convention}
\newtheorem*{aside}{Aside}
\newtheorem*{notation}{Notation}
\newtheorem*{term}{Terminology}
\newtheorem*{background}{Background}
\newtheorem*{further}{Further reading}
\newtheorem*{sources}{Sources}

\makeatletter
\def\th@plain{%
  \thm@notefont{}% same as heading font
  \itshape % body font
}
\def\th@definition{%
  \thm@notefont{}% same as heading font
  \normalfont % body font
}
\makeatother

\makeatletter
\renewcommand*\env@matrix[1][*\c@MaxMatrixCols c]{%
  \hskip -\arraycolsep
  \let\@ifnextchar\new@ifnextchar
  \array{#1}}
\makeatother
\pgfplotsset{unit circle/.style={width=4cm,height=4cm,axis lines=middle,xtick=\empty,ytick=\empty,axis equal,enlargelimits,xmax=1,ymax=1,xmin=-1,ymin=-1,domain=0:pi/2}}
\DeclareMathOperator{\Ima}{Im}
\newcommand{\A}{\mathcal A}
\newcommand{\C}{\mathbb C}
\newcommand{\E}{\vec E}
\newcommand{\CP}{\mathbb CP}
\newcommand{\F}{\mathcal F}
\newcommand{\G}{\vec G}
\renewcommand{\H}{\vec H}
\newcommand{\HP}{\mathbb HP}
\newcommand{\K}{\mathbb K}
\renewcommand{\L}{\mathcal L}
\newcommand{\M}{\mathbb M}
\newcommand{\N}{\mathbb N}
\renewcommand{\O}{\mathbf O}
\newcommand{\OP}{\mathbb OP}
\renewcommand{\P}{\mathcal P}
\newcommand{\Q}{\mathbb Q}
\newcommand{\I}{\mathbb I}
\newcommand{\R}{\mathbb R}
\newcommand{\RP}{\mathbb RP}
\renewcommand{\S}{\mathbf S}
\newcommand{\T}{\mathbf T}
\newcommand{\X}{\mathbf X}
\newcommand{\Z}{\mathbb Z}
\newcommand{\B}{\mathcal{B}}
\newcommand{\1}{\mathbf{1}}
\newcommand{\ds}{\displaystyle}
\newcommand{\ran}{\right>}
\newcommand{\lan}{\left<}
\newcommand{\bmat}[1]{\begin{bmatrix} #1 \end{bmatrix}}
\renewcommand{\a}{\vec{a}}
\renewcommand{\b}{\vec b}

\renewcommand{\c}{\mathscr{C}}
\renewcommand{\d}{\mathscr{D}}
\newcommand{\e}{\mathscr{E}}
\newcommand{\y}{\mathscr{Y}}

\newcommand{\h}{\vec h}
\newcommand{\f}{\vec f}
\newcommand{\g}{\vec g}
\renewcommand{\i}{\vec i}
\renewcommand{\j}{\vec j}
\renewcommand{\k}{\vec k}
\newcommand{\n}{\vec n}
\newcommand{\p}{\vec p}
\newcommand{\q}{\vec q}
\renewcommand{\r}{\vec r}
\newcommand{\s}{\vec s}
\renewcommand{\t}{\vec t}
\renewcommand{\u}{\vec u}
\renewcommand{\v}{\vec v}
\newcommand{\w}{\vec w}
\newcommand{\x}{\vec x}
\newcommand{\z}{\vec z}
\newcommand{\0}{\vec 0}
\DeclareMathOperator*{\Span}{span}
\DeclareMathOperator*{\GL}{GL}
\DeclareMathOperator{\rng}{range}
\DeclareMathOperator{\gemu}{gemu}
\DeclareMathOperator{\almu}{almu}
\newcommand{\Char}{\mathsf{char}}
\DeclareMathOperator{\id}{Id}
\DeclareMathOperator{\im}{Im}
\DeclareMathOperator{\graph}{Graph}
\DeclareMathOperator{\gal}{Gal}
\DeclareMathOperator{\tr}{Tr}
\DeclareMathOperator{\norm}{N}
\DeclareMathOperator{\aut}{Aut}
\DeclareMathOperator{\Int}{Int}
\DeclareMathOperator{\ext}{Ext}
\DeclareMathOperator{\stab}{Stab}
\DeclareMathOperator{\orb}{Orb}
\DeclareMathOperator{\inn}{Inn}
\DeclareMathOperator{\out}{Out}
\DeclareMathOperator{\op}{op}
\DeclareMathOperator{\fix}{Fix}
\DeclareMathOperator{\ab}{ab}
\DeclareMathOperator{\sgn}{sgn}
\DeclareMathOperator{\syl}{syl}
\DeclareMathOperator{\Syl}{Syl}
\DeclareMathOperator{\ob}{ob}
\DeclareMathOperator{\mor}{mor}
\DeclareMathOperator{\ar}{Ar}
\DeclareMathOperator{\ZFC}{ZFC}

\linespread{1.3}

% info for header block in upper right hand corner
\name{Perry Hart}
\class{$K$-theory reading seminar}
\assignment{UPenn}
\duedate{September 26, 2018}

%Talk #3

\begin{document}

\begin{abstract}
We introduce the concept of a natural transformation in category theory, leading to equivalences and adjunctions. The main sources for this talk are the following.
\begin{itemize}
\item $n$Lab.
\item John Rognes's \textit{Lecture Notes on Algebraic $K$-Theory}, Ch. 3.
\item Peter Johnstone's lecture notes for ``Category Theory" (Mathematical Tripos Part III, Michaelmas 2015), Ch. 1.
\end{itemize}
\end{abstract}

\smallskip

\section{Natural transformations}


Let $\c$ and $\d$ be categories and $F$ and $G$ be functors $\c \to \d$.  A \textit{natural transformation} $\phi :F \Rightarrow G$  is a function $A \mapsto f_A$ from $\ob \c$ to $\mor \d$ such that $f_A$ is a map  $F(A) \to G(A)$ and the following diagram commutes for any morphism $h: A \to B$ in $\c$.

\[
\begin{tikzcd}[row sep=large, column sep = large]
FA \arrow[r, "F{h}"] \arrow[d, "f_A", swap]
& FB \arrow[d, "f_B"] \\
GA \arrow[r, "G{h}"']
& GB
\end{tikzcd}
\]
In symbols, this may be written as $f_Bh_{\ast} = h_{\ast}f_A$, where $f_A$ is called a \textit{component} of $\phi$.


\begin{note}\label{rmk}
If every $f_A$ is an isomorphism, then the maps $\left(f_A\right)^{{-1}}$ define a natural transformation $G \Rightarrow F$.
\end{note}

If each $f_A$ is an isomorphism, then we say that $\phi$ is a \textit{natural isomorphism}.
Note that if $\d$ is a groupoid (i.e., a category in which every morphism is an isomorphism), then $\phi$ must be a natural isomorphism.

\medskip

Let $F$, $G$, and $H$ be functors $\c \to \d$. The \textit{identity natural transformation} $\id_F : F \Rightarrow F$ is given by $A \mapsto \id_{F(A)}$. Moreover, given natural transformations $\phi: F \to G$ and $\psi: G \to H$, define the \textit{composite natural transformation} $\psi \circ \phi$ by $A \mapsto (\psi \circ \phi)_A \coloneqq \psi_{A} \circ \phi_{A}$.

\begin{lemma}
A natural transformation $\phi: F \Rightarrow G$ is a natural isomorphism iff it has an inverse $\phi^{{-1}} : G \Rightarrow F$.
\end{lemma}
\begin{proof}
This follows from \cref{rmk} along with our definition of a composite natural transformation.
\end{proof}

\begin{exmp} $ $
\begin{enumerate}
\item Let $R$ and $S$ be commutative rings. Any ring homomorphism $f: R \to S$ induces a ring homomorphism $\GL_n(f): \GL_n(R) \to \GL_n(S)$ satisfying $$f(\det(A)) = \det\left(\GL_n(f)(A)\right).$$ By viewing $\GL_n$ and $R \mapsto R^{\ast}$ as functors from $\mathbf{Ring}$ to $\mathbf{Grp}$ and $\det_R : \GL_n(R) \to R^{\ast}$ as a morphism in $\mathbf{Grp}$, we see that $\det_R$ defines a natural transformation $\phi : \GL_n \Rightarrow f^{\ast}$ where $f^{\ast}$ denotes $f\restriction_{R^{\ast}} :R^{\ast} \to S^{\ast}$.  

\[
\begin{tikzcd}[row sep=large, column sep = large]
\GL_n(R) \arrow[r, "\GL_n(f)"] \arrow[d, "\det_R", swap]
& \GL_n(S) \arrow[d, "\det_S"] \\
R^{\ast} \arrow[r, "f^{\ast}"']
& S^{\ast}
\end{tikzcd}
\]
\item Consider the power set functor $\P: \mathbf{Set} \to \mathbf{Set}$ defined  on objects  by $A \mapsto \P(A)$ and on morphisms $g$  by $\P{g}(S) = g(S)$. Then the function $f_A: A \to \P(A)$ given by $a \mapsto \{a\}$ defines a natural transformation $\phi: \id_{\mathbf{Set}} \Rightarrow \P$.
\item 
Set $\c = \d = \mathbf{Grp}$, $F= \id_{\c}$, and $G =(-)^{\ab}$. Then given a group $H$, the natural projection $f: H \twoheadrightarrow H^{\ab}$ induces a natural transformation $\phi: F \Rightarrow G$.
\item 
We can view preorders $\left(P, \leq\right)$ and $\left(Q, \leq\right)$ as small categories and functors $F, G: P \to Q$ as order-preserving functions. Then there is a unique natural transformation $\phi: F \Rightarrow G$ iff $F(x) \leq G(x)$ for every $x\in P$.
\item 
The inversion isomorphism from a group $G$ to its opposite group $G^{\op}$ defines a natural transformation $\phi: \id_{\mathbf{Grp}} \Rightarrow \left((-)^{\op}: \mathbf{Grp} \to \mathbf{Grp}\right)$. In this sense, $G$ is naturally isomorphic to $G^{\op}$.
\end{enumerate}
\end{exmp}

\smallskip

\begin{definition}
Let $\c$ and $\d$ be categories with $\c$ small. The $\textit{functor category}$ $\mathbf{Fun}(\c, \d)\coloneqq \d^\c$ has functors $F: \c \to \d$ as objects and natural transformations as morphisms. 
\end{definition}


\begin{remark}
Any Grothendieck universe models $\ZFC$, in particular $\mathsf{Replacement}$. This ensure that for any two functors $F, G: \c \to \d$, the class of natural transformation $\phi: F \Rightarrow G$ is a set. This means that $\mathbf{Fun}(\c, \d)$ is locally small, a condition of our definition of a category.
\end{remark}

\begin{definition}
Given a category $\c$, the \textit{arrow category} $\ar(\c)$ of $\c$ has as objects morphisms $f: X_0 \to X_1$ in $\c$ and as morphisms $M: \left(f: X_0 \to X_1\right) \to \left(g: Y_0 \to Y_1\right)$ the pairs $\left(M_0, M_1\right)$ of morphisms $M_0 : X_0 \to Y_0$ and $M_1 : X_1 \to Y_1$ such that
\[
\begin{tikzcd}[row sep=large, column sep = large]
X_0 \arrow[r, "f"] \arrow[d, "M_0", swap]
& X_1 \arrow[d, "M_1"] \\
Y_0 \arrow[r, "g"']
& Y_1
\end{tikzcd}
\]
commutes. 
\end{definition}

\begin{note} $ $
\begin{enumerate}
\item $\ar(\c) \cong \mathbf{Fun}([1], \c)$.
\item $\mathbf{Fun}(\c \times \d, \e) \cong \mathbf{Fun}(\c, \mathbf{Fun}( \d, \e))$. 
\end{enumerate}
\end{note}

\section{Equivalences}

Usually, it is useful to make our notion of \textit{sameness} between categories weaker than \textit{isomorphism}.

\begin{definition}
A functor $F: \c \to \d$ is an $\textit{equivalence}$ if there is a functor $G: \d \to \c$, called the \textit{quasi-inverse of $F$}, such that $F \circ G \cong \id_{\c}$ and $G \circ F \cong \id_{\d}$. In this case, we say that $F$ and $G$ are \textit{equivalent categories}. Moreover, we say that a property of $\c$ is \textit{categorical} if it is invariant under categorical equivalence.
\end{definition}

\begin{exmp}
Let $k$ be a field. Let the category $\mathbf{Mat}_k$  have natural numbers as objects and morphisms $n \to p$ given by $p \times n$ matrices over $k$. Let $\mathbf{fdMod}$ denote the category of finite-dimensional vector spaces with linear maps as morphisms. These two categories are equivalent. Indeed, send the natural number $n$ to $k^n$ in one direction and the space $V$ to $\dim V$ in the other direction.    
\end{exmp}


\begin{definition}
A functor $F : \c \to \d$ is \textit{essentially surjective} if for each object $Z$ of  $\d$, there is some object $Y$ of $\c$ such that $F(Y) \cong Z$.
\end{definition}

\begin{theorem}\label{equiv}
A functor is an equivalence iff it is full, faithful, and essentially surjective.\footnote{Theorem 3.2.10 (Rognes).}
\end{theorem}


\begin{definition}
A \textit{skeleton} of $\c$ is a full subcategory $\c' \subset \c$ such that each element of $\ob \c$ is isomorphic to exactly one element of $\ob \c'$.
\end{definition}

An application of \cref{equiv} yields the following result.

\begin{lemma}\label{e1}
Let $\c'$ be a skeleton of $\c$. Then the inclusion functor  $\c' \hookrightarrow \c$ is an equivalence.
\end{lemma}

\smallskip

\begin{lemma}\label{e2}
Any two skeleta $\c', \c'' \subset \c$ are isomorphic.
\end{lemma}
\begin{proof}
Define $F: \c' \to \c''$ on objects by $F(X) =Y$ where $X\cong Y$ via a chosen isomorphism $h_X$ and on morphisms $f\in \c(X, Y)$ by $F(f) = h_Y \circ f \circ (h_X)^{{-1}}$.
To get $F^{{-1}}$,  define $G: \c'' \to \c'$ by similarly choosing an isomorphism $\left(h_X\right)^{{-1}}$ for each $X\in \ob{\c''}$.
\end{proof}

\begin{remark}
Both \cref{e1} and \cref{e2} are logically equivalent to the axiom of choice, as is the statement that every category admits a skeleton.
\end{remark}


\section{Adjunctions}

\begin{definition} $ $
\begin{enumerate}
\item Let $Z \in \ob \c$. Define the contravariant functor $\y_Z: \c^{\op} \to \mathbf{Set}$  on objects by $Y \mapsto \c(Y, Z)$ and on morphisms by sending $f: X \to Y$ in $\c$ to the map $f^{\ast} : \c(Y, Z) \to \c(X, Z)$ given by $g\mapsto gf$. 

We call $\c(-, Z)\coloneqq \y^Z$ the set-valued functor \textit{represented by $Z$} in $\c$.
\item Let $X \in \ob \c$. Define the functor $\y^X : \c \to \mathbf{Set}$ on objects by $Y \mapsto \c(X, Y)$  and on morphisms by sending $g: Y \to Z$ to the map $g_{\ast} : \c(X, Y) \to \c(X, Z)$ given by $f \mapsto gf$. 

We call $\c(X, -)\coloneqq\y^X$ the set-valued functor \textit{corepresented by $X$} in $\c$.
\end{enumerate}
\end{definition}

\smallskip

A functor of the form $\c \times \c' \to \d$ is called a \textit{bifunctor}.
In particular, define $\c(-, -): \c^{\op} \times \c \to \mathbf{Set}$ on objects by $\left(X, X'\right) \to \c(X, X')$ and on morphisms by sending $\left(f, f'\right) : \left(X, X'\right) \to \left(Y, Y'\right)$ to the map $\c(f, f') : \c(X, X') \to \c(Y, Y')$ given by $g \mapsto {f'}{g}{f}$.

\medskip

Let $\c$ and $\d$ be categories and $F : \c \to \d$ and $G: \d \to \c$ be functors. 


\begin{definition}[Kan]
Consider the set-valued bifunctors $ \d(F(-), -), \c(-, G(-)): \c^{\op} \times \d \to \mathbf{Set}$. An \textit{adjunction from $F$ to $G$} is a natural isomorphism $$\phi : \d(F(-), -) \Rightarrow \c(-, G(-)).$$
If such a $\phi$ exists, then we say that $\left(F, G\right)$ is an \textit{adjoint pair (of functors)}. 
\end{definition}

\smallskip


Note that $\phi$ is natural in the sense that for any map $c: X' \to X$ in $\c$ and $d: Y \to Y'$ in $\d$, the square

\[
\begin{tikzcd}[row sep=large, column sep = large]
\d(F{X}, Y) \arrow[r, "\phi_{X, Y}"] \arrow[d, "c^{\ast}d_{\ast}", swap]
& \c(X, G{Y}) \arrow[d, "c^{\ast}d_{\ast}"] \\
\d(F{X'}, Y')\arrow[r, "\phi_{X', Y'}"']
& \c(X', G{Y'})  
\end{tikzcd}
\] commutes in $\mathbf{Set}$.

\begin{exmp}
Let $\left(P, \leq\right)$ and $\left(Q, \leq\right)$ be preorders. An adjoint pair $\left(F: P \to Q, G: Q \to P\right)$ is precisely a pair of order-preserving functions such that 
\[
F{x} \leq y \iff x \leq G{y}
\] for all $x\in P$ and $y \in Q$. In order theory, such a pair is called a \textit{Galois connection}. 
\end{exmp}

\begin{prop}
Left and right adjoints are both unique up to unique isomorphism.
\end{prop}

\begin{term}
We call $F$ the \textit{left adjoint} to $G$ and $G$ the \textit{right adjoint} to $F$. In symbols, $F \dashv G$.
\end{term}

\begin{note}
It is straightforward to check that any adjoint triple $F \dashv G \dashv H$ yields two new adjunctions:
\begin{gather*}
G{F} \dashv G{H}
\\ F{G} \dashv H{G}
\end{gather*}
\end{note}



\begin{definition}
Given an adjunction $\phi: \d(F(-), -) \Rightarrow \c(-, G(-))$, define the \textit{unit morphism } $$\eta_X =\phi_{X, F{X}}\left(\id_{F{X}}\right) \in \c(X, GF(X))$$ and the \textit{counit morphism} $$\epsilon_Y =\phi_{G{Y}, Y}^{{-1}}\left(\id_{G{Y}}\right) \in \d(FG(Y), Y).$$
\end{definition}

\begin{lemma}
The unit morphisms $\left(\eta_X\right)_{X\in \ob{\c}}$ define a natural transformation $\eta: \id_{\c} \Rightarrow GF$, and the counit morphisms $\left(\epsilon_Y\right)_{Y\in \ob{\d}}$ define a natural transformation $\epsilon: FG \Rightarrow \id_{\d}$.
\end{lemma}
\begin{proof}
For simplicity, let us just prove that $\epsilon$ is a natural transformation. We must check that
\[
\begin{tikzcd}[column sep=large]
FG(Y) \arrow[d, "\epsilon_Y"'] \arrow[r, "FG(y)"] & FG(Y') \arrow[d, "\epsilon_{Y'}"] \\
Y \arrow[r, "y"']                             & Y'                               
\end{tikzcd}
\] commutes for any map $y: Y\to Y'$ in $\d$. By the naturality of $\phi$, we have that
\begin{align*}
y\circ \epsilon_Y  & =  y\circ \phi^{{-1}}\left(\id_{G{Y}}\right)
\\ & = \phi^{{-1}}\left(G{y} \circ \id_{G{Y}}\right)
\\ & =  \phi^{{-1}}\left(\id_{G{Y'}} \circ G{y}\right)
\\ & =   \phi^{{-1}}\left( \id_{G{Y'}}\right) \circ FG(y) 
\\ & = \epsilon_{Y'} \circ FG(y)
,\end{align*} as required.
\end{proof}

Moreover, one can verify that the unit and counit of $\phi$ satisfy the \textit{triangle identities},
\begin{align}
\epsilon_{F{X}} \circ F \eta_{X} &=1_{F{X}} \label{tri1} \tag{$\vartriangle_1$}
\\ G \epsilon_{Y} \circ \eta_{G{Y}} &=1_{G{Y}}, \label{tri2} \tag{$\vartriangle_2$}
\end{align}
for any $X\in \ob{\c}$ and $Y\in \ob{\d}$.

\smallskip

Conversely, suppose that $F$ and $G$ come equipped with two natural transformations
\begin{gather*}
\eta: \id_{\c} \Rightarrow GF
\\ \epsilon: FG \Rightarrow \id_{\d}
\end{gather*}
satisfying the triangle identities. Then we get an adjunction $\phi$ from $F$ to $G$ with component 
\[
\phi_{X, Y} : \d(F{X}, Y) \to \c(X, G{Y}), \ \quad f \mapsto G{f} \circ \eta_X.
\] 
Indeed, define $\psi_{X,Y} : \c(X, G{Y})\to  \d(F{X}, Y)$ by $g\mapsto \epsilon_Y \circ F{g}$. We have that 
\begin{align*}
\psi_{X,Y}\left(\phi_{X,Y}(f)\right) & = \psi_{X,Y}\left(G{f} \circ \eta_X\right)
\\ & = \epsilon_{Y} \circ F\left(G{f} \circ \eta_X\right)
\\ & = \epsilon_{Y} \circ F(G{f}) \circ F{\eta_X}
\\ & = f\circ \epsilon_{F{X}}\circ F{\eta_X} \tag{naturality of $\epsilon$}
\\ & =  f \tag{\eqref{tri1}}.
\end{align*}
Likewise, we have that $\phi_{X,Y}\left(\psi_{X,Y}(g)\right) = g$. Hence  $\phi_{X,Y}$ is a natural isomorphism in both $X$ and $Y$ with inverse $\psi_{X,Y}$.


\smallskip

Even so,  $\overset{F}{\underset{G}{\c \rightleftarrows \d}}$ need \emph{not} be an equivalence of categories, as $\eta$ and $\epsilon$ may not be isomorphisms. Further, a given equivalence $\overset{L}{\underset{R}{\c \rightleftarrows \d}}$ of categories need \emph{not} be an adjunction, as its associated natural transformations
\begin{gather*}
\eta': \id_{\c} \Rightarrow RL
\\ \epsilon': LR \Rightarrow \id_{\d}
\end{gather*}
may not satisfy the triangle inequalities. Nevertheless, $\left(L, R\right)$ \emph{is} an adjoint pair with unit $\eta'$ and counit another natural transformation defined in terms of $\eta'$ and $\epsilon'$. By symmetry, $\left(R, L\right)$ is also an adjoint pair.

\smallskip


\begin{exmp}[Monad]
Let $\left(\c, \otimes, 1\right)$ be a monoidal category. A \textit{monoid} in $\c$ is an object $M$ equipped with a \textit{multiplication} map $\mu : M \otimes M \to M$ and a \textit{unit} map $\eta : 1 \to M$ that satisfy certain coherence properties expressing that $\mu$ is associative and that $\eta$ is a two-sided identity. Given two monoids $\left(M, \mu, \eta\right)$ and $\left(M', \mu', \eta'\right)$ in $\c$, a map $f : M \to M'$ in $\c$ is a \textit{morphism of monoids} if it satisfies 
\[
f \circ \mu = \mu' \circ \left(f \otimes f\right) \quad \quad f \circ \eta = \eta'.
\]
A \textit{comonoid} $N$ in $\c$ is a monoid in $\c^{\op}$,  equipped with a \textit{comultiplication} map $\delta :  N\to N^2$ and a \textit{counit} map $\epsilon :N\to 1$. 

For example, a monoid  in the monoidal category $\left(\mathsf{End}(\c), \circ, \id_{\c}\right)$ of endofunctors of $\c$ is called a \textit{monad on $\c$}. A comonoid in $\mathsf{End}(\c)$ is called a \textit{comonad on $\c$}.

Explicitly, a monad on $\c$ consists of an endofunctor $T: \c \to \c$ together with two natural transformations $\eta :\id_{\c} \to Y$ and $\mu : T^2 \to T$ such that the following diagrams commute:
\[
\begin{tikzcd}
T^3 \arrow[d, "\mu_T"'] \arrow[r, "T{\mu}"] & T^2 \arrow[d, "\mu"] &  & T \arrow[rd, equal] \arrow[r, "\eta_T"] & T^2 \arrow[d, "\mu"] & T \arrow[ld, equal] \arrow[l, "T{\eta}"'] \\
T^2 \arrow[r, "\mu"']                       & T                    &  &                                  & T                    &                                   
\end{tikzcd}
.\] These are precisely the \textit{associativity} and \textit{unit} laws, respectively. Now, let $\left(F: \c \to \d, G: \d \to \c \right)$ be an adjoint pair with unit $\eta :\id_{\c} :G\circ F$ and counit $\epsilon : F\circ G \to \id_{\d}$.  We then  have a natural transformation $\left(G\circ F\right)^2 \to G\circ F$ given componentwise by
\[
G\left(\epsilon_{F{X}}\right) : G{F{G{F{X}}}}  \to G{F{X}}
\] One can check that $\left(G \circ F, \eta, G{\epsilon_F}\right)$ is a monad on $\c$.

Dually, a comonad $R :\c \to \c$ on $\c$ satisfies the relations
\begin{gather*}
\delta_{R} \circ \delta=R \delta \circ \delta\\
\epsilon_{R} \circ \delta= \id_{R}=R \epsilon \circ \delta.
\end{gather*}
Moreover, any adjoint pair $\left(F: \c \to \d, G: \d \to \c \right)$ with unit $\eta$ and counit $\epsilon$ induces a comonad $\left(G, \epsilon, \delta\right)$ on $\d$ where
\begin{align*}
G& \equiv F\circ G : \d \to \d
\\ \delta &\equiv F{\eta_G} : G \to G^2.
\end{align*}
\end{exmp}

\begin{theorem}
The category of monoids in $\c$ is equivalent to the category of $\c$-enriched categories with one object.
\end{theorem}

\smallskip

\begin{exmp}\label{U1} $ $
\begin{enumerate}[label=(\arabic*)]
\item The forgetful functor $U: \mathbf{Grp} \to \mathbf{Set}$ has a left adjoint $F: \mathbf{Set} \to \mathbf{Grp}$ sending a set to the free group generated by $A$. 
\item Let $R$ be a ring. The forgetful functor $U: R{-}\mathbf{Mod}\to \mathbf{Set}$ has a left adjoint $R(-)$ sending a set $S$ to $\bigoplus_{s\in S} R$, the free $R$-module generated by $S$.
\end{enumerate}
\end{exmp}



The forgetful functor has no right adjoint in either \cref{U1}(1) or \cref{U1}(2). It does, however, have one in the following setting.


\begin{exmp}
The forgetful functor $U: \mathbf{Top} \to \mathbf{Set}$ has a left adjoint that sends a set to the same set equipped with the discrete topology.  It also has a right adjoint that sends a set to the same set equipped with the indiscrete topology.
\end{exmp}

\smallskip

\begin{definition}
A subcategory $\c \subset \d$ is \textit{reflective} if the inclusion functor has a left adjoint and is \textit{coreflective} if the inclusion functor has a right adjoint. 
\end{definition}

\begin{exmp} $ $
\begin{enumerate}
\item The full subcategory $\mathbf{Ab}\subset \mathbf{Grp}$ is reflexive as the inclusion functor is right adjoint to $\left(-\right)^{\ab}$.
\item
Let $\mathbf{Ab}_T\subset \mathbf{Ab}$ denote the subcategory of torsion groups. This is coreflective as the inclusion functor is right adjoint to the functor sending an abelian group to its torsion subgroup.
\end{enumerate}
\end{exmp}

\end{document}
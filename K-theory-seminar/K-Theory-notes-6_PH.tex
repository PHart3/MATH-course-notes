\documentclass[10pt,letterpaper,cm]{nupset}
\usepackage[margin=1.2in]{geometry}
\usepackage{graphicx}
 \usepackage{enumitem}
 \usepackage{stmaryrd}
 \usepackage{bm}
\usepackage{amsfonts}
\usepackage{amssymb}
\usepackage{pgfplots}
\usepackage{amsmath,amsthm}
\usepackage{lmodern}
\usepackage{tikz-cd}
\usepackage{faktor}
\usepackage{xfrac}
\usepackage{mathtools}
\usepackage{bm}
\usepackage{ dsfont }
\usepackage{mathrsfs}
\usepackage{hyperref}
\hypersetup{colorlinks=true, linkcolor=red,          % color of internal links (change box color with linkbordercolor)
    citecolor=green,        % color of links to bibliography
    filecolor=magenta,      % color of file links
    urlcolor=cyan           }

\usepackage{thmtools}
\usepackage[capitalise]{cleveref} 
    
\theoremstyle{definition}
\newtheorem{definition}{Definition}
\newtheorem{exmp}[definition]{Example}
\newtheorem{non-exmp}[definition]{Non-example}
\newtheorem{note}[definition]{Note}

\theoremstyle{theorem}
\newtheorem{theorem}{Theorem}
\newtheorem{lemma}[theorem]{Lemma}
\newtheorem{prop}[theorem]{Proposition}
\newtheorem{corollary}[theorem]{Corollary}
\newtheorem*{claim}{Claim}
\newtheorem{exercise}[theorem]{Exercise}

\theoremstyle{remark}
\newtheorem{remark}{Remark}
\newtheorem*{todo}{To do}
\newtheorem*{conv}{Convention}
\newtheorem*{aside}{Aside}
\newtheorem*{notation}{Notation}
\newtheorem*{term}{Terminology}
\newtheorem*{background}{Background}
\newtheorem*{further}{Further reading}
\newtheorem*{sources}{Sources}

\makeatletter
\def\th@plain{%
  \thm@notefont{}% same as heading font
  \itshape % body font
}
\def\th@definition{%
  \thm@notefont{}% same as heading font
  \normalfont % body font
}
\makeatother


\makeatletter
\renewcommand*\env@matrix[1][*\c@MaxMatrixCols c]{%
  \hskip -\arraycolsep
  \let\@ifnextchar\new@ifnextchar
  \array{#1}}
\makeatother
\pgfplotsset{unit circle/.style={width=4cm,height=4cm,axis lines=middle,xtick=\empty,ytick=\empty,axis equal,enlargelimits,xmax=1,ymax=1,xmin=-1,ymin=-1,domain=0:pi/2}}
\DeclareMathOperator{\Ima}{Im}
\newcommand{\A}{\mathcal A}
\newcommand{\C}{\mathbb C}
\newcommand{\E}{\vec E}
\newcommand{\CP}{\mathbb CP}
\newcommand{\F}{\mathbb F}
\newcommand{\G}{\vec G}
\renewcommand{\H}{\mathbb H}
\newcommand{\HP}{\mathbb HP}
\newcommand{\K}{\mathbb K}
\renewcommand{\L}{\mathcal L}
\newcommand{\M}{\mathbb M}
\newcommand{\N}{\mathbb N}
\renewcommand{\O}{\mathbf O}
\newcommand{\OP}{\mathbb OP}
\renewcommand{\P}{\mathbf P}
\newcommand{\Q}{\mathbb Q}
\newcommand{\I}{\mathbb I}
\newcommand{\R}{\mathbb R}
\newcommand{\RP}{\mathbb RP}
\renewcommand{\S}{\mathbf S}
\newcommand{\T}{\mathbf T}
\newcommand{\X}{\mathbf X}
\newcommand{\Z}{\mathbb Z}
\newcommand{\B}{\mathcal{B}}
\newcommand{\1}{\mathbf{1}}
\newcommand{\ds}{\displaystyle}
\newcommand{\ran}{\right>}
\newcommand{\lan}{\left<}
\newcommand{\bmat}[1]{\begin{bmatrix} #1 \end{bmatrix}}

\renewcommand{\a}{\mathscr{A}}
\renewcommand{\b}{\mathscr{B}}
\renewcommand{\c}{\mathscr{C}}
\renewcommand{\d}{\mathscr{D}}
\newcommand{\e}{\mathscr{E}}
\newcommand{\y}{\mathscr{Y}}
\renewcommand{\j}{\mathscr{J}}

\newcommand{\h}{\vec h}
\newcommand{\f}{\vec f}
\newcommand{\g}{\vec g}
\renewcommand{\i}{\vec i}
\renewcommand{\k}{\vec k}
\newcommand{\n}{\vec n}
\newcommand{\p}{\vec p}
\newcommand{\q}{\vec q}
\renewcommand{\r}{\vec r}
\newcommand{\s}{\vec s}
\renewcommand{\t}{\vec t}
\renewcommand{\u}{\vec u}
\renewcommand{\v}{\vec v}
\newcommand{\w}{\vec w}
\newcommand{\x}{\vec x}
\newcommand{\z}{\vec z}
\newcommand{\0}{\vec 0}
\DeclareMathOperator*{\Span}{span}
\DeclareMathOperator*{\GL}{GL}
\DeclareMathOperator*{\SL}{SL}
\DeclareMathOperator*{\SO}{SO}
\DeclareMathOperator*{\SU}{SU}
\DeclareMathOperator{\rng}{range}
\DeclareMathOperator{\gemu}{gemu}
\DeclareMathOperator{\almu}{almu}
\newcommand{\Char}{\mathsf{char}}
\DeclareMathOperator{\id}{Id}
\DeclareMathOperator{\im}{im}
\DeclareMathOperator{\graph}{Graph}
\DeclareMathOperator{\gal}{Gal}
\DeclareMathOperator{\tr}{Tr}
\DeclareMathOperator{\norm}{N}
\DeclareMathOperator{\aut}{Aut}
\DeclareMathOperator{\Int}{Int}
\DeclareMathOperator{\ext}{Ext}
\DeclareMathOperator{\stab}{Stab}
\DeclareMathOperator{\orb}{Orb}
\DeclareMathOperator{\inn}{Inn}
\DeclareMathOperator{\out}{Out}
\DeclareMathOperator{\op}{op}
\DeclareMathOperator{\fix}{Fix}
\DeclareMathOperator{\ab}{ab}
\DeclareMathOperator{\sgn}{sgn}
\DeclareMathOperator{\syl}{syl}
\DeclareMathOperator{\Syl}{Syl}
\DeclareMathOperator{\ob}{ob}
\DeclareMathOperator{\mor}{mor}
\DeclareMathOperator{\iso}{iso}
\DeclareMathOperator{\ar}{Ar}
\DeclareMathOperator{\red}{red}
\DeclareMathOperator{\colim}{colim}
\DeclareMathOperator{\ZFC}{ZFC}
\DeclareMathOperator{\set}{\mathbf{Set}}
\DeclareMathOperator{\Ab}{\mathbf{Ab}}
\DeclareMathOperator{\Cmon}{\mathbf{CMon}}
\DeclareMathOperator{\spec}{Spec}
\DeclareMathOperator{\rank}{rank}
\DeclareMathOperator{\rk}{rk}
\DeclareMathOperator{\diag}{diag}
\DeclareMathOperator{\Ar}{Ar}
\DeclareMathOperator{\Sp}{Sp}
\DeclareMathOperator{\cone}{cone}
\DeclareMathOperator{\ch}{\mathbf{Ch}}
\DeclareMathOperator{\Mod}{\mathbf Mod}

% info for header block in upper right hand corner
\name{Perry Hart}
\class{Homotopy and K-theory seminar}
\assignment{Talk \#16}
\duedate{November 14, 2018}

\begin{document}

\begin{abstract}
We continue doing higher Waldhausen $K$-theory. The main sources for this talk are the following.
\begin{itemize}
\item $n$Lab.
\item Charles Weibel's \textit{The $K$-book: an introduction to algebraic $K$-theory},  Ch. V.2.
\item John Rognes's \textit{Lecture Notes on Algebraic $K$-Theory}, Ch. 8.
\end{itemize}
\end{abstract}

\smallskip

Recall that $\lvert{wS_{\bullet} \c}\rvert$ is an $H$-space via the map $$\coprod: \lvert{wS_{\bullet} \c}\rvert \times \lvert{wS_{\bullet} \c}\rvert \cong \lvert{wS_{\bullet} \c \times  wS_{\bullet} \c}\rvert\to \lvert{wS_{\bullet} \c}\rvert.$$ This produces an $H$-space structure $\left(K(\c), +\right)$.


\begin{definition}
Let $\b$ and $\c$ be Waldhausen categories. We say that $F' \rightarrowtail F \twoheadrightarrow F''$ is a \textit{short exact sequence} or \textit{cofiber sequence of exact functors} if every $F'(B) \rightarrowtail F(B) \twoheadrightarrow F''(B)$ is a cofiber sequence and $F(A) \cup_{F'(A)} F'(B) \rightarrowtail F(B)$ is a cofibration in $\c$ for every $A \rightarrowtail B$ in $\b$.
\end{definition}

Let $\c$ be a Waldhausen category. Let $(\eta): A \rightarrowtail B \twoheadrightarrow C$ be an object in $S_2\c$. Define the source $s$, target $t$, and quotient $q$ functors $S_2\c \to \c$ by $s(\eta) = A$, $t(\eta) = B$, and $q(\eta) = C$. Then $s \rightarrowtail t \twoheadrightarrow q$ is a cofiber sequence of functors. Since defining a cofiber sequence of exact functors $\b \to \c$ is equivalent to defining an exact functor $\b \to S_2 \c$, we may restrict our attention to $s \rightarrowtail t \twoheadrightarrow q$ when proving things about a given cofiber sequence of exact functors $\b \to \c$. We say that $S_2\c$ is universal in this sense.

\begin{theorem}[Extension theorem]
Let $\c$ be Waldhausen. The exact functor $(s, q) : S_2 \c \to \c \times \c$ induces  a homotopy $K(S_2\c) \simeq K(\c) \times K(\c)$. The functor $\coprod : (A, B) \to (A \rightarrowtail A \coprod B \twoheadrightarrow  B)$ is a homotopy inverse.
\end{theorem}
\begin{proof}
Let $\c^w_m$ denote the category of $m$-length sequences of weak equivalences. For each $n$, define $s_n\c^w_m$ as the commutative diagram
\[
\begin{tikzcd}
 X_1^0 \arrow[d, "\sim"'] \arrow[r, tail] & X_2^0 \arrow[r, tail] \arrow[d, "\sim"'] & \cdots \arrow[r, tail] & X_n^0 \arrow[d, "\sim"'] \\
X_1^1 \arrow[d, "\sim"'] \arrow[r, tail] & X_2^1 \arrow[r, tail] \arrow[d, "\sim"'] & \cdots \arrow[r, tail] & X_n^1 \arrow[d, "\sim"'] \\
  \vdots \arrow[d, "\sim"'] & \vdots \arrow[d, "\sim"'] &  & \vdots \arrow[d, "\sim"'] \\
 X_1^m \arrow[r, tail] & X_2^m \arrow[r, tail] & \cdots \arrow[r, tail] & X_n^m
\end{tikzcd}
.\]
This is naturally isomorphic to an $(m, n)$-bisimplex in $N_{\bullet}w S_{\bullet}\c$, which is thus isomorphic to the bisimplicial set $s_{\bullet}\c^w_{(-)}$. One can show that the source $s$ and quotient $q$ functors  $S_2 \c \to \c$ give a homotopy equivalence $s \times q : s_{\bullet}S_2(\c^w_m) \to s_{\bullet}\c^w_m \times s_{\bullet}\c^w_m$ for each $m$. Thus, we get a homotopy equivalence  $$s_{\bullet}S_2(\c^w_{(-)}) \simeq s_{\bullet}\c^w_{(-)} \times s_{\bullet}\c^w_{(-)}$$ between bisimplicial sets. But we already have that $s_{\bullet}\c^w_{(-)} \cong N_{\bullet}w S_{\bullet}\c$, completing the proof.
\end{proof}

\begin{theorem}[Additivity theorem]
Let $F' \rightarrowtail F \twoheadrightarrow F''$ be a short exact sequence of exact functors $\b \to \c$. Then $F_{\ast} \simeq F_{\ast}' + F_{\ast}''$ as maps $K(\b) \to K(\c)$. Hence $F_{\ast} = F_{\ast}' + F_{\ast}''$ as maps $K_i(\b) \to K_i(\c)$. 
\end{theorem}
\begin{proof}
As $S_2\c$ is universal, it suffices to prove that $t_{\ast}  \simeq s_{\ast} + q_{\ast}$. Notice that the two composites $$\c \times \c \overset{\coprod}{\longrightarrow} S_2 \c  \underset{s \coprod q}{\overset{t}{\rightrightarrows}} \c$$ are the same. The extension theorem implies that $K(\coprod): K(\c) \times K(\c) \to K(S_2 \c)$ is a homotopy equivalence. Since the $H$-space structure on $K(\c)$ is induced by $\coprod$, we get $t_{\ast}  \simeq s_{\ast} + q_{\ast}$. 
\end{proof}

\begin{definition}
Let $\c$ be Waldhausen. We say that a sequence $\ast \to A_n \to \cdots \to A_0 \to \ast$ is \textit{admissibly exact} if each morphism in the sequences can be written as a cofiber sequence $A_{i+1} \twoheadrightarrow B_i \rightarrowtail A_i$. 
\end{definition}

\begin{corollary}
Suppose that $\ast \to F^0 \to F^1 \to \cdots \to F^n \to \ast$ is an admissibly exact sequence of exact functors $\b \to \c$. Then $\sum_i ({-}1)^iF_{\ast}^i = 0$ as maps $K_i(\b) \to K_i(\c)$.
\end{corollary}
\begin{proof}
Induct on $n$.
\end{proof}

\begin{corollary}
Let $F' \rightarrowtail F \twoheadrightarrow F''$ be a short exact sequence of exact functors $\b \to \c$. Then $$F_{\ast}'' \simeq F_{\ast} - F_{\ast} \simeq 0.$$ This implies that the homotopy fiber of $F_{\ast}'' : K(\b) \to K(\c)$ is homotopy equivalent to $K(\b) \vee \Omega K(\c)$.
\end{corollary}

\begin{definition}
Let $\c$ be a Waldhausen category. Recall the arrow category $\Ar(\c)$ of $\c$ consisting of morphisms in $\c$ as objects and commutative squares as morphisms. Let $s$ and $t$ denote the source and target functors $\Ar(\c)\to \c$, respectively. 

A functor $T: \Ar(\c) \to \c$ is a \textit{(mapping) cylinder functor} on $\c$ if it comes equipped with natrual transformations $j_1 :  s \Rightarrow T$, $j_2 : t \Rightarrow T$, and $p: T \Rightarrow t$ such that for any $f: A \to B$, we have the commutative diagram
\[
\begin{tikzcd}
A \arrow[r, "j_1"] \arrow[rd, "f"'] & T(f) \arrow[d, "p"] & B \arrow[l, "j_2"'] \arrow[ld, "="] \\
 & B & 
\end{tikzcd}
. \]
Moreover, $T$ must satisfy the following axioms.
\begin{enumerate}
\item $T$ sends every initial morphism $\ast \to A$ to $A$ for any $A \in \ob \c$.
\item $j_1 \coprod j_2 : A \coprod B \rightarrowtail T(f)$ is a cofibration for any $f: A \to B$.
\item Given a morphism $(a,b) : f \to f'$ in $\Ar(\c)$, if both $a$ and $b$ are w.e. in $\c$, then so is $T(f) \to T(f').$
\item Given a morphism $(a,b) : f \to f'$ in $\Ar(\c)$, if both $a$ and $b$ are cofibrations in $\c$, then so is $T(f) \to T(f')$. Also, the map $A' \coprod_A T(f) \coprod_B B' \to T(f')$ induced by axiom 2 is a cofibration in $\c$.
\item {(Cylinder Axiom)} The map $p: T(f) \to B$ is a w.e. in $\c$.
\end{enumerate}
\begin{definition} Let $T$ be a cylinder functor on $\c$.
\begin{enumerate}
\item We call $T(A \to \ast)$ the \textit{cone} of $A$, denoted by $\cone(A)$.
\item We call $\faktor{\cone(A)}{A}$ the \textit{suspension} of $A$, denoted by $\Sigma A$. 
\end{enumerate}
\end{definition}
\begin{corollary}
The induced suspension map $\Sigma : K(\c) \to K(\c)$ is a homotopy inverse for the $H$-space $K(\c)$.
\end{corollary}
\begin{proof}
Note that axiom 3 gives us a cofiber sequence $A \rightarrowtail \cone(A) \twoheadrightarrow \Sigma A$. Therefore, $1 \rightarrowtail \cone \twoheadrightarrow \Sigma$ is an exact sequence of functors. By the cylinder axiom, we know that $\cone$ is null-homotopic. It follows by the additivity theorem that $\Sigma_{\ast} + 1 =  \cone_{\ast} = \ast$.
\end{proof}
\end{definition}

Let $\c$ be a category with cofibrations. Equip it with two Waldhausen subcategories $v(\c)$ and $w(\c)$ of weak equivalences such that $v(\c) \subset w(\c)$. Assume that $(\c, w)$ admits a cylinder functor. Suppose that $w(\c)$ is saturated and closed under extensions. 
\begin{notation}
Let $\c^w$ denote the Waldhausen subcategory of $(\c, v)$ consisting of any $A$ where $\ast \to A$ is in $w(\c)$.
\end{notation}\reversemarginpar{\marginpar{\small{Are the initial morphisms the only w.e.?}}}

\begin{theorem}[Waldhausen localization theorem]
The sequence $$ K(A^w) \to K(\c, v) \to K(\c, w) $$ is a homotopy fibration sequence.
\end{theorem}
\begin{proof}
Recall that a small bicategory is a bisimplicial set such that each row/column is the nerve of a category. Note that $v_{(-)}w_{(-)}\c$ is a bicategory whose bimorphisms are commutative squares of the form
\[
\begin{tikzcd}
(-) \arrow[r, "w'"] \arrow[d, "v"'] & (-) \arrow[d, "v'"] \\
(-) \arrow[r, "w"] & (-)
\end{tikzcd}.
\]
It turns out that treating $w \c$ as a bicategory with a single vertical morphism proves that $w \c \simeq v_{(-)}w_{(-)}\c$. This gives $wS_n \c \simeq v_{(-)}w_{(-)}S_n\c$ for each $n$.

\medskip

 Now, let $v_{(-)}\text{co}w_{(-)}\c$ denote the subcategory of the above squares where the horizontal maps are also cofibrations. One can show that the inclusion $v_{(-)}\text{co}w_{(-)}\c \subset v_{(-)}w_{(-)}\c$ is a homotopy equivalence. Since each $S_n \c$ inherits a cylinder functor from $\c$, we simplicial bi-subcategory $v_{(-)}\text{co}w_{(-)}S_{\bullet}\c$ such that the inclusion intro $v_{(-)}w_{(-)}S_{\bullet}\c$ is a homotopy equivalence. We have now obtained the following diagram.
\[
\begin{tikzcd}
vS_{\bullet}C^w \arrow[r] & vS_{\bullet}C \arrow[r] \arrow[d] & v_{(-)}\text{co}w_{(-)}S_{\bullet}C \arrow[d, "\simeq"] \\
 & wS_{\bullet}C \arrow[r, "\simeq"] & v_{(-)}w_{(-)}S_{\bullet}C
\end{tikzcd}
\]

\medskip

 It therefore suffices to show that the top row is a fibration.\reversemarginpar{\marginpar{\small{What about the left vertical morphism?}}} You can do this by using the relative $K$-theory space construction. See IV.8.5.3 and V.2.1 (Weibel).
\end{proof}

\begin{definition}
Let $\a$ be an exact category embedded in an abelian category $\b$ and let $\ch^b(\a)$ denote the category of bounded chain complexes in $\a$. One can verify that $\ch^b(\a)$ is Waldhausen where the cofibrations $A_{\bullet} \rightarrowtail B_{\bullet}$ are precisely the degree-wise admissible monomorphisms (i.e., those giving a short exact sequence $A_n \to B_n \to \faktor{B_n}{A_n}$ in $\a$ for each $n$) and the w.e. are precisely the chain maps which are quasi-isomorphisms of complexes in $\ch(\b)$.
\end{definition}

\begin{theorem}[Gillet-Waldhausen]
Let $\a$ be an exact category closed under kernels of surjections. Then the exact inclusion $\a \to \ch^b(\a)$ induces a homotopy equivalence $K(\a) \simeq K\ch^b(\a)$. Hence $$K_i(\a) = K_i \ch^b(\a)$$ for every $i$.
\end{theorem}
\begin{proof}
Apply the localization theorem. See V.2.2 (Weibel).
\end{proof}

\begin{definition}
Let $F: \a \to \b$ be an exact functor between Waldhausen categories. We say that $F$ satisfies the \textit{approximate lifting property} if for any map $b: F(A) \to B$ in $\b$, there is some map $a : A \to A'$ in $\a$ and some w.e. $b' : F(A') \simeq B$ in $\b$ so that
\[
\begin{tikzcd}
F(A') \arrow[r, "\sim", dashed] & B \\
F(A) \arrow[u, "F(a)", dashed] \arrow[ru, "b"] & 
\end{tikzcd}.
\]
commutes. In this way, we can lift to  a w.e.
\end{definition}

\begin{prop}
Let $F: \a \to \b$ be an exact functor between Waldhausen categories such that the following hold.
\begin{enumerate}
\item $F$ satisfies the approximate lifting property.
\item $\a$ admits a cylinder functor.
\item A morphism $f$ in $\a$ is a w.e. iff $F(f)$ is a w.e. in $\b$.
\end{enumerate}
Then $wF : w \a \to w\b$ is a homotopy equivalence.
\end{prop}

\begin{corollary}[Waldhausen approximation theorem]
With the same conditions as before, we have $$K(\a) \simeq K(\b).$$
\end{corollary}
\begin{proof}
One can show that each functor $S_n \a \to S_n \b$ is exact and also has the approximate lifting property.  The previous proposition thus gives degree-wise homotopy equivalence between the bisimplicial map $wS_{\bullet}\a \to wS_{\bullet} \b$, which is enough. 
\end{proof}

\begin{definition}
Let $\a$ be an abelian category $\ch(\a)$ denote the category of chain complexes over $\a$. We say that a complex $C_{\bullet}$ is \textit{homologically bounded} if only finitely many $H_i(C_j)$ are nonzero. Let $\ch_{\pm}^{hb}$ denote the subcategory of bounded below (respectively, bounded above) complexes. 
\end{definition}

\begin{exmp}
Let $\a$ be an abelian category. By homology theory, we have that $\ch^b(\a) \subset \ch_{-}^{hb}(\a)$ and $\ch_+^{hb}(\a)\subset \ch^{hb}(\a)$ have the approximate lifting property. We also have that $\ch^b(\a) \subset \ch_+^{hb}(\a)$ and $\ch_+^{hb}(\a)\subset \ch^{hb}(\a)$ satisfy the dual of the approximate lifting property. Thus, we can apply the approximation theorem and Gillet-Waldhausen to see that $$  K(\a) \simeq K\ch^b(\a) \simeq  K\ch_{-}^{hb}\simeq K\ch_{+}^{hb}(\a) \simeq K \ch^{hb}(\a).$$
\end{exmp}

\begin{definition}
 A \textit{symmetric spectrum} $\X$ in topological spaces in a sequence of based $\Sigma_n$-spaces $(X_n)$ endowed with structure maps $\sigma : X_n \land S^1 \to X_{n+1}$ such that $\sigma^k : X_n \land S^k \to X_{n+k}$ is $(\Sigma_{n}\times \Sigma_{k})$-equivariant for any $n,k\geq 0$, where $S^k \coloneqq \underbrace{S^1 \land \cdots \land S^1}_{k\text{-times}}$. A map $\mathbf{f} : \x \to \mathbf{Y}$ of symmetric spectra is a sequence $\left(f_n : X_n \to Y_n\right)$ of based $\Sigma_n$-equivariant maps such that for each $n\geq 0$, the square
\[
\begin{tikzcd}
X_n \land S^1 \arrow[d, "\sigma"'] \arrow[r, "f_n \land \id"] & Y_n \land S^1 \arrow[d, "\sigma"] \\
X_{n+1} \arrow[r, "f_{n+1}"] & Y_{n+1}
\end{tikzcd}
\]
commutes. Let $\Sp^{\Sigma}$ denote the category of symmetric spectra in topological spaces.
\end{definition}

\begin{definition}
Let $(\c, w\c)$ be a Waldhausen category. The \textit{external $n$-fold $S_{\bullet}$-construction} on $\c$ is the $n$-multisimplicial Waldhausen category $$\left(S_{\bullet}\cdots S_{\bullet}\c, wS_{\bullet} \cdots S_{\bullet} \c\right).$$ It multidegree $(q_1, \ldots, q_n)$, it has as objects the diagrams  $X: \Ar[q_1] \times \cdots \times \Ar[q_n] \to \c$ such that
\begin{enumerate}
\item $X((i_1, j_1), \ldots, (i_n, j_n)) = \ast$ if $i_k = j_k$ for some $1\leq k \leq n$.
\item $X(\ldots, (i_t, j_t), \ldots) \rightarrowtail X(\ldots, (i_t, k_t), \ldots) \twoheadrightarrow X(\ldots, (j_t, k_t), \ldots)$ is a cofiber sequence  in the $(n-1)$-fold iterated $S_{\bullet}$-construction for any $i_t \leq j_t \leq k_t$ in $[q_t]$.
\end{enumerate}
\end{definition}

\begin{definition}
Let $\left(\c, w\c\right)$ be a Waldhausen category. The \textit{internal $n$-fold $S_{\bullet}$-construction} on $\c$ is the simplicial Waldhausen category $$(S_{\bullet}^{(n)}\c, wS_{\bullet}^{(n)} \c).$$ It has as $q$-simplices the functor categories $(S_q \cdots S_q \c, wS_q \cdots S_q \c)$ whose objects are the $\left(\Ar[q]\right)^n$-shaped diagrams  $X: \left(\Ar[q]\right)^n\to \c$ such that
\begin{enumerate}
\item $X((i_1, j_1), \ldots, (i_n, j_n)) = \ast$ if $i_k = j_k$ for some $1\leq k \leq n$.
\item $X(\ldots, (i_t, j_t), \ldots) \rightarrowtail X(\ldots, (i_t, k_t), \ldots) \twoheadrightarrow X(\ldots, (j_t, k_t), \ldots)$ is a cofiber sequence  in the $(n-1)$-fold iterated $S_{\bullet}$-construction for any $i_t \leq j_t \leq k_t$ in $[q]$.
\end{enumerate}
Note that $\Sigma_n$ acts on $S_{\bullet}^{(n)}\c$ by $(\pi \cdot X)(\ldots, (i_t, j_t), \ldots) = X(\ldots, (i_{\pi^{{-}1}(t)}, j_{\pi^{{-}1}(t)}), \ldots)$.
\end{definition}

\begin{definition}
The \textit{(symmetric) algebraic $K$-theory  spectrum $\mathbf{K}(\c, w)$} of a small Waldhausen category $\left(\c, w\c\right)$ has $n$-th space $K(\c, w)_n = \lvert{wS_{\bullet}^{(n)}\c}\rvert$ based at $\ast$. There is a $\Sigma_n$-action on $K(\c, w)_n$ induced by permuting the order of the internal $S_{\bullet}$-constructions. Moreover, we have $$\lvert{wS_{\bullet}^{(n)}\c}\rvert \land S^1 \cong \lvert{wS_{\bullet}^{(n)}S_{\bullet}\c}\rvert^{(1)} \subset \lvert{wS_{\bullet}^{(n)}S_{\bullet}\c}\rvert \cong \lvert{wS_{\bullet}^{(n+1)}\c}\rvert$$, where $^{(1)}$ denotes the $1$-skeleton with respect to the last simplicial direction. This determines the structure map $\sigma$. Then $\sigma^k$ is $\left(\Sigma_{n} \times \Sigma_{k}\right)$-invariant.
\end{definition}


\begin{theorem}
For any $i\geq 0$, we have that $K_i(\c, w) = \pi_{i+1}K(\c, w)_1 \cong \pi_i \mathbf{K}(\c, w)$.
\end{theorem}
\begin{proof}
See Lemma 8.7.4 (Rognes).
\end{proof}


In this way, we encode our algebraic $K$-theory in an infinite loop space.


\end{document}
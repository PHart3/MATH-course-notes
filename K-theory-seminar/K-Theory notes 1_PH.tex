\documentclass[10pt,letterpaper,cm]{nupset}
\usepackage[margin=1in]{geometry}
\usepackage{graphicx}
 \usepackage{enumitem}
 \usepackage{stmaryrd}
 \usepackage{bm}
\usepackage{amsfonts}
\usepackage{amssymb}
\usepackage{pgfplots}
\usepackage{amsmath,amsthm}
\usepackage{lmodern}
\usepackage{tikz-cd}
\usepackage{faktor}
\usepackage{xfrac}
\usepackage{mathtools}
\usepackage{bm}
\usepackage{ dsfont }
\usepackage{mathrsfs}
\theoremstyle{definition}
\newtheorem*{definition}{Definition}
\newtheorem{exmp}{Example}
\newtheorem{nonexmp}{non-Example}
\newtheorem{remark}{Remark}
\newtheorem{theorem}{Theorem}
\newtheorem{corollary}{Corollary}
\newtheorem{lemma}{Lemma}
\newtheorem{exercise}{Exercise}
\makeatletter
\renewcommand*\env@matrix[1][*\c@MaxMatrixCols c]{%
  \hskip -\arraycolsep
  \let\@ifnextchar\new@ifnextchar
  \array{#1}}
\makeatother
\pgfplotsset{unit circle/.style={width=4cm,height=4cm,axis lines=middle,xtick=\empty,ytick=\empty,axis equal,enlargelimits,xmax=1,ymax=1,xmin=-1,ymin=-1,domain=0:pi/2}}
\DeclareMathOperator{\Ima}{Im}
\newcommand{\A}{\mathcal A}
\newcommand{\C}{\mathbb C}
\newcommand{\E}{\vec E}
\newcommand{\CP}{\mathbb CP}
\newcommand{\F}{\mathcal F}
\newcommand{\G}{\vec G}
\renewcommand{\H}{\vec H}
\newcommand{\HP}{\mathbb HP}
\newcommand{\K}{\mathbb K}
\renewcommand{\L}{\mathcal L}
\newcommand{\M}{\mathbb M}
\newcommand{\N}{\mathbb N}
\renewcommand{\O}{\mathbf O}
\newcommand{\OP}{\mathbb OP}
\renewcommand{\P}{\mathcal P}
\newcommand{\Q}{\mathbb Q}
\newcommand{\I}{\mathbb I}
\newcommand{\R}{\mathbb R}
\newcommand{\RP}{\mathbb RP}
\renewcommand{\S}{\mathbf S}
\newcommand{\T}{\mathbf T}
\newcommand{\X}{\mathbf X}
\newcommand{\Z}{\mathbb Z}
\newcommand{\B}{\mathcal{B}}
\newcommand{\1}{\mathbf{1}}
\newcommand{\ds}{\displaystyle}
\newcommand{\ran}{\right>}
\newcommand{\lan}{\left<}
\newcommand{\bmat}[1]{\begin{bmatrix} #1 \end{bmatrix}}
\renewcommand{\a}{\vec{a}}
\renewcommand{\b}{\vec b}

\renewcommand{\c}{\mathscr{C}}
\renewcommand{\d}{\mathscr{D}}
\newcommand{\e}{\mathscr{E}}
\newcommand{\y}{\mathscr{Y}}

\newcommand{\h}{\vec h}
\newcommand{\f}{\vec f}
\newcommand{\g}{\vec g}
\renewcommand{\i}{\vec i}
\renewcommand{\j}{\vec j}
\renewcommand{\k}{\vec k}
\newcommand{\n}{\vec n}
\newcommand{\p}{\vec p}
\newcommand{\q}{\vec q}
\renewcommand{\r}{\vec r}
\newcommand{\s}{\vec s}
\renewcommand{\t}{\vec t}
\renewcommand{\u}{\vec u}
\renewcommand{\v}{\vec v}
\newcommand{\w}{\vec w}
\newcommand{\x}{\vec x}
\newcommand{\z}{\vec z}
\newcommand{\0}{\vec 0}
\DeclareMathOperator*{\Span}{span}
\DeclareMathOperator*{\GL}{GL}
\DeclareMathOperator{\rng}{range}
\DeclareMathOperator{\gemu}{gemu}
\DeclareMathOperator{\almu}{almu}
\newcommand{\Char}{\mathsf{char}}
\DeclareMathOperator{\id}{Id}
\DeclareMathOperator{\im}{Im}
\DeclareMathOperator{\graph}{Graph}
\DeclareMathOperator{\gal}{Gal}
\DeclareMathOperator{\tr}{Tr}
\DeclareMathOperator{\norm}{N}
\DeclareMathOperator{\aut}{Aut}
\DeclareMathOperator{\Int}{Int}
\DeclareMathOperator{\ext}{Ext}
\DeclareMathOperator{\stab}{Stab}
\DeclareMathOperator{\orb}{Orb}
\DeclareMathOperator{\inn}{Inn}
\DeclareMathOperator{\out}{Out}
\DeclareMathOperator{\op}{op}
\DeclareMathOperator{\fix}{Fix}
\DeclareMathOperator{\ab}{ab}
\DeclareMathOperator{\sgn}{sgn}
\DeclareMathOperator{\syl}{syl}
\DeclareMathOperator{\Syl}{Syl}
\DeclareMathOperator{\ob}{ob}
\DeclareMathOperator{\mor}{mor}
\DeclareMathOperator{\ar}{Ar}
\DeclareMathOperator{\ZFC}{ZFC}

% info for header block in upper right hand corner
\name{Perry Hart}
\class{Homotopy and K-theory seminar}
\assignment{Talk \#3}
\duedate{September 26, 2018}

\begin{document}

\begin{abstract}
More basic category theory. The main sources for these notes are nLab, Rognes, Ch. 3, and Peter Johnstone's Part III lecture notes (Michaelmas 2015), Ch. 1.
\end{abstract}

\begin{definition}
Let $\c$ and $\d$ be categories and $F,G: \c \to \d$ be functors.  A \textit{natural transformation} $\phi :F \Rightarrow G$  is a function $A \mapsto f_A$ from $\ob \c$ to $\mor \d$ such that $f_A : F(A) \to G(A)$ and the following diagram commutes for any morphism $f: A \to B$.

\begin{center}
\begin{tikzcd}[row sep=large, column sep = large]
FA \arrow[r, "Ff"] \arrow[d, "f_A", swap]
& FB \arrow[d, "f_B"] \\
GA \arrow[r, "Gf"]
& GB
\end{tikzcd}
\end{center}
In symbols, this may be written as $f_Bf_{\ast} = f_{\ast}f_A$, where $f_A$ and $f_B$ are called the \textit{components} of $\phi$.
\end{definition}

\begin{remark}
If every $f_A$ is an isomorphism, then the $(f_A)^{-1}$ define a natural transformation between the same two functors.
\end{remark}

\begin{definition}
Let $F, G, H: \c \to \d$ be functors. The \textit{identity natural transformation} $\id_F : F \Rightarrow F$ is given by $A \mapsto \id_{F(A)}$. Moreover, given natural transformations $\phi: F \to G$ and $\psi: G \to H$, define the \textit{composite natural transformation} $\psi \circ \phi$ by $A \mapsto (\psi \circ \phi)_A := \psi_{A} \circ \phi_{A}$.
\end{definition}


\begin{definition}
If each $f_A$ is an isomorphism, then we call $\phi: F \cong G$ a \textit{natural isomorphism}.
\end{definition}

\begin{remark}
If $\d$ is a groupoid, then $\phi$ must be a natural isomorphism.
\end{remark}

\begin{lemma}
A natural transformation $\phi: F \Rightarrow G$ is a natural isomorphism iff it has an inverse $\phi^{-1} : G \Rightarrow F$.
\end{lemma}
\begin{proof}
This follows from Remark 1 and the definition of composite natural transformation.
\end{proof}

\begin{exmp}
Let $R$ and $S$ be commutative rings. Any ring homomorphism $f: R \to S$ induces a ring homomorphism $\GL_n(f): \GL_n(R) \to \GL_n(S)$ which satisfies $f(\det(A)) = \det(\GL_n(f)(A))$. Viewing $\GL_n$ and $R \mapsto R^{\ast}$ as functors from $\mathbf{Rng}$ to $\mathbf{Grp}$ and $\det_R : \GL_n(R) \to R^{\ast}$ as a morphism in $\mathbf{Grp}$, we see that $\det_R$ defines a natural transformation $\phi : \GL_n \Rightarrow f^{\ast}$, where $f^{\ast}$ denotes $f\restriction_{R^{\ast}} R^{\ast} \to S^{\ast}$.  

\begin{center}
\begin{tikzcd}[row sep=large, column sep = large]
\GL_n(R) \arrow[r, "\GL_n(f)"] \arrow[d, "\det_R", swap]
& \GL_n(S) \arrow[d, "\det_S"] \\
R^{\ast} \arrow[r, "f^{\ast}"]
& S^{\ast}
\end{tikzcd}
\end{center}
\end{exmp}

\begin{exmp}
Recall the power set functor $P: \mathbf{Set} \to \mathbf{Set}$ given by $A \mapsto P(A)$ and $Pg(S) = g(S)$ where $g: A \to B$ is a function and $S\subset A$. Then the function $f_A: A \to P(A)$ given by $a \mapsto \{a\}$ defines a natural transformation $\phi: \id_{\mathbf{Set}} \Rightarrow P$.
\end{exmp}

\begin{exmp}
Set $\c = \d = \mathbf{Grp}$, $F= \id_{\c}$, and $G$ equal to the abelianization functor. Then given a group $H$, the homomorphism $f: H \to H^{\ab}$ defines a natural transformation $\phi: F \Rightarrow G$.
\end{exmp}

\begin{exmp}
Consider the preorders $(P, \leq)$ and $(Q, \leq)$ as small categories where functors $F, G: P \to Q$ are order-preserving functions. Then there is a unique natural transformation $\phi: F \Rightarrow G$ iff $F(x) \leq G(x)$ for every $x\in P$.
\end{exmp}

\begin{exmp}
The inversion isomorphism from a group $G$ to $G^{\op}$ defines a natural transformation $\phi: \id_{\mathbf{Grp}} \Rightarrow (^{\op}: \mathbf{Grp} \to \mathbf{Grp})$. In other words, $G$ is naturally isomorphic to $G^{\op}$.
\end{exmp}

\begin{definition}
Let $\c$ and $\d$ be categories with $\c$ small. The $\textit{functor category}$ $\mathbf{Fun}(\c, \d):= \d^\c$ has functors $F: \c \to \d$ as objects and natural transformations as morphisms. 
\end{definition}

\begin{remark}
Given functors $F, G: \c \to \d$, why is the class of natural transformation $\phi: F \Rightarrow G$ necessarily a set?
A $G$-Universe models $\ZFC$, in particular $\mathsf{Replacement}$.
\end{remark}

\begin{definition}
Given a category $\c$, the \textit{arrow category} $\ar(\c)$ of $\c$ has as objects morphisms $f: X_0 \to X_1$ in $\c$ and as morphisms $M: (f: X_0 \to X_1) \to (g: Y_0 \to Y_1)$ the pairs $M=(M_0, M_1)$ of morphisms $M_0 : X_0 \to Y_0$ and $M_1 : X_1 \to Y_1$ such that the following commutes. 

\begin{center}
\begin{tikzcd}[row sep=large, column sep = large]
X_0 \arrow[r, "f"] \arrow[d, "M_0", swap]
& X_1 \arrow[d, "M_1"] \\
Y_0 \arrow[r, "g"]
& Y_1
\end{tikzcd}
\end{center}

\begin{remark}$\ar(\c) \cong \mathbf{Fun}([1], \c)$.
\end{remark}
\end{definition}

\begin{lemma}
$\mathbf{Fun}(\c \times \d, \e) \cong \mathbf{Fun}(\c, \mathbf{Fun}( \d, \e))$ via currying. 
\end{lemma}

\begin{definition}
A functor $F: \c \to \d$ is an $\textit{equivalence}$ if there is a functor $G: \d \to \c$ such that $F \circ G \cong \id_{\c}$ and $G \circ F \cong \id_{\d}$. In this case, we say that $F$ and $G$ are \textit{equivalent categories}. Moreover, we say that a property of $\c$ is \textit{categorical} if it is invariant under such equivalence.
\end{definition}

\begin{exmp}
Let $k$ be a field. Let the category $\mathbf{Mat}_k$  have natural numbers as objects and morphisms $n \to p$ given by $p \times n$ matrices over $k$. Let $\mathbf{fdMod}$ denote the category of finite-dimensional vector spaces with linear maps as morphisms. These two categories are equivalent. Send nat $n$ to $k^n$ in one direction and the space $V$ to $\dim V$ in the other direction.    
\end{exmp}

\begin{definition}
A functor $F : \c \to \d$ is \textit{essentially surjective} if for each object $Z$ of  $\d$, there is some object $Y$ of $\c$ such that $F(Y) \cong Z$.
\end{definition}

\begin{theorem}
A functor is an equivalence iff it is full, faithful, and essentially surjective. 
\end{theorem}
\begin{proof}
See Rognes, Theorem 3.2.10.
\end{proof}

\begin{definition}
A \textit{skeleton} of $\c$ is a full subcategory $\c' \subset \c$ such that each element of $\ob \c$ is isomorphic to exactly one element of $\ob \c'$.
\end{definition}

\begin{lemma}
With notation as before, $\c'$ and $\c$ are equivalent categories via the inclusion functor.
\end{lemma}
\begin{proof}
Apply Theorem 1.
\end{proof}

\begin{lemma}
Any two skeleta $\c', \c'' \subset \c$ are isomorphic.
\end{lemma}
\begin{proof}
Define $F: \c' \to \c''$ by $F(X) =Y$ where $h_X: X \cong Y$ and $F(f) = h_Y \circ f \circ (h_X)^{-1}$ for $f\in \c(X, Y)$. 
To get $F^{-1}$, similarly define $G: \c'' \to \c'$ by choosing $(h_X)^{-1}$.
\end{proof}

\begin{remark}
The previous two lemmas are equivalent to the axiom of choice, as is the statement that every category admits a skeleton.
\end{remark}

\begin{definition}
Fix $X \in \ob \c$. Define the functor $\y^X : \c \to \mathbf{Set}$ by $Y \mapsto \c(X, Y)$  and mapping each morphism $g: Y \to Z$ to $g_{\ast} : \c(X, Y) \to \c(X, Z)$ given by $f \mapsto gf$. We call $\c(X, -):=\y^X$ the set-valued functor \textit{corepresented} by $X$ in $\c$.
\end{definition}

\begin{definition}
Fix $Z \in \ob \c$. Define the contravariant functor $\y_Z: \c^{\op} \to \mathbf{Set}$ by $Y \mapsto \c(Y, Z)$ and mapping each morphism $f: X \to Y$ in $\c$ to $f^{\ast} : \c(Y, Z) \to \c(X, Z)$ given by $g\mapsto gf$. We call $\c(-, Z):= \y^Z$ the set-valued functor \textit{represented} by $Z$ in $\c$.
\end{definition}

\begin{definition}
A functor $F: \c \times \c' \to \d$ is also called a \textit{bifunctor}.
\end{definition}

\begin{exmp}
Let $\c$ be a category. Define $\c(-, -): \c^{\op} \times \c \to \mathbf{Set}$ by $(X, X') \to \c(X, X')$ and mapping each morphism $(f, f') : (X, X') \to (Y, Y')$ to $\c(f, f') : \c(X, X') \to \c(Y, Y')$ given by $g \mapsto f'gf$.
\end{exmp}

\begin{definition}
This is due to Dan Kan. Let $\c$ and $\d$ be categories and $F : \c \to \d$ and $G: \d \to \c$ be functors. Consider the set-valued bifunctors $ \d(F(-), -), \c(-, G(-)): \c^{\op} \times \d \to \mathbf{Set}$. An $\textit{adjunction}$ between $F$ and $G$ is a natural isomorphism $\phi : \d(F(-), -) \Rightarrow \c(-, G(-))$. If such $\phi$ exists, then we say that $(F, G)$ is an \textit{adjoint pair} or functors. We also call $F$ the \textit{left adjoint} to $G$ and $G$ the \textit{right adjoint} to $F$. 

\begin{remark}
For each $c: X' \to X$ and $d: Y \to Y'$, the following commutes.

\begin{center}
\begin{tikzcd}[row sep=large, column sep = large]
\d(F(X), Y) \arrow[r, "\phi_{X, Y}"] \arrow[d, "c^{\ast}d_{\ast}", swap]
& \c(X, G(Y)) \arrow[d, "c^{\ast}d_{\ast}"] \\
\d(F(X'), Y')\arrow[r, "\phi_{X', Y'}"]
& \c(X', G(Y'))  
\end{tikzcd}
\end{center}

\end{remark}

\end{definition}

\begin{exmp}
The forgetful functor $U: \mathbf{Grp} \to \mathbf{Set}$ admits a left adjoint $F: \mathbf{Set} \to \mathbf{Grp}$ which maps a set to the free group generated by $A$. The adjunction is the natural bijection $\mathbf{Set}(A, U(G)) \cong \mathbf{Grp}(F(A), G)$.
\end{exmp}

\begin{exmp}
Let $R$ be a ring. The forgetful functor $U: R-\mathbf{Mod}\to \mathbf{Set}$ admits a left adjoint $R(-)$ sending a set $S$ to $\bigoplus_{s\in S} R$, the free $R$-module generated by $S$. The adjunction is the natural bijection $\mathbf{Set}(S, U(M)) \cong R-\mathbf{Mod}(R(S), M)$.
\end{exmp}

\begin{remark}
Rognes says that $U$ does not admit a right adjoint in either of the previous two examples.
\end{remark}

\begin{exmp}
The forgetful functor $U: \mathbf{Top} \to \mathbf{Set}$ has left adjoint that sends a set to the same set equipped with the discrete topology.  It also has a right adjoint via the functor sending a set to the same set equipped with the indiscrete topology.
\end{exmp}

\begin{exmp}
Let $\mathbf{CMon}$ be the category of commutative monoids. Given $M\in \ob \mathbf{CMon}$, we can construct the completion, or Grothendieck group, $G(M)$ on $M\times M$ as follows. Define addition on $M \times M$ component-wise and say that $(m_1, m_2) \sim (n_1, n_2)$ if $m_1 + m_2 + k = m_2 +n_1 +k$ for some $k\in M$. Set $G(M)$ as $(\faktor{M\times M}{\sim}, +)$.
\\ \\
Then $G: \mathbf{CMon} \to \mathbf{Ab}$ is a functor. This is left adjoint to the forgetful functor $U: \mathbf{Ab} \to \mathbf{CMon}$.
\end{exmp}

\begin{remark}
Read Rognes, Definition 3.4.8, where he constructs the group completion $K(M)$ of non-commutative monoids $M$. It turns out that $K(M)$ is realized as the fundamental group of an important classifying space. 
\end{remark}

\begin{definition}
A subcategory $\c \subset \d$ is \textit{reflective} if the inclusion functor is a right adjoint and is \textit{coreflective} if the inclusion functor is a left adjoint. 
\end{definition}

\begin{exmp}
$\mathbf{Ab}\subset \mathbf{CMond}$ is reflective by Example 11. 
\end{exmp}

\begin{exmp}
$\mathbf{Ab}\subset \mathbf{Grp}$ is reflective. 
\end{exmp}

\begin{exmp}
Let $\mathbf{Ab}_T\subset \mathbf{Ab}$ denote the category of torsion groups. This is coreflective via the functor sending an abelian group to its torsion subgroup  because any homomorphism $f: A \to B$ where $A$ is torsion has $f(A) \subset B_T$.
\end{exmp}

\begin{definition}
Given an adjunction $\phi: \d(F(-), -) \Rightarrow \c(-, G(-))$, define the \textit{unit morphism} $$\eta_X =\phi_{X, F(X)}(\id_{F(X)})$$ and the \textit{counit morphism} $$\epsilon_Y =\phi_{G(Y), Y}^{-1}(\id_{G(Y)}).$$
\end{definition}

\begin{lemma}
Given an adjunction $\phi$, the unit morphisms $\eta_X$ define a natural transformation $\eta: \id_{\c} \Rightarrow GF$ and the counit morphisms $\eta_Y$ define a natural transformation $\epsilon: FG \Rightarrow \id_{\d}$.
\end{lemma}

\end{document}
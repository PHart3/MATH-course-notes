\documentclass[10pt,letterpaper,cm]{nupset}
\usepackage[margin=1.2in]{geometry}
\usepackage{graphicx}
 \usepackage{enumitem}
 \usepackage{stmaryrd}
 \usepackage{bm}
\usepackage{amsfonts}
\usepackage{amssymb}
\usepackage{pgfplots}
\usepackage{amsmath,amsthm}
\usepackage{lmodern}
\usepackage{tikz-cd}
\usepackage{faktor}
\usepackage{xfrac}
\usepackage{mathtools}
\usepackage{bm}
\usepackage{ dsfont }
\usepackage{mathrsfs}
\usepackage{hyperref}
\hypersetup{colorlinks=true, linkcolor=red,          % color of internal links (change box color with linkbordercolor)
    citecolor=green,        % color of links to bibliography
    filecolor=magenta,      % color of file links
    urlcolor=cyan           }

\usepackage{thmtools}
\usepackage[capitalise]{cleveref} 
    
\theoremstyle{definition}
\newtheorem{definition}{Definition}
\newtheorem{exmp}[definition]{Example}
\newtheorem{non-exmp}[definition]{Non-example}
\newtheorem{note}[definition]{Note}

\theoremstyle{theorem}
\newtheorem{theorem}[definition]{Theorem}
\newtheorem{lemma}[definition]{Lemma}
\newtheorem{corollary}[definition]{Corollary}
\newtheorem{prop}[definition]{Proposition}
\newtheorem{conj}[definition]{Conjecture}
\newtheorem*{claim}{Claim}
\newtheorem{exercise}[definition]{Exercise}

\theoremstyle{remark}
\newtheorem{remark}[definition]{Remark}
\newtheorem*{todo}{To do}
\newtheorem*{conv}{Convention}
\newtheorem*{aside}{Aside}
\newtheorem*{notation}{Notation}
\newtheorem*{term}{Terminology}
\newtheorem*{background}{Background}
\newtheorem*{further}{Further reading}
\newtheorem*{sources}{Sources}

\makeatletter
\def\th@plain{%
  \thm@notefont{}% same as heading font
  \itshape % body font
}
\def\th@definition{%
  \thm@notefont{}% same as heading font
  \normalfont % body font
}
\makeatother

\makeatletter
\renewcommand*\env@matrix[1][*\c@MaxMatrixCols c]{%
  \hskip -\arraycolsep
  \let\@ifnextchar\new@ifnextchar
  \array{#1}}
\makeatother
\pgfplotsset{unit circle/.style={width=4cm,height=4cm,axis lines=middle,xtick=\empty,ytick=\empty,axis equal,enlargelimits,xmax=1,ymax=1,xmin=-1,ymin=-1,domain=0:pi/2}}
\DeclareMathOperator{\Ima}{Im}
\newcommand{\A}{\mathcal A}
\newcommand{\C}{\mathbb C}
\newcommand{\E}{\vec E}
\newcommand{\CP}{\mathbb CP}
\newcommand{\F}{\mathbb F}
\newcommand{\G}{\vec G}
\renewcommand{\H}{\mathbb H}
\newcommand{\HP}{\mathbb HP}
\newcommand{\K}{\mathbb K}
\renewcommand{\L}{\mathcal L}
\newcommand{\M}{\mathbb M}
\newcommand{\N}{\mathbb N}
\renewcommand{\O}{\mathbf O}
\newcommand{\OP}{\mathbb OP}
\renewcommand{\P}{\mathbf P}
\newcommand{\Q}{\mathbb Q}
\newcommand{\I}{\mathbb I}
\newcommand{\R}{\mathbb R}
\newcommand{\RP}{\mathbb RP}
\renewcommand{\S}{\mathbf S}
\newcommand{\T}{\mathbf T}
\newcommand{\X}{\mathbf X}
\newcommand{\Z}{\mathbb Z}
\newcommand{\B}{\mathcal{B}}
\newcommand{\1}{\mathbf{1}}
\newcommand{\ds}{\displaystyle}
\newcommand{\ran}{\right>}
\newcommand{\lan}{\left<}
\newcommand{\bmat}[1]{\begin{bmatrix} #1 \end{bmatrix}}

\renewcommand{\a}{\mathscr{A}}
\renewcommand{\b}{\mathscr{B}}
\renewcommand{\c}{\mathscr{C}}
\renewcommand{\d}{\mathscr{D}}
\newcommand{\e}{\mathscr{E}}
\newcommand{\y}{\mathscr{Y}}
\renewcommand{\j}{\mathscr{J}}

\newcommand{\h}{\vec h}
\newcommand{\f}{\vec f}
\newcommand{\g}{\vec g}
\renewcommand{\i}{\vec i}
\renewcommand{\k}{\vec k}
\newcommand{\n}{\vec n}
\newcommand{\p}{\vec p}
\newcommand{\q}{\vec q}
\renewcommand{\r}{\vec r}
\newcommand{\s}{\vec s}
\renewcommand{\t}{\vec t}
\renewcommand{\u}{\vec u}
\renewcommand{\v}{\vec v}
\newcommand{\w}{\vec w}
\newcommand{\x}{\vec x}
\newcommand{\z}{\vec z}
\newcommand{\0}{\vec 0}
\DeclareMathOperator*{\Span}{span}
\DeclareMathOperator*{\GL}{GL}
\DeclareMathOperator*{\SL}{SL}
\DeclareMathOperator*{\SO}{SO}
\DeclareMathOperator*{\SU}{SU}
\DeclareMathOperator{\rng}{range}
\DeclareMathOperator{\gemu}{gemu}
\DeclareMathOperator{\almu}{almu}
\newcommand{\Char}{\mathsf{char}}
\DeclareMathOperator{\id}{Id}
\DeclareMathOperator{\im}{im}
\DeclareMathOperator{\graph}{Graph}
\DeclareMathOperator{\gal}{Gal}
\DeclareMathOperator{\tr}{Tr}
\DeclareMathOperator{\norm}{N}
\DeclareMathOperator{\aut}{Aut}
\DeclareMathOperator{\Int}{Int}
\DeclareMathOperator{\co}{\mathtt{co}}
\DeclareMathOperator{\ext}{Ext}
\DeclareMathOperator{\stab}{Stab}
\DeclareMathOperator{\orb}{Orb}
\DeclareMathOperator{\inn}{Inn}
\DeclareMathOperator{\out}{Out}
\DeclareMathOperator{\op}{op}
\DeclareMathOperator{\fix}{Fix}
\DeclareMathOperator{\ab}{ab}
\DeclareMathOperator{\sgn}{sgn}
\DeclareMathOperator{\syl}{syl}
\DeclareMathOperator{\Syl}{Syl}
\DeclareMathOperator{\ob}{ob}
\DeclareMathOperator{\mor}{mor}
\DeclareMathOperator{\iso}{iso}
\DeclareMathOperator{\ar}{Ar}
\DeclareMathOperator{\red}{red}
\DeclareMathOperator{\colim}{colim}
\DeclareMathOperator{\ZFC}{ZFC}
\DeclareMathOperator{\set}{\mathbf{Set}}
\DeclareMathOperator{\Ab}{\mathbf{Ab}}
\DeclareMathOperator{\Cmon}{\mathbf{CMon}}
\DeclareMathOperator{\spec}{Spec}
\DeclareMathOperator{\rank}{rank}
\DeclareMathOperator{\rk}{rk}
\DeclareMathOperator{\diag}{diag}
\DeclareMathOperator{\Ar}{Ar}
\DeclareMathOperator{\Mod}{\mathbf Mod}

\linespread{1.3}

% info for header block in upper right hand corner
\name{Perry Hart}
\class{$K$-theory reading seminar}
\assignment{UPenn}
\duedate{November 12, 2018}

%Talk #15

\begin{document}

\begin{abstract}
We begin higher Waldhausen $K$-theory. The main sources for this talk are the following.
\begin{itemize}
\item $n$Lab.
\item Charles Weibel's \textit{The $K$-book: an introduction to algebraic $K$-theory},  Ch. IV.8.
\item John Rognes's \textit{Lecture Notes on Algebraic $K$-Theory}, Ch. 8.
\end{itemize}
For the original development, see Friedhelm Waldhausen's \textit{Algebraic K-theory of spaces} (1985). 
\end{abstract}

\smallskip

Our goal is to construct the $K$-theory $K(\c)$ of a Waldhausen category $\c$ as a based loop space $\Omega Y$ endowed with a loop completion map $ \iota : \left\lvert{w{\c}}\right\rvert \to K(\c)$ where $w{\c}$ denotes the subcategory of weak equivalences. This will produce a function $\ob \c \to \left\lvert{w{\c}}\right\rvert \to \Omega Y$. Further, we'll require of $K(\c)$ certain limit and coherence properties, thereby making $K(\c)$ the underlying infinite loop space of a spectrum $\mathbf{K}(\c)$, called the \textit{algebraic $K$-theory spectrum} of $\c$.

\medskip

\begin{definition}

 Let $\c$ be a category with cofibrations. Let the \textit{extension category} $S_2\c$ have as objects the cofiber sequences in $\left(\c, \co{\c}\right)$ and as morphisms the triples $\left(f', f, f''\right)$ of maps in $\c$ such that
\[ \label{eqn:trips}
\begin{tikzcd}
X' \arrow[r, tail] \arrow[d, "f'"] & X \arrow[r, two heads] \arrow[d, "f"] & X'' \arrow[d, "f''"] \\
Y' \arrow[r, tail] & Y \arrow[r, two heads] & Y''
\end{tikzcd}
\tag{$\star$}
\] 
commutes. This is pointed at $\ast \rightarrowtail \ast \twoheadrightarrow \ast$.
\end{definition}

\begin{definition}
Suppose that $\c$ is Waldhausen. Consider any  triple $\left(f', f, f''\right)$ as in \eqref{eqn:trips} with the property that whenever $f'$ and $f''$ are weak equivalences, then so is $f$. In this case, we say $\c$ is \textit{extensional} or \textit{closed under extensions}.
\end{definition}


Say that the morphism $\left(f', f, f''\right)$ is a cofibration if $f'$, $f''$, and $Y' \cup_{X'} X \to Y$ are cofibrations in $\c$. Say that  the same triple is a weak equivalence if $f'$, $f$, and $f''$ are weak equivalences in $\c$. This makes $S_2 \c$ into a Waldhausen category.


\begin{definition}
Let $q\geq 0$. Let the \textit{arrow category $\Ar[q]$ on $\left[q\right]$} have as objects ordered pairs $\left(i, j\right)$ with $i\leq j \leq q$ and as morphisms commutative diagrams of the form
\[
\begin{tikzcd}
i \arrow[d, "\leq"'] \arrow[r, "\leq"] & j \arrow[d, "\leq"] \\
i' \arrow[r, "\leq"'] & j'
\end{tikzcd}.
\] 
\end{definition}

We view $\left[q\right]$ as a full subcategory of $\Ar[q]$ via the embedding $\left[q\right] \overset{k \mapsto \left(0, k\right)}{\xrightarrow{\hspace*{1cm}}} \Ar[q]$.

\begin{note} $ $
\begin{enumerate}
\item Any triple $i\leq j \leq k$ determines the morphisms $\left(i, j\right) \to \left(i, k\right)$ and $\left(i, k\right) \to \left(j, k\right)$. Conversely, any morphism in the arrow category is a composite of such triples.
\item $\Ar[q] \cong \mathbf{Fun}([1], [q])$ with each pair $\left(i, j\right)$ identified with the functor satisfying $0 \mapsto i$ and $1 \mapsto j$.
\end{enumerate}
\end{note}

\begin{exmp}
The category $\Ar[2]$ is generated by the commutative diagram
\[
\begin{tikzcd}
{(0, 0)} \arrow[r] & {(0,1)} \arrow[r] \arrow[d] & {(0,2)} \arrow[d] \\
 & {(1, 1)} \arrow[r] & {(1,2)} \arrow[d] \\
 &  & {(2,2)}
\end{tikzcd} .
\]
\end{exmp}

\smallskip

Let $\c$ be a category with cofibrations and $q\in \Z_{\geq 0}$. Define $S_q\c$ as the full subcategory of $\mathbf{Fun}(\Ar[q], \c)$ generated by $X: \Ar[q] \to \c$ such that
\begin{enumerate}
\item $X_{j, j} = \ast$ for each $j \in [q]$.
\item $X_{i, j} \rightarrowtail X_{i, k} \twoheadrightarrow X_{j, k}$ is a cofiber sequence for any $i < j < k$ in $[q]$. Equivalently, if $i\leq j\leq k$ in $[q]$, then the square
\[
\begin{tikzcd}
{X_{i,j}} \arrow[d, two heads] \arrow[r, tail] & {X_{i,k}} \arrow[d, two heads] \\
{X_{j,j}=\ast} \arrow[r, tail] & {X_{j,k}}
\end{tikzcd}
\]
is a pushout. 
\end{enumerate}
This is pointed at the constant diagram at $\ast$.


\begin{note}\label{note}
A generic object in $S_q\c$ looks like
\[ \label{eqn:D} 
\begin{tikzcd}
\ast \arrow[r, tail] & X_1 \arrow[r, tail] \arrow[d, two heads] & \cdots \arrow[r, tail] & X_{q-1} \arrow[r, tail] \arrow[d, two heads] & X_q \arrow[d, two heads] \\
 & \ast \arrow[r, tail] & \cdots \arrow[r, tail] & \faktor{X_{q-1}}{X_1} \arrow[d, two heads] \arrow[r, tail] & \faktor{X_q}{X_1} \arrow[d, two heads] \\
 &  & \ddots & \vdots \arrow[d, two heads] & \vdots \arrow[d, two heads] \\
 &  &  & \ast \arrow[r, tail] & \faktor{X_q}{X_{q-1}} \arrow[d, two heads] \\
 &  &  &  & \ast
\end{tikzcd}. \tag{$\dagger$}
\] where $X_q$ corresponds to $X_{0, q}$ and $\faktor{X_j}{X_i}$ to $X_{i, j}$ for any $1 \leq i \leq j \leq q$. 
\end{note}

\begin{definition}
Let $\left(\c, \co{\c}\right)$ be a category with cofibrations. Let $\co{S_q}\c \subset S_q \c$ consist of the morphisms $f: X \rightarrowtail Y$ of $\Ar[q]$-shaped diagrams such that for each $1\leq j \leq q$ we have
\[
\begin{tikzcd}
{X_{0, j-1}} \arrow[d, "{f_{0, j-1}}"'] \arrow[r, tail] & {X_{0, j}} \arrow[d] \arrow[rdd, "{f_{0, j}}", bend left] &  \\
{Y_{0, j-1}} \arrow[r, tail] \arrow[rrd, tail, bend right] & {X_{0, j} \cup_{X_{0, j-1}} Y_{0, j-1}} \arrow[rd, dashed, tail] &  \\
 &  & {Y_{0, j}}
\end{tikzcd}.
\]
\end{definition}

\begin{prop}
If $f: X \to Y$ is a cofibration of $S_q\c$, then 
\[
\begin{tikzcd}
{X_{i, j}} \arrow[d, "{f_{i,j}}"', tail] \arrow[r, tail] & {X_{i, k}} \arrow[d, "{f_{i,k}}", tail] \\
{Y_{i, j}} \arrow[r, tail] & {Y_{i,k}}
\end{tikzcd}
\] for any $i \leq j \leq k$ in $[q]$.\footnote{Lemma 8.3.12 (Rognes).}
\end{prop}

\begin{lemma}
$\left(S_q\c, \co{S_1} \c\right)$ is a category with cofibrations. 
\end{lemma}
\begin{proof}
First notice that the composite of two cofibrations $g \circ f : X \to Y \to Z$ is a cofibration thanks to the commutative diagram 
\[
\begin{tikzcd}
{X_{0, j-1}} \arrow[r, tail] \arrow[d, "{f_{0, j-1}}"'] & {X_{0, j}} \arrow[d] \arrow[rd, "{f_{0, j}}", bend left] &  &  \\
{Y_{0, j-1}} \arrow[d, "{g_{0, j-1}}"'] \arrow[r, tail] & {X_{0, j} \cup_{X_0, j-1} Y_{0, j-1}} \arrow[d] \arrow[r, tail] & {Y_{0, j}} \arrow[d] \arrow[rd, "{f_{0, j}}", bend left] &  \\
{Z_{0, j-1}} \arrow[r, tail] & {X_{0, j} \cup_{X_{0, j-1}} Z_{0, j-1}} \arrow[r, tail] & {Y_{0, j} \cup_{Y_{0, j-1}} Z_{0, j-1}} \arrow[r, tail] & {Z_{0, j}}
\end{tikzcd}.
\]
It's clear that any isomorphism or initial morphism in $S_q \c$ is a cofibration.


To see that axiom W2 is satisfied, let $f: X \to Y$ and $g : X \to Z$ be morphisms in $S_q \c$. It's easy to verify that each component $f_{i, j}: X_{i, j} \to Y_{i, j}$ is a cofibration. Thus, each pushout of the form $W_{i, j} \coloneqq Y_{i,j} \cup_{X_{i,j}} Z_{i, j}$ exists.
These form a functor $W: \Ar[q] \to \c$. If $i < j < k$, then we have a cofiber sequence $W_{i,j} \rightarrowtail W_{i, k} \twoheadrightarrow W_{j,k}$ because  $W_{i,j} \rightarrowtail W_{i, k}$ factors as the composite of two cofibrations as follows.
\[
\begin{tikzcd}
{Z_{i,j}} \arrow[d, "{f_{i,j} \cup \id}"'] \arrow[r, tail] & {Z_{i,k}} \arrow[d, "{f_{i,j} \cup \id}"] &  \\
{Y_{i,j} \cup_{X_{i,j}} Z_{i,j}} \arrow[r, tail] & {Y_{i,j} \cup_{X_{i,j}} Z_{i,k}} \arrow[r, tail] & {Y_{i,k} \cup_{X_{i,k}} Z_{i,k}} \\
 & {Y_{i,j} \cup_{X_{i,j}} X_{i,k}} \arrow[u, "{\id \cup g_{i,k}}"] \arrow[r, tail] & {Y_{i,k}} \arrow[u, "{\id \cup g_{i,k}}"']
\end{tikzcd}
\] The fact that colimits commute with each other ensures that $W_{j,k} \cong \faktor{W_{i,k}}{W_{i,j}}$. Hence $W$ is the pushout of $f$ and $g$. To verify that this is a cofibration, we must check that the pushout map $W_{0, j-1} \cup_{Z_{0, j-1}} Z_{0, j} \to W_{0, j}$ is a cofibration. But this follows from the pushout square
\[
\begin{tikzcd}
{Y_{0, j-1}\cup_{X_{0, j-1}} X_{0, j}} \arrow[d] \arrow[r, tail] & {Y_{0, j}} \arrow[d] \\
{Y_{0, j-1}\cup_{X_{0, j-1}} Z_{0, j}} \arrow[r, tail] & {Y_{0, j}\cup_{X_{0, j}} Z_{0, j}}
\end{tikzcd}
.\]
\end{proof}

\begin{definition}
Let $\left(\c, w{\c}\right)$ be a Waldhausen category. Let $w S_q\c \subset S_q \c$ consist of the morphisms $f: X \overset{\sim}{\longrightarrow} Y$ of $\Ar[q]$-shaped diagrams such that the component $f_{0, j} : X_{0, j} \to Y_{0, j}$ is a weak equivalence in $\c$ for each $1\leq j \leq q$.
\end{definition}

\begin{prop}
Let $f$ be a weak equivalence in $S_q \c$.  Each component $f_{i, j}: X_{i, j} \to Y_{i, j}$ is a weak equivalence in $\c$.
\end{prop}
\begin{proof}
Apply the Gluing axiom to the diagram
\[
\begin{tikzcd}
{X_{0, j}} \arrow[d, "\cong"'] & {X_{0, i}} \arrow[l, tail] \arrow[r] \arrow[d, "\cong"'] & \ast \arrow[d, "="'] \\
{Y_{0, j}} & {Y_{0, i}} \arrow[l, tail] \arrow[r] & \ast
\end{tikzcd}.
\] Then $X_{i, j} \cong X_{0, j} \cup_{X_{0, i}} \ast \overset{\sim}{\longrightarrow} Y_{0, j} \cup_{Y_{0, i}} \ast \cong Y_{i, j}$, as desired.
\end{proof}

\begin{lemma}
$\left(S_q \c, wS_q \c\right)$ is a Waldhausen category. 
\end{lemma}

\begin{definition}
Let $\c$ be a category with cofibrations. If $\alpha : [p] \to [q]$, then define $\alpha^{\ast} : S_q \c \to S_p \c$ by
$$\alpha^{\ast}(X: \Ar[q] \to \c) = X \circ \Ar(\alpha) : \Ar[p] \to \Ar[q] \to \c.$$
\end{definition}

It's easy to check that this satisfies the two conditions of a diagram in $S_p \c$.
Moreover, the face maps $d_i$ are obtained by deleting the row $X_{i, -}$ and the column containing $X_i$ in \eqref{eqn:D} and then reindexing as necessary. The degeneracy maps $s_i$ are given by duplicating $X_i$ and then reindexing such that $X_{i+1, i} =0$.\reversemarginpar{\marginpar{\small{Not sure that the $s_i$ work.}}}

\begin{prop}
Let $\left(\c, w{\c}\right)$ be a Waldhausen category. Each functor $\alpha^{\ast}: S_q \c \to S_p \c$ is exact, so that $\left(S_{\bullet}\c, wS_{\bullet} \c\right)$ is a simplicial Waldhausen category.
\end{prop}	


The nerve $N_{\bullet}w S_{\bullet}\c$ is a bisimplicial set with $(p,q)$-bisimplices the diagrams of the form
\[
\begin{tikzcd}
\ast \arrow[r, tail] & X_1^0 \arrow[d, "\sim"'] \arrow[r, tail] & X_2^0 \arrow[r, tail] \arrow[d, "\sim"'] & \cdots \arrow[r, tail] & X_q^0 \arrow[d, "\sim"'] \\
\ast \arrow[r, tail] & X_1^1 \arrow[d, "\sim"'] \arrow[r, tail] & X_2^1 \arrow[r, tail] \arrow[d, "\sim"'] & \cdots \arrow[r, tail] & X_q^1 \arrow[d, "\sim"'] \\
 & \vdots \arrow[d, "\sim"'] & \vdots \arrow[d, "\sim"'] &  & \vdots \arrow[d, "\sim"'] \\
\ast \arrow[r, tail] & X_1^p \arrow[r, tail] & X_2^p \arrow[r, tail] & \cdots \arrow[r, tail] & X_q^p
\end{tikzcd}
\]
such that $X^k_{i,j} \cong \faktor{X_j^k}{X_i^k}$ for every $i\leq j\leq q$ and $k\in \left[p\right]$. 


\begin{lemma}
There is a natural map $N_{\bullet} w{\c} \land \Delta_{\bullet}^1 \to N_{\bullet} w S_{\bullet} \c$, which automatically induces a based map $\sigma : \Sigma \left\lvert{w{\c}}\right\rvert\to \left\lvert{w S_{\bullet} \c}\right\rvert$ of classifying spaces.
\end{lemma}
\begin{proof}
We can treat $N_{\bullet} w S_{\bullet} \c$ as the simplicial set $[q] \mapsto N_{\bullet} w S_q \c$. This defines a right skeletal structure on $N_{\bullet} w S_{\bullet} \c$. 


 If $q = 0$, then $w S_0 \c = S_0 \c = \ast$, so that $N_{\bullet} w S_0\c = \ast$ as well. If $q= 1$, then
$w S_1 \c \cong w{\c}$. Thus, the right $1$-skeleton is equal to $N_{\bullet} w{\c}  \land \Delta_{\bullet}^1$, which in turn must be equal to the image $I$ of the canonical map $$\coprod_{q\leq 1} N_{\bullet} w S_q \c \times \Delta_{\bullet}^q \to N_{\bullet} w S_{\bullet} \c.$$ Now, the degeneracy map $s_0$ collapses $\{\ast\} \times \Delta_{\bullet}^1$, and the face maps $d_0$ and $d_1$ collapse $ N_{\bullet} w{\c} \times \partial{\Delta_{\bullet}^1}$. 
Therefore, $I$ must equal $$N_{\bullet} w{\c}  \land \Delta_{\bullet}^1  = \frac{N_{\bullet} w{\c} \times \Delta_{\bullet}^1}{\{\ast\} \times \Delta_{\bullet}^1 \cup N_{\bullet} w{\c} \times \partial{\Delta_{\bullet}^1}}.$$  We have defined a natural inclusion map $\lambda : N_{\bullet} w{\c} \land \Delta_{\bullet}^1 \to  N_{\bullet} w S_{\bullet} \c$.

Since $\Delta_{\bullet}^1$ is isomorphic to the unit interval and the map $\lambda$ agrees on the endpoints, we can pass to $S^1$ during the suspension. Hence $\lambda$  induces the desired map $\sigma$.\footnote{This is a tentative explanation due to Thomas Brazelton.}
\end{proof}

\begin{note}
Axiom W3 implies that $w{\c}$ is closed under coproducts, making $\left\lvert{wS_{\bullet} \c}\right\rvert$ into an $H$-space via the map $$\coprod: \left\lvert{wS_{\bullet} \c}\right\rvert \times \left\lvert{wS_{\bullet} \c}\right\rvert \cong \left\lvert{wS_{\bullet} \c \times  wS_{\bullet} \c}\right\rvert\to \left\lvert{wS_{\bullet} \c}\right\rvert.$$
\end{note}

\begin{definition}
Let $\left(\c, w{\c}\right)$ be a Waldhausen category. Define the \textit{algebraic $K$-theory space} $$K(\c, w) = \Omega \left\lvert{ N_{\bullet} wS_{\bullet} \c}\right\rvert.$$ 
\end{definition}

\begin{note}
We have a right adjoint $\iota: \left\lvert{w{\c}}\right\rvert \to K(\c, w)$ to the based map $\sigma$.
\end{note}

\medskip


Let $F : \left(\c, w{\c}\right) \to \left(\d, w{\d}\right)$ be an exact functor. Let $$K(F) = \Omega\left\lvert{wS_{\bullet}F}\right\rvert : K(\c, w) \to K(\d, w).$$ This is the \textit{algebraic $K$-theory functor} $K : \mathbf{Wald} \to \mathbf{Top_{\ast}}.$

\medskip

Note that any exact category $\a$ is a Waldhausen category with cofibrations the admissible exact sequences and weak equivalences the isomorphisms. Waldhausen showed that $\left\lvert{i{S_{\bullet}\a}}\right\rvert$ (where $i({-})$ denotes the isomorphism category) and $BQ\a$ are homotopy equivalent. Therefore, our current definition of higher algebraic $K$-theory agrees with Quillen's.


\begin{exmp}
Let $R$ be a ring. Define the \textit{algebraic $K$-theory space of $R$} as $$K(R) = K(\P(R), i)$$ where the weak equivalences are precisely the injective $R$-linear maps with projective cokernel and the cofibrations are precisely the $R$-linear maps.
\end{exmp}

\begin{exmp}
Assume that $\c$is a small Waldhausen category where $w{\c}$ consists of the isomorphisms in $\c$. If $s_n\c$ denotes the set of objects of $S_n \c$, then we get a simplicial set $s_{\bullet} \c$. Waldhausen showed that the inclusion map $\left\lvert{s_{\bullet} \c}\right\rvert \hookrightarrow \left\lvert{iS_{\bullet} \c}\right\rvert$ is a homotopy equivalence. This makes $\Omega |s_{\bullet} \c|$ into a so-called simplicial model for $K(\c, w)$.
\end{exmp}

\begin{remark}
Since $wS_0 \c = \ast$ and every simplex of degree $n >0$ is attached to $\ast$, it follows that the classifying space $\left\lvert{w S_{\bullet} \c}\right\rvert$ is connected. Therefore, we preserve any homotopical information when passing to the loop space.
\end{remark}

\begin{definition}
The \textit{$i$-th algebraic $K$-group} is $K_i(\c, w) \equiv \pi_iK(\c, w)$ for each $i\geq 0$. 
\end{definition}

\begin{prop}\label{P4}
$\pi_1\left\lvert{w S_{\bullet} \c}\right\rvert \cong K_0(\c, w)$.
\end{prop}

\begin{lemma}\label{L8}
The group $K_0(\c, w)$ is generated by all elements $\left[X\right]$ such that 
\begin{itemize}
\item $\left[X'\right] + \left[X''\right] = \left[X\right]$ for every cofiber sequence $X' \rightarrowtail X \twoheadrightarrow X''$ and
\item  $\left[X\right] = \left[Y\right]$ for every weak equivalence $X \overset{\sim}{\longrightarrow} Y$.
\end{itemize}
\end{lemma}
\begin{proof}
In light of \cref{P4}, it suffices to compute $\pi_1\left\lvert{N_{\bullet}w S_{\bullet} \c}\right\rvert$ based at the $(0,0)$-bisimplex $\ast$. For this, just notice the CW structure of $\left\lvert{N_{\bullet}w S_{\bullet} \c}\right\rvert$,  with $1$-cells the $(0,1)$-bisimplices and $2$-cells the $(0,2)$-bisimplices $X' \rightarrowtail X \twoheadrightarrow X''$ and the $(1,1)$-bisimplices $X \overset{\sim}{\longrightarrow} Y$, which are attached to the $1$-cells $X$ and $Y$. Any cell of dimension $n>2$ is irrelevant to computing $\pi_1$.
\end{proof}


As a result, we obtain functors $$K_i : \mathbf{Wald} \to \mathbf{Top_{\ast}} \to \mathbf{Ab}$$ known as the \textit{algebraic $K$-group functors}. Indeed, thanks to \cref{P4}, we know that $$K_i(\c, w) = \pi_{i+1}\left\lvert{w S_{\bullet} \c}\right\rvert,$$ which is abelian for $i\geq 1$. Moreover, note that if $X' \rightarrowtail X' \vee X'' \twoheadrightarrow X''$ and $X'' \rightarrowtail X' \vee X'' \twoheadrightarrow X'$ are cofiber sequences, then \cref{L8} implies that $$\left[X'\right] + \left[X''\right] = \left[X' \vee X''\right] = \left[X'' + X'\right].$$ Hence $K_0(\c, w)$ is also abelian.


\begin{exmp}
Let $X$ be a CW complex and $\mathcal{R}(X)$ denote the category of CW complexes $Y$  obtained  by attaching at least one cell to $X$ so that $X$ is a retract of $Y$. Equip this with cofibrations in the form of cellular inclusions fixing $X$ and weak equivalence in the form of homotopy equivalences. This makes $\mathcal{R}(X)$ into a Waldhausen category. 

\smallskip
If $\mathcal{R}_f(X)$ denotes the subcategory of those $Y$ obtained by attaching finitely many cells, then we denote  $K(\mathcal{R}_f(X))$ by $A(X)$.
\end{exmp}

\begin{prop}
$A_0(X)\cong \Z$.
\end{prop}

\begin{definition}
If $\b$ is a Waldhausen subcategory of $\c$, then it is \textit{cofinal in $\c$} if for any $X \in \ob \c$, there is some $X' \in \ob \c$ such that $X \coprod X' \in \ob \b$.
\end{definition}

\begin{theorem}
Let $\left(\b, w\right)$ be cofinal in $\left(\c, w\right)$ and closed under extensions. Assume that $K_0(\b) = K_0(\c)$. Then $wS_{\bullet}\b \to wS_{\bullet}\c$ is a homotopy equivalence.
\end{theorem}

\smallskip

It follows that $K_i(\b) \cong K_i(\c)$ for every $i\geq 0$.

\end{document}
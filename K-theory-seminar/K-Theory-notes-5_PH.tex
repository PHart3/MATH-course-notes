\documentclass[10pt,letterpaper,cm]{nupset}
\usepackage[margin=1in]{geometry}
\usepackage{graphicx}
 \usepackage{enumitem}
 \usepackage{stmaryrd}
 \usepackage{bm}
\usepackage{amsfonts}
\usepackage{amssymb}
\usepackage{pgfplots}
\usepackage{amsmath,amsthm}
\usepackage{lmodern}
\usepackage{tikz-cd}
\usepackage{faktor}
\usepackage{xfrac}
\usepackage{mathtools}
\usepackage{bm}
\usepackage{ dsfont }
\usepackage{mathrsfs}
\usepackage{hyperref}
\hypersetup{colorlinks=true, linkcolor=red,          % color of internal links (change box color with linkbordercolor)
    citecolor=green,        % color of links to bibliography
    filecolor=magenta,      % color of file links
    urlcolor=cyan           }

\usepackage{thmtools}
\usepackage[capitalise]{cleveref} 
    
\theoremstyle{definition}
\newtheorem{definition}{Definition}
\newtheorem{exmp}[definition]{Example}
\newtheorem{non-exmp}[definition]{Non-example}
\newtheorem{note}[definition]{Note}

\theoremstyle{theorem}
\newtheorem{theorem}{Theorem}
\newtheorem{lemma}[theorem]{Lemma}
\newtheorem{prop}[theorem]{Proposition}
\newtheorem{corollary}[theorem]{Corollary}
\newtheorem*{claim}{Claim}
\newtheorem{exercise}[theorem]{Exercise}

\theoremstyle{remark}
\newtheorem{remark}{Remark}
\newtheorem*{todo}{To do}
\newtheorem*{conv}{Convention}
\newtheorem*{aside}{Aside}
\newtheorem*{notation}{Notation}
\newtheorem*{term}{Terminology}
\newtheorem*{background}{Background}
\newtheorem*{further}{Further reading}
\newtheorem*{sources}{Sources}

\makeatletter
\def\th@plain{%
  \thm@notefont{}% same as heading font
  \itshape % body font
}
\def\th@definition{%
  \thm@notefont{}% same as heading font
  \normalfont % body font
}
\makeatother

\makeatletter
\renewcommand*\env@matrix[1][*\c@MaxMatrixCols c]{%
  \hskip -\arraycolsep
  \let\@ifnextchar\new@ifnextchar
  \array{#1}}
\makeatother
\pgfplotsset{unit circle/.style={width=4cm,height=4cm,axis lines=middle,xtick=\empty,ytick=\empty,axis equal,enlargelimits,xmax=1,ymax=1,xmin=-1,ymin=-1,domain=0:pi/2}}
\DeclareMathOperator{\Ima}{Im}
\newcommand{\A}{\mathcal A}
\newcommand{\C}{\mathbb C}
\newcommand{\E}{\vec E}
\newcommand{\CP}{\mathbb CP}
\newcommand{\F}{\mathbb F}
\newcommand{\G}{\vec G}
\renewcommand{\H}{\mathbb H}
\newcommand{\HP}{\mathbb HP}
\newcommand{\K}{\mathbb K}
\renewcommand{\L}{\mathcal L}
\newcommand{\M}{\mathbb M}
\newcommand{\N}{\mathbb N}
\renewcommand{\O}{\mathbf O}
\newcommand{\OP}{\mathbb OP}
\renewcommand{\P}{\mathbf P}
\newcommand{\Q}{\mathbb Q}
\newcommand{\I}{\mathbb I}
\newcommand{\R}{\mathbb R}
\newcommand{\RP}{\mathbb RP}
\renewcommand{\S}{\mathbf S}
\newcommand{\T}{\mathbf T}
\newcommand{\X}{\mathbf X}
\newcommand{\Z}{\mathbb Z}
\newcommand{\B}{\mathcal{B}}
\newcommand{\1}{\mathbf{1}}
\newcommand{\ds}{\displaystyle}
\newcommand{\ran}{\right>}
\newcommand{\lan}{\left<}
\newcommand{\bmat}[1]{\begin{bmatrix} #1 \end{bmatrix}}

\renewcommand{\a}{\mathscr{A}}
\renewcommand{\b}{\mathscr{B}}
\renewcommand{\c}{\mathscr{C}}
\renewcommand{\d}{\mathscr{D}}
\newcommand{\e}{\mathscr{E}}
\newcommand{\y}{\mathscr{Y}}
\renewcommand{\j}{\mathscr{J}}

\newcommand{\h}{\vec h}
\newcommand{\f}{\vec f}
\newcommand{\g}{\vec g}
\renewcommand{\i}{\vec i}
\renewcommand{\k}{\vec k}
\newcommand{\n}{\vec n}
\newcommand{\p}{\vec p}
\newcommand{\q}{\vec q}
\renewcommand{\r}{\vec r}
\newcommand{\s}{\vec s}
\renewcommand{\t}{\vec t}
\renewcommand{\u}{\vec u}
\renewcommand{\v}{\vec v}
\newcommand{\w}{\vec w}
\newcommand{\x}{\vec x}
\newcommand{\z}{\vec z}
\newcommand{\0}{\vec 0}
\DeclareMathOperator*{\Span}{span}
\DeclareMathOperator*{\GL}{GL}
\DeclareMathOperator*{\SL}{SL}
\DeclareMathOperator*{\SO}{SO}
\DeclareMathOperator*{\SU}{SU}
\DeclareMathOperator{\rng}{range}
\DeclareMathOperator{\gemu}{gemu}
\DeclareMathOperator{\almu}{almu}
\newcommand{\Char}{\mathsf{char}}
\DeclareMathOperator{\id}{Id}
\DeclareMathOperator{\im}{im}
\DeclareMathOperator{\graph}{Graph}
\DeclareMathOperator{\gal}{Gal}
\DeclareMathOperator{\tr}{Tr}
\DeclareMathOperator{\norm}{N}
\DeclareMathOperator{\aut}{Aut}
\DeclareMathOperator{\Int}{Int}
\DeclareMathOperator{\ext}{Ext}
\DeclareMathOperator{\stab}{Stab}
\DeclareMathOperator{\orb}{Orb}
\DeclareMathOperator{\inn}{Inn}
\DeclareMathOperator{\out}{Out}
\DeclareMathOperator{\op}{op}
\DeclareMathOperator{\fix}{Fix}
\DeclareMathOperator{\ab}{ab}
\DeclareMathOperator{\sgn}{sgn}
\DeclareMathOperator{\syl}{syl}
\DeclareMathOperator{\Syl}{Syl}
\DeclareMathOperator{\ob}{ob}
\DeclareMathOperator{\mor}{mor}
\DeclareMathOperator{\iso}{iso}
\DeclareMathOperator{\ar}{Ar}
\DeclareMathOperator{\red}{red}
\DeclareMathOperator{\colim}{colim}
\DeclareMathOperator{\ZFC}{ZFC}
\DeclareMathOperator{\set}{\mathbf{Set}}
\DeclareMathOperator{\Ab}{\mathbf{Ab}}
\DeclareMathOperator{\Cmon}{\mathbf{CMon}}
\DeclareMathOperator{\spec}{Spec}
\DeclareMathOperator{\rank}{rank}
\DeclareMathOperator{\rk}{rk}
\DeclareMathOperator{\diag}{diag}
\DeclareMathOperator{\Ar}{Ar}
\DeclareMathOperator{\Mod}{\mathbf Mod}

% info for header block in upper right hand corner
\name{Perry Hart}
\class{Homotopy and K-theory seminar}
\assignment{Talk \#15}
\duedate{November 12, 2018}

\begin{document}

\begin{abstract}
We begin higher Waldhausen $K$-theory. The main sources for this talk are the following.
\begin{itemize}
\item \textit{nLab}.
\item Charles Weibel's \textit{The} K\textit{-book: an introduction to algebraic} K\textit{-theory}.  Chapter IV.8.
\item John Rognes's \textit{Lecture Notes on Algebraic K-Theory}, Ch. 8.
\end{itemize}
For the original development, see Friedhelm Waldhausen's \textit{Algebraic K-theory of spaces} (1985), 318-419. 
\end{abstract}


Let $\c$ be a Waldhausen category. Our goal is to construct the $K$-theory $K(\c)$ of $\c$ as a based loop space $\Omega Y$ endowed with a loop completion map $ \iota : |w \c| \to K(\c)$ where $w \c$ denotes the subcategory of weak equivalences. This will produce a function $\ob \c \to |w \c| \to \Omega Y$. Further, we'll require of $K(\c)$ certain limit and coherence properties, eventually rendering $K(\c)$ the underlying infinite loop space of a spectrum $\mathbf{K}(\c)$, called the algebraic $K$-theory spectrum of $\c$.


\begin{definition}
Let $\c$ be a category equipped with a subcategory $co(\c)$ of morphisms called \textit{cofibrations}. The pair $(\c, co\c)$ is a \textit{category with cofibrations} if the following conditions hold.
\begin{enumerate}
\item (W0) Every isomorphism in $\c$ is a cofibration.
\item (W1) There is a base point $\ast$ in $\c$ such that the unique morphism $\ast \rightarrowtail A$ is a cofibration for any $A \in \ob \c$.
\item (W2) We have a \textit{cobase change}
\[
\begin{tikzcd}
A \arrow[d] \arrow[r, tail] & B \arrow[d, dotted] \\
C \arrow[r, dotted, tail] & B \cup_A C
\end{tikzcd}.
\]
\end{enumerate}
\end{definition}


We see that $B \coprod C$ always exists as the pushout $B \cup_{\ast} C$ and that the cokernel of any $i : A \rightarrowtail B$ exists as $B \cup_A \ast$ along $A \to \ast$. We call $A \rightarrowtail  B \twoheadrightarrow \faktor{B}{A}$ a \textit{cofiber sequence}.


\begin{definition}
A \textit{Waldhausen category} $\c$ is a category with cofibrations together with a subcategory $w\c$ of morphisms called \textit{weak equivalences} such that every isomorphism in $\c$ is a w.e. and the following ``Gluing axiom'' holds.
\begin{enumerate}
\item (W3) For any diagram
\[
\begin{tikzcd}
C \arrow[d, "\sim"'] & A \arrow[d, "\sim"'] \arrow[r, tail] \arrow[l] & B \arrow[d, "\sim"'] \\
C' & A' \arrow[r, tail] \arrow[l] & B'
\end{tikzcd}, \]
the induced map $B \cup_A C \to  B' \cup_{A'} C'$ is a w.e.
\end{enumerate}
\end{definition}

\begin{definition}
A Waldhausen category $(\c, w)$ is \textit{saturated} if whenever $fg$ makes sense and is a w.e., then $f$ is a w.e. iff $g$ is. 
\end{definition}

\begin{definition}
We now introduce the main concept to be generalized.
\\ \\ Let $\c$ be a category with cofibrations. Let the \textit{extension category} $S_2\c$ have as objects the cofiber sequences in $(\c, co\c)$ and as morphisms the triples $(f', f, f'')$ such that
\[
\begin{tikzcd}
X' \arrow[r, tail] \arrow[d, "f'"] & X \arrow[r, two heads] \arrow[d, "f"] & X'' \arrow[d, "f''"] \\
Y' \arrow[r, tail] & Y \arrow[r, two heads] & Y''
\end{tikzcd}
\] 
commutes. This is pointed at $\ast \rightarrowtail \ast \twoheadrightarrow \ast$.
\end{definition}

\begin{definition}
Suppose an arbitrary triple $(f', f, f'')$ as above has the property that whenever $f'$ and $f''$ are w.e., then so is $f$. Then we say $\c$ is \textit{extensional} or \textit{closed under extensions}.
\end{definition}


Say that the morphism $(f', f, f'')$ is a cofibration if $f'$, $f''$, and $Y' \cup_{X'} X \to Y$ are cofibrations in $\c$. Say that  the same triple is a weak equivalence if $f'$, $f$, and $f''$ are w.e. in $\c$. This makes $S_2 \c$ into a Waldhausen category.


\begin{definition}
Let $q\geq 0$. Let the \textit{arrow category $\Ar[q]$ on $[q]$} have as objects ordered pairs $(i, j)$ with $i\leq j \leq q$ and as morphisms commutative diagrams of the form
\[
\begin{tikzcd}
i \arrow[d, "\leq"'] \arrow[r, "\leq"] & j \arrow[d, "\leq"] \\
i' \arrow[r, "\leq"'] & j'
\end{tikzcd}.
\] We view $[q]$ a full subcategory of $\Ar[q]$ via the embedding $[q] \overset{k \mapsto (0, k)}{\xrightarrow{\hspace*{1cm}}} \Ar[q]$.
\end{definition}

\begin{remark} $ $
\begin{enumerate}
\item Any triple $i\leq j \leq k$ determines the morphisms $(i, j) \to (i, k)$ and $(i, k) \to (j, k)$. Conversely, any morphism in the arrow category is a composition of such triples.
\item $\Ar[q] \cong \mathbf{Fun}([1], [q])$ by identifying each pair $(i, j)$ with the functor satisfying $0 \mapsto i$ and $1 \mapsto j$.
\end{enumerate}
\end{remark}

\begin{exmp}
The category $\Ar[2]$ is generated by the commutative diagram
\[
\begin{tikzcd}
{(0, 0)} \arrow[r] & {(0,1)} \arrow[r] \arrow[d] & {(0,2)} \arrow[d] \\
 & {(1, 1)} \arrow[r] & {(1,2)} \arrow[d] \\
 &  & {(2,2)}
\end{tikzcd} .
\]
\end{exmp}

\begin{definition}
Let $\c$ be a category with cofibrations and $q\geq 0$. Define $S_q\c$ as the full subcategory of $\mathbf{Fun}(\Ar[q], \c)$ generated by $X: \Ar[q] \to \c$ such that
\begin{enumerate}
\item $X_{j, j} = \ast$ for each $j \in [q]$.
\item $X_{i, j} \rightarrowtail X_{i, k} \twoheadrightarrow X_{j, k}$ is a cofiber sequence for any $i < j < k$ in $[q]$. Equivalently, if $i\leq j\leq k$ in $[q]$, then the square
\[
\begin{tikzcd}
{X_{i,j}} \arrow[d, two heads] \arrow[r, tail] & {X_{i,k}} \arrow[d, two heads] \\
{X_{j,j}=\ast} \arrow[r, tail] & {X_{j,k}}
\end{tikzcd}
\]
is a pushout. 
\end{enumerate}
This is pointed at the constant diagram at $\ast$.
\end{definition}

\begin{note}\label{note}
A generic object in $S_q\c$ looks like
\[ \tag{$\ast$}
\begin{tikzcd}
\ast \arrow[r, tail] & X_1 \arrow[r, tail] \arrow[d, two heads] & \cdots \arrow[r, tail] & X_{q-1} \arrow[r, tail] \arrow[d, two heads] & X_q \arrow[d, two heads] \\
 & \ast \arrow[r, tail] & \cdots \arrow[r, tail] & \faktor{X_{q-1}}{X_1} \arrow[d, two heads] \arrow[r, tail] & \faktor{X_q}{X_1} \arrow[d, two heads] \\
 &  & \ddots & \vdots \arrow[d, two heads] & \vdots \arrow[d, two heads] \\
 &  &  & \ast \arrow[r, tail] & \faktor{X_q}{X_{q-1}} \arrow[d, two heads] \\
 &  &  &  & \ast
\end{tikzcd}.
\] where $X_q$ corresponds to $X_{0, q}$ and $\faktor{X_j}{X_i}$ to $X_{i, j}$ for any $1 \leq i \leq j \leq q$. 
\end{note}

\begin{definition}
Let $(\c, co\c)$ be a category with cofibrations. Let $coS_q\c \subset S_q \c$ consist of the morphisms $f: X \rightarrowtail Y$ of $\Ar[q]$-shaped diagrams such that for each $1\leq j \leq q$ we have
\[
\begin{tikzcd}
{X_{0, j-1}} \arrow[d, "{f_{0, j-1}}"'] \arrow[r, tail] & {X_{0, j}} \arrow[d] \arrow[rdd, "{f_{0, j}}", bend left] &  \\
{Y_{0, j-1}} \arrow[r, tail] \arrow[rrd, tail, bend right] & {X_{0, j} \cup_{X_{0, j-1}} Y_{0, j-1}} \arrow[rd, dashed, tail] &  \\
 &  & {Y_{0, j}}
\end{tikzcd}.
\]
\end{definition}

\begin{prop}
If $f: X \to Y$ is a cofibration of $S_q\c$, then 
\[
\begin{tikzcd}
{X_{i, j}} \arrow[d, "{f_{i,j}}"', tail] \arrow[r, tail] & {X_{i, k}} \arrow[d, "{f_{i,k}}", tail] \\
{Y_{i, j}} \arrow[r, tail] & {Y_{i,k}}
\end{tikzcd}
\] for any $i \leq j \leq k$ in $[q]$.
\end{prop}
\begin{proof}
See Rognes, Lemma 8.3.12.
\end{proof}

\begin{lemma}
$(S_q\c, coS_1 \c)$ is a category with cofibrations. 
\end{lemma}
\begin{proof}
First notice that the composite of two cofibrations $g \circ f : X \to Y \to Z$ is a cofibration because we have
\[
\begin{tikzcd}
{X_{0, j-1}} \arrow[r, tail] \arrow[d, "{f_{0, j-1}}"'] & {X_{0, j}} \arrow[d] \arrow[rd, "{f_{0, j}}", bend left] &  &  \\
{Y_{0, j-1}} \arrow[d, "{g_{0, j-1}}"'] \arrow[r, tail] & {X_{0, j} \cup_{X_0, j-1} Y_{0, j-1}} \arrow[d] \arrow[r, tail] & {Y_{0, j}} \arrow[d] \arrow[rd, "{f_{0, j}}", bend left] &  \\
{Z_{0, j-1}} \arrow[r, tail] & {X_{0, j} \cup_{X_{0, j-1}} Z_{0, j-1}} \arrow[r, tail] & {Y_{0, j} \cup_{Y_{0, j-1}} Z_{0, j-1}} \arrow[r, tail] & {Z_{0, j}}
\end{tikzcd}.
\]
It's clear that any isomorphism or initial morphism in $S_q \c$ is a cofibration.
\\ \\
To see that (W2) is satisfied, let $f: X \to Y$ and $g : X \to Z$ be morphisms in $S_q \c$. It's easy to verify that each component $f_{i, j}: X_{i, j} \to Y_{i, j}$ is a cofibration. Thus, each pushout $W_{i, j} \coloneqq Y_{i,j} \cup_{X_{i,j}} Z_{i, j}$ exists.
These form a functor $W: \Ar[q] \to \c$. If $i < j < k$, then we have $W_{i,j} \rightarrowtail W_{i, k} \twoheadrightarrow W_{j,k}$ because the left morphism factors as the composite of two cofibrations
\[
\begin{tikzcd}
{Z_{i,j}} \arrow[d, "{f_{i,j} \cup \id}"'] \arrow[r, tail] & {Z_{i,k}} \arrow[d, "{f_{i,j} \cup \id}"] &  \\
{Y_{i,j} \cup_{X_{i,j}} Z_{i,j}} \arrow[r, tail] & {Y_{i,j} \cup_{X_{i,j}} Z_{i,k}} \arrow[r, tail] & {Y_{i,k} \cup_{X_{i,k}} Z_{i,k}} \\
 & {Y_{i,j} \cup_{X_{i,j}} X_{i,k}} \arrow[u, "{\id \cup g_{i,k}}"] \arrow[r, tail] & {Y_{i,k}} \arrow[u, "{\id \cup g_{i,k}}"']
\end{tikzcd}
.\] The fact that colimits commute confirms that $W_{j,k} \cong \faktor{W_{i,k}}{W_{i,j}}$ Hence $W$ is the pushout of $f$ and $g$. To verify that this is a cofibration, we must check that the pushout map $W_{0, j-1} \cup_{Z_{0, j-1}} Z_{0, j} \to W_{0, j}$ is a cofibration. But this follows from the pushout square
\[
\begin{tikzcd}
{Y_{0, j-1}\cup_{X_{0, j-1}} X_{0, j}} \arrow[d] \arrow[r, tail] & {Y_{0, j}} \arrow[d] \\
{Y_{0, j-1}\cup_{X_{0, j-1}} Z_{0, j}} \arrow[r, tail] & {Y_{0, j}\cup_{X_{0, j}} Z_{0, j}}
\end{tikzcd}
.\]
\end{proof}

\begin{definition}
Let $(\c, w \c)$ be a Waldhausen category. Let $w S_q\c \subset S_q \c$ consist of the morphisms $f: X \overset{\sim}{\longrightarrow} Y$ of $\Ar[q]$-shaped diagrams such that the component $f_{0, j} : X_{0, j} \to Y_{0, j}$ is a w.e. in $\c$ for each $1\leq j \leq q$.
\end{definition}

\begin{prop}
Let $f$ be a w.e. in $S_q \c$.  Each component $f_{i, j}: X_{i, j} \to Y_{i, j}$ is a w.e. in $\c$.
\end{prop}
\begin{proof}
Apply the Gluing axiom to the diagram
\[
\begin{tikzcd}
{X_{0, j}} \arrow[d, "\cong"'] & {X_{0, i}} \arrow[l, tail] \arrow[r] \arrow[d, "\cong"'] & \ast \arrow[d, "="'] \\
{Y_{0, j}} & {Y_{0, i}} \arrow[l, tail] \arrow[r] & \ast
\end{tikzcd}.
\] Then $X_{i, j} \cong X_{0, j} \cup_{X_{0, i}} \ast \overset{\sim}{\longrightarrow} Y_{0, j} \cup_{Y_{0, i}} \ast \cong Y_{i, j}$, as desired.
\end{proof}

\begin{lemma}
$(S_q \c, wS_q \c)$ is a Waldhausen category. 
\end{lemma}

\begin{definition}
Let $\c$ be a category with cofibrations. If $\alpha : [p] \to [q]$, then define $\alpha^{\ast} : S_q \c \to S_p \c$ by
$$\alpha^{\ast}(X: \Ar[q] \to \c) = X \circ \Ar(\alpha) : \Ar[p] \to \Ar[q] \to \c.$$
It's easy to check that this satisfies the two conditions of a diagram in $S_p \c$.
Moreover, the face maps $d_i$ are given by deleting the row $X_{i, -}$ and the column containing $X_i$ in $(\ast)$ of \cref{note} and then reindexing as necessary. The degeneracy maps $s_i$ are given by duplicating $X_i$ and then reindexing such that $X_{i+1, i} =0$. {[[Not sure the $s_i$ work.]]} 
\end{definition}

\begin{prop}
Let $(\c, w \c)$ be a Waldhausen category. Each functor $\alpha^{\ast}: S_q \c \to S_p \c$ is exact, so that $(S_{\bullet}\c, wS_{\bullet} \c)$ is a simplicial Waldhausen category.
\end{prop}	


The nerve $N_{\bullet}w S_{\bullet}\c$ is a bisimplicial set with $(p,q)$-bisimplices the diagrams of the form
\[
\begin{tikzcd}
\ast \arrow[r, tail] & X_1^0 \arrow[d, "\sim"'] \arrow[r, tail] & X_2^0 \arrow[r, tail] \arrow[d, "\sim"'] & \cdots \arrow[r, tail] & X_q^0 \arrow[d, "\sim"'] \\
\ast \arrow[r, tail] & X_1^1 \arrow[d, "\sim"'] \arrow[r, tail] & X_2^1 \arrow[r, tail] \arrow[d, "\sim"'] & \cdots \arrow[r, tail] & X_q^1 \arrow[d, "\sim"'] \\
 & \vdots \arrow[d, "\sim"'] & \vdots \arrow[d, "\sim"'] &  & \vdots \arrow[d, "\sim"'] \\
\ast \arrow[r, tail] & X_1^p \arrow[r, tail] & X_2^p \arrow[r, tail] & \cdots \arrow[r, tail] & X_q^p
\end{tikzcd}
\]
such that $X^k_{i,j} \cong \faktor{X_j^k}{X_i^k}$ for every $i\leq j\leq q$ and $k\in [p]$. 


\begin{lemma}
There is a natural map $N_{\bullet} w \c \land \Delta_{\bullet}^1 \to N_{\bullet} w S_{\bullet} \c$, which automatically induces a based map $\sigma : \Sigma |w \c|\to |w S_{\bullet} \c|$ of classifying spaces.
\end{lemma}
\begin{proof}
We can treat $N_{\bullet} w S_{\bullet} \c$ as the simplicial set $[q] \mapsto N_{\bullet} w S_q \c$. This defines a right skeletal structure on $N_{\bullet} w S_{\bullet} \c$. 
\\ \\ If $q = 0$, then $w S_0 \c = S_0 \c = \ast$, so that $N_{\bullet} w S_0\c = \ast$ as well. If $q= 1$, then
$w S_1 \c \cong w \c$. Thus, the right $1$-skeleton is equal to $N_{\bullet} w \c  \land \Delta_{\bullet}^1$, which in turn must be equal to the image $I$ of the canonical map $$\coprod_{q\leq 1} N_{\bullet} w S_q \c \times \Delta_{\bullet}^q \to N_{\bullet} w S_{\bullet} \c.$$ Now, the degeneracy map $s_0$ collapses $\{\ast\} \times \Delta_{\bullet}^1$, and the face maps $d_0$ and $d_1$ collapse $ N_{\bullet} w \c \times \partial{\Delta_{\bullet}^1}$. 
Therefore, $I$ must equal $$N_{\bullet} w \c  \land \Delta_{\bullet}^1  = \frac{N_{\bullet} w \c \times \Delta_{\bullet}^1}{\{\ast\} \times \Delta_{\bullet}^1 \cup N_{\bullet} w \c \times \partial{\Delta_{\bullet}^1}}.$$  We have defined a natural inclusion map $\lambda : N_{\bullet} w \c \land \Delta_{\bullet}^1 \to  N_{\bullet} w S_{\bullet} \c$.
\\  \\ Since $\Delta_{\bullet}^1$ is isomorphic to the unit interval and the map $\lambda$ agrees on the endpoints, we can pass to $S^1$ during the suspension. Hence $\lambda$ immediately induces the desired map $\sigma$. {[[This is a tentative explanation offered by Thomas Brazelton.]]}
\end{proof}

\begin{remark}
The axiom (W3) implies that $w\c$ is closed under coproducts, making $|wS_{\bullet} \c|$ into an $H$-space via the map $$\coprod: |wS_{\bullet} \c| \times |wS_{\bullet} \c| \cong |wS_{\bullet} \c \times  wS_{\bullet} \c|\to |wS_{\bullet} \c|.$$
\end{remark}

\begin{definition}
Let $(\c, w\c)$ be a Waldhausen category. Define the \textit{algebraic $K$-theory space} $$K(\c, w) = \Omega | N_{\bullet} wS_{\bullet} \c|.$$ Then we have a right adjoint $\iota: |w \c | \to K(\c, w)$ to the based map $\sigma$.
\\ \\ Moreover,
let $F : (\c, w \c) \to (\d, w\d)$ be an exact functor. Then set $K(F) = \Omega | wS_{\bullet}F| : K(\c, w) \to K(\d, w)$. We have thus defined the \textit{algebraic $K$-theory functor} $K : \mathbf{Wald} \to \mathbf{Top_{\ast}}.$
\end{definition}


Recall that any exact category $\a$ is a Waldhausen category with cofibrations the admissible exact sequences and w.e. the isomorphisms. Waldhausen showed that $|iS_{\bullet}\a|$ (where $i$ denotes the iso category) and $BQ\a$ are homotopy equivalent. Hence our current definition of higher algebraic $K$-theory agrees with Quillen's.


\begin{exmp}
Let $R$ be a ring. Define the \textit{algebraic $K$-theory space of $R$} as $$K(R) = K(\P(R), i)$$ where the w.e. $i$ are precisely the injective $R$-linear maps with projective cokernel and the cofibrations are precisely the $R$-linear maps.
\end{exmp}

\begin{exmp}
Assume that $\c$is a small Waldhausen category where $w\c$ consists of the isomorphisms in $\c$. If $s_n\c$ denotes the set of objects of $S_n \c$, then we get a simplicial set $s_{\bullet} \c$. Waldhausen showed that the inclusion $|s_{\bullet} \c| \hookrightarrow |iS_{\bullet} \c|$ is a homotopy equivalence. This makes $\Omega |s_{\bullet} \c|$ into a so-called simplicial model for $K(\c, w)$.
\end{exmp}

Since $wS_0 \c = \ast$ and every simplex of degree $n >0$ is attached to $\ast$, it follows that the classifying space $|w S_{\bullet} \c|$ is connected. Therefore, we preserve any homotopical information when passing to the loop space.

\begin{definition}
Define the \textit{$i$-th algebraic $K$-group} as $K_i(\c, w) = \pi_iK(\c, w)$ for each $i\geq 0$. 
\end{definition}

\begin{prop}\label{P4}
$\pi_1|w S_{\bullet} \c| \cong K_0(\c, w)$.
\end{prop}

\begin{lemma}
The group $K_0(\c, w)$ is generated by $[X]$ for every $X \in \ob \c$ such that $[X'] + [X''] = [X]$ for every cofiber sequence $X' \rightarrowtail X \twoheadrightarrow X''$ and $[X] = [Y]$ for every w.e. $X \overset{\sim}{\longrightarrow} Y$.
\end{lemma}
\begin{proof}
We compute $\pi_1|N_{\bullet}w S_{\bullet} \c|$ based at the $(0,0)$-bisimplex $\ast$. Notice that $|N_{\bullet}w S_{\bullet} \c|$ has a CW structure {[[this is reasonable visually]]} with $1$-cells the $(0,1)$-bisimplices and $2$-cells the $(0,2)$-bisimplices $X' \rightarrowtail X \twoheadrightarrow X''$ and the $(1,1)$-bisimplices $X \overset{\sim}{\longrightarrow} Y$, which are attached to the $1$-cells $X$ and $Y$. Any cell of dimension $n>2$ is irrelevant to computing $\pi_1$.
\end{proof}

\begin{corollary}
We obtain the functors $K_i : \mathbf{Wald} \to \mathbf{Top_{\ast}} \to \mathbf{Ab}$, called the \textit{algebraic $K$-group functors}.
\end{corollary}
\begin{proof}
By \cref{P4}, we know that $K_i(\c, w) = \pi_{i+1}|w S_{\bullet} \c|$, which is abelian for $i\geq 1$. Moreover, note that if $X' \rightarrowtail X' \vee X'' \twoheadrightarrow X''$ and $X'' \rightarrowtail X' \vee X'' \twoheadrightarrow X'$ are cofiber sequences, then the previous lemma implies that $[X'] + [X''] = [X' \vee X''] = [X'' + X']$. Hence $K_0(\c, w)$ is also abelian.
\end{proof}

\begin{exmp}
Let $X$ be a CW complex and $\mathcal{R}(X)$ denote the category of CW complexes $Y$  obtained from $X$ by attaching at least one cell such that $X$ is a retract of $Y$. Equip this with cofibrations in the form of cellular inclusions fixing $X$ and w.e. in the form of homotopy equivalences. This makes $\mathcal{R}(X)$ into a Waldhausen category. If $\mathcal{R}_f(X)$ denotes the subcategory of those $Y$ obtained by attaching finitely many cells, then we write $A(X)\coloneqq K(\mathcal{R}_f(X))$.
\begin{lemma}
$A_0(X)\cong \Z$.
\end{lemma}
\begin{proof}
Weibel leaves this proof an an exercise.
\end{proof}
\end{exmp}

\begin{definition}
If $\b$ is a Waldhausen subcategory of $\c$, then it is \textit{cofinal in $\c$} is for any $X \in \ob \c$, there is some $X' \in \ob \c$ such that $X \coprod X' \in \ob \b$.
\end{definition}

\begin{theorem}
Let $(\b, w)$ be cofinal in $(\c, w)$ and closed under extensions. Assume that $K_0(\b) = K_0(\c)$. Then $wS_{\bullet}\b \to wS_{\bullet}\c$ is a homotopy equivalence. Therefore, $K_i(\b) \cong K_i(\c)$ for every $i\geq 0$.
\end{theorem}

\end{document}
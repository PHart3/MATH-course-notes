\documentclass[10pt,letterpaper,cm]{nupset}
\usepackage[margin=1in]{geometry}
\usepackage{graphicx}
 \usepackage{enumitem}
 \usepackage{stmaryrd}
 \usepackage{bm}
\usepackage{amsfonts}
\usepackage{amssymb}
\usepackage{pgfplots}
\usepackage{amsmath,amsthm}
\usepackage{lmodern}
\usepackage{tikz-cd}
\usepackage{faktor}
\usepackage{xfrac}
\usepackage{mathtools}
\usepackage{bm}
\usepackage{ dsfont }
\usepackage{mathrsfs}
\usepackage{hyperref}
\hypersetup{colorlinks=true, linkcolor=red,          % color of internal links (change box color with linkbordercolor)
    citecolor=green,        % color of links to bibliography
    filecolor=magenta,      % color of file links
    urlcolor=cyan           }

\usepackage{thmtools}
\usepackage[capitalise]{cleveref} 
    
\theoremstyle{definition}
\newtheorem{definition}{Definition}[section]
\newtheorem{exmp}[definition]{Example}
\newtheorem{non-exmp}[definition]{Non-example}
\newtheorem{note}[definition]{Note}

\theoremstyle{theorem}
\newtheorem{theorem}[definition]{Theorem}
\newtheorem{lemma}[definition]{Lemma}
\newtheorem{corollary}[definition]{Corollary}
\newtheorem{prop}[definition]{Proposition}
\newtheorem{conj}[definition]{Conjecture}
\newtheorem*{claim}{Claim}
\newtheorem{exercise}[definition]{Exercise}

\theoremstyle{remark}
\newtheorem{remark}[definition]{Remark}
\newtheorem*{todo}{To do}
\newtheorem*{conv}{Convention}
\newtheorem*{aside}{Aside}
\newtheorem*{notation}{Notation}
\newtheorem*{term}{Terminology}
\newtheorem*{background}{Background}
\newtheorem*{further}{Further reading}
\newtheorem*{sources}{Sources}

\makeatletter
\def\th@plain{%
  \thm@notefont{}% same as heading font
  \itshape % body font
}
\def\th@definition{%
  \thm@notefont{}% same as heading font
  \normalfont % body font
}
\makeatother

\makeatletter
\renewcommand*\env@matrix[1][*\c@MaxMatrixCols c]{%
  \hskip -\arraycolsep
  \let\@ifnextchar\new@ifnextchar
  \array{#1}}
\makeatother
\pgfplotsset{unit circle/.style={width=4cm,height=4cm,axis lines=middle,xtick=\empty,ytick=\empty,axis equal,enlargelimits,xmax=1,ymax=1,xmin=-1,ymin=-1,domain=0:pi/2}}
\DeclareMathOperator{\Ima}{Im}
\newcommand{\A}{\mathcal A}
\newcommand{\C}{\mathbb C}
\newcommand{\E}{\vec E}
\newcommand{\CP}{\mathbb CP}
\newcommand{\F}{\mathbb F}
\newcommand{\G}{\vec G}
\renewcommand{\H}{\vec H}
\newcommand{\HP}{\mathbb HP}
\newcommand{\K}{\mathbb K}
\renewcommand{\L}{\mathcal L}
\newcommand{\M}{\mathbb M}
\newcommand{\N}{\mathbb N}
\renewcommand{\O}{\mathbf O}
\newcommand{\OP}{\mathbb OP}
\renewcommand{\P}{\mathbf P}
\newcommand{\Q}{\mathbb Q}
\newcommand{\I}{\mathbb I}
\newcommand{\R}{\mathbb R}
\newcommand{\RP}{\mathbb RP}
\renewcommand{\S}{\mathbf S}
\newcommand{\T}{\mathbf T}
\newcommand{\X}{\mathbf X}
\newcommand{\Z}{\mathbb Z}
\newcommand{\B}{\mathcal{B}}
\newcommand{\1}{\mathbf{1}}
\newcommand{\ds}{\displaystyle}
\newcommand{\ran}{\right>}
\newcommand{\lan}{\left<}
\newcommand{\bmat}[1]{\begin{bmatrix} #1 \end{bmatrix}}
\renewcommand{\a}{\vec{a}}
\renewcommand{\b}{\vec b}

\renewcommand{\c}{\mathscr{C}}
\renewcommand{\d}{\mathscr{D}}
\newcommand{\e}{\mathscr{E}}
\newcommand{\y}{\mathscr{Y}}
\renewcommand{\j}{\mathscr{J}}

\newcommand{\h}{\vec h}
\newcommand{\f}{\vec f}
\newcommand{\g}{\vec g}
\renewcommand{\i}{\vec i}
\renewcommand{\k}{\vec k}
\newcommand{\n}{\vec n}
\newcommand{\p}{\vec p}
\newcommand{\q}{\vec q}
\renewcommand{\r}{\vec r}
\newcommand{\s}{\vec s}
\renewcommand{\t}{\vec t}
\renewcommand{\u}{\vec u}
\renewcommand{\v}{\vec v}
\newcommand{\w}{\vec w}
\newcommand{\x}{\vec x}
\newcommand{\z}{\vec z}
\newcommand{\0}{\vec 0}
\DeclareMathOperator*{\Span}{span}
\DeclareMathOperator*{\GL}{GL}
\DeclareMathOperator{\rng}{range}
\DeclareMathOperator{\gemu}{gemu}
\DeclareMathOperator{\almu}{almu}
\newcommand{\Char}{\mathsf{char}}
\DeclareMathOperator{\id}{Id}
\DeclareMathOperator{\im}{im}
\DeclareMathOperator{\graph}{Graph}
\DeclareMathOperator{\gal}{Gal}
\DeclareMathOperator{\tr}{Tr}
\DeclareMathOperator{\nilrad}{nilradical}
\DeclareMathOperator{\norm}{N}
\DeclareMathOperator{\aut}{Aut}
\DeclareMathOperator{\Int}{Int}
\DeclareMathOperator{\ext}{Ext}
\DeclareMathOperator{\stab}{Stab}
\DeclareMathOperator{\orb}{Orb}
\DeclareMathOperator{\inn}{Inn}
\DeclareMathOperator{\out}{Out}
\DeclareMathOperator{\op}{op}
\DeclareMathOperator{\fix}{Fix}
\DeclareMathOperator{\ab}{ab}
\DeclareMathOperator{\sgn}{sgn}
\DeclareMathOperator{\syl}{syl}
\DeclareMathOperator{\Syl}{Syl}
\DeclareMathOperator{\ob}{ob}
\DeclareMathOperator{\mor}{mor}
\DeclareMathOperator{\iso}{iso}
\DeclareMathOperator{\ar}{Ar}
\DeclareMathOperator{\vect}{\mathbf{Vect}}
\DeclareMathOperator{\topp}{\mathrm{top}}
\DeclareMathOperator{\red}{red}
\DeclareMathOperator{\colim}{colim}
\DeclareMathOperator{\ZFC}{ZFC}
\DeclareMathOperator{\set}{\mathbf{Set}}
\DeclareMathOperator{\Ab}{\mathbf{Ab}}
\DeclareMathOperator{\Cmon}{\mathbf{CMon}}
\DeclareMathOperator{\spec}{Spec}
\DeclareMathOperator{\rank}{rank}
\DeclareMathOperator{\rk}{rk}
\DeclareMathOperator{\wh}{Wh}

\linespread{1.3}

% info for header block in upper right hand corner
\name{Perry Hart}
\class{$K$-theory reading seminar}
\assignment{UPenn}
\duedate{October 19, 2018}

%Talk #9

\begin{document}

\begin{abstract}
We begin low-dimensional $K$-theory, which consists of the groups $K_0(-)$, $K_1(-)$, and $K_2(-)$. Specifically, we describe $K_0$ for rings and for topological spaces. The main sources for this talk are the following.
\begin{itemize}
\item $n$Lab.
\item Charles Weibel's \textit{The $K$-book: an introduction to algebraic $K$-theory}, Chapters I and II.
\item Eric M. Friedlander's \textit{An Introduction to $K$-theory}, Chapter 1.
\end{itemize}
\end{abstract}

\smallskip

\section{$K_0$ for rings}

The forgetful functor $U: \Ab \to \Cmon$ admits a left adjoint $K: \Cmon \to \Ab$, called the \textit{group completion} functor. Specifically, for any commutative monoid $\left(C, +\right)$, we call the abelian group $K(C)$ the \textit{Grothendieck group of $C$}, which is constructed as follows.

\medskip

 Consider $S\coloneqq \faktor{C \times C}{\sim}$ where $(a_1, b_1) \sim (a_2, b_2)$ if $$a_1 + b_2 +k = b_1 + a_2 +k$$ for some $k\in C$. Note that $\sim = {\sim'}$ where $(a_1, b_1) \sim' (a_2, b_2)$ if $$\left(a_1 + k_1, b_1 + k_1\right) = \left(a_2 +k_2, b_2 + k_2\right)$$ for some $\left(k_1, k_2\right) \in C\times C$. Now set $K(C) = \left(S, +\right)$, where $+$ is inherited from $C$ and acts componentwise on equivalence classes. Our definition of ${\sim'}$ makes it clear that $\left[a_1, b_1\right]^{{-1}} = \left[b_1, a_1\right]$.

\begin{prop}
The inclusion $C \hookrightarrow K(C)$ given by $x \mapsto \left[x\right]\coloneqq \left[x, 0\right]$ is injective iff $C$ is a cancellation monoid.
\end{prop}

\begin{lemma}[Universal property of $K({-})$] 
Let $B$ be an abelian group and $f: A \to B$ be a monoid homomorphism. Then we have a commutative diagram 
\[
\begin{tikzcd}[column sep=large]
A \arrow[rd, "f"] \arrow[d, hook] &  \\
K(A) \arrow[swap, r, "\exists! \tilde{f}", dashed] & B
\end{tikzcd}.
\]
\end{lemma}
\begin{proof}
Define $\tilde{f}$ by $\left[a_1, b_1\right]\mapsto f(a_1) - f(b_1)$.
\end{proof}

\begin{lemma}\label{L2}
$K(C_1 \times C_2) \cong K(C_1) \times K(C_2)$.
\end{lemma}

\begin{definition}
A submonoid $L$ of $C$ is \textit{cofinal} if for any $c\in C$, there is some $c' \in C$ such that $c + c' \in L$.
\end{definition}

\begin{prop}\label{prop2} Let $L$ be cofinal in a commutative monoid $C$.
\begin{enumerate}
\item Any element of $K(C)$ can be written as $\left[m\right]-\left[n\right]$ for some $m, n\in  C$.
\item $K(L) \leq K(C)$.
\item Any element of $K(C)$ can be written as $\left[m\right] - \left[l\right]$ for some $m\in C$ and $l\in L$.
\item If $[m] = [m']$, then $m + l = m' +l$ for some $l\in L$.
\end{enumerate}
\end{prop}

\begin{exmp}\label{Z} $ $
\begin{enumerate}
\item $K(\N) \cong \Z$ via the mapping $\left[a_1, b_1\right] \mapsto a_1 - b_1$.
\item $K(\Z^{\times}) \cong \Q^{\times}$ via the mapping $\left[a_1, b_1\right] \mapsto \frac{a_1}{b_1}$.
\end{enumerate}
\end{exmp}

\smallskip


Let $R$ be a unital ring. Let $\left(\P(R), \oplus, \otimes_R\right)$ denote the semiring  of (isomorphism classes of) finitely generated projective $R$-modules. Let $K_0(R) = K(\P(R))$.

\begin{lemma}
$\P(R_1 \times R_2) \cong \P(R_1) \times \P(R_2)$.  
\end{lemma}

Therefore, $K_0$ can be computed componentwise by \cref{L2}.

\smallskip

Now, $K_0(-)$ defines a functor from $\mathbf{Ring}$ to $\Ab$. Let $f: R \to S$ be a ring homomorphism and $P$ be a finitely generated projective $R$-module. Define the group map $K_0(f)$ as follows.
\begin{enumerate}[label=(\arabic*)]
\item Construct  the base extension $S\otimes_R P$ of $P$. This is the \textit{unique} $S$-module  compatible with the $R$-module structure on $S$ induced by $f$, and its action is given by $$\left(s', s \otimes p\right) \mapsto  s's \times p.$$ This is also an $R$-module with $f(r) \cdot t \coloneqq r\cdot t$ for $t\in S \otimes_R P$. We know that $P \oplus Q$ is free for some $R$-module $Q$. Since $$S\otimes_R \left(P \oplus Q\right) \cong_S \left(S \otimes_R P\right)\oplus \left(S \otimes_R Q\right)$$ and $P\oplus Q$ is free over $S$ via $f$, it follows that  $S \otimes_R P$ is a finitely generated projective $S$-module.
\item We've just defined a monoid homomorphism $\tilde{f} : \P(R) \to \P(S)$.
\item Apply the universal property of $K$ to find the filler
\[
\begin{tikzcd}
\mathbf{P}(R) \arrow[d, hook] \arrow[r, "\tilde{f}"] & \mathbf{P}(S) \arrow[d, hook] \\
K(\mathbf{P}(R)) \arrow[r, "f_{\ast}"', dotted] & K(\mathbf{P}(S))
\end{tikzcd},
\]
and set $K_0(f) = f_{\ast}$.
\end{enumerate}

\smallskip

\begin{theorem}[Eilenberg swindle]
Suppose $P \oplus Q =R^n$ as $R$-modules. Then $$P \oplus R^{\infty} \cong P \oplus  \left(Q \oplus P\right) \oplus \left(Q \oplus P\right) \oplus \cdots \cong \left(P \oplus Q\right) \oplus \left(P \oplus Q\right) \oplus \cdots \cong R^{\infty}.$$
\end{theorem}

Therefore, if we added $R^{\infty}$ to $\P(R)$, then we would have $\left[P\right] = 0$ for each finitely generated projective $P$.

\smallskip

\begin{exmp}
If $R$ is a field, then $\P(R) \cong \N$ and, by \cref{Z}, $K_0(R) \cong \Z$. 
\end{exmp}

We can generalize this phenomenon a bit.

\begin{definition}
A ring $R$ has the \textit{invariant basis property (IBP)} if $R^n \not \cong R^m$ whenever $n \ne m$. 
\end{definition}

Note that any commutative ring has the IBP.

\begin{definition}
An $R$-module $P$ is \textit{stably free of rank $n-m$} if $P \oplus R^m \cong R^n$.
\end{definition}

\begin{lemma}\label{L4}
The map $f: \N \to \P(R)$ defined by $n \mapsto R^n$ induces a homomorphism $\phi : \Z \to K_0(R)$.
\begin{enumerate}
\item $\phi$ is injective iff $R$ has the IBP.
\item Suppose $R$ has IBP. Then $K_0(R) \cong \Z$ iff every finitely generated projective $R$-module is stably free.
\end{enumerate}
\end{lemma}
\begin{proof} $ $
\begin{enumerate}
\item By \cref{prop2}(4), we know that $\left[P\right] = \left[Q\right]$ in $K_0(R)$ iff $P\oplus R^m \cong Q\oplus R^m$ for some $m$. 
\item $\left[P\right]= \left[R^n\right]$ iff $P$ is stably free.
\end{enumerate}
\end{proof}


\begin{exmp}
Suppose that $R$ is commutative. There is a ring homomorphism $R\to F$ with $F$ a field. Then the induced map $K_0(R) \to K_0(F) \cong \Z$ sends $[R]$ to $1$. Also, the map $\phi : \Z \to K_0(R)$ is injective by \cref{L4}. Letting $K= \ker(K_0(R) \to \Z)$, we get a split exact sequence of abelian groups
\[
\begin{tikzcd}
1 \arrow[r] & K \arrow[r] & K_0(R) \arrow[r] & \mathbb Z \arrow[r] & 1
\end{tikzcd},
\]
so that $K_0(R) \cong \Z \oplus K$.
\end{exmp}

\begin{exmp}
A ring $R$ is a \textit{flasque} if there exist an $R$-bimodule $M$ which is also a finitely generated projective on one side and a bimodule isomorphism $R\oplus M \cong M$. In this case, since $$P \oplus \left(P \otimes_R M\right) \cong P \otimes_R \left(R \oplus M\right) \cong P \otimes_R M,$$ we see that $K_0(R) =0$.
\end{exmp}

\begin{exmp}
A module is \textit{semisimple} if it is the direct sum of simple modules. A ring $R$ is  \textit{semisimple} if it a semisimple $R$-module. Notice that any semisimple module is both Noetherian and Artinian and that any module over a semisimple ring is semisimple. 

 Suppose $R$ is semisimple with summands $V_1, \ldots, V_m$. Then any finitely generated $R$-module has the form $\bigoplus_{i=1}^m V_i^{l_i}$, where each integer $l_i$ is uniquely determined thanks to the Krull-Remak-Schmidt theorem. Hence $\P(R) \cong \N^m$, and $K_0(R) \cong \Z^m$.
\end{exmp}

\begin{exmp}
A ring $R$ is \textit{von Neumann regular} if it satisfies $$\left(\forall{r} \right)\left(\exists{x_r}\right)\left(rx_rr=r\right).$$ As it turns out, any one-sided ideal in $R$ is generated by an idempotent element. Let $\faktor{E}{\sim}$ denote the set of idempotent elements in $R$ modulo the equivalence relation where $e_1 \sim e_2$ if the two generate the same ideal. Then  $\faktor{E}{\sim}$  forms a lattice where the join and meet correspond to the addition and intersection of ideals, respectively.

\smallskip

 Kaplansky (1998) proved that any projective $R$-module is some direct sum of $\left(e\right)$ with $e$ idempotent. It follows that $\faktor{E}{\sim}$ determines $K_0(R)$.
\end{exmp}

\begin{prop}\label{Krull}
Let $R$ be a commutative ring. TFAE
\begin{enumerate}
\item $R_{\red}\coloneqq \faktor{R}{\nilrad(R)}$ is a commutative von Neumann regular ring.
\item $R$ has (Krull) dimension $0$.
\item $\spec(R)$ is compact, Hausdorff, and totally disconnected.
\end{enumerate}
\end{prop}

\begin{lemma}\label{L5}
If $I\subset R$ is nilpotent, then it's not hard to show that $\P\left(\faktor{R}{I}\right) \cong \P(R)$, hence $K_0(R) \cong K_0\left(\faktor{R}{I}\right).$
\end{lemma}

\begin{definition}
Let $R$ be a commutative ring. The \textit{rank} of a finitely generated projective $R$-module $P$ at a prime ideal $\frak{p}$ is the function $$\rk: \spec(R) \to \N,\  \quad \frak{p} \mapsto \dim_{R_{\frak{p}}} (P \otimes R_{\frak{p}}).$$
\end{definition}

\begin{prop} The rank of a finitely generated projective module is
\begin{enumerate}[label=(\alph*)]
\item continuous and
\item a semiring homomorphism.
\end{enumerate}
\end{prop}

\begin{definition}
An $R$-module $M$ is a \textit{componentwise free module} if we have $R = \prod_{i=1}^n R_i$ and $M \cong \prod_{i=1}^n R_i^{c_i}$ for some integers $c_i$.
\end{definition}

Note that $M$ must be projective in this case.

\begin{lemma}\label{L6}
Let $R$ be a commutative ring. The monoid $L$ of finitely generated componentwise free $R$-modules is isomorphic to $\left[\spec(R), \N\right]$.
\end{lemma}
\begin{proof}
Let $f: \spec(R) \to \N$ be continuous. By some point-set topology, we see that $\im f$ is finite, say $\left\{n_1, \ldots, n_c\right\}$. It's also possible to write $R= R_1 \times \cdots \times R_c$. Then $R^f \coloneqq R_1^{n_1} \times \cdots \times R_c^{n_c}$ is a finitely generated componentwise free $R$-module. Moreover, $f \mapsto R^f$ has inverse $\rk$ restricted to componentwise free modules.
\end{proof}


\begin{theorem}[Pierce]
If $R$ is a $0$-dimensional commutative ring, then $K_0(R) \cong \left[\spec(R), \Z\right]$ where $\left[X, Y\right]$ denotes the semiring of continuous maps $f: X\to Y$.
\end{theorem}
\begin{proof}
We have that $R_{\red}$ is a commutative von Neumann regular ring by \cref{Krull}. Any ideal $(d)$ in $R_{\red}$  where $d$ is idempotent is componentwise free. By Kaplansky, every object $X$ of $\P(R)$ is therefore componentwise free. Therefore, 
\begin{gather*}
\P(R_{\red}) \cong \left[\spec(R_{\red}), \N\right]
\\ \Downarrow
\\ K_0(R_{\red}) \cong \left[\spec(R_{\red}), \Z\right]
\end{gather*}
As $\spec(R_{\red})$ is homeomorphic to $\spec(R)$, it follows by \cref{L5} that $$K_0(R)\cong \left[\spec(R_{\red}), \Z\right] \cong \left[\spec(R), \Z\right].$$
\end{proof}

\smallskip

When $R$ is commutative, let $H_0(R) = \left[\spec(R), \Z\right]$. If $R$ is Noetherian, then $H_0(R) \cong \Z^c$ where $c <\infty$ denotes the number of components of $H_0(R)$. If $R$ is a domain, then $H_0(R)$ is connected, implying $H_0(R) \cong \Z$.

The submonoid $L\subset \P(R)$ of componentwise free modules is cofinal, so that $K(L) \leq K_0(R)$. Moreover, $K(L) \cong H_0(R)$ by \cref{L6}.

The rank of a projective module induces a homomorphism $\rank: K_0(R) \to H_0(R)$. Since $\rank(R^f) = f$ for any $R^f \in L$, we see that
\[
\begin{tikzcd}
1 \arrow[r] & H_0(R) \cong K(L) \arrow[r, hook] & K_0(R) \arrow[r, "\rank"] & H_0(R) \arrow[r] & 1
\end{tikzcd}
\]
splits. This implies that $$K_0(R) \cong H_0(R) \oplus \widetilde{K}_0(R),$$ where $\widetilde{K}_0(R)$ denotes $\ker(\rank)$.


\begin{exmp}
The \textit{Whitehead group} of a group $G$ is the quotient $$\wh_0(G) \equiv \faktor{K_0(\Z[G])}{\Z},$$ where $\Z[G]$ denotes the group ring of $G$ over $\Z$. The augmentation map $f: \Z[G] \to \Z$ induces a split exact sequence 
\[
\begin{tikzcd}
1 \arrow[r] & \wh_0(G) \arrow[r] & K_0(\Z[G]) \arrow[r] & \underbrace{K_0(\Z)}_{\Z} \arrow[r] & 1
\end{tikzcd}.
\]
 Hence $K_0(\Z[G]) \cong \Z \oplus \wh_0(G)$. Due to \cref{swan}, if $G$ is finite, then $\wh_0(G) \cong \widetilde{K}_0(\Z[G])$  and $\Z \cong H_0(\Z)$. 
\end{exmp}

\smallskip

\begin{definition} $ $
\begin{enumerate}
\item A category $\c$ is \textit{preadditive} if each of its $\hom$-sets is an abelian group.
\item A functor $F: \c \to \d$ of preadditive categories is \textit{additive} if $F: \c(X, Y) \to \d(FX, FY)$ is a group homomorphism for any $X, Y \in \ob \c$.
\end{enumerate}
\end{definition}

\begin{definition}
The rings $R$ and $S$ are \textit{Morita equivalent} if there exists an additive equivalence between $\mathbf{Mod}_R$ and $\mathbf{Mod}_S$. 
\end{definition}

\begin{theorem}
If $R$ and $S$ are Morita equivalent, then $K_0(R) \cong K_0(S)$.
\end{theorem}

\bigskip



Our results thus far can be extended to symmetric monodical categories because these come equipped with a notion of direct sum that enabled our Grothendieck construction. 

\begin{definition}
A \textit{symmetric monoidal category} $S$ is equipped with a functor $\square : S \times S \to S$, a base object $e$, and four natural isomorphisms expressing commutativity, associativity, and the property that $e$ acts as an identity. These four isomorphisms must also satisfy certain coherence properties.
\end{definition}

\begin{exmp} The following are examples of a symmetric monoidal category.	.
\begin{enumerate}
\item Any $k$-vector space where $\square \coloneqq \otimes_k$.
\item Any category with finite coproducts where $s \square t\coloneqq s \amalg t$.
\item The category of pointed topological spaces where $s \square t \coloneqq s \wedge t $ and $e\coloneqq S^0$.
\end{enumerate} 
\end{exmp}

Suppose that the class of isomorphism classes of objects of a category $S$ is a set and denote it by $S^{\iso}$. If $S$ is symmetric monoidal, then $\left(S^{\iso}, \square\right)$ is an abelian monoid with identity element $e$. In this case, we define the \textit{Grothendieck group of $S$} as $K_0(S)$.

\section{Topological $K$-theory}

\begin{notation}
$\F$ stands for either $\R$ or $\C$.
\end{notation}

\begin{definition}
Let $f: F \to X$ and $g: G \to X$ be vector bundles. 
\begin{enumerate}
\item The \textit{Whitney sum} of $f$ and $g$ is the vector bundle $F \oplus G$ on $X$ whose fiber at $x \in X$ is precisely $F_x \oplus G_x$.  
\item The \textit{tensor product bundle} $F \otimes G$ is defined similarly.
\end{enumerate}
\end{definition}

\begin{definition}
A \textit{vector bundle homomorphism} from $\phi : E_1 \to X_1$ to $\psi : E_2 \to X_2$ is a pair of maps $f: E_1 \to E_2$ and $g: X_1 \to X_2$ such that
\begin{enumerate}[label=(\roman*)]
\item the square
\[
\begin{tikzcd}
E_1 \arrow[r, "f"] \arrow[d, "\phi"'] & E_2 \arrow[d, "\psi"] \\
X_1 \arrow[r, "g"'] & X_2
\end{tikzcd}
\] commutes and
\item for each $x \in X_1$, the map $f \restriction_{\phi^{{-1}}(x)} : \phi^{{-1}}(x) \to \psi^{{-1}}(g(x))$ is linear.
\end{enumerate}
\end{definition}

\begin{definition}[Topological $K$-groups]
Let $\left(\vect_{\F}(X), \oplus\right)$ denote the abelian monoid of (isomorphism classes of) $\F$-vector bundles on a paracompact space $X$. 
\begin{itemize}
\item $KU(X) \equiv K(\vect_{\C}(X))$ 
\item $KO(X) \equiv K(\vect_{\R}(X)).$
\end{itemize}
\end{definition}

Note that these are commutative rings with identity. 

\smallskip

\begin{center}
 \textbf{We apply the notation $K_{\topp}(-)$ to topological spaces when we wish to omit the base field.}
\end{center}

\medskip

Both $KU(-)$ and $KO(-)$ define contravariant functors $\mathbf{Top} \to \Ab$. Let $f: Y \to X$ be a map of spaces and $\phi : E \to X$ be a vector bundle. Recall the pullback $f^{\ast}E = \left\{\left(y, e\right) \in Y \times E : f(y) = \phi(e)\right\}$ of $E$ in $\mathbf{Top}$.  Define the vector bundle $f^{\ast}(\phi) : f^{\ast}E \to Y$ as the appropriate restriction of the projection map $\pi : Y \times E \to Y$. The assignment $\phi \mapsto f^{\ast}(\phi)$ defines a morphism  $\vect_{\F}(X) \to \vect_{\F}(Y)$ of monoids. In turn, the universal property of $K$ induces a unique morphism $f^{\ast}: K_{\topp}(X) \to K_{\topp}(Y)$.


\begin{lemma}
If $X$ and $Y$ are homotopy equivalent, then $K(X) \cong K(Y)$.
\end{lemma}
\begin{proof}
Apply the homotopy invariance theorem (HIT), which states that if $Y$ is paracompact and $f, g: Y \to X$ are homotopic, then $f^{\ast}E \cong g^{\ast}E$ for any vector bundle $E$ over $X$.
\end{proof}

\begin{exmp} $ $
\begin{enumerate}
\item $K_{\topp}(\ast) = \Z$.
\item If $X$ is contractible, then the HIT implies that $KO(X) = KU(X) = \Z$
\item According to I.4.9 of \textit{The K-book}, we have
\begin{align*}
KO(S^1) & \cong \Z \times C_2 
\\ KU(S^1) &  \cong \Z \\
KO(S^2) & \cong \Z \times C_2  
\\  KU(S^2) & \cong \Z \times \Z
\\ KO(S^3) & \cong KU(S^3) \cong \Z 
\\ KO(S^4) & \cong KU(S^4) \cong \Z \times \Z
\end{align*}
\end{enumerate}
\end{exmp}


\begin{definition}
The \textit{dimension} of a vector bundle $E$ over $X$ is the continuous homomorphism $\widehat{\dim}(E) : X \to \N$ given by $x \mapsto \dim(E_x)$.
\end{definition}

\begin{definition}
A vector bundle $p: E \to X$ is a \textit{componentwise trivial bundle} if  $X =\coprod_{i\in S} X_i$ where $S$ is a set, each $X_i$ is a clopen component of $X$, and $p\restriction_{p^{{-1}}(X_i)}$ is trivial.
In this case, if $S$ is finite, then we say that $E$ has \textit{finite type}.
\end{definition}

\begin{lemma}
The submonoid of componentwise trivial bundles over $X$ is isomorphic to $\left[X, \N\right]$.
\end{lemma}
\begin{proof}
Send a given map $f: X \to \N$ to $T^f \coloneqq \coprod_{i \in \N}\left(f^{{-1}}(i) \times \F\right)$. Conversely, if $E$ is a componentwise trivial bundle, then $E \cong T^{\widehat{\dim}(E)}$.
\end{proof}


Thus, the submonoid  of trivial bundles and the submonoid of componentwise trivial bundles are naturally isomorphic to $\N$ and $\left[X, \N\right]$, respectively.  When $X$ is compact, these are cofinal in $\vect_{\F}(X)$ thanks to the following theorem (proven using Riemannian geometry). 

\begin{theorem}[Subbundle]
Let $p: E \to X$ be a vector bundle such that $X$ is paracompact.
\begin{enumerate}[label=(\alph*)]
\item For any subbundle $F$ of $E$, there is a subbundle $F^{\perp}$ of $E$ such that $E \cong F \oplus F^{\perp}$.
\item $E$ has finite type if and only if there is another bundle $E'$ such that $E\oplus E'$ is trvial.
\end{enumerate}
\end{theorem}

We now can deduce that $$\Z \leq \left[X, \Z\right] \leq K_{\topp}(X).$$

\medskip

\begin{note} $ $
\begin{enumerate}
\item We have a split exact sequence. 
\[
\begin{tikzcd}
1 \arrow[r] & \widetilde{K}_{\topp}(X) \arrow[r] & K_{\topp}(X) \arrow[r, "\widehat{\dim}"'] & {[X, \mathbb Z]} \arrow[r] \arrow[l, hook, bend right] & 1
\end{tikzcd},
\] where $\widetilde{K}_{\topp}(X)$ denotes $\ker\left(\widehat{\dim}\right)$.
\item The map of monoids $\vect_{\R}(X) \to \vect_{\C}(X)$ given by $\left[E\right] \mapsto \left[E \otimes \C\right]$ extends by universality to a homomorphism $KO(X) \to KU(X)$. Likewise, the forgetful functor $\vect_{\C}(X) \to \vect_{\R}(X)$ extends to a homomorphism $KU(X) \to KO(X)$.
\end{enumerate}
\end{note}

\medskip

Finally, to state a nice early connection between algebraic and topological $K$-theory, let $X$ be a compact Hausdorff space and $\mathcal C(X, \F)$ denote the ring of continuous functions $X \to \F$. For any vector bundle $p: E \to X$ over $\F$, set $$\Gamma(X, E) = \left\{s: X \to E : p \circ s = \id_X\right\},$$ the vector space of global sections of $E$.

\begin{theorem}[Swan]\label{swan}
The mapping $E \mapsto \Gamma(X, E)$ induces isomorphisms $KO(X) \cong K_0(\mathcal C(X, \R))$ and $KU(X) \cong K_0( \mathcal C(X, \C))$.
\end{theorem}



\end{document}
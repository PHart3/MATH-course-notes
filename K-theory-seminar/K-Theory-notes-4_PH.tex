\documentclass[10pt,letterpaper,cm]{nupset}
\usepackage[margin=1in]{geometry}
\usepackage{graphicx}
 \usepackage{enumitem}
 \usepackage{stmaryrd}
 \usepackage{bm}
\usepackage{amsfonts}
\usepackage{amssymb}
\usepackage{pgfplots}
\usepackage{amsmath,amsthm}
\usepackage{lmodern}
\usepackage{tikz-cd}
\usepackage{faktor}
\usepackage{xfrac}
\usepackage{mathtools}
\usepackage{bm}
\usepackage{ dsfont }
\usepackage{mathrsfs}
\usepackage{hyperref}
\hypersetup{colorlinks=true, linkcolor=red,          % color of internal links (change box color with linkbordercolor)
    citecolor=green,        % color of links to bibliography
    filecolor=magenta,      % color of file links
    urlcolor=cyan           }

\usepackage{thmtools}
\usepackage[capitalise]{cleveref} 
    
\theoremstyle{definition}
\newtheorem{definition}{Definition}[section]
\newtheorem{exmp}[definition]{Example}
\newtheorem{non-exmp}[definition]{Non-example}
\newtheorem{note}[definition]{Note}

\theoremstyle{theorem}
\newtheorem{theorem}[definition]{Theorem}
\newtheorem{lemma}[definition]{Lemma}
\newtheorem{corollary}[definition]{Corollary}
\newtheorem{prop}[definition]{Proposition}
\newtheorem{conj}[definition]{Conjecture}
\newtheorem*{claim}{Claim}
\newtheorem{exercise}[definition]{Exercise}

\theoremstyle{remark}
\newtheorem{remark}[definition]{Remark}
\newtheorem*{todo}{To do}
\newtheorem*{conv}{Convention}
\newtheorem*{aside}{Aside}
\newtheorem*{notation}{Notation}
\newtheorem*{term}{Terminology}
\newtheorem*{background}{Background}
\newtheorem*{further}{Further reading}
\newtheorem*{sources}{Sources}

\makeatletter
\def\th@plain{%
  \thm@notefont{}% same as heading font
  \itshape % body font
}
\def\th@definition{%
  \thm@notefont{}% same as heading font
  \normalfont % body font
}
\makeatother

\makeatletter
\renewcommand*\env@matrix[1][*\c@MaxMatrixCols c]{%
  \hskip -\arraycolsep
  \let\@ifnextchar\new@ifnextchar
  \array{#1}}
\makeatother
\pgfplotsset{unit circle/.style={width=4cm,height=4cm,axis lines=middle,xtick=\empty,ytick=\empty,axis equal,enlargelimits,xmax=1,ymax=1,xmin=-1,ymin=-1,domain=0:pi/2}}
\DeclareMathOperator{\Ima}{Im}
\newcommand{\A}{\mathcal A}
\newcommand{\C}{\mathbb C}
\newcommand{\E}{\vec E}
\newcommand{\CP}{\mathbb CP}
\newcommand{\F}{\mathbb F}
\newcommand{\G}{\vec G}
\renewcommand{\H}{\mathbb H}
\newcommand{\HP}{\mathbb HP}
\newcommand{\K}{\mathbb K}
\renewcommand{\L}{\mathcal L}
\newcommand{\M}{\mathbb M}
\newcommand{\N}{\mathbb N}
\renewcommand{\O}{\mathbf O}
\newcommand{\OP}{\mathbb OP}
\renewcommand{\P}{\mathbf P}
\newcommand{\Q}{\mathbb Q}
\newcommand{\I}{\mathbb I}
\newcommand{\R}{\mathbb R}
\newcommand{\RP}{\mathbb RP}
\renewcommand{\S}{\mathbf S}
\newcommand{\T}{\mathbf T}
\newcommand{\X}{\mathbf X}
\newcommand{\Z}{\mathbb Z}
\newcommand{\B}{\mathcal{B}}
\newcommand{\1}{\mathbf{1}}
\newcommand{\ds}{\displaystyle}
\newcommand{\ran}{\right>}
\newcommand{\lan}{\left<}
\newcommand{\bmat}[1]{\begin{bmatrix} #1 \end{bmatrix}}

\renewcommand{\a}{\mathscr{A}}
\renewcommand{\b}{\mathscr{B}}
\renewcommand{\c}{\mathscr{C}}
\renewcommand{\d}{\mathscr{D}}
\newcommand{\e}{\mathscr{E}}
\newcommand{\y}{\mathscr{Y}}
\renewcommand{\j}{\mathscr{J}}

\newcommand{\h}{\vec h}
\newcommand{\f}{\vec f}
\newcommand{\g}{\vec g}
\renewcommand{\i}{\vec i}
\renewcommand{\k}{\vec k}
\newcommand{\n}{\vec n}
\newcommand{\p}{\vec p}
\newcommand{\q}{\vec q}
\renewcommand{\r}{\vec r}
\newcommand{\s}{\vec s}
\renewcommand{\t}{\vec t}
\renewcommand{\u}{\vec u}
\renewcommand{\v}{\vec v}
\newcommand{\w}{\vec w}
\newcommand{\x}{\vec x}
\newcommand{\z}{\vec z}
\newcommand{\0}{\vec 0}
\DeclareMathOperator*{\Span}{span}
\DeclareMathOperator*{\GL}{GL}
\DeclareMathOperator*{\SL}{SL}
\DeclareMathOperator*{\SO}{SO}
\DeclareMathOperator*{\SU}{SU}
\DeclareMathOperator{\rng}{range}
\DeclareMathOperator{\gemu}{gemu}
\DeclareMathOperator{\almu}{almu}
\newcommand{\Char}{\mathsf{char}}
\DeclareMathOperator{\id}{Id}
\DeclareMathOperator{\im}{im}
\DeclareMathOperator{\graph}{Graph}
\DeclareMathOperator{\gal}{Gal}
\DeclareMathOperator{\tr}{Tr}
\DeclareMathOperator{\norm}{N}
\DeclareMathOperator{\aut}{Aut}
\DeclareMathOperator{\Int}{Int}
\DeclareMathOperator{\ext}{Ext}
\DeclareMathOperator{\stab}{Stab}
\DeclareMathOperator{\orb}{Orb}
\DeclareMathOperator{\inn}{Inn}
\DeclareMathOperator{\out}{Out}
\DeclareMathOperator{\op}{op}
\DeclareMathOperator{\fix}{Fix}
\DeclareMathOperator{\ab}{ab}
\DeclareMathOperator{\cone}{cone}
\DeclareMathOperator{\sgn}{sgn}
\DeclareMathOperator{\syl}{syl}
\DeclareMathOperator{\Syl}{Syl}
\DeclareMathOperator{\ob}{ob}
\DeclareMathOperator{\mor}{mor}
\DeclareMathOperator{\iso}{iso}
\DeclareMathOperator{\ar}{Ar}
\DeclareMathOperator{\co}{\mathtt{co}}
\DeclareMathOperator{\red}{red}
\DeclareMathOperator{\colim}{colim}
\DeclareMathOperator{\ZFC}{ZFC}
\DeclareMathOperator{\set}{\mathbf{Set}}
\DeclareMathOperator{\Ab}{\mathbf{Ab}}
\DeclareMathOperator{\Cmon}{\mathbf{CMon}}
\DeclareMathOperator{\spec}{Spec}
\DeclareMathOperator{\rank}{rank}
\DeclareMathOperator{\rk}{rk}
\DeclareMathOperator{\st}{St}
\DeclareMathOperator{\wh}{Wh}
\DeclareMathOperator{\diag}{diag}
\DeclareMathOperator{\Mod}{\mathbf Mod}

\linespread{1.3}

% info for header block in upper right hand corner
\name{Perry Hart}
\class{$K$-theory seminar}
\assignment{Talk \#10}
\duedate{October 24, 2018}

\begin{document}

\begin{abstract}
We continue to look at low-dimensional $K$-theory, finishing our description of $K_0(-)$ and then defining $K_1(-)$, and $K_2(-)$ for rings. The main sources for this talk are the following.
\begin{itemize}
\item $n$Lab.
\item Charles Weibel's \textit{The $K$-book: an introduction to algebraic $K$-theory}, Chapters II and III.
\item Eric M. Friedlander's \textit{An Introduction to $K$-theory}, Chapter 1.
\item \url{http://people.math.harvard.edu/~lurie/281notes/Lecture3-Whitehead.pdf}.
\end{itemize}
\end{abstract}

\smallskip

\section{$K_0$ of a Waldhausen category}

\begin{definition}
Let $\c$ be a category equipped with a ``subcategory'' $\co{\c}$ of morphisms called \textit{cofibrations}. The pair $\left(\c, \co\right)$ is a \textit{category with cofibrations} if the following conditions hold.
\begin{enumerate}
\item (W0) Every isomorphism in $\c$ is a cofibration.
\item (W1) There is a base point $0$ in $\c$ such that the unique morphism $0 \rightarrowtail A$ for every $A \in \ob \c$.
\item (W2) We have
\[
\begin{tikzcd}
A \arrow[d] \arrow[r, tail] & B \arrow[d, dotted] \\
C \arrow[r, dotted, tail] & B \cup_A C
\end{tikzcd}.
\]
\end{enumerate}
\end{definition}


Note that $B \coprod C$ always exists as the pushout $B \cup_0 C$ and that the cokernel of any $i : A \rightarrowtail B$ exists as $B \cup_A 0$ along $A \to 0$. We call $A \rightarrowtail  B \twoheadrightarrow \faktor{B}{A}$ a \textit{cofibration sequence}.


\begin{definition}
A \textit{Waldhausen category} $\c$ is a category with cofibrations together with a subcategory $w(\c)$ of morphisms called \textit{weak equivalences} such that every isomorphism in $\c$ is a w.e. and the following ``glueing axiom'' holds.
\begin{enumerate}
\item (W3) For any diagram
\[
\begin{tikzcd}
C \arrow[d, "\sim"'] & A \arrow[d, "\sim"'] \arrow[r, tail] \arrow[l] & B \arrow[d, "\sim"'] \\
C' & A' \arrow[r, tail] \arrow[l] & B'
\end{tikzcd}, \]
the induced map $B \cup_A C \to  B' \cup_{A'} C'$ is a w.e.
\end{enumerate}
\end{definition}

\begin{definition}
Let $\c$ be a Waldhausen category. Define $K_0(\c)$ as the abelian group generated by $[C]$ for each object $C$ of $\c$ such that
\begin{enumerate}
\item $[C] = [C']$ if there some w.e. from $C$ to $C'$
\item $[C] = [B] + \left[\faktor{C}{B}\right]$ for every $B \rightarrowtail  C \twoheadrightarrow \faktor{C}{B}$
\item The weak equivalence classes of objects in $\c$ is a set.
\end{enumerate}
\end{definition}

\begin{prop} $ $
\begin{enumerate}
\item $[0] = 0$.
\item $\left[B \coprod C\right] = [B] +[C]$.
\item $\left[B \cup_A C\right] = [B]+[C]-[A]$.
\item $[C]= 0$ whenever $0 \simeq C$.
\end{enumerate}
\end{prop}

\begin{exmp}
Let $\mathcal{R}_f(\ast)$ denote the category of finite CW complexes. Here, cofibrations and weak equivalences correspond to cellular inclusions  and weak homotopy equivalences, respectively. It is known that $K_0(\mathcal{R}_f) \cong \Z$.
\end{exmp}

\begin{definition}
if $\c$ and $\d$ are Waldhausen, then a functor $F: \c \to \d$ is \textit{exact} if
\begin{enumerate}[label=(\alph*)] 
\item preserves base points, cofibrations, and weak equivalences and 
\item for any $A \rightarrowtail B$, $FB \cup_{FA} FC \to F(B\cup_A C)$ is an isomorphism. 
\end{enumerate}
\end{definition}

Note that $F$ induces a group map $K_0(F) :K_0(\c) \to K_0(\d)$.

\begin{theorem}
Let $F : \a \to \b$ be an exact functor. Assume the following.
\begin{enumerate}[label=(\arabic*)]
\item A morphism $f$ is a w.e. iff $F(f)$ is a w.e.
\item For any morphism $b : FA \to B$ in $\b$, there is some $a: A \rightarrowtail A'$ in $\a$ and a w.e. $b' : FA' \overset{\sim}{\longrightarrow} B$ in $\b$ such that $b = b' \circ F(a)$. Moreover, we may choose $a$ to be a w.e. whenever $b$ is a w.e.
\end{enumerate}
Then $F$ induces an isomorphism $K_0(\a) \cong K_0(\b)$.
\end{theorem}
\begin{proof}
Apply condition (2) to any $0 \rightarrowtail B$ to get $FA' \overset{\sim}{\longrightarrow} B$. If this is a w.e., then there is some $A \overset{\sim}{\longrightarrow} A'$. Hence there is a bijection between the set $W$ of w.e. classes of objects of $\a$ and that in $\b$. 
\medskip

 The group $K_0(\b)$ is given by the free abelian group $\Z[W]$ modulo the relation $$[C] = [B] + \left[\faktor{C}{B}\right].$$ Let $FA \overset{\sim}{\longrightarrow}  B$. Then applying condition (2) yields the diagram
\[
\begin{tikzcd}
0 \arrow[d, no head] \arrow[d, no head] & FA \arrow[d, "\sim"'] \arrow[r, tail] \arrow[l] & FA' \arrow[d, "\sim"'] \\
0 & B \arrow[r, tail] \arrow[l] & C
\end{tikzcd}.
\]
Apply the glueing axiom to see that $F\left(\faktor{A'}{A}\right) \to \faktor{C}{B}$ is a w.e. Hence $[C] = [B] + \left[\faktor{C}{B}\right]$
 holds iff $[A'] = [A] + \left[\faktor{A'}{A}\right]$ holds.
\end{proof}


\section{$K_1$ for rings}


Let $R$ be a unital ring. Recall that direct limits in $\Mod_R$ always exist. Let $$K_1(R) = \GL(R)^{\ab}$$ where $\GL(R) \equiv\colim_{n \in \N} \GL(n, R)$.

\begin{note}[Universal property of $K$]
The universal property of $\ab: \mathbf{Grp} \to \Ab$ induces the universal property of $K_1$ that any homomorphism $f: \GL(R) \to H$ with $H$ abelian has $f = g \circ \pi$ for some unique $g: K_1(R) \to H$.
\end{note}

\begin{prop}
Any ring map $f: R \to S$ induces a natural map $\GL(R) \to \GL(S)$. Hence $K_1$ is a functor $\mathbf{Rng} \to \Ab$.
\end{prop}


Thanks to Whitehead, we know that the commutator subgroup $\left[\GL(R), \GL(R)\right]$ is equal to $E(R) = \bigcup_n E_n(R)$, the group of elementary matrices $E_{i, j}(r)$ where $r \in R$ and $i\ne j$. Thus, $K_1(R)$ can be viewed as the ``stabilized'' group of automorphisms of the trivial projective module modulo trivial automorphisms.


\begin{exmp}
If $F$ is a field, then $K_1(F) = F^{\times}$.
\end{exmp}
\begin{proof}
It is each to check that $E_n(F) \cong \SL_n(F)$ for any $n\in \N$. Therefore, $E(F) \cong \SL(R)$.
\end{proof}

\begin{prop}
Suppose $R$ is commutative. Consider the sequence $R^{\times} \cong \GL(1, R) \to \GL(R) \to K_1(R)$. This induces a natural split exact sequence.
\[
\begin{tikzcd}
1 \arrow[r] & SK_1(R) \arrow[r, hook] & K_1(R) \arrow[r, "\det"] & R^{\times} \arrow[r] & 1,
\end{tikzcd}
\]
where $SK_1(R)$ denotes $\ker(\det)$.
\end{prop}

This means that $K_1(R) \cong R^{\times} \times SK_1(R)$.

\begin{exmp}
Suppose $R$ is a Euclidean domain. Then $SK_1(R) =1$, so that $K_1(R) \cong R^{\times}$.
\end{exmp}

\begin{lemma}
Let $D$ be a division ring. Then $K_1(D) \cong \faktor{\GL_n(D)}{E_n(D)}$ for any $n\geq 3$.
\end{lemma}
\begin{proof}
Any invertible matrix over $D$ is reducible (a la Gaussian elimination) to a diagonal matrix of the form $(r, 1, \ldots, 1)$. Moreover, $E_n(D)\unlhd \GL_n(D)$ for each $n$. In particular, Dieudonn\'e (1943) showed that  $\faktor{\GL_n(D)}{E_n(D)} \cong \faktor{D^{\times}}{(D^{\times})'}$ for any $n\ne 2$. 
\end{proof}

\smallskip

Now, suppose that $R$ is Noetherian of dimension $d$, so that $E_n(R)\unlhd \GL_n(R)$ for any $n\geq d+2$. 

\begin{prop}[Vaserstein]
$K_1(R) \cong \faktor{\GL_n(R)}{E_n(R)}$ for any $n \geq d+2$. 
\end{prop}

\medskip


Let $D$ be a $d$-dimensional division algebra over the field $F\coloneqq Z(D)$. We know that $d =n^2$ for some integer $n$. By Zorn's lemma, there is some maximal subfield $E\subset D$ such that $[E : F] = n$. Then $D \otimes_F E \cong M_n(E)$, where $M_n$ denotes the $n$-dimensional matrix ring over $E$. Any field with this property is called a \textit{splitting field} for $D$.

Let $E'$ be a splitting field for $D$. For any $r \in \N$, the inclusions $D \hookrightarrow M_n(E')$ and $M_r(D) \hookrightarrow M_{nr}(E')$ induce maps $D^{\times}\subset \GL_n(E') \overset{\det}{\longrightarrow} (E')^{\times}$ and $\GL_r(D)\to \GL_{nr}(E')  \overset{\det}{\longrightarrow} (E')^{\times}$ whose images are contained in $F^{\ast}$. The induced maps are called the \textit{reduced norms} $N_{\red}$ for $D$.

\begin{exmp}
If $D = \H$, then $N_{\red}$ is the square of the usual norm. It induces an isomorphism $K_1(\H) \cong \R_+^{\times}$.
\end{exmp}

\medskip

Let $R$ be a commutative Banach algebra over $\mathbb{F} \in \left\{\R, \C\right\}$ (i.e., a Banach space equipped with a commutative bilinear multiplication map $m : R \times R \to R$ such that $\left\lVert{m(a,b)}\right\rVert \leq \left\lVert{a}\right\rVert\cdot \left\lVert{b}\right\rVert$). Recall that both $\GL_n(R)$ and $\SL_n(R)$ are topological groups as subspaces of $\R^{n^2}$. 

\begin{prop}\label{prev}
We have that $E_n(R)$ is the path component of the identity matrix $I_n$ for any $n\geq 2$.
\end{prop}

\begin{corollary}
We may identify $SK_1(R)$ with the set $\pi_0\SL(R)$. 
\end{corollary}
\begin{proof}
Note that $E(R)\leq \SL_(R)$. By the third isomorphism theorem, we get $$ \faktor{\GL(R)}{E(R)}\big / \faktor{\SL(R)}{E(R)} \cong \faktor{\GL(R)}{\SL(R)}.$$ Thus, we get the short exact sequence 
\[
\begin{tikzcd}
1 \arrow[r] & \faktor{\SL(R)}{E(R)} \arrow[r] & \faktor{\GL(R)}{E(R)}\cong K_1(R) \arrow[r] & \faktor{\GL(R)}{\SL(R)} \cong R^{\times} \arrow[r] & 1
\end{tikzcd}
\]
By \cref{prev}, we know that $\faktor{\SL(R)}{E(R)} \cong \pi_0\SL(R)$, yielding a short exact sequence.
\[
\begin{tikzcd}
1 \arrow[r] & \pi_0\SL(R) \arrow[r] & K_1(R) \arrow[r, "\det"] & R^{\times} \arrow[r] & 1
\end{tikzcd}.
\]
\end{proof}
\begin{exmp} $ $
If $X$ is compact, then 
\begin{align*}
SK_1(\R^X) &\leftrightarrow \left[X, \SL(\R)\right] \cong \left[X, \SO\right]
\\ SK_1(\C^X) &\leftrightarrow \left[X, \SL(\C)\right] \cong \left[X, \SU\right].
\end{align*}
In particular, $SK_1(\R^{S^1}) \leftrightarrow \pi_1 \SO \cong C_2$.
\end{exmp}

\smallskip


Let $P$ be a finitely generated projective $R$-module. Any choice of isomorphism $P \oplus Q \cong R^n$ induces a group map $$\aut(P) \to \aut(P) \oplus \aut(Q) \cong \aut(R^n) \cong \GL(n,R).$$ The group map $\aut(P) \to \GL(R)$ is independent of our choice of isomorphism up to inner automorphism of $\GL(R)$. Therefore, there is a well-defined homomorphism $\Phi: \aut(R) \to K_1(R)$.


\begin{lemma}
Suppose that $R$ is commutative and $T$ is an $R$-algebra. Then $K_1(T)$ has a natural module structure over $K_0(R)$.
\end{lemma}
\begin{proof}
For any $P \in \P(R)$ and $m\in \N$, consider the homomorphism $\Phi : \aut(P \otimes T^m) \to K_1(R\otimes T).$ For any $\beta \in \GL_m(T)$, let $$\left[P\right] \cdot \beta = \Phi(1_P \otimes \beta).$$This action factors through $K_0(R)$ and $K_1(T)$, inducing an operation $K_0(R) \times K_1(T) \to K_1(R \otimes S)$. Now, since $T$ is an $R$-algebra, there is a ring map $R\otimes T \to T$. The induced composite $K_0(R) \times K_1(T) \to K_1(R \otimes T) \to K_1(T)$ is the desired module structure.
\end{proof}

\medskip

As it turns out, $K_1(R)$ is completely determined by the category $\P(R)$. This means that $K_1$ is invariant under Morita equivalence, just as $K_0$ is.

\begin{theorem}
 if $R$ and $S$ are Morita equivalent, then $K_1(R) \cong K_1(R)$. 
\end{theorem}

\medskip


For an application of $K_1$ to manifold theory, let $\pi$ be a finitely generated group. Define the \textit{Whitehead group} $\wh(\pi)$ of $\pi$ as the cokernel of the map $\pi \times \{\pm 1\}\to K_1(\Z\pi)$ given by $\left(g, \pm 1\right)\mapsto \begin{bmatrix} \pm g\end{bmatrix}$.

\begin{definition} Suppose that $W$, $M$, and $N$ are compact manifolds (possibly smooth or piecewise-linear). Suppose that $M$ and $N$ are without boundary. Let $\dim(M)=\dim(N) =n$ and $\dim(W) =n+1$.
\begin{enumerate}
\item We say that $W$ is a \textit{cobordism of $M$ and $N$} if $\partial{W}\cong M \coprod N$.
\item We say that $W$ is an \textit{$h$-cobordism of $M$ and $N$} if it is a cobordism of $M$ and $N$ and the inclusion maps $i_M : M \hookrightarrow \partial{W}$ and $i_N: N\hookrightarrow \partial{W}$ are homotopy equivalences.
\end{enumerate}
\end{definition}

Let $R$ be a ring. A \textit{based chain complex over $R$} is a bounded chain complex
\[
\begin{tikzcd}
\cdots \arrow[r] & F_{n+1} \arrow[r, "d_{n+1}"] & F_n \arrow[r, "d_n"] & F_{n-1} \arrow[r] & \cdots
\end{tikzcd}
\] of finitely generated free $R$-modules together with a choice $B_n$ of basis (ordered in a predetermined way) for each $F_n$. The \textit{Euler characteristic of $\left(F_{\ast}, d_n\right)$} is the finite sum
\[
\chi(F_{\ast}) \equiv \sum_{n}({-1})^n\left\lvert{B_n}\right\rvert.
\] If $F_{\ast}$ is acyclic, then it is contractible, so that there is some map $h:F_{\ast} \to F_{\ast +1}$ such that $dh+hd =\id_{F_{\ast}}$. In this case, one can check that
\[
d+h : \bigoplus_nF_{2n} \to \bigoplus_nF_{2n+1}.
\] is an isomorphism of free $R$-modules. If $\chi(F_{\ast}) =0$, then this yields an element $\underbrace{\rho(F_{\ast}) \coloneqq \left[d+h\right]}_{\textit{Reidemeister torsion}}$ of $\faktor{K_1(R)}{\left\{\pm 1\right\}}$, which is independent of our choice of null-homotopy $h$. 

\smallskip

Suppose that $f: X_{\ast} \to Y_{\ast}$ is a quasi-isomorphism of based chain complexes over $R$. Then $\cone(f)$ is an acyclic based chain complex over $R$. Further, if $\chi(X_{\ast}) = \chi(Y_{\ast})$, then $\chi(\cone(f)) =0$, in which case we may  define the \textit{torsion of $f$} as the element $\rho(\cone(f))$ of $\faktor{K_1(R)}{\left\{\pm 1\right\}}$.

Now, suppose that $f: X \to Y$ is a homotopy equivalence of finite connected CW complexes. Since these are locally contractible, they admit respective universal covering spaces $\widetilde{X}$ and $\widetilde{Y}$. If $f$ is a cellular map, then it induces a map
\[
\lambda_f: C_{\ast}(\widetilde{X}; \Z) \to C_{\ast}(\widetilde{Y}; \Z)
\] of cellular chain complexes, which must be a quasi-isomorphism since $f$ is assumed to be a homotopy equivalence. Note that $C_{\ast}(\widetilde{X}; \Z)$ and $C_{\ast}(\widetilde{Y}; \Z)$ may be viewed as based chain complexes over $\Z{\pi_1(Y)}$. In this case, the \textit{Whitehead torsion $\tau(f)$ of $f$} is the image of the torsion of $\lambda_f$ under the natural projection $\faktor{K_1(\Z{\pi_1(Y)})}{\left\{\pm 1\right\}} \twoheadrightarrow \wh(\Z{\pi_1(Y)})$.

\begin{theorem}[$s$-cobordism]
Suppose that $W$, $M$, and $N$ are compact manifolds and that $W$ is an $h$-cobordism of $M$ and $N$. If $\dim(M) \geq 5$, then $\left(W, M, N\right) \cong \left(M \times \left[0, 1\right], M \times 0, M \times 1\right)$ iff $\tau(i_M)$ vanishes.
\end{theorem}

\begin{corollary}[Generalized Poincar\'e conjecture]
Let $M$ be an $n$-manifold that is homotopy equivalent to $S^n$. If $n\geq 5$, then $M$ is homeomorphic to $S^n$.
\end{corollary}

\medskip

\begin{definition}
Let $I$ be an ideal in $R$. Define $\GL(I)$ as the kernel of the map $\GL(R) \to \GL\left(\faktor{R}{I}\right)$. Moreover, define $E(R, I)$ as the smallest normal subgroup of $E(R)$ that contains $E_{i, j}(r)$ for any $r\in I$ and $i\ne j$.
\end{definition}

\begin{prop}
$\left[\GL(I), \GL(I)\right] \subset E(R, I)\unlhd \GL(I)$
\end{prop}

\begin{definition}
The \textit{relative group} $K_1(R, I)$ is the the abelian group $\faktor{\GL(I)}{E(R, I)}$.
\end{definition}

\begin{remark}
Swan has shown that a ring homomorphism $f: R\to S$ mapping the ideal $I$ isomorphically to the ideal $J$ need \emph{not} induce an isomorphism $K_1(R, I) \to K_1(S, J)$.
\end{remark}

\begin{prop}
We have an exact sequence 
\[
\begin{tikzcd}
{K_1(R, I)} \arrow[r] & K_1(R) \arrow[r] & K_1\left(\faktor{R}{I}\right) \arrow[r] & K_0(I) \arrow[r] & K_0(R) \arrow[r] & K_0\left(\faktor{R}{I}\right)
\end{tikzcd}.\footnote{Section III.2.3  (Weibel).}
\]
\end{prop}

\section{$K_2$ for rings}

\begin{definition}
Let $n\geq 3$ and $R$ be a ring. The \textit{Steinberg group} $\st_n(R)$ is the group generated by the symbols $x_{ij}(r)$ with $1\leq i\ne j\leq n$ and $r\in R$ that satisfy the following relations.
\begin{enumerate}[label=(\roman*)]
\item $$x_{ij}(r)x_{ij}(s) = x_{ij}(r+s)$$
\item 
\[
\left[x_{ij}(r), x_{kl}(s)\right] = 
\begin{cases}
1 & j\ne k, \quad i\ne l \\
x_{il}(rs) & j= k, \quad i\ne l \\
x_{kj}(-sr) & j \ne k, \quad i=l
\end{cases} \].
\end{enumerate}
\end{definition}

\smallskip

We have a natural group surjection $\phi_n : \st_n(R) \to E_n(R)$ given by $x_{ij}(r) \mapsto E_{ij}(r)$. Moreover, there is a group map $\st_n(R) \hookrightarrow \st_{n+1}(R)$. Since $\st(R)\coloneqq \colim_n \st_n(R)$ exists, the $\phi_n$  form a group epimorphism $\phi : \st(R) \to E(R)$.
Let \begin{align*}
K_2(n, R) &= \ker \phi_n
\\K_2(R)  &= \ker \phi.
\end{align*} 

Note that
$K_2(-)$ is a functor  $\mathbf{Rng} \to \Ab$. Furthermore, we have an exact sequence
\[
\begin{tikzcd}
1 \arrow[r] & K_2(R) \arrow[r] & \st(R) \arrow[r, "\phi"] & \GL(R) \arrow[r] & K_1(R) \arrow[r] & 1
\end{tikzcd}.
\]


\begin{lemma}
$K_2(R) \cong Z(\st(R))$.
\end{lemma}
\begin{proof}
The fact that $K_2(R) \supset Z(\st(R))$ follows from the fact that $Z(E(R))$ is trivial. The reverse containment is easy but more tedious to prove. See III.5.2.1 (Weibel).
\end{proof}


\begin{exmp} A certain sort of Euclidean algorithm yields the following computations.
\begin{enumerate}
\item $K_2(\Z) \cong C_2$
\item $K_2(\Z[i]) =1$
\item $K_2(F) \cong K_2(F[t])$ when $F$ is a field
\end{enumerate}
\end{exmp}

\smallskip


\begin{theorem}
 Suppose that $R$ is Noetherian of dimension $d$. Then $K_2(n, R) \cong K_2(R)$ for any $n\geq d +3$.
\end{theorem}

\begin{theorem}
If $R$ and $S$ are Morita equivalent, then $K_2(R) \cong K_2(R)$.
\end{theorem}

\begin{exmp}
Let $n\in \Z_{\geq 1}$. Let $R$ be any ring and let $S= M_n(R)$. These are Morita equivalent, so that $$K_i(R) \cong K_i(M_n(R))$$ for each $i=0, 1, 2$. Indeed, in one direction, define $F: M \mapsto M^n$. In the other direction, define $G: M \mapsto e_{11}M$ where $e_11$ denotes the matrix with $1$ in position $(1, 1)$ and $0$ elsewhere. Define the natural isomorphism $\id_{\Mod_R} \Rightarrow G\circ F$ by the components $f_M : M \to \left\{(m, 0, \ldots, 0) : m \in M\right\}$. Further, define the natural isomorphism $\id_{\Mod_S} \Rightarrow F\circ G$ by the components  $g_M : M \to (e_{11}M)^n$ given by $m\mapsto \left(e_{11}m, \ldots, e_{1n}m\right)$. Hence $\Mod_R$ and $\Mod_S$ are equivalent, hence Morita equivalence as they are preadditive.
\end{exmp}

\begin{lemma}
Let $R$ be a commutative Banach algebra. Then there is a surjection from $K_2(R)$ onto $\pi_1\SL(R)$.\footnote{III.5.9 (Weibel).} 
\end{lemma}

\begin{exmp}
There is a surjection $K_2(\R) \to \pi_1 \SL_(\R) \cong \pi_1\SO \cong C_2$. Hence $K_2(\R)$ is nontrivial.
\end{exmp}

\begin{theorem}[Matsumoto 1969]
Let $F$ be a field. Then $K_2(F)$ is isomorphic to the free abelian group with system of generators $\{a, b\}$  satisfying the following relations.
\begin{enumerate}[label=(\roman*)]
\item $\{ac, b\} = \{a, b\}\{c, b\}$
\item $\{a, bd\} = \{a, b\}\{a, d\}$
\item $\{a, 1-a\}=1$ when $a \ne 1 \ne 1-a$.
\end{enumerate}
\end{theorem}

\begin{term}
The $\{a,b\}$ are called \textit{Steinberg symbols}.
\end{term}

\smallskip

Suppose that $A, B \in E(F)$ commute. Write $\phi(a) =A$ and $\phi(b) = B$. Then define $$A \bigstar B =[a,b] \in K_2(R).$$ If $a,b\in F$, then we can alternatively define the Steinberg symbol $$\{a, b\} = \begin{bmatrix}
    r & & \\
    & r^{{-1}} & \\
    & & 1
  \end{bmatrix}
 \bigstar
  \begin{bmatrix}
    s & & \\
    & 1 & \\
    & & s^{{-1}}
  \end{bmatrix}
 .$$

\begin{corollary}
$K_2(\F_p^n) =1$ for any prime $p$ and any integer $n\geq 1$.
\end{corollary}
\begin{proof}
The proof is entirely computational. See III.6.1.1 (Weibel).
\end{proof}

\begin{prop}
If $F \supset \Q(t)$, then $\left\lvert{K_2(F)}\right\rvert = \left\lvert{F}\right\rvert$.
\end{prop}

\end{document}
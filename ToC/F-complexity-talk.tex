\documentclass{beamer}
\newcommand\bmmax{0}
\newcommand\hmmax{0}
\usepackage[utf8]{inputenc}
\usepackage{graphicx}
\usepackage{tikz-cd}
\usepackage{pgfplots}
\usepackage{amsfonts}
\usepackage{amssymb}
\usepackage{amsmath,amsthm}
\usepackage{lmodern}
\usepackage{faktor}
\usepackage{extpfeil}
\usepackage{xfrac}
\usepackage{mathtools}
\usepackage{bm}
\usepackage{xcolor,pict2e}
\usepackage{soul}
\usepackage{mathpartir}
\usepackage{ dsfont }
\usepackage{mathrsfs}
\usepackage{hyperref}
\usepackage{url}
\usepackage{quiver}
\usepackage{thmtools}
\usepackage[capitalise]{cleveref}
\usepackage{scrextend}
\usepackage{mdframed}
\usepackage{adjustbox}
\usepackage{comment}
\usepackage{tcolorbox}
\usetikzlibrary{calc,decorations.pathmorphing,shapes}
\newcommand{\vequiv}{\rotatebox[origin=c]{-90}{$\equiv$}}

\makeatletter
\renewcommand*\env@matrix[1][*\c@MaxMatrixCols c]{%
  \hskip -\arraycolsep
  \let\@ifnextchar\new@ifnextchar
  \array{#1}}
\makeatother
\DeclareMathOperator{\Ima}{Im}
\newcommand{\A}{\mathcal A}
\newcommand{\C}{\mathbb C}
\newcommand{\D}{\mathcal D}
\newcommand{\E}{\vec E}
\newcommand{\CP}{\mathbb CP}
\newcommand{\F}{\mathbb F}
\newcommand{\G}{\vec G}
\renewcommand{\H}{\mathbb H}
\newcommand{\HP}{\mathbb HP}
\newcommand{\K}{\mathcal K}
\renewcommand{\L}{\mathcal L}
\newcommand{\RI}{\mathcal R}
\newcommand{\M}{\mathbb M}
\renewcommand{\O}{\mathbf O}
\newcommand{\OP}{\mathbb OP}
\renewcommand{\P}{\mathbf P}
\newcommand{\Q}{\mathbb Q}
\newcommand{\I}{\mathbb I}
\newcommand{\un}{\mathbf{U}}
\newcommand{\R}{\mathbb R}
\newcommand{\RP}{\mathbb RP}
\renewcommand{\S}{\mathbf S}
\newcommand{\X}{\mathbf X}
\newcommand{\Z}{\mathbb Z}
\newcommand{\B}{\mathcal{B}}
\newcommand{\ds}{\displaystyle}
\newcommand{\ran}{\right>}
\newcommand{\lan}{\left<}
\newcommand{\bmat}[1]{\begin{bmatrix} #1 \end{bmatrix}}
\newcommand{\sra}{\shortrightarrow}
\newcommand{\hooklongrightarrow}{\lhook\joinrel\longrightarrow}

\newcommand{\h}{\vec h}
\newcommand{\f}{\vec f}
\newcommand{\g}{\vec g}
\renewcommand{\i}{\vec i}
\renewcommand{\k}{\vec k}
\newcommand{\n}{\vec n}
\newcommand{\p}{\vec p}
\newcommand{\q}{\vec q}
\renewcommand{\r}{\vec r}
\newcommand{\s}{\vec s}
\renewcommand{\t}{\vec t}
\renewcommand{\u}{\vec u}
\renewcommand{\v}{\vec v}
\newcommand{\w}{\vec w}
\newcommand{\x}{\vec x}
\newcommand{\z}{\vec z}
\DeclareMathOperator*{\Span}{span}
\DeclareMathOperator*{\GL}{GL}
\DeclareMathOperator*{\SL}{SL}
\DeclareMathOperator*{\SO}{SO}
\DeclareMathOperator*{\SU}{SU}
\DeclareMathOperator{\rng}{range}
\DeclareMathOperator{\ft}{ft}
\DeclareMathOperator{\gemu}{gemu}
\DeclareMathOperator{\almu}{almu}
\DeclareMathOperator{\Char}{\mathsf{char}}
\DeclareMathOperator{\im}{im}
\DeclareMathOperator{\graph}{Graph}
\DeclareMathOperator{\gal}{Gal}
\DeclareMathOperator{\tr}{Tr}
\DeclareMathOperator{\norm}{N}
\DeclareMathOperator{\aut}{Aut}
\DeclareMathOperator{\Int}{Int}
\DeclareMathOperator{\ext}{Ext}
\DeclareMathOperator{\stab}{Stab}
\DeclareMathOperator{\orb}{Orb}
\DeclareMathOperator{\inn}{Inn}
\DeclareMathOperator{\out}{Out}
\DeclareMathOperator{\fix}{Fix}
\DeclareMathOperator{\ab}{ab}
\DeclareMathOperator{\sgn}{sgn}
\DeclareMathOperator{\syl}{syl}
\DeclareMathOperator{\Syl}{Syl}
\DeclareMathOperator{\ob}{Ob}
\DeclareMathOperator{\mor}{Mor}
\DeclareMathOperator{\sym}{sym}
\DeclareMathOperator{\red}{red}
\DeclareMathOperator{\ev}{ev}
\DeclareMathOperator{\colimm}{\mathsf{colim}}
\DeclareMathOperator{\limm}{\mathsf{lim}}
\DeclareMathOperator{\Ty}{Ty}
\DeclareMathOperator{\Tm}{Tm}
\DeclareMathOperator{\plus}{\mathtt{+}}



%Our function symbols for our signature (not including universe closure symbols, \mathsf{0}, or lambda symbol):
\newcommand{\J}{\mathsf{J}}
\newcommand{\id}{\mathsf{Id}}
\newcommand{\refl}{\mathsf{refl}}
\newcommand{\app}{\mathsf{app}}
\renewcommand{\split}{\mathsf{split}}
\newcommand{\ind}{\mathsf{ind}}
\newcommand{\pair}{\mathsf{pair}}
\newcommand{\case}{\mathsf{case}}
\newcommand{\U}{\mathcal{U}}
\newcommand{\el}{\mathsf{el}}
\newcommand{\univv}{\mathsf{univ}}
\newcommand{\0}{\mathbf{0}}
\newcommand{\1}{\mathbf{1}}
\newcommand{\2}{\mathbf{2}}

%Primitive strings in our meta-language:
\DeclareMathOperator{\ctx}{\mathtt{ctx}}
\DeclareMathOperator{\type}{\mathtt{type}}

%Our abbreviations used in our meta-language:

\DeclareMathOperator{\inv}{\mathtt{inv}}
\DeclareMathOperator{\isprop}{\mathtt{is\_prop}}
\DeclareMathOperator{\retr}{\mathtt{retr}}
\DeclareMathOperator{\sect}{\mathtt{sec}}
\DeclareMathOperator{\ac}{\mathtt{ac}}
\DeclareMathOperator{\assoc}{\mathtt{assoc}}
\DeclareMathOperator{\lunit}{\mathtt{l\_unit}}
\DeclareMathOperator{\runit}{\mathtt{r\_unit}}
\DeclareMathOperator{\linv}{\mathtt{l\_inv}}
\DeclareMathOperator{\rinv}{\mathtt{r\_inv}}
\DeclareMathOperator{\concat}{\mathtt{concat}}
\DeclareMathOperator{\cons}{\mathtt{cons}}
\DeclareMathOperator{\comp}{\mathtt{comp}}
\DeclareMathOperator{\idmap}{\mathtt{idmap}}
\DeclareMathOperator{\transport}{\mathtt{transport}}
\DeclareMathOperator{\ap}{\mathtt{ap}}
\DeclareMathOperator{\apd}{\mathtt{apd}}
\DeclareMathOperator{\happly}{\mathtt{hApply}}
\DeclareMathOperator{\isequiv}{\mathtt{is\_equiv}}
\DeclareMathOperator{\hfiber}{\mathtt{hFiber}}
\DeclareMathOperator{\iscont}{\mathtt{is\_contr}}
\DeclareMathOperator{\iscontmap}{\mathtt{is\_contr\_map}}
\DeclareMathOperator{\iso}{\mathtt{iso}}
\DeclareMathOperator{\hpyconcat}{\mathtt{htpy\_concat}}
\DeclareMathOperator{\pr}{\mathtt{pr}}
\DeclareMathOperator{\isset}{\mathtt{is\_set}}
\DeclareMathOperator{\isgrp}{\mathtt{is\_group}}
\DeclareMathOperator{\isoeq}{\mathtt{iso\_eq}}
\DeclareMathOperator{\homm}{\mathtt{hom}}
\DeclareMathOperator{\equiveq}{\mathtt{idtoequiv}}

\DeclareMathOperator{\fv}{FV}

\DeclareMathOperator{\cdtt}{\mathrm{CDTT}}
\DeclareMathOperator{\univ}{\mathrm{Univ}}
\DeclareMathOperator{\wfe}{\mathrm{WFE}}
\DeclareMathOperator{\sfe}{\mathrm{FE}}
\DeclareMathOperator{\uip}{\mathrm{UIP}}
\DeclareMathOperator{\err}{\mathrm{ERR}}




\DeclareMathOperator{\op}{op}
\DeclareMathOperator{\sset}{\mathbf{sSet}}
\DeclareMathOperator{\expr}{exp}
\DeclareMathOperator{\set}{\mathbf{Set}}
\DeclareMathOperator{\Ab}{\mathbf{Ab}}
\DeclareMathOperator{\Cmon}{\mathbf{CMon}}
\DeclareMathOperator{\spec}{Spec}
\DeclareMathOperator{\rank}{rank}
\DeclareMathOperator{\rk}{rk}
\DeclareMathOperator{\hocolimm}{hocolim}
\DeclareMathOperator{\holimm}{holim}
\DeclareMathOperator{\diag}{diag}
\DeclareMathOperator{\Ar}{Arr}
\DeclareMathOperator{\pb}{Pb}
\DeclareMathOperator{\trr}{tr}
\DeclareMathOperator{\sk}{sk}
\DeclareMathOperator{\Sk}{Sk}
\DeclareMathOperator{\Lan}{Lan}
\DeclareMathOperator{\fp}{fp}

\makeatletter
\newcommand{\colim}{\gen@colim{\colimm}}
\newcommand{\gen@colim}[1]{%
  \@ifnextchar_{\gen@@colim{#1}}{\mathbin{#1}}%
}
\def\gen@@colim#1_#2{%
  \mathpalette\gen@@@colim{{#1}{#2}}%
}
\newcommand\gen@@@colim[2]{\mathbin{\gen@@@@colim#1#2}}
\newcommand\gen@@@@colim[3]{%
  \ifx#1\displaystyle
    \mathop{#2}\limits_{#3}%
  \else
    {#2}_{#3}%
  \fi
}
\makeatother

\makeatletter
\newcommand{\hocolim}{\gen@colim{\hocolimm}}
\newcommand{\gen@hocolim}[1]{%
  \@ifnextchar_{\gen@@hocolim{#1}}{\mathbin{#1}}%
}
\def\gen@@hocolim#1_#2{%
  \mathpalette\gen@@@hocolim{{#1}{#2}}%
}
\newcommand\gen@@@hocolim[2]{\mathbin{\gen@@@@hocolim#1#2}}
\newcommand\gen@@@@hocolim[3]{%
  \ifx#1\displaystyle
    \mathop{#2}\limits_{#3}%
  \else
    {#2}_{#3}%
  \fi
}
\makeatother

\makeatletter
\newcommand{\holim}{\gen@colim{\holimm}}
\newcommand{\gen@holim}[1]{%
  \@ifnextchar_{\gen@@holim{#1}}{\mathbin{#1}}%
}
\def\gen@@holim#1_#2{%
  \mathpalette\gen@@@holim{{#1}{#2}}%
}
\newcommand\gen@@@holim[2]{\mathbin{\gen@@@@holim#1#2}}
\newcommand\gen@@@@holim[3]{%
  \ifx#1\displaystyle
    \mathop{#2}\limits_{#3}%
  \else
    {#2}_{#3}%
  \fi
}
\makeatother

\newcommand{\RomanNumeralCaps}[1]
    {\MakeUppercase{\romannumeral #1}}

\renewcommand{\a}{\mathscr{A}}
\renewcommand{\b}{\mathscr{B}}
\renewcommand{\c}{\mathscr{C}}
\renewcommand{\d}{\mathscr{D}}
\newcommand{\e}{\mathscr{E}}
\renewcommand{\j}{\mathscr{J}}
\newcommand{\y}{\mathscr{Y}}
\newcommand{\rr}{\mathscr{R}}

\newcommand{\N}{\mathbb N}
\newcommand{\T}{\mathbb T}

\DeclareMathOperator{\Sp}{Sp}
\DeclareMathOperator{\Hom}{Hom}
\DeclareMathOperator{\Fun}{Fun}
\DeclareMathOperator{\cone}{cone}
\DeclareMathOperator{\idd}{id}
\DeclareMathOperator{\fib}{\mathnormal{Fib}}
\DeclareMathOperator{\cof}{\mathnormal{Cof}}
\DeclareMathOperator{\we}{\mathnormal{W}}
\DeclareMathOperator{\dom}{dom}
\DeclareMathOperator{\cod}{cod}
\DeclareMathOperator{\cell}{cell}
\DeclareMathOperator{\ret}{ret}
\DeclareMathOperator{\rel}{rel}
\DeclareMathOperator{\rlp}{rlp}
\DeclareMathOperator{\llp}{llp}
\DeclareMathOperator{\nondeg}{nondeg}
\DeclareMathOperator{\trans}{\mathnormal{trans}}
\DeclareMathOperator{\ch}{\mathbf{Ch}}
\DeclareMathOperator{\Mod}{\mathbf Mod}



\newcounter{sarrow}
\newcommand\xrsquigarrow[1]{%
\stepcounter{sarrow}%
\mathrel{\begin{tikzpicture}[baseline= {( $ (current bounding box.south) + (0,-0.1ex) $ )}]
\node[inner sep=.5ex] (\thesarrow) {$\scriptstyle #1$};
\path[draw,<-,decorate,
  decoration={zigzag,amplitude=0.7pt,segment length=1.2mm,pre=lineto,pre length=4pt}] 
    (\thesarrow.south east) -- (\thesarrow.south west);
\end{tikzpicture}}%
}


\def \RightTirName #1{\rm\hbox {\hskip 1ex (#1)}}

\newcommand{\bi}{\begin{itemize}}
\newcommand{\ei}{\end{itemize}}

\newcommand{\be}{\begin{enumerate}}
\newcommand{\ee}{\end{enumerate}}

\newcommand{\bmp}{\begin{mathpar}}
\newcommand{\emp}{\end{mathpar}}

\setlength{\parindent}{0pt}

\makeatletter
\def\th@plain{%
  \thm@notefont{}% same as heading font
  \itshape % body font
}
\def\th@definition{%
  \thm@notefont{}% same as heading font
  \normalfont % body font
}
\makeatother

\setbeamertemplate{itemize items}[circle]

%Information to be included in the title page:
\title{The complexity of type inference for System F}
\author{Perry Hart}
\institute{UMN PL Seminar}
\date{March 29, 2023}

\AtBeginSection[]
{
  \begin{frame}
    \frametitle{Table of Contents}
    \tableofcontents[currentsection]
  \end{frame}
}

\usepackage{multirow}


\usecolortheme{seahorse}


\setbeamertemplate{sidebar right}{}
\setbeamertemplate{footline}{%
\hfill\usebeamertemplate***{navigation symbols}
\hspace{1cm}\insertframenumber{}/\inserttotalframenumber}
\logo{\includegraphics[scale=.4]{logo-ic}}

\begin{document}

\frame{\titlepage}

\begin{frame}
\frametitle{Overview}

Let $\mathbb{T}$ denote a type system. 

\medskip

\textbf{Type inference:}  Given a raw term $t$ of $\mathbb{T}$, decide whether $t$ is typable.

\medskip

\begin{theorem}
\begin{enumerate}
\item Type inference for STLC is decidable in polynomial time.
\item Type inference for Core ML is decidable in $O(2^n)$ time and is $\mathsf{EXPTIME}$-hard under logspace reduction.
\end{enumerate}
\end{theorem}

\end{frame}

\begin{frame}

The more expressive the language, the harder the decision problem.

\smallskip

\begin{example}
Deciding whether a propositional sentence is true is $\mathsf{PTIME}$-complete.

\medskip

Deciding whether $\exists{x}.\varphi(x)$ with $\varphi(x)$ a propositional formula is $\mathsf{NP}$-complete.
\end{example}

\bigskip

\begin{theorem}[Henglein and Mairson]
Type inference for System F is $\mathsf{EXPTIME}$-hard under logspace reduction. 
\end{theorem}

\begin{corollary}
Type inference for System F requires exponential time.
\end{corollary}

\end{frame}

\begin{frame}
\frametitle{Proof strategy}

For each integer $k \geq 1$, find a logspace reduction
\[
\text{total TM $M$ and string $x$ of length $n$} \ \ \mapsto \ \  \text{raw term $\Psi_k(M, x)$}
,\]
where $M$ rejects $x$ in $2^{n^k}$ steps if and only if $\Psi_k(M, x)$ is non-typable.

\bigskip

\emph{Such a reduction is known for Core ML.}

\smallskip

\begin{block}{Difficulty}
The non-typable term $\Psi_k(M, x)$ in Core ML could be typable in System F.
\end{block}
\end{frame}

\begin{frame}
Construct a term $A_k(M,x)$ of System F in logspace such that 
\begin{align*}
&   \text{$A_k(M,x) \to^{\ast} \mathbf{false}$ when $M$ accepts $x$ in $2^{n^k}$ steps}
\\ & \text{$A_k(M,x) \to^{\ast} \mathbf{true}$ when $M$ rejects $x$ in $2^{n^k}$ steps}.
\end{align*} 

\smallskip

Take
 \[\adjustbox{scale =.9}{$
\Psi_k(M,x) \ \coloneqq \  \left(\Lambda{\alpha}.\lambda(x : \alpha).xx\right)\left(A_k(M,x)\left(\Lambda{\alpha}.\lambda(x : \alpha).x\right)\left(\Lambda{\alpha}.\lambda(y : \alpha).yy\right) \right)$}
\] \pause

\smallskip

If $A_k(M,x) \to^{\ast} \mathbf{false}$, then 
\[
\Psi_k(M,x) \ \longrightarrow^{\ast} \ \underbrace{\left(\Lambda{\alpha}.\lambda(x : \alpha).xx\right)\left(\Lambda{\alpha}.\lambda(y : \alpha).yy\right)}_{\text{non-terminating}}
.\]

\smallskip

If $A_k(M,x) \to^{\ast} \mathbf{true}$, then we must show that $\Psi_k(M,x)$ is typable.
\end{frame}

\begin{frame}
\frametitle{Encoding Turing machines with System F}

Let $M$ be a TM.

\medskip

\medskip

The $n$-th configuration $\mathsf{ID}_{M,x}(n)$ of $M$ on input $x$ is the triple $\left(q, L, R\right)$ where, after exactly $n$ steps, 
\begin{itemize}
\item $q$ is the state of $M$, 
\item $L$ is the (finite) list of symbols to the left of the read head, and 
\item $R$ the list of symbols at or to the right of the read head. 
\end{itemize}
\end{frame}

\begin{frame}
We can find a $\lambda$-term $\delta_M$ in logspace such that
\[
\delta_M \ \overline{\mathsf{ID}_{M,x}(n)} \ \  \longrightarrow^{\ast} \ \  \overline{\mathsf{ID}_{M,x}(n+1)}
\] for all $n \in \N$. \pause

\bigskip

Moreover, the term  
\[
 \left(\overline{2^{n^k}}[T_{\mathsf{ID}_{M,x}}] \ \delta_M\right) \overline{\mathsf{ID}_{M,x}(0)}
\] has normal form $\overline{\mathsf{ID}_{M,x}(2^{n^k})}$. If $\left(q, L, R\right) = \mathsf{ID}_{M,x}(2^{n^k})$, then take
\[
A_k(M,x) \ \coloneqq \ \overline{q}[\mathsf{Bool}] \ \mathit{false}  \cdots \mathit{false} \ \underbrace{\mathit{true} \cdots \mathit{true}}_{\text{$\left\lvert{Q_F}\right\rvert$ copies}}
.\]
\end{frame}

\begin{frame}
If $M$ accepts $x$ in $2^k$ steps, then $A_k(M,x)$ reduces to $\mathbf{true}$. 

\medskip

Therefore,
 \[\adjustbox{scale =.9}{$
 \left(\lambda(x : \tau).x[\tau]x\right)\left(A_k(M,x)[\tau][\tau \to \tau]\left(\Lambda{\alpha}.\lambda(x : \alpha).x\right)\left(\lambda(y : \tau).y[\tau]y\right) \right)$}
\tag{$\tau \coloneqq \forall{\alpha}.\alpha \to \alpha$}
\] has type $\tau$, so that $\Psi_k(M,x)$ is typable. \pause

\bigskip

Since $A_k(M,x)$ can be constructed in logspace, so can $\Psi_k(M,x)$.
\end{frame}

\begin{frame}
\frametitle{Related questions}

\begin{itemize}
\item The decidability of type inference for System F was an open question until 2012, then answered in the negative by Fujita and Schubert.

\bigskip

\item Pfenning proved that partial type inference for System F is undecidable.

\bigskip

\item The set of all strongly normalizing terms of System F is $\mathsf{RE}$-complete (hence undecidable).

\end{itemize}

\end{frame}

\end{document}